\section{Erinnerung und Plan}

\begin{frame}
  {Letzte Woche | Grammatik}
  \onslide<+->
  \begin{itemize}[<+->]
    \item Kompositionalität
      \begin{itemize}[<+->]
        \item Größere sprachliche Einheiten sind verstehbar,\\
          weil sie aus kleineren Einheiten regelhaft zusammengesetzt werden. 
      \end{itemize}
      \Zeile
    \item Grammatikalität
      \begin{itemize}[<+->]
        \item Ein Satz ist grammatisch relativ zu einer Grammatik,\\
          wenn er den Regeln dieser Grammatik entspricht.
      \end{itemize}
      \Zeile
    \item Akzeptabilität
      \begin{itemize}[<+->]
        \item Ein Satz ist akzeptabel, wenn Sprecher ihn als akzeptabel finden.\\
          Unsicherheiten in den Urteilen deuten darauf hin,\\
          dass die kognitive Grammatik entweder unscharf ist\\
          oder wir nicht immer perfekt darauf zugreifen können.
      \end{itemize}
  \end{itemize}
\end{frame}


\begin{frame}
  {Diese Woche}
  \onslide<+->
  \begin{itemize}[<+->]
    \item Kern und Peripherie
    \Zeile
    \item Regel, Regularität und Norm
      \Zeile
    \item \citet[Kap.~1]{Schaefer2018b}
  \end{itemize}
\end{frame}


\section{Kern und Peripherie}

\begin{frame}
  {Kern und Peripherie}
  \pause
\begin{exe}
  \ex\label{ex:kernundperipherie022}
    \begin{xlist}
      \ex \alert{Baum, Haus, Matte, Döner, Angst, Öl, Kutsche, \ldots}
      \ex \rot{System, Kapuze, Bovist, Schlamassel, Marmelade, Melodie, \ldots}
    \end{xlist}
    \pause
    \ex
    \begin{xlist}
      \ex \alert{geht, läuft, lacht, schwimmt, liest, \ldots}
      \ex \rot{kann, muss, will, darf, soll, mag}
    \end{xlist}
    \pause
    \ex
    \begin{xlist}
      \ex \alert{des Hundes, des Geistes, des Tisches, des Fußes, \ldots}
      \ex \rot{des Schweden, des Bären, des Prokuristen, des Phantasten, \ldots}
    \end{xlist}
  \end{exe}
  \pause
  \vspace{\baselineskip}
  \Large
  \centering
  \alert{Hohe Typenhäufigkeit} vs.\ \rot{niedrige Typenhäufigkeit}.  
\end{frame}

\begin{frame}
  {Zwei verschiedene Häufigkeiten}
  \pause
  \Large\begin{block}{Typenhäufigkeit}
    Wie viele \alert{verschiedene} Realisierungen (=~Typen)\\
    einer Sorte linguistischer Einheiten gibt es?
  \end{block}

  \pause
  \vspace{\baselineskip}
  
  \begin{block}{Tokenhäufigkeit}
    Wie häufig sind die \alert{ggf.\ identischen} Realisierungen\\
    (=~Tokens) einer Sorte linguistischer Einheiten?
  \end{block}
\end{frame}


\section{Regel, Regularität, Norm}

\begin{frame}
  {Regel vs.\ Regularität bzw.\ Generalisierung}
  \pause
  \begin{exe}
    \ex
    \begin{xlist}
      \ex{Relativsätze und eingebettete \textit{w}-Sätze werden nicht\\
    durch Komplementierer eingeleitet.}
      \pause
      \ex{\textit{fragen} ist ein schwaches Verb.}
      \pause
      \ex{\textit{zurückschrecken} bildet das Perfekt mit dem Hilfsverb \textit{sein}.}
      \pause
      \ex{Im Aussagesatz steht vor dem finiten Verb genau ein Satzglied.}
      \pause
      \ex{In Kausalsätzen mit \textit{weil} steht das finite Verb an letzter Stelle.}
    \end{xlist}
  \end{exe}
\end{frame}


\begin{frame}
  {Normkorm? Regularitätenkonform?}
  \pause
  \begin{exe}
    \ex
    \begin{xlist}
      \ex Dann sieht man auf der ersten Seite \alert{wann, wo und wer} \rot{dass} kommt.
      \pause
      \ex Er \rot{frägt} nach der Uhrzeit.
      \pause
      \ex Man \rot{habe} zu jener Zeit nicht vor Morden \alert{zurückgeschreckt}.
      \pause
      \ex \rot{Der Universität} \alert{zum Jubiläum} gratulierte auch Bundesminister Dorothee Wilms, die in den fünfziger Jahren in Köln studiert hatte.
      \pause
      \ex Das ist Rindenmulch, \alert{weil} hier \rot{kommt} noch ein Weg.
    \end{xlist}
  \end{exe}
\end{frame}


\begin{frame}
  {Regel und Regularität}
  \pause
  \begin{block}{Regularität}
    Eine grammatische Regularität innerhalb eines Sprachsystems liegt dann vor, wenn sich Klassen von Symbolen unter vergleichbaren Bedingungen gleich (und damit vorhersagbar) verhalten.
  \end{block}

  \pause
  \vspace{0.5\baselineskip}

  \begin{block}{Regel}
    Eine grammatische Regel ist die Beschreibung einer Regularität, die in einem normativen Kontext geäußert wird.
  \end{block}

  \pause
  \vspace{0.5\baselineskip}
  
  \begin{block}{Generalisierung}
    Eine grammatische Generalisierung ist eine durch Beobachtung zustandegekommene Beschreibung einer Regularität.
  \end{block}
\end{frame}

\begin{frame}
  {Regel vs.\ Regularität bzw.\ Generalisierung}
  Was ist dann der Status dieser Feststellungen?\\
  \Zeile
  \begin{exe}
    \ex
    \begin{xlist}
      \ex{Relativsätze und eingebettete \textit{w}-Sätze werden nicht\\
    durch Komplementierer eingeleitet.}
      \ex{\textit{fragen} ist ein schwaches Verb.}
      \ex{\textit{zurückschrecken} bildet das Perfekt mit dem Hilfsverb \textit{sein}.}
      \ex{Im Aussagesatz steht vor dem finiten Verb genau ein Satzglied.}
      \ex{In Kausalsätzen mit \textit{weil} steht das finite Verb an letzter Stelle.}
    \end{xlist}
  \end{exe}
\end{frame}



\begin{frame}
  {Norm ist Beschreibung}
  \pause
  \begin{itemize}[<+->]
    \item Norm als Grundkonsens
    \item Sprache und Norm im Wandel
    \item Norm und Situation (Register, Stil, \dots)
    \item Variation in der Norm
      \vspace{\baselineskip}
    \item \alert{Wichtigkeit der Norm, insbesondere im schulischen Deutschunterricht}
  \end{itemize}
\end{frame}


