\begin{frame}
  {Merken Sie sich unbedingt Folgendes!}
  \onslide<+->
  \onslide<+->
  \centering 
  \rot{\large Fragen}\\\Halbzeile\onslide<+->
  \rot{\large Sie}\\\Halbzeile\onslide<+->
  \rot{\Large mich}\\\Halbzeile\onslide<+->
  \rot{\LARGE niemals,}\\\Halbzeile\onslide<+->
  \rot{\huge was}\\\Halbzeile\onslide<+->
  \rot{\huge in}\\\Halbzeile\onslide<+->
  \rot{\huge der}\\\Halbzeile\onslide<+->
  \rot{\huge Klausur}\\\Halbzeile\onslide<+->
  \rot{\Huge drankommt}\onslide<+->%
  \rot{\Huge !}\onslide<+->%
  \rot{\Huge !}\onslide<+->%
  \rot{\Huge !}\onslide<+->%
\end{frame}


\section{Anforderungen}

\begin{frame}
  {Anforderungen an Sie}
  \onslide<+->
  \onslide<+->
  \centering 
  Im Prinzip gibt es nur eine einzige Anforderung:\\
  \onslide<+->
  \Doppelzeile
  \alert{\Large Ich erwarte, dass Sie sich für das Fach interessieren,\\
    \Viertelzeile
  für dessen Studium Sie sich entschieden haben.}
\end{frame}

\begin{frame}
  {Sie studieren Germanistik als Fachwissenschaft?}
  \onslide<+->
  \onslide<+->
  Damit haben Sie sich für eine typische \alert{Philologie} entschieden.\\
  \Zeile
  \begin{itemize}[<+->]
    \item Laut Duden: \alert{"`\textit{Wissenschaft, die sich mit der Erforschung von Texten\\
      in einer bestimmten Sprache beschäftigt; Sprach- und Literaturwissenschaft}"'}.
      \Halbzeile
    \item Ursprung der Germanistik\slash Deutschen Philologie: bekannte Philologen\\
      im 19.~Jahrhundert, z.\,B.\ Jakob (1785--1863) und Wilhelm Grimm (1786--1859)
    \Halbzeile
  \item Typische Leistungen \grau{(ganz anders als z.\,B.\ AVL)}:
      \begin{itemize}[<+->]
        \item Erforschung der aktuellen Sprache (Grammatik; heute auch Semantik, Pragmatik usw.)
        \item Beschreibende Dialektologie
        \item Erforschung älterer Sprachstufen (Mediävistik)
        \item Anschluss der Germanistik an die Indogermanistik
        \item Erstellung von Wörterbüchern und Grammatiken
        \item Beschäftigung mit Literatur (Literaturgeschichte; Editionen)
        \item Literaturwissenschaft -- insbesondere im Sinn der hermeneutischen \\
          Textinterpretation -- war lange nachrangig (eher 20.~Jahrhundert).
      \end{itemize}
  \end{itemize}
\end{frame}

\begin{frame}
  {Sie studieren Deutsch Lehramt?}
  \onslide<+->
  \onslide<+->
  Sie möchten den \alert{Bildungsspracherwerb} von Kindern\slash Jugendlichen fördern.\\
  Die Anforderungen an Sie ergeben sich aus den \alert{Zielkompetenzen Ihrer Schüler}.\\
  \onslide<+->
  \Zeile
  \begin{block}
    {Zielkompetenzen \textit{Deutsch} 5.--11.~Klasse (Thüringer RLP 2019; S.~7)}
    \begin{enumerate}[<+->]
      \item Texte rezipieren
      \item Texte produzieren
      \item \alert{Über Sprache, Sprachverwendung und Sprachenlernen reflektieren}
    \end{enumerate}
  \end{block}
\end{frame}

\begin{frame}
  {Was bedeutet das?}
  \onslide<+->
  \onslide<+->
  Im Sinne der Literatur der letzten Jahrzehnte: Erwerb von \alert{Bildungssprache}\\
  \Zeile
  \begin{itemize}[<+->]
    \item Sprache in \alert{Lehr-, Lern-, Bildungskontexten}
    \item Erforderlich in \alert{beruflichen Situationen} und \alert{vielen Alltagssituationen}
    \item Darstellung \alert{komplexer Sachverhalte}
    \item Darstellung \alert{situationsunabhängiger Sachverhalte} (Dekontextualisierung)
    \item Darstellung \alert{hypothetischer Sachverhalte} (Dekontextualisierung)
    \item \alert{Argumentierendes Sprechen}
    \item Aufbau \alert{kohärenter Texte}
    \item Stark, aber nicht exklusiv an \alert{Schriftsprache} gebunden
    \item \gruen{Mit konkreten sprachlichen\slash grammatischen Formen verknüpft}\\
      \grau{\footnotesize{\citet{GogolinLange2011,Feilke2012,Feilke2019} usw.}}
    \item \rot{Nicht alle lernen die gleiche Sprache gleich erfolgreich!}\\
      \grau{\footnotesize{\citet{Dabrowska1997,DabrowskaStreet2006,Dabrowska2012,Dabrowska2015,Dabrowska2018} usw.}}
  \end{itemize}
\end{frame}

\begin{frame}
  {Wie erwirbt man Bildungssprache?}
  \onslide<+->
  \onslide<+->
  Typischerweise mit dem Erwerb der \alert{Schriftsprache} und durch \alert{Sprachreflexion}:\\
  \textit{Kompletter Umbau der Grammatik\slash Sprache des Kindes durch Deutschunterricht}\\
    \grau{\footnotesize\citet{Eisenberg2004,Bredel2013}}\\
  \Halbzeile
  \begin{itemize}[<+->]
    \item Wissen darüber, \alert{welche sprachlichen Anforderungen in\\
      sprachgebundenen Aufgaben stecken} \grau{\citep[4]{Eisenberg2004}}
    \item Den \alert{eigenen Sprachgebrauch} in ein Verhältnis [\ldots] zu den \alert{Möglichkeiten\\
      der Sprache überhaupt} [\ldots] setzen \grau{\citep[167]{Ossner2007}}
    \item Einsicht in fundamentale sprachliche Regularitäten und in die \alert{Strukturiertheit sprachlicher Phänomene} \grau{\citep[72]{Portmanntselikas2011}}
    \item \grau{Siehe auch \citet{SchaeferSayatz2017a,Schaefer2018}}
  \end{itemize}
  \Halbzeile
  \centering
  \onslide<+->
  \rot{[Dieses] durch \textit{Reflexion über Sprache} [\ldots] [geforderte besondere Verhältnis zur Sprache] können die Schüler nicht entwickeln, wenn es die Lehrer nicht haben.} \grau{\citep[23]{Eisenberg2004}}
\end{frame}

\begin{frame}
  {Zu vermittelnde Kompetenzen (nur Bereich 3 \textit{Sprache reflektieren})}
  \onslide<+->
  \onslide<+->
  \rot<6->{Dafür haben Sie \alert<1-5>{nur sieben Jahre Zeit}\onslide<6->{?}} (RLP Thüringen 2019, S.~25)\\
  \onslide<+->
  \Halbzeile
  \begin{itemize}\tiny
    \item Wörter nach Wortarten unterscheiden, deren Formmerkmale anwenden und Bezüge zur Satzkonstruktion herstellen,
    \item Wortbausteine bestimmen und nutzen,
    \item Wortfamilien und Wortfelder bilden,
    \item Satzarten sicher unterscheiden,
    \item Satzzeichen und Kommas bei Aufzählungen setzen,
    \item Zeichen der wörtlichen Rede sicher setzen,
    \item Satzglieder bestimmen,
    \item grundlegende sprachliche Strukturen und Fachbegriffe verwenden,
    \item sprachliche Verständigung reflektieren,
    \item Unterschiede gesprochener und geschriebener Sprache reflektieren,
    \item sein sprachliches Wissen auf das Lernen einer Fremdsprache übertragen und umgekehrt,
    \item mit Sprache spielerisch und experimentell umgehen,
    \item Sprache situationsangemessen und bewusst anwenden,
    \item durch selbstständiges Üben sein Sprachwissen festigen,
    \item über sprachliche und nicht sprachliche Phänomene nachdenken,
    \item Toleranz gegenüber fremden Sprachen zeigen,
    \item seine Kompetenzentwicklung einschätzen.
  \end{itemize}
  \onslide<+->
  \onslide<+->
  \onslide<+->
  \Halbzeile
  \centering 
  \rot{\Large Falsch! Das soll die Lernausgangslage nach der vierten Klasse sein!}
\end{frame}

\begin{frame}
  {Wie klappt das mit der Grammatik-Ausbildung im Studium?}
  \onslide<+->
  \onslide<+->
  \centering 
  \alert{\large 48 \% der Befragten fühlen sich durch Ausbildung nicht hinreichend\\
  auf den Grammatikunterricht vorbereitet}\\
  \Viertelzeile
  \grau{\footnotesize\citet[76]{TopalovicDuenschede2014}, eine Studie mit 1.017 Lehrkräften}\\
  \Zeile
  \onslide<+->
  \rot{\large Und der Grammatikunterricht ist sogar nur \ul{eine} der konkreten Aufgaben.}\\
  \raggedright
  \Zeile
  \begin{itemize}[<+->]
    \item \rot{Bewerten} sprachlicher Leistungen
    \item \rot{Erklären} der Bewertung
    \item Auf Basis der Grammatik \rot{Bildungssprache} vermitteln
    \item \rot{Lernprobleme erkennen} und Schüler \rot{individuell fördern}
    \item \grau{Eine Weitervermittlung von Schülern zum Schreib-\slash Lese-Förderunterricht,\\
      sobald etwas nicht funktioniert, ist keine Förderleistung Ihrerseits.}
    \item \rot{Deutsche Sprache vermitteln} bei nicht-deutscher Erstsprache
  \end{itemize}
\end{frame}

\begin{frame}
  {Gewissensfrage}
  \Doppelzeile
  \onslide<+->
  \onslide<+->
  \centering 
  \LARGE
  \alert{Sind Sie überzeugt, dass Sie diese Aufgaben\\
  mit Ihrem Wissen über deutsche Grammatik\\
  verantwortungsvoll und souverän bewältigen können?}
\end{frame}

\begin{frame}
  {Lehramtsstudierende und Schulaufgaben}
  \onslide<+->
  \onslide<+->
  Abschneiden von Lehramtsstudierenden in einem Test,\\
  der aus Grammatikaufgaben der 5.--9.~Klasse bestand\\
  \grau{\footnotesize\citet{SchaeferSayatz2017a}}\\
  \onslide<+->
  \centering 
  \includegraphics[height=0.7\textheight]{graphics/schaefersayatz2017a}
\end{frame}

\begin{frame}
  {Lehramtsstudierende und Schulaufgaben}
  \onslide<+->
  \onslide<+->
  Abschneiden von Lehramtsstudierenden in einem Test,\\
  der aus Grammatikaufgaben der 5.--9.~Klasse bestand\\
  \grau{\footnotesize\citet{SchaeferSayatz2017a}}\\
  \onslide<+->
  \centering 
  \includegraphics[height=0.7\textheight]{graphics/schaefersayatz2017c}
\end{frame}

\begin{frame}
  {Sprachkompetenz | IQB-Bericht 2023}
  \onslide<+->
  \onslide<+->
  PISA ist doof, Institutionen wie das \alert{IQB} sind besser!\\
  \grau{\footnotesize\url{https://www.iqb.hu-berlin.de/bt/BT2022/Bericht}}\\
  \onslide<+->
  \Doppelzeile
  \centering
  Bundesweit verfehlen Neuntklässler die Mindeststandards \ldots\\
  \Halbzeile
  \onslide<+->
  \rot{im Bereich \textit{Lesen} zu 32,5 \%}\\
  \onslide<+->
  \Halbzeile
  \rot{im Bereich \textit{Zuhören} zu 34,4 \%}\\
  \onslide<+->
  \Halbzeile
  \rot{im Bereich \textit{Orthografie} zu 22,3 \%}\\
  \onslide<+->
  \Halbzeile
  \rot{Und über Bildungssprache reden wir hier noch gar nicht!}\\
\end{frame}

% \begin{frame}
%   {Sprachkompetenz | Realschulabsolventen}
%   \onslide<+->
%   \onslide<+->
%   Aus einem "`Märchen"', geschrieben von Realschulabsolventen 2024:\\
%   \onslide<+->
%   \Zeile
%   \textit{Ein Präsident hatte einen Sohn, der wahr gut anzusehen aber auch sehr verwöhnt und ausgelassen, dass ihn keine seiner Freundinnen gut genug war. Er machte mit einer nach der anderen Schluss und redete diese noch dazu schlecht. Einmal ließ sein Vater ein großes Fest veranstalten und ließ Leute von überall, aus der Welt einladen, um so Frauen zu finden die ihn heiraten wollten. Sie wurden alle in einer Reihe nach ihren Ansehen geordnet aufgestellt: Zuerst die Präsidentenkinder, dann die reichen Schnösel, die Politiker, Wohlhabende und die Normalos, zuletzt die Harzer.}\\
%   \onslide<+->
%   \Zeile
%   \centering 
%   Die eigentliche Frage ist: \rot{Wo war die Lehrperson?}
% \end{frame}
% 
% \begin{frame}
%   {Sprachkompetenz | Hausarbeiten in der Germanistik 1-1}
%   \onslide<+->
%   \onslide<+->
%   \centering 
%   \includegraphics[width=0.9\textwidth]{graphics/ha1a}
% \end{frame}
% 
% \begin{frame}
%   {Sprachkompetenz | Hausarbeiten in der Germanistik 1-2}
%   \centering 
%   \includegraphics[width=0.9\textwidth]{graphics/ha1b}
% \end{frame}
% 
% \begin{frame}
%   {Sprachkompetenz | Hausarbeiten in der Germanistik 1-3}
%   \centering 
%   \includegraphics[height=\textheight]{graphics/ha1c}
% \end{frame}
% 
% \begin{frame}
%   {Sprachkompetenz | Hausarbeiten in der Germanistik 2}
%   \centering 
%   \includegraphics[width=0.9\textwidth]{graphics/ha2a}
% \end{frame}


\begin{frame}
  {Unterrichtsrealität | Kasus 1}
  Es reicht nicht, die Aufgaben gemäß Schulbuch durchzuarbeiten!\\
  \Halbzeile
  \centering 
  \onslide<+->
  \onslide<+->
  \includegraphics[height=0.7\textheight]{graphics/kasusschule1}\\
  \grau{\tiny Aus: \citet[36--37]{Gramzowemden2002}, zitiert nach \citet[257--258]{Bredel2013}}
\end{frame}

\begin{frame}
  {Unterrichtsrealität | Kasus 2}
  \centering 
  \includegraphics[height=0.8\textheight]{graphics/kasusschule2}\\
  \grau{\tiny Aus: \citet[36--37]{Gramzowemden2002}, zitiert nach \citet[257--258]{Bredel2013}} 
\end{frame}

\begin{frame}
  {Unterrichtsrealität | Kasus 3}
  \centering 
  \includegraphics[height=0.8\textheight]{graphics/kasusschule3}\\
  \grau{\tiny Aus: \citet[36--37]{Gramzowemden2002}, zitiert nach \citet[257--258]{Bredel2013}} 
\end{frame}

\begin{frame}
  {Wie soll es dann gehen?}
  \onslide<+->
  \onslide<+->
  \centering 
  Durch den oben illustrierten Unterricht werden die Zielkompetenzen \rot{nicht} vermittelt!\\
  \onslide<+->
  \Zeile
  \Large Anderes Problem: starker \rot{Normpositivismus}\\
  anstelle von \gruen{Erklärungskompetenz}\\
  \Zeile
  \onslide<+->
  \normalsize
  Schon bei \citet{Braun1979}: Lehrende entscheiden \rot{inkonsistent}\\
  und \rot{viel zu leichtfertig} auf Normverstoß!
\end{frame}

\begin{frame}
  {Normpositivismus | Präpositionen in der Umgangssprache}
  \onslide<+->
  \onslide<+->
  Sind die Formen \alert{\textit{durchs}, \textit{ans}, \textit{ins}, \textit{am}, \textit{im}, \textit{vom}} usw.\ umgangssprachlich\\
   und sollten schriftsprachlich aufgelöst werden (\textit{durch das} usw.)?\\
  \Halbzeile
  \onslide<+->
  Kennen Sie das Grundgesetz?\\
  \Halbzeile
  \onslide<+->
  \begin{quote}\footnotesize
    \textbf{Präambel}\\
    
    \rot{Im} Bewußtsein seiner Verantwortung vor Gott und den Menschen, \gruen{von dem} Willen beseelt, als gleichberechtigtes Glied in einem vereinten Europa dem Frieden der Welt zu dienen, hat sich das Deutsche Volk kraft seiner verfassungsgebenden Gewalt dieses Grundgesetz gegeben. Die Deutschen in den Ländern Baden-Württemberg, Bayern, Berlin, Brandenburg, Bremen, Hamburg, Hessen, Mecklenburg-Vorpommern, Niedersachsen, Nordrhein-Westfalen, Rheinland-Pfalz, Saarland, Sachsen, Sachsen-Anhalt, Schleswig-Holstein und Thüringen haben in freier Selbstbestimmung die Einheit und Freiheit Deutschlands vollendet. Damit gilt dieses Grundgesetz für das gesamte Deutsche Volk.
  \end{quote}
  \onslide<+->
  \Halbzeile
  \centering 
  \rot{Erklären Sie das Phänomen! Können Sie das?}
\end{frame}

\begin{frame}
  {Normpositivismus | Der Imperativ}
  \onslide<+->
  \onslide<+->
  \centering 
  \textit{Viele können den Imperativ gar nicht mehr richtig!}\\
  \Zeile
  \onslide<+->
  \alert{\LARGE\textit{Die sagen "`Gebe!"' und "`Nehme!"' und sowas.}\\}
  \onslide<+->
  \Zeile
  \rot{Erklären Sie, wie der "`Fehler"' zustande kommt,\\
    was das System der normkorrekten Formen ist,\\
    und wie die regionale Verteilung ist.}\\
  \onslide<+->
  \Halbzeile
  \rot{Zu sagen, wie es Ihrer Meinung nach richtig heißt, ist keine Erklärung!}
\end{frame}

\begin{frame}
  {Normpositivismus | Verona Pooth (ehemals Feldbusch)}
  \onslide<+->
  \onslide<+->
  \centering 
  In der Werbung der Telefonauskunft 11880:\\
  \Zeile
  \onslide<+->
  \alert{\LARGE\textit{Da werden Sie geholfen!}}\\
  \onslide<+->
  \Zeile
  Das fanden alle unglaublich witzig, weil es "`falsch"' war.\\
  \onslide<+->
  \Zeile
  \centering 
  \rot{Erklären Sie, warum das kein Standard ist! Können Sie das?}\\
  \onslide<+->
  \Halbzeile
  \rot{Zu sagen, wie es Ihrer Meinung nach richtig heißt, ist keine Erklärung!}
\end{frame}




\section{Diese Vorlesung}

\begin{frame}
  {Aber was soll dann diese Art von Grammatik?}
  \onslide<+->
  \onslide<+->
  \alert{Warum behandelt diese Vorlesung nicht die Aufgaben aus der 5.--11.~Klasse?}\\
  \onslide<+->
  \Zeile
  \textit{Das Verhältnis zur Sprache, das als ‚metasprachliche Kompetenz‘, durch ‚Reflexion über Sprache‘ und ‚Transfer von explizitem zu implizitem Wissen‘ von den meisten Lehrplänen gefordert wird, können die Schüler nicht entwickeln, wenn es die Lehrer nicht haben.}\\
  \Viertelzeile
  \grau{\citet[23]{Eisenberg2004}}\\
  \Doppelzeile 
  \centering
  \onslide<+->
  \gruen{\Large Weil Sie viel mehr wissen müssen als Ihre Schüler!}\\
  \Viertelzeile
  \onslide<+->
  \alert{Das gilt für alle Bereiche, also auch für das Sprachsystem des Deutschen.}
\end{frame}

\begin{frame}
  {Ablauf dieser Vorlesung}
  \onslide<+->
  \onslide<+->
  Es gilt für jede Woche derselbe Plan.\\
  \grau{\footnotesize\url{https://rolandschaefer.net/archives/2446}}\\
  \Zeile
  \begin{enumerate}[<+->]\Large
    \item Sie sehen sich das entsprechende \alert{Video} an.
    \item Sie bearbeiten die \alert{Übungen} dazu.
    \item Sie stellen in der Präsenzveranstaltung\\
      \alert{Fragen zum Inhalt und den Übungen}.
    \item Sie vertiefen die Inhalte durch die \alert{Lektüre des Buchs}.
    \item Sie schärfen Ihre Kompetenz mit den \alert{Übungen im Buch}.
      \Halbzeile
    \item[ ] Vorher werden Sie einmal eingenordet mit dem \alert{Einstiegstest}.
  \end{enumerate}
\end{frame}

\begin{frame}
  {Warum war das heute wichtig?}
  \onslide<+->
  \onslide<+->
  Für mich: Ich finde es wichtig, Ihnen zu sagen, warum in meinem Lehrbereich\\
  die Inhalte und die Anforderungen so sind, wie sie sind (\alert{Fairnessgründe}).\\
  \onslide<+->
  \Zeile
  Für Sie:\\
  \Halbzeile
  \begin{itemize}[<+->]
    \item Sie wissen jetzt, \alert{warum der Stoff wichtig ist}.
    \item Es ist mit der zentralste Stoff für Ihre primäre Lehraufgabe.
    \item \alert{Verinnerlichen Sie das für Ihre Motivation!}
      \begin{itemize}[<+->]
        \item \alert{Lernen Sie von Anfang an!}
        \item Lernen Sie nicht mit dem Ziel einer 4,0.
        \item \alert{Machen Sie sich klar, warum Grammatik relevant ist.}
        \item Wählen Sie auch im restlichen Studium nicht immer den einfachsten Weg.
          \Viertelzeile
        \item \rot{Evaluieren Sie permanent Ihren Berufswunsch.}
          \Viertelzeile
        \item \rot{Denken Sie daran, dass Sie in dieser Vorlesung sitzen,\\
          weil Sie sich freiwillig dazu entschieden haben!}
        \item Alles andere folgt aus den \gruen{Ansprüchen Ihrer zukünftigen Schüler}.
      \end{itemize}
  \end{itemize}
\end{frame}


\begin{frame}
  {Deutsche Syntax | Plan}
  \rot{Alle} angegebenen Kapitel\slash Abschnitte aus \rot{\citet{Schaefer2018b}} sind \rot{Klausurstoff}!\\
  \Halbzeile
  \begin{enumerate}
    \item \alert{Grammatik und Grammatik im Lehramt} \rot{(Kapitel 1 und 3)}
    \item Grundbegriffe \rot{(Kapitel 2)}
    \item Wortklassen \rot{(Kapitel 6)}
    \item Konstituenten und Satzglieder \rot{(Kapitel 11 und Abschnitt 12.1)}
    \item Nominalphrasen \rot{(Abschnitt 12.3)}
    \item Andere Phrasen \rot{(Abschnitte 12.2 und 12.4--12.7)}
    \item Verbphrasen und Verbkomplex \rot{(Abschnitte 12.8)}
    \item Sätze \rot{(Abschnitte 12.9 und 13.1--13.3)} 
    \item Nebensätze \rot{(Abschnitt 13.4)}
    \item Subjekte und Prädikate \rot{(Abschnitte 14.1--14.3)}
    \item Passive und Objekte \rot{(14.4 und 14.5)}
    \item Syntax infiniter Verbformen \rot{(Abschnitte 14.7--14.9)}
  \end{enumerate}
  \Halbzeile
  \centering 
  \url{https://langsci-press.org/catalog/book/224}
\end{frame}
