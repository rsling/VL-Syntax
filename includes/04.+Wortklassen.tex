\section{Überblick}

\begin{frame}
  {Nächste Woche | Wortklassen}
  \begin{itemize}[<+->]
    \item Möglichkeiten, Wortklassen zu definieren
      \Halbzeile
    \item syntaktisch definierte Wortklassen
      \Halbzeile
    \item \citet[Kapitel~6]{Schaefer2018b}
  \end{itemize}
\end{frame}


\section{Syntaktische Wörter}

\begin{frame}
  {Wort und Wortform I}
  \pause
  \begin{exe}
    \ex
    \begin{xlist}
      \ex (der) Tisch
      \pause
      \ex (den) Tisch
      \pause
      \ex (dem) Tisch\alert{e}
      \pause
      \ex (des) Tisch\alert{es}
      \pause
      \ex (die) Tisch\alert{e}
      \pause
      \ex (den) Tisch\alert{en}
    \end{xlist}
  \end{exe}
  \pause
  \begin{exe}
    \ex
    \begin{xlist}
      \ex Der \_\_\_\ ist voll hässlich.
      \pause
      \ex Ich kaufe den \_\_\_ nicht.
      \pause
      \ex Wir speisten am \_\_\_\ des Bundespräsidenten.
      \pause
      \ex Der Preis des \_\_\_\ ist eine Unverschämtheit.
      \pause
      \ex Die \_\_\_\ kosten nur noch die Hälfte.
      \pause
      \ex Mit den \_\_\_\ können wir nichts mehr anfangen.
    \end{xlist}
  \end{exe}
\end{frame}

\begin{frame}
  {Wort und Wortform II}
  \pause
  \begin{block}{Wortform}
    Eine \alert{Wortform} ist eine in syntaktischen Strukturen auftretende und in diesen Strukturen nicht weiter zu unterteilende Einheit.
    [\ldots]
  \end{block}
  \Zeile
  \pause
  \begin{block}{Lexikalisches Wort}
    Das (\alert{lexikalische}) \alert{Wort} ist eine Repräsentation von lexikalisch (bedeutungsmäßig) zusammengehörigen Wortformen.
    [\ldots]
  \end{block}
\end{frame}

\begin{frame}
  {Syntaktisches Wort}
  \onslide<+->
  \onslide<+->
  Ein \alert{syntaktisches Wort} ist eine \alert{Wortform}\\
  im syntaktischen Kontext.\\
  \Zeile
  \onslide<+->
  Ein syntaktisches Wort ist immer \alert{für alle Merkmale spezifiziert},\\
  auch wenn man ihm (morphologisch) nicht die volle Spezifikation ansieht.\\
  \Zeile
  \onslide<+->
  \begin{exe}
    \ex \alert{Ein [Mitglied]\Sub{Nom, Sg, Neut}} widersprach dem Beschluss.
    \onslide<+->
    \ex Wir überzeugten \alert{ein [Mitglied]\Sub{Akk, Sg, Neut}}, dem Beschluss zuzustimmen.
  \end{exe}
\end{frame}

\section{Methode}

\begin{frame}
  {Klassische Grundschul-Wortarten}
  \onslide<+->
    \begin{itemize}[<+->]
      \item Dingwort
      \item Tuwort, Tätigkeitswort
      \item Wiewort, Eigenschaftswort
      \item Umstandswort
    \end{itemize}
    \onslide<+->
    \Zeile
    Überwiegend \alert{bedeutungsbasiert}!
\end{frame}

\begin{frame}
  {Ein paar neue Wortarten nach Bedeutungen I}
  \pause
  \begin{itemize}[<+->]
    \item \alert{Bewegungsverben}: \textit{laufen}, \textit{springen}, \textit{fahren}, \dots
    \item \alert{Zustandsverben}: \textit{duften}, \textit{wohnen}, \textit{liegen}, \dots
      \Halbzeile
    \item \alert{Konkreta}: \textit{Haus}, \textit{Buch}, \textit{Blume}, \textit{Stier}, \dots
    \item \alert{Abstrakta}: \textit{Konzept}, \textit{Glaube}, \textit{Wunder}, \textit{Kausalität}, \dots
      \Halbzeile
    \item \alert{Zählsubstantive}: \textit{Keks}, \textit{Student}, \textit{Mikrobe}, \textit{Kneipe}, \dots
    \item \alert{Stoffsubstantive}: \textit{Wasser}, \textit{Wein}, \textit{Zement}, \textit{Mehl}, \dots
  \end{itemize}
\end{frame}

\begin{frame}
  {Ein paar neue Wortarten nach Bedeutungen II}
  \pause
  Aber Moment mal\dots\\
  \pause
  \Zeile
  \begin{exe}
   \ex
   \begin{xlist}
     \ex[ ]{\alert{Wein} kann lecker sein.}
     \ex[ ]{\alert{Ein Keks kann} lecker sein.}
     \ex[*]{\rot{Keks} kann lecker sein.}
     \ex[ ]{\alert{Kekse können} lecker sein.}
   \end{xlist}
    \pause
    \ex
    \begin{xlist}
      \ex Johanna hätte gerne \alert{einen Keks}.
      \ex Johanna hätte gerne \alert{einen Wein}.
    \end{xlist}
  \end{exe}
  \pause
  \Zeile
  Es gibt hier durchaus auch \alert{formale} Unterschiede.
\end{frame}


\begin{frame}
  {Syntaktische Klassifikation}
  \pause
  \begin{exe}
    \ex
    \begin{xlist}
      \ex[]{Alexandra spielt schnell \alert{und} präzise.}
      \pause
      \ex[*]{Alexandra spielt schnell \alert{obwohl} präzise.}
      \pause
      \ex[]{Alexandra \alert{und} Dzsenifer spielen eine gute Saison.}
      \pause
      \ex[*]{Alexandra \alert{obwohl} Dzsenifer spielen eine gute Saison.}
    \end{xlist}
    \pause
    \Zeile
    \ex
    \begin{xlist}
      \ex[]{Alexandra spielt herausragend,\\
        \alert{obwohl} der Leistungsdruck hoch ist.}
      \pause
      \ex[*]{Alexandra spielt herausragend,\\
        \alert{und} der Leistungsdruck hoch ist.}
    \end{xlist}
  \end{exe}
    \pause
    \Zeile
    Alles nur wegen der Bedeutung?
\end{frame}

\begin{frame}
  {Syntaktische Klassifikation}
  \pause
  \begin{center}
    \Large Wörter lassen sich in Kategorien einordnen, je nachdem\\
    \alert{in welchen syntaktischen Kontexten sie auftreten}.
  \end{center}
  \Zeile
  \pause
  \begin{itemize}[<+->]
    \item Konjunktionen: zwischen zwei gleichartigen Satzteilen
    \item Komplementierer: am Anfang bestimmter Nebensätze
  \end{itemize}
\end{frame}



\begin{frame}[fragile]
  {Filter}
  Mittels syntaktischer Klassifikation können wir den rechten Arm des Wortklassenbaums aufbauen (nicht-flektierbare Wörter).\\
  \Zeile
  \hspace{0.25\textwidth}\scalebox{0.6}{
    \begin{minipage}{0.5\textwidth}  
      \centering 
    \begin{forest}
      /tikz/every node/.append style={font=\footnotesize},
      for tree={l sep=2em, s sep=2.5em},
      [\textit{Wort}, intrme, {visible on=<6->}, for children={visible on=<7->}
        [{Hat  Numerus?}, decide, for children={visible on=<8->}
          [\textit{flektierbar}, intrme, yes, {visible on=<9->}, for children={visible on=<11->}
            [{Ist finit  flektierbar?}, decide, {visible on=<11->}, for children={visible on=<12->}
              [\textbf{Verb}, finall, yes, {visible on=<13->}]
              [\textit{Nomen}, intrme, no, {visible on=<14->}]
            ]
          ]
          [\textit{nicht flektierbar}, intrme, no, {visible on=<10->}, for children={visible on=<15->}
            [{Hat Valenz-\slash  Kasusrektion?}, decide, {visible on=<15->}, for children={visible on=<16->}
              [\textbf{Präposition}, finall, yes, {visible on=<17->}]
              [\textit{andere}, intrme, no, {visible on=<18->}]
            ]
          ]
        ]
      ]
    \end{forest}
   \end{minipage}
   }
\end{frame}


\section{Wortklassen}

\begin{frame}
  {Präpositionen flektieren nicht und regieren Kasus}
  \pause
  \begin{exe}
    \ex
    \begin{xlist}
      \ex{\alert<3->{Mit} \rot<4->{dem kaputten Rasen} ist nichts mehr anzufangen.}
      \pause
      \pause
      \pause
      \ex{\alert<6->{Angesichts} \rot<7->{des kaputten Rasens} wurde das Spiel abgesagt.}
    \end{xlist}
  \end{exe}
  \pause
  \pause
  \pause
  \Zeile
  \begin{block}{Rektion}
    In einer Rektionsrelation werden durch die regierende Einheit (das \alert{Regens}) Werte für bestimmte Merkmale\slash Werte (und damit ggf.\ auch die Form) beim regierten Element (dem \alert{Rectum}) verlangt.\\
  \end{block}
  \Zeile
  \pause
  \begin{block}{Präposition}
    Präpositionen kasusregieren eine obligatorische Nominalphrase.
  \end{block}
\end{frame}

\begin{frame}
  {Komplementierer}
  \pause
  \begin{exe}
    \ex
    \begin{xlist}
      \ex[]{Ich glaube, [\alert<3->{dass} dieser Nebensatz ein Verb \alert<4->{enthält}].}
      \ex[]{[\alert<6->{Während} die Spielzeit \alert<7->{läuft}], zählt jedes Tor.}
      \ex[]{Es fällt ihnen schwer [\rot<8->{zu laufen}].}
      \ex[\rot<11->{*}]{[\alert<9->{Obwohl} kein Tor \alert<10->{fiel}].}
    \end{xlist}
  \end{exe}
  \Zeile
  \pause
  \pause
  \pause
  \pause
  \pause
  \pause
  \pause
  \pause
  \pause
  \pause
  \begin{block}{Komplementierer}
    Komplementierer leiten Nebensätze ein.\\
    Die Rede von der \textit{unterordnenden Konjunktion} ist ungeschickt.
  \end{block}
\end{frame}

\begin{frame}
  {Nicht-flektierbare Wörter im "`Vorfeld"'}
  \pause
  Was steht im unabhängigen Aussagesatz am Satzanfang?\\
  \pause
  {\rot{Antworten Sie nie mehr mit "`das Subjekt"'!}}
  \pause
  \begin{exe}
    \ex\label{ex:adverbenadkopulasundpartikeln038}
    \begin{xlist}
      \ex[ ]{\alert<5->{Gestern} hat der FCR Duisburg gewonnen.}
      \pause
      \pause
      \ex[ ]{\alert<7->{Erfreulicherweise} hat der FCR Duisburg gestern gewonnen.}
      \pause
      \pause
      \ex[ ]{\alert<9->{Oben} finden wir andere Beispiele.}
      \pause
      \pause
      \ex[*]{\alert<11->{Doch} ist das aber nicht das Ende der Saison.}
      \pause
      \pause
      \ex[*]{\alert<13->{Und} ist die Saison zuende.}
      \pause
      \pause
    \end{xlist}
    \ex\label{ex:adverbenadkopulasundpartikeln044} Das ist aber \alert{doch} nicht das Ende der Saison.
  \end{exe}
  \pause
  \Viertelzeile
  \begin{block}{Adverb}
    Adverben sind die übriggebliebenen nicht-flektierbaren Wörter, die im Vorfeld stehen können.
  \end{block}
\end{frame}


\begin{frame}
  {Konjunktionen}
  \onslide<+->
  \onslide<+->
  \begin{exe}
    \ex
    \begin{xlist}
      \ex Wir \alert{laufen} \rot{und} \alert{springen}.
      \ex Ich bin allergisch gegen \alert{Haselnüsse} \rot{und} \alert{Bananen}.
      \ex \alert{Kommst du jetzt} \rot{oder} \alert{sollen wir schon gehen}?
      \ex \alert{Erschöpft}, \rot{aber} \alert{zufrieden} lief sie über die Ziellinie.
    \end{xlist}
  \end{exe}
  \onslide<+->
  \begin{block}{Kunjunktion}
    Eine Konjunktion (\textit{und}, \textit{oder}, \textit{aber}, \textit{sondern}, \ldots) verbindet zwei Konstituenten A und B, die sich syntaktisch gleich verhalten. Die Gesamtheit [A Konjunktion B] verhält sich ebenso.
  \end{block}
\end{frame}



\section{Ausblick}

\begin{frame}
  {Syntaktische Baupläne}
  \onslide<+->
  \begin{itemize}[<+->]
    \item Wozu Syntaxregeln?
    \item Baupläne und konkrete Analysen
    \item Konstituententests
      \Zeile
    \item \citet[Kapitel 11]{Schaefer2018b}
  \end{itemize}
\end{frame}
