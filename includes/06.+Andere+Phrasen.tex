
\section{Überblick}

\begin{frame}
  {Andere Phrasentypen}
  \onslide<+->
  \begin{itemize}[<+->]
    \item Adjektivphrasen
    \item Präpositionalphrasen
    \item Adverbphrasen
    \item Koordination
    \item Komplementiererphrase
      \Zeile
    \item \citet[12.2,12.4--12.7]{Schaefer2018b}
  \end{itemize}
\end{frame}

\section[AP]{Adjektivphrasen}

\begin{frame}
  {Gradierungselemente vor dem Adjektiv}
  \onslide<+->
  \onslide<+->
  \begin{exe}
    \ex[]{\label{ex:adjektivphrase081} die [\gruen{sehr} \alert{angenehme}] Stimmung}
    \ex[]{\label{ex:adjektivphrase082} die [\gruen{ziemlich} \alert{angenehme}] Stimmung}
    \ex[]{\label{ex:adjektivphrase083} die [\gruen{wenig} \alert{angenehme}] Stimmung}
      \onslide<+->
      \Zeile
      \ex[]{\label{ex:adjektivphrase085} die [\gruen{[über alle Maßen]} \alert{angenehme}] Stimmung}
      \ex[]{\label{ex:adjektivphrase086} die [\gruen{[ja mal wieder so rein gar nicht]} \alert{angenehme}] Stimmung}
  \end{exe}
\end{frame}

\begin{frame}
  {Modifizierer | noch vor Gradierungselementen}
  \onslide<+->
  \onslide<+->
  \begin{exe}
    \ex\label{ex:adjektivphrase087}
    \begin{xlist}
      \ex{\label{ex:adjektivphrase088} die [\braun{[seit gestern]} \gruen{sehr} \alert{angenehme}] Stimmung}
      \ex{\label{ex:adjektivphrase089} das [\braun{[in Hessen]} \gruen{überaus} \alert{beliebte}] Getränk}
    \end{xlist}
    \onslide<+->
    \Zeile
    \ex[*]{\label{ex:adjektivphrase090}die [\gruen{sehr} \braun{[seit gestern]} \alert{angenehme}] Stimmung}
  \end{exe}
\end{frame}

\begin{frame}
  {Adjektivphrase | Baumbeispiel}
  \onslide<+->
  \onslide<+->
  \centering
  \begin{forest}
    [AP, calign=last
      [PP, tier=preterminal
        [\it seit gestern, narroof]
      ]
      [Ptkl, tier=preterminal
        [\it sehr]
      ]
      [\bf A, tier=preterminal
        [\it angenehme]
      ]
    ]
  \end{forest}
\end{frame}

\begin{frame}
  {Ergänzungen in der AP}
  \onslide<+->
  \onslide<+->\begin{exe}
  \ex\label{ex:adjektivphrase092}
  \begin{xlist}
    \ex[]{\label{ex:adjektivphrase093} die [\tuerkis{[auf ihre Tochter]} \alert{stolze}] Frau}
    \ex[*]{die [\alert{stolze} \tuerkis{[auf ihre Tochter]}] Frau}
    \Zeile
    \onslide<+->
    \ex[]{die [\tuerkis{[über ihre Tochter]} \alert{verwunderte}] Frau}
    \ex[*]{die [\alert{verwunderte} \tuerkis{[über ihre Tochter]}] Frau}
    \onslide<+->
    \Zeile
    \ex[]{die [\tuerkis{[ihres Lieblingseises]} \alert{überdrüssige}] Frau}
    \ex[*]{die [\alert{überdrüssige} \tuerkis{[ihres Lieblingseises]}] Frau}
  \end{xlist}
\end{exe}
\end{frame}

\begin{frame}
  {Ziemlich volle AP}
  \onslide<+->
  \onslide<+->
  \centering
  \begin{forest}
    [AP, calign=last
      [PP, tier=preterminal
        [\it seit gestern, narroof]
      ]
      [PP, tier=preterminal
        [\it auf ihre Tochter, narroof, name=AufIhreTochter]
      ]
      [Ptkl, tier=preterminal
        [\it sehr]
      ]
      [\bf A, tier=preterminal
        [\it stolze]
        {\draw [->, bend left=30] (.south) to (AufIhreTochter);}
      ]
    ]
  \end{forest}
\end{frame}

\begin{frame}
  {Adjektivphrase | Schema}
  \onslide<+->
  \onslide<+->
  \centering 
  \begin{forest}
    phrasenschema
    [AP, Ephr, calign=child, calign child=2
      [Modifizierer, Eopt, Erec, Emult [Ergänzungen, Eopt, Erec, name=Apergaenzi]]
      [Gradierungselement, Eopt]
      [A, Ehd]
      {\draw [->, bend left=45] (.south) to (Apergaenzi.south);}
    ]
  \end{forest}
\end{frame}

\section[PP]{Präpositionalphrasen}

\begin{frame}
  {Präpositionalphrasen | Beispiele}
  \onslide<+->
  \onslide<+->
  Erinnerung | \alert{Präpositionen haben eine einstellige Valenz.}\\
  \onslide<+->
  \Zeile
  \begin{exe}
    \ex\label{ex:normalepp096}
    \begin{xlist}
      \ex{[\alert{Auf} \orongsch{[dem Tisch]}] steht Ischariots Skulptur.}
      \ex{[\gruen{[Einen Meter]} \alert{unter} \orongsch{[der Erde]}] ist die Skulptur versteckt.}
    \end{xlist}
    \onslide<+->
    \Zeile
    \ex{\label{ex:normalepp097} Seit der EM springt Christina [\gruen{weit} \alert{über} \orongsch{[ihrem früheren Niveau]}].}
  \end{exe}{}
\end{frame}

\begin{frame}
  {Baumbeispiel | PP mit Maßangabe}
  \onslide<+->
  \onslide<+->
  \centering
  \begin{forest}
    [PP, calign=child, calign child=2
      [NP, tier=preterminal
        [\it einen Meter, narroof]
      ]
      [\bf P, tier=preterminal
        [\it unter]
      ]
      [NP, tier=preterminal
        [\it der Erde, narroof]
      ]
    ]
  \end{forest}
\end{frame}


\begin{frame}
  {Präpositionalphrase | Schema}
  \onslide<+->
  \onslide<+->
  \centering
  \begin{forest}
    phrasenschema
    [PP, Ephr, calign=child, calign child=2
      [Modifizierer, Eopt]
      [P, Ehd, name=Ppkopf]
      [NP, Eobl]
      {\draw [<-, bend left=45] (.south) to (Ppkopf.south);}
    ]
  \end{forest}
\end{frame}


\section[AdvP]{Adverbphrasen}

\begin{frame}
  {Adverbphrasen}
  \onslide<+->
  \onslide<+->
  \alert{Adverben} | Präpositionen mit \alert{nullstelliger Valenz}.\\
  \Zeile
  \onslide<+->
  \begin{exe}
    \ex{\label{ex:adverbphrase106} Ischariot malt [\orongsch{sehr} \alert{oft}].}
      \Halbzeile
      \ex{\label{ex:adverbphrase107} Ischariot schwimmt [\orongsch{weit} \alert{draußen}].}
      \Halbzeile
      \ex{\label{ex:adverbphrase108} Ischariot verreist [\orongsch{sehr} \alert{wahrscheinlich}].}
  \end{exe}
\end{frame}


\begin{frame}
  {Baumbeispiel | AdvP mit Modifizierer}
  \onslide<+->
  \onslide<+->
  \centering
  \begin{forest}
    [AdvP, calign=last
      [Ptkl, tier=preterminal
        [\it sehr]
      ]
      [\bf Adv, tier=preterminal
        [\it oft]
      ]
    ]
  \end{forest}
\end{frame}


\begin{frame}
  {Adverbphrase | Schema}
  \onslide<+->
  \onslide<+->
  \centering
  \begin{forest}
    phrasenschema
    [AdvP, Ephr, calign=last
      [Modifizierer, Eopt]
      [Adv, Ehd]
    ]
  \end{forest}
\end{frame}



\section{Koordination}

\begin{frame}
  {Koordination | Beispiele}
  \onslide<+->
  \onslide<+->
  \alert{Koordination} | Gleiches mit Gleichem zu Gleichem verbinden.\\
  \onslide<+->
  \Zeile
  \begin{exe}
    \ex\label{ex:koordination006}
    \begin{xlist}
      \ex{Ihre Freundin möchte [\orongsch{Kuchen} \alert{und} \orongsch{Sahne}].}
      \Halbzeile
      \onslide<+->
      \ex{[\orongsch{[Es ist Sonntag]} \alert{und} \orongsch{[die Zeit wird knapp]}].}
      \Halbzeile
      \onslide<+->
      \ex{Hast du das Teepulver [\orongsch{auf} \alert{oder} \orongsch{neben}]\\
    den Tatami-Matten verstreut?}
    \end{xlist}
  \end{exe}
\end{frame}


\begin{frame}
  {Koordination von Substantiven (oder NPs?)}
  \centering
  \begin{forest}
    [\textbf{N}, calign=child, calign child=2
      [\textbf{N}, tier=preterminal
        [\it Kuchen]
      ]
      [Konj, tier=preterminal
        [\it und]
      ]
      [\textbf{N}, tier=preterminal
        [\it Sahne]
      ]
    ]
  \end{forest}
\end{frame}

\begin{frame}
  {Koordination von Sätzen}
  \centering
  \begin{forest}
    [S, calign=child, calign child=2
      [S, tier=preterminal
        [\it Es ist Sonntag, narroof]
      ]
      [Konj, tier=preterminal
        [\it und]
      ]
      [S, tier=preterminal
        [\it die Zeit wird knapp, narroof]
      ]
    ]
  \end{forest}
\end{frame}

\begin{frame}
  {Koordination von Präpositionen}
  \centering
  \begin{forest}
    [\textbf{P}, calign=child, calign child=2
      [\textbf{P}, tier=preterminal
        [\it auf]
      ]
      [Konj, tier=preterminal
        [\it oder]
      ]
      [\textbf{P}, tier=preterminal
        [\it neben]
      ]
    ]
  \end{forest}
\end{frame}

\begin{frame}
  {Koordination | Schema}
  \onslide<+->
  \onslide<+->
  Die Koordination selber ist kein Kopf!\\
  \onslide<+->
  \Zeile
  \centering
  \begin{forest}
    phrasenschema
    [$\kappa$, Ephr
      [$\kappa$, Eobl]
      [Konj, Eopt]
      [$\kappa$, Eobl]
    ]
  \end{forest}
\end{frame}


\section[KP]{Komplementiererphrase}

\begin{frame}
  {Komplementiererphrasen = eingeleitete Nebensätze}
  \pause
  \begin{exe}
    \ex\label{ex:komplementiererphrase111}
    \begin{xlist}
      \ex[]{\label{ex:komplementiererphrase112} Der Arzt möchte, [dass [der Privatpatient die Rechnung \alert{bezahlt}]].}
      \pause
      \ex[*]{\label{ex:komplementiererphrase113} Der Arzt möchte, [dass [der Privatpatient \rot{bezahlt} die Rechnung]].}
      \pause
      \ex[*]{\label{ex:komplementiererphrase114} Der Arzt möchte, [dass [\rot{bezahlt} der Privatpatient die Rechnung]].}
    \end{xlist}
  \end{exe}
  \pause
  \Halbzeile
  \centering
  \begin{forest}
    [KP, calign=first
      [\bf K, tier=preterminal
        [\it dass, name=Kpkopf]
      ]
      [\alert{VP}, tier=preterminal
        [\it der Kassenpatient \alert{geht}, narroof]
      ]
    ]
  \end{forest}\\
  \pause
  \Zeile
  \alert{Verb-Letzt-Stellung!}\\
\end{frame}



\begin{frame}
  {Komplementiererphrase | Schema}
  \begin{center}
    \begin{forest}
      phrasenschema
      [KP, Ephr, calign=first
        [K, Ehd, name=Kpkopf]
        [VP, Eobl]
        {\draw [bend left=45, <-] (.south) to (Kpkopf.south);}
      ]
    \end{forest}
  \end{center}
  \onslide<+->
  \Zeile
  \alert{Aber wie sieht die VP aus?}\\
  \Viertelzeile
  \onslide<+->
  \orongsch{Und was ist mit unabhängigen Sätzen?}
\end{frame}


\section{Vorschau}

