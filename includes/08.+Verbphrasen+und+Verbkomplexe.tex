\section{Überblick}

\begin{frame}
  {Verbphrasen und Verbkomplexe}
  \onslide<+->
  \begin{itemize}[<+->]
    \item Verbphrasen mit Verb-Letzt-Stellung
    \item Scrambling | Stellungsfreiheit in der VP
      \Halbzeile
    \item Verbkomplexe | Verbketten am Ende der VP
      \Halbzeile
    \item systematische syntaktische Analysen
      \Zeile
    \item \citet[12.8]{Schaefer2018b}
  \end{itemize}
\end{frame}


\section{Verbphrasen}

\begin{frame}
  {Beispiele für Verbphrasen}
  \pause
  \begin{exe}
  \ex
    \begin{xlist}
      \ex{dass [Ischariot \alert{malt}]}
      \pause
      \Halbzeile
      \ex{dass [Ischariot [das Bild] \alert{malt}]}
      \pause
      \Halbzeile
      \ex{dass [Ischariot [dem Arzt] [das Bild] \alert{verkauft}]}
      \pause
      \Halbzeile
      \ex{dass [Ischariot [wahrscheinlich] [dem Arzt]\\
        { }[heimlich] [das Bild] schnell \alert{verkauft}]}
    \end{xlist}
  \end{exe}
\end{frame}

\begin{frame}
  {VP mit einstelliger Valenz}
  \centering 
  \scalebox{1}{
  \begin{forest}
    [VP, calign=last
      [NP, tier=preterminal
        [\it Ischariot, narroof]
      ]
      [\bf V, tier=preterminal
        [\it malt]
      ]
    ]
  \end{forest}
  }
\end{frame}

\begin{frame}
  {VP mit zweistelliger Valenz}
  \centering 
  \scalebox{1}{
  \begin{forest}
    l sep+=1em
    [VP, calign=last
      [NP, tier=preterminal
        [\it Ischariot, narroof]
      ]
      [NP, tier=preterminal
        [\it das Bild, narroof]
      ]
      [\bf V, tier=preterminal
        [\it malt]
      ]
    ]
  \end{forest}
  }
\end{frame}

\begin{frame}
  {VP mit dreistelliger Valenz}
  \centering 
  \scalebox{1}{
  \begin{forest}
    l sep+=2em
    [VP, calign=last
      [NP, tier=preterminal
        [\it Ischariot, narroof]
      ]
      [NP, tier=preterminal
        [\it dem Arzt, narroof]
      ]
      [NP, tier=preterminal
        [\it das Bild, narroof]
      ]
      [\bf V, tier=preterminal
        [\it verkauft]
      ]
    ]
  \end{forest}
  }
\end{frame}

\begin{frame}
  {VP mit einstelliger Valenz und Adverbialen}
  \centering 
  \scalebox{1}{
  \begin{forest}
    l sep+=3em
    [VP, calign=last
      [NP, tier=preterminal
        [\it Ischariot, narroof]
      ]
      [AdvP, tier=preterminal
        [\it wahrscheinlich, narroof]
      ]
      [NP, tier=preterminal
        [\it dem Arzt, narroof]
      ]
      [AdvP, tier=preterminal
        [\it heimlich, narroof]
      ]
      [NP, tier=preterminal
        [\it das Bild, narroof]
      ]
      [AdvP, tier=preterminal
        [\it schnell, narroof]
      ]
      [\bf V, tier=preterminal
        [\it verkauft]
      ]
    ]
  \end{forest}
  }
\end{frame}

\begin{frame}
  {Achtung! Scrambling!}
  \onslide<+->
  \onslide<+->
  \alert{Scrambling} | Die Phrasen innerhalb der VP können\\
  nahezu beliebig umsortiert werden.\\
  \Zeile
  \onslide<+->
  \begin{exe}
    \ex dass dem Arzt Ischariot wahrscheinlich\\
    schnell das Bild verkauft
    \ex dass Ischariot wahrscheinlich schnell\\
    dem Arzt das Bild verkauft
    \ex dass Ischariot wahrscheinlich\\
    das Bild schnell dem Arzt verkauft
    \ex \ldots
  \end{exe}
  \onslide<+->
  \Zeile
  Die Umstellungen haben \alert{semantische und pragmatische Effekte},\\
  aber syntaktisch sind sie alle möglich.
\end{frame}

\section{Verbkomplexe}

\begin{frame}
  {Warum Verbkomplexe?}
  \pause
  \begin{exe}
    \ex{\label{ex:verbkomplex121} dass der Junge ein Eis \alert{[isst]}}
    \pause
    \ex\label{ex:verbkomplex122}
    \begin{xlist}
      \ex{\label{ex:verbkomplex123} dass der Junge ein Eis \alert{[essen wird]}}
      \pause
      \ex{\label{ex:verbkomplex124} dass das Eis \alert{[gegessen wird]}}
      \pause
      \ex{\label{ex:verbkomplex125} dass die Freundin das Eis \alert{[kaufen wollen wird]}}
    \end{xlist}
  \end{exe}
  \Zeile
  \pause
  Deutsch: \alert{Verben werden miteinander kombiniert, um Tempora,\\
  Modalität, Diathese usw.\ zu kodieren.}\\
\end{frame}


\begin{frame}
  {Verbkomplexe und Statusrektion}
  \pause
  \Halbzeile
  \centering
  \scalebox{0.7}{
    \begin{forest}
      [\bf V, tier=preterminal
        [\it isst, baseline]
      ]
    \end{forest}
  }
  \hspace{1em}\scalebox{0.7}{
    \begin{forest}
      [\bf V\Sub{B+A}, calign=last, bluetree
        [\bf V\Sub{B}, tier=preterminal, gruentree
          [\it essen\\(1.~Status)]
        ]
        [\bf V\Sub{A}, tier=preterminal
          [\it wird, baseline]
          {\draw [->, trueblue, bend left=30] (.south) to (!uu11.south);}
        ]
      ]
    \end{forest}
  }
  \hspace{1em}\scalebox{0.7}{
    \begin{forest}
      [\bf V\Sub{B+A}, calign=last, bluetree
        [\bf V\Sub{B}, tier=preterminal, gruentree
          [\it gegessen\\(3.~Status)]
        ]
        [\bf V\Sub{A}, tier=preterminal
          [\it wird, baseline]
          {\draw [->, trueblue, bend left=30] (.south) to (!uu11.south);}
        ]
      ]
    \end{forest}
  }
  \hspace{1em}\scalebox{0.7}{
    \begin{forest}
      [\bf V\Sub{C+B+A}, calign=last, bluetree
        [\bf V\Sub{C+B}, calign=last, gruentree
          [\bf V\Sub{C}, tier=preterminal, orongschtree
            [\it kaufen\\(1.~Status)]
          ]
          [\bf V\Sub{B}, tier=preterminal
            [\it wollen\\(1.~Status)]
            {\draw [->, gruen, bend left=30] (.south) to (!uu11.south);}
          ]
        ]
        [\bf V\Sub{A}, tier=preterminal
          [\it wird, baseline]
          {\draw [->, trueblue, bend left=30] (.south) to (!uu121.south);}
        ]
      ]
    \end{forest}
  }
  \pause
  \Halbzeile
  \begin{itemize}[<+->]
    \item Buchstaben (im Buch Zahlen): \alert{Verb A} regiert \gruen{Verb B} regiert \orongsch{Verb C}
    \item Numerierung: Status
      \begin{itemize}[<+->]
        \item 1.~Status: Infinitiv ohne \textit{zu}
        \item 2.~Status: Infinitiv mit \textit{zu}
        \item 3.~Status: Partizip
      \end{itemize}
    \item infinite Verbformen: solche, die von anderen Verben regiert werden
%    \item "`Partizip 1"' keine infinite Verbform (Derivation zum Adjektiv)
  \end{itemize}
\end{frame}


\begin{frame}
  {Verbkomplex und Rektion in der VP}
  \pause
  Die Hilfsverben \textit{heben} die \alert{Valenz-Anforderungen}\\
  lexikalischer Verben zu sich \textit{an}.\\
  \pause
  \centering
  \adjustbox{max width=0.5\textwidth}{%
    \begin{forest}
      l sep+=2em
      [VP, calign=last
        [NP, tier=preterminal, name=subj
          [\it die Freundin, narroof]
        ]
        [NP, tier=preterminal, name=obj
          [\it das Eis, narroof]
        ]
        [\bf V\Sub{\orongsch{C}\gruen{+B}\alert{+A}}, calign=last, bluetree, name=CBAnode
          [\bf V\Sub{\orongsch{C}\gruen{+B}}, calign=last, gruentree, name=CBnode
            [\bf V\Sub{C}, tier=preterminal, orongschtree
              [\it kaufen]
            ]
            {\draw [->, orongsch, bend left=15] (.north) to node [above, near start] {\tiny{(Nom, Akk)}} (CBnode.west);}
            [\bf V\Sub{B}, tier=preterminal
              [\it wollen]
              {\draw [->, gruen, bend left=30] (.south) to node [below] {\footnotesize{1.~Status}} (!uu11.south);}
            ]
          ]
          {\draw [->, orongsch, bend left=15] (.north) to node [above, near start] {\tiny{(Nom, Akk)}} (CBAnode.west);}
          [\bf V\Sub{A}, tier=preterminal
            [\it wird]
            {\draw [->, trueblue, bend left=30] (.south) to node [below, near start] {\footnotesize{1.~Status}} (!uu121.south);}
          ]
        ]
        {\draw [->, orongsch, bend right=80] (.north) to node [above] {\footnotesize{Nom}} (subj.north);}
        {\draw [->, orongsch, bend right=60] (.north) to node [above] {\footnotesize{Akk}} (obj.north);}
      ]
    \end{forest}
  }
\end{frame}


\begin{frame}
  {Verbphrase und Verbkomplex | Schemata}
  \centering
  \begin{forest}
    phrasenschema
    [VP, Ephr, calign=last
      [Angaben, Emult, Eopt, Erec [Ergänzungen, Eopt, Emult, Erec, name=Vpergaenzi]]
      [V, Ehd]
      {\draw [->, bend left=90] (.south) to (Vpergaenzi.south);}
    ]
  \end{forest}
  \pause\hspace{2em}
  \begin{forest}
    phrasenschema
    [V\Sub{j+i}, Ephr, , calign=last
      [V\Sub{j}, Eopt, name=Vkkopf]
      [V\Sub{i}, Ehd]
      {\draw [->, bend left=30] (.south) to (Vkkopf.south);}
    ]
  \end{forest}
\end{frame}

\section{Analyse}

\begin{frame}
  {Systematische syntaktische Analyse | Schritt 1}
  \centering
  \begin{forest}
    [, phantom, s sep=1em
      [\bf K, tier=preterminal [\it dass]]
      [\bf N, tier=preterminal [\it Frida]]
      [Art, tier=preterminal [\it den]]
      [\bf A, tier=preterminal [\it heißen]]
      [\bf N, tier=preterminal [\it Kaffee]]
      [\bf Adv, tier=preterminal [\it gerne]]
      [\bf V, tier=preterminal [\it trinken]]
      [\bf V, tier=preterminal [\it möchte]]
    ]
  \end{forest}
\end{frame}

\begin{frame}
  {Systematische syntaktische Analyse | Schritt 2}
  \centering
  \begin{forest}
    [, phantom, s sep=1em
      [\bf K, tier=preterminal [\it dass]]
      [NP
        [\bf N, tier=preterminal [\it Frida]]
      ]
      [Art, tier=preterminal [\it den]]
      [AP
        [\bf A, tier=preterminal [\it heißen]]
      ]
      [\bf N, tier=preterminal [\it Kaffee]]
      [AdvP
        [\bf Adv, tier=preterminal [\it gerne]]
      ]
      [\bf V, tier=preterminal [\it trinken]]
      [\bf V, tier=preterminal [\it möchte]]
    ]
  \end{forest}
\end{frame}

\begin{frame}
  {Systematische syntaktische Analyse | Schritt 3}
  \centering
  \begin{forest}
    [, phantom, s sep=0.5em
      [\bf K, tier=preterminal [\it dass]]
      [NP
        [\bf N, tier=preterminal [\it Frida]]
      ]
      [NP, calign=last
        [Art [\it den, tier=terminal]]
        [AP
          [\bf A, tier=preterminal [\it heißen]]
        ]
        [\bf N, tier=preterminal [\it Kaffee]]
      ]
      [AdvP
        [\bf Adv, tier=preterminal [\it gerne]]
      ]
      [\bf V, tier=preterminal [\it trinken]]
      [\bf V, tier=preterminal [\it möchte, tier=terminal]]
    ]
  \end{forest}
\end{frame}

\begin{frame}
  {Systematische syntaktische Analyse | Schritt 4}
  \centering
  \begin{forest}
    [, phantom, s sep=0.5em
      [\bf K, tier=preterminal [\it dass]]
      [NP
        [\bf N, tier=preterminal [\it Frida]]
      ]
      [NP, calign=last
        [Art [\it den, tier=terminal]]
        [AP
          [\bf A, tier=preterminal [\it heißen]]
        ]
        [\bf N, tier=preterminal [\it Kaffee]]
      ]
      [AdvP
        [\bf Adv, tier=preterminal [\it gerne]]
      ]
      [\bf V, calign=last
        [\bf V, tier=preterminal [\it trinken]]
        [\bf V, tier=preterminal [\it möchte, tier=terminal]]
      ]
    ]
  \end{forest}
\end{frame}

\begin{frame}
  {Systematische syntaktische Analyse | Schritt 5}
  \centering
  \begin{forest}
    [, phantom, s sep=0.5em
      [\bf K, tier=preterminal [\it dass]]
      [VP, calign=last, l sep=4em
        [NP
          [\bf N, tier=preterminal [\it Frida]]
        ]
        [NP, calign=last
          [Art [\it den, tier=terminal]]
          [AP
            [\bf A, tier=preterminal [\it heißen]]
          ]
          [\bf N, tier=preterminal [\it Kaffee]]
        ]
        [AdvP
          [\bf Adv, tier=preterminal [\it gerne]]
        ]
        [\bf V, calign=last
          [\bf V, tier=preterminal [\it trinken]]
          [\bf V, tier=preterminal [\it möchte, tier=terminal]]
        ]
      ]
    ]
  \end{forest}
\end{frame}

\begin{frame}
  {Systematische syntaktische Analyse | Schritt 6}
  \centering
  \begin{forest}
    [KP, calign=first
      [\bf K, tier=preterminal [\it dass]]
      [VP, calign=last, l sep=4em
        [NP
          [\bf N, tier=preterminal [\it Frida]]
        ]
        [NP, calign=last
          [Art [\it den, tier=terminal]]
          [AP
            [\bf A, tier=preterminal [\it heißen]]
          ]
          [\bf N, tier=preterminal [\it Kaffee]]
        ]
        [AdvP
          [\bf Adv, tier=preterminal [\it gerne]]
        ]
        [\bf V, calign=last
          [\bf V, tier=preterminal [\it trinken]]
          [\bf V, tier=preterminal [\it möchte, tier=terminal]]
        ]
      ]
    ]
  \end{forest}
\end{frame}


\section{Vorschau}

\begin{frame}
  {Form und Funktion von Sätzen}
  \onslide<+->
  \begin{itemize}[<+->]
    \item Was ist ein "`unabhängiger Satz"'?\\
      \grau{Funktion unabhängiger Sätze}
    \item Hypotaxe und komplexe Sachverhalte
    \item Komplementatz, Adverbialsatz, Relativsatz\\
      in Relation zum Matrixsatz (Semantik)
      \Zeile
    \item Überblick über die Syntax des unabhängigen Satzes\\
      \grau{Feldermodell}
      \Zeile
    \item \citet[13.1, 13.2]{Schaefer2018b}
  \end{itemize}
\end{frame}
