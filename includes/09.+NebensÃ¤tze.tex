\section{Überblick}

\begin{frame}
  {Nebensätze und unabhängige Sätze}
  \begin{itemize}[<+->]
    \item \alert{Relativsätze} | interne und externe Beziehungen des Relativelements
    \item \alert{Objektsätze} | Rektion und Stellung
    \Zeile
    \item \alert{Feldermodell} | alternative Beschreibung deutscher Saztsyntax
  \end{itemize}
\end{frame}


\section{Relativsätze}

\begin{frame}
  {Relativsätze als etwas andere VL-Sätze}
  \pause
  Das \tuerkis{Relativelement} wird nach links gestellt. Das \alert{Verb} bleibt rechts.\\
  \pause
  \begin{center}
    \adjustbox{max width=0.4\textwidth}{%
      \begin{minipage}{0.7\textwidth}
      \begin{forest}
        [NP, calign=child, calign child=2
          [Art, tier=preterminal
            [\it einen]
          ]
          [\bf N, tier=preterminal
            [\it Tofu]
          ]
          [RS, calign=first
            [NP\Sub{1}, tier=preterminal, tuerkistree
              [\it der, narroof, name=BeweDer]
            ]
            [VP, calign=last
              [NP, tier=preterminal, baseline
                [\it mir, narroof]
              ]
              [\Ti, tier=preterminal, tuerkistree]
              {\draw[dotted, tuerkis, thick, ->] (.south) |- ++(0,-4.5em) -| (BeweDer.south);}
              [Ptkl, tier=preterminal
                [nicht]
              ]
              [\bf V, calign=last
                [\bf V, tier=preterminal
                  [\it geschmeckt]
                ]
                [\bf V, tier=preterminal, bluetree
                  [\it hat]
                ]
              ]
            ]
          ]
        ]
      \end{forest}
    \end{minipage}
    }
    \pause
    \hspace{0.1\textwidth}\adjustbox{max width=0.35\textwidth}{%
      \begin{minipage}{0.35\textwidth}
      \begin{forest}
        [RS, Ephr
          [XP\UpSub{relativ}{1}, Eobl, baseline, tuerkis]
          [VP\\{[\ldots\tuerkis{\Ti}\ldots]}, Eobl]
        ]
      \end{forest}
      \end{minipage}
    }
  \end{center}
  \pause
  \Halbzeile
  \begin{itemize}[<+->]
    \item Relativelement
      \begin{itemize}[<+->]
        \item \alert{Bedeutung}: Bezugs-Substantiv
        \item \alert{Genus, Numerus}: Kongruenz mit Bezugs-Substantiv
        \item \alert{Kasus\slash PP-Form}: gemäß Status als Ergänzung\slash Angabe im RS
      \end{itemize}
  \end{itemize}
\end{frame}


\begin{frame}
  {Komplexe Einbettung des Relativelements}
  \pause
  Das \tuerkis{Relativelement} als pränominaler Genitiv nimmt die Matrix-NP mit.\\
  \pause
  \Halbzeile
  \centering
  \begin{forest}
    [NP, calign=child, calign child=2
      [Art, tier=preterminal
        [\it der]
      ]
      [\bf N, tier=preterminal
        [\it Tofu]
        {\draw [->, bend right=30] (.south) to node [below, near start] {\footnotesize\textsc{Genus,Numerus}} (RekDessen.south);}
      ]
      [RS, calign=first
        [NP\Sub{A}, calign=last, tuerkistree
          [NP\Sub{B}, tier=preterminal
            [\it dessen, narroof, name=RekDessen]
          ]
          [\bf N\Sub{A}, tier=preterminal
            [\it Geschmack, name=RekGeschmack]
            {\draw [->, bend left=25] (.south) to node [below, near start] {\footnotesize\textsc{Kasus}} (RekDessen.south);}
          ]
        ]
        [VP, calign=last
          [NP, tier=preterminal
            [\it ich, narroof]
          ]
          [\Ti, tuerkistree]
          [\bf V, tier=preterminal
            [\it mag]
            {\draw [->, bend left=15] (.south) to node [below, near start] {\footnotesize\textsc{Kasus}} (RekGeschmack.south);}
          ]
        ]
      ]
    ]
  \end{forest}
\end{frame}



\section{Objektsätze}

\begin{frame}
  {Objektsätze}
  \pause
  \begin{exe}
    \ex{\label{ex:komplementsaetze127} Michelle weiß, [\rot{dass} die Corvette nicht anspringen wird].}
    \pause
    \ex\label{ex:komplementsaetze128}
    \begin{xlist}
      \ex{\label{ex:komplementsaetze129} Michelle will wissen, [\rot{wer} die Corvette gewartet hat].}
      \pause
      \ex{\label{ex:komplementsaetze130} Michelle will wissen, [\rot{ob} die Corvette gewartet wurde].}
    \end{xlist}
  \end{exe}
  \pause
  \Halbzeile
  \alert{Achtung: \textit{ob} ist eigentlich nur ein w-Wort ohne w (vgl.\ engl.\ \textit{whether}).}\\
  \pause
  \Halbzeile
\end{frame}

\begin{frame}
  {Regierende Verben und Alternationen}
  \pause
  \alert{Drei primäre Muster}, welche Satz-Objekte Verben regieren.\\
  \pause\Halbzeile
  \begin{exe}
    \ex\label{ex:komplementsaetze131}
    \begin{xlist}
      \ex[]{\label{ex:komplementsaetze132} Michelle behauptet, \alert{dass} die Corvette nicht anspringt.}
      \pause
      \ex[*]{\label{ex:komplementsaetze133} Michelle behauptet, \rot{wie\slash ob} die Corvette nicht anspringt.}
    \end{xlist}
    \pause
    \ex\label{ex:komplementsaetze134}
    \begin{xlist}
      \ex[*]{\label{ex:komplementsaetze135} Michelle untersucht, \rot{dass} der Vergaser funktioniert.}
      \pause
      \ex[]{\label{ex:komplementsaetze136} Michelle untersucht, \alert{wie\slash ob} der Vergaser funktioniert.}
    \end{xlist}
    \pause
    \ex\label{ex:komplementsaetze137}
    \begin{xlist}
      \ex[]{\label{ex:komplementsaetze138} Michelle hört, \alert{dass} die Nockenwelle läuft.}
      \pause
      \ex[]{\label{ex:komplementsaetze139} Michelle hört, \alert{wie\slash ob} die Nockenwelle läuft.}
    \end{xlist}
  \end{exe}
  \pause\Halbzeile
  Außerdem: \textit{dass} alterniert oft mit \textit{zu}-Infinitiv.\\
  \pause
  \Halbzeile
  \begin{exe}
  \ex\label{ex:komplementsaetze140}
  \begin{xlist}
    \ex{\label{ex:komplementsaetze141} Michelle glaubt, [\alert{dass} sie das Geräusch erkennt].}
    \pause
    \ex{\label{ex:komplementsaetze142} Michelle glaubt, [das Geräusch \alert{zu} erkennen].}
  \end{xlist}
  \end{exe}
\end{frame}

\begin{frame}
  {Stellung von Adverbial- und Komplementsätzen}
  \pause
  \Halbzeile
  \begin{exe}
  \ex\label{ex:komplementsaetze146}
  \begin{xlist}
    \ex[]{\label{ex:komplementsaetze147} \alert{[Dass sie unseren Kuchen mag]}, hat Sarah uns eröffnet.}
    \pause
    \ex[]{\label{ex:komplementsaetze148} Sarah hat uns eröffnet, \alert{[dass sie unseren Kuchen mag]}.}
    \pause
    \ex[?]{\label{ex:komplementsaetze149} Sarah hat uns, \rot{[dass sie unseren Kuchen mag]}, eröffnet.}
  \end{xlist}
    \pause

  \ex\label{ex:komplementsaetze150}
  \begin{xlist}
    \ex[]{\label{ex:komplementsaetze151} \alert{[Ob Pavel unseren Kuchen mag]}, haben wir uns oft gefragt.}
    \pause
    \ex[]{\label{ex:komplementsaetze152} Wir haben uns oft gefragt, \alert{[ob Pavel unseren Kuchen mag]}.}
    \pause
    \ex[?]{\label{ex:komplementsaetze153} Wir haben uns, \rot{[ob Pavel unseren Kuchen mag]}, oft gefragt.}
  \end{xlist}
    \pause
  \ex\label{ex:komplementsaetze154}
  \begin{xlist}
    \ex[]{\label{ex:komplementsaetze155} \alert{[Wer die Rosinen geklaut hat]}, wollen wir endlich wissen.}
    \pause
    \ex[]{\label{ex:komplementsaetze156} Wir wollen endlich wissen, \alert{[wer die Rosinen geklaut hat]}.}
    \pause
    \ex[?]{\label{ex:komplementsaetze157} Wir wollen, \rot{[wer die Rosinen geklaut hat]}, endlich wissen.}
  \end{xlist}
  \end{exe}
  \pause
  \Halbzeile
  \begin{itemize}[<+->]
    \item Fast immer Bewegung nach links oder Rechtsversetzung \alert{hinter VK}!\\
  \end{itemize}
  \pause
\end{frame}


\begin{frame}
  {Was heißt \orongsch{Rechtsversetzung}?}
  \onslide<+->
  \onslide<+->
  Ähnliche bisher wenig beachtete Strukturen | Rechtsversetzung von \rot{PPs}\\
  \onslide<+->
  \Halbzeile
  \begin{exe}
    \ex{Ich habe den Schrank \alert{zurückgebracht} \rot{ins Wohnzimmer}.\label{ex:schrank}}
    \ex{Wir würden viel \alert{geben} \rot{für den Frieden}.\label{ex:frieden}}
  \end{exe}
  \onslide<+->
  \Halbzeile
  Einfachste Modellierung | \alert{Adjunktionsbewegung} rechts an die Phrase (hier VP)\\
  \onslide<+->
  \Halbzeile
  \centering
  \centering
  \scalebox{0.7}{\begin{forest}
    [S, calign=child, calign child=2
      [NP\Sub{2}, tier=preterminal
        [\it Wir, narroof]
      ]
      [V\Sub{1}, tier=preterminal
        [\it würden, narroof]
      ]
      [VP, calign=first
        [VP, calign=last
          [\Tii, tier=preterminal]
          [NP, tier=preterminal
            [\it viel, narroof, tier=terminal]
          ]
          [\alert{\Tiii}, tier=preterminal]
          [V, calign=last
            [V, tier=preterminal
              [\it geben, tier=terminal]
            ]
            [\Ti, tier=terminal]
          ]
        ]
        [PP\Sub{3}, tier=preterminal, bluetree
          [\it für den Frieden, narroof, tier=terminal]
        ]
      ]
    ]
  \end{forest}}
\end{frame}

\begin{frame}
  {Rechtsadjunktion eines Nebensatzes}
  \onslide<+->
  \onslide<+->
  \centering 
  \scalebox{1}{\begin{forest}
    [S, calign=child, calign child=2
      [NP\Sub{2}, tier=preterminal
        [\it Wir, narroof]
      ]
      [V\Sub{1}, tier=preterminal
        [\it haben, narroof]
      ]
      [VP, calign=first
        [VP, calign=last
          [\Tii, tier=preterminal]
          [NP, tier=preterminal
            [\it uns, narroof]
          ]
          [AdvP, tier=preterminal
            [\it oft, narroof, tier=terminal]
          ]
          [\alert{\Tiii}, tier=preterminal]
          [V, calign=last
            [V, tier=preterminal
              [\it gefragt, tier=terminal]
            ]
            [\Ti, tier=terminal]
          ]
        ]
        [KP\Sub{3}, tier=preterminal, bluetree
          [\it ob Pavel unseren Kuchen mag, narroof, tier=terminal]
        ]
      ]
    ]
  \end{forest}}
  
\end{frame}

\begin{frame}
  {Korrelate bei Komplementsätzen}
  \onslide<+->
  \onslide<+->
  Komplementsätze werden also meistens aus der VP herausbewegt.\\
  \Halbzeile
  Anstelle des Nebensatzes kann ein optionales \alert{Korrelat} stehen.\\
  \Zeile
  \onslide<+->
  \begin{exe}
    \ex\label{ex:komplementsaetze158}
    \begin{xlist}
      \ex{\label{ex:komplementsaetze159} Sarah hat \alert{es} uns eröffnet, [dass sie unseren Kuchen mag].}
      \ex{\label{ex:komplementsaetze160} Wir haben \alert{es} uns gefragt, [ob Pavel unseren Kuchen mag].}
      \ex{\label{ex:komplementsaetze161} Wir wollen \alert{es} wissen, [wer die Rosinen geklaut hat].}
    \end{xlist}
  \end{exe}
\end{frame}


\begin{frame}
  {Korrelate bei Subjektsätzen}
  \onslide<+->
  \onslide<+->
  Subjektskorrelate, immer \alert{vor} dem Subjektsatz.\\
  \Zeile
  \begin{exe}
  \ex\label{ex:komplementsaetze166}
  \begin{xlist}
    \ex[ ]{\label{ex:komplementsaetze167} \alert{Es} hat uns gefreut, [dass Sarah unseren Kuchen mochte].}
    \ex[ ]{\label{ex:komplementsaetze168} Uns hat \alert{es} gefreut, [dass Sarah unseren Kuchen mochte].}
    \ex[ ]{\label{ex:komplementsaetze169} Uns hat gefreut, [dass Sarah unseren Kuchen mochte].}
    \ex[*]{\label{ex:komplementsaetze170} [Dass Sarah unseren Kuchen mochte], hat \rot{es} uns gefreut.}
  \end{xlist}
\end{exe}
\end{frame}

\begin{frame}
  {Obligatorische Korrelate von Präpositionalobjekten}
  \onslide<+->
  \onslide<+->
  Objektsätze können auch Präpositionalobjekte vertreten.\\
  \onslide<+->
  \Zeile
  \begin{exe}
  \ex\label{ex:komplementsaetze162}
  \begin{xlist}
    \ex[]{\label{ex:komplementsaetze163} Ich weise [auf den leckeren Kuchen] hin.}
    \Halbzeile
    \ex[]{\label{ex:komplementsaetze164} Ich weise \alert{darauf} hin, [dass der Kuchen lecker ist].}
    \ex[*]{\label{ex:komplementsaetze165} Ich weise hin, [dass der Kuchen lecker ist].}
  \end{xlist}
  \end{exe}
  \onslide<+->
  \Zeile
  Vertritt der Objektsatz ein Präpositionalobjekt, ist das Korrelat manchmal obligatorisch.
\end{frame}

\section{Feldermodell}

\begin{frame}
  {Das topologische Satzmodell}
  \onslide<+->
  \onslide<+->
  \begin{itemize}[<+->]
    \item (Neben-)Sätze werden eingeteilt in \alert{Felder} und die \rot{Satzklammer} \\
      \Halbzeile
      \alert{Vorfeld} | \rot{linke Klammer} | \alert{Mittelfeld} | \rot{rechte Klammer} | \alert{Nachfeld}\\
      \Halbzeile
      \grau{\ldots\ und ggf.\ weitere Felder}
      \Zeile
    \item angeblich eine vereinfachte Analyse deutscher Syntax
    \item keine hierarchische Struktur, nur topologische Anordnung
    \item nicht ordentlich rekursiv
      \Zeile
    \item \rot{führt bei komplexeren Sätzen prinzipiell zu 0 Punkten in Klausuren}
    \item meines Erachtens überflüssig, aber populär in bestimmten Didaktiken
  \end{itemize}
\end{frame}

  
\begin{frame}
  {Felder im unabhängigen Aussagesatz}
  \centering 
    \resizebox{\textwidth}{!}{
      \begin{tabular}{cp{0.1em}cp{0.1em}cp{0.1em}c}
        \textbf{Vf} && \textbf{LSK} && \textbf{Mf} && \textbf{RSK} \\
        \cmidrule{1-1}\cmidrule{3-3}\cmidrule{5-5}\cmidrule{7-7}
        &&&&&&\\
        \alert{irgendeine Konstituente} && \alert{finites Verb} && \alert{(Rest)} && \alert{infinite Verben} \\
        &&&&&&\\
        \orongsch{\textit{das Bild}} && \orongsch{\textit{hat}} && \orongsch{\textit{Ischariot wahrscheinlich}} && \orongsch{\textit{verkauft}} \\
      \end{tabular}
    }
\end{frame}

\begin{frame}
  {Felder im eingeleiteten Nebensatz}
  \centering 
  \resizebox{\textwidth}{!}{
    \begin{tabular}{cp{0.1em}cp{0.1em}cp{0.1em}c}
      \textbf{Vf} && \textbf{LSK} && \textbf{Mf} && \textbf{RSK} \\
      \cmidrule{1-1}\cmidrule{3-3}\cmidrule{5-5}\cmidrule{7-7}
        &&&&&&\\
        \alert{(leer)} && \alert{Komplementierer} && \alert{(Rest)} && \alert{Verbkomplex} \\
        &&&&&&\\
        && \orongsch{\textit{dass}} && \orongsch{\textit{Ischariot das Bild wahrscheinlich}} && \orongsch{\textit{verkauft hat}} \\
    \end{tabular}
  }
\end{frame}

\begin{frame}
  {Felder im Ja\slash Nein-Fragesatz}
  \centering 
    \begin{tabular}{cp{0.1em}cp{0.1em}cp{0.1em}c}
      \textbf{Vf} && \textbf{LSK} && \textbf{Mf} && \textbf{RSK} \\
      \cmidrule{1-1}\cmidrule{3-3}\cmidrule{5-5}\cmidrule{7-7}
        &&&&&&\\
      \alert{(leer)} && \alert{finites Verb} && \alert{(Rest)} && \alert{infinite Verben} \\
        &&&&&&\\
      && \orongsch{\textit{hat}} && \orongsch{\textit{Ischariot das Bild}} && \orongsch{\textit{verkauft}} \\
    \end{tabular}
\end{frame}

\begin{frame}
  {Felder im Relativsatz}
  \centering 
  \resizebox{\textwidth}{!}{
    \begin{tabular}{cp{0.1em}cp{0.1em}cp{0.1em}c}
      \textbf{Vf} && \textbf{LSK} && \textbf{Mf} && \textbf{RSK} \\
      \cmidrule{1-1}\cmidrule{3-3}\cmidrule{5-5}\cmidrule{7-7}
        &&&&&&\\
      \alert{Relativpronomen} && \alert{(leer)} && \alert{(Rest)} && \alert{Verbkomplex} \\
        &&&&&&\\
      \orongsch{\textit{dem}} &&&& \orongsch{\textit{Ischariot das Bild wahrscheinlich}} && \orongsch{\textit{verkauft hat}} \\
    \end{tabular}
  }
\end{frame}

\begin{frame}
  {Felderanalyse mit Nachfeld} 
  \centering 
  \resizebox{\textwidth}{!}{
    \begin{tabular}{cp{0.1em}cp{0.1em}cp{0.1em}cp{0.1em}c}
      \textbf{Vf} && \textbf{LSK} && \textbf{Mf} && \textbf{RSK} && \textbf{Nf} \\
      \cmidrule{1-1}\cmidrule{3-3}\cmidrule{5-5}\cmidrule{7-7}\cmidrule{9-9}
        &&&&&&&&\\
      \orongsch{\textit{Ischariot}} && \orongsch{\textit{hat}} && \orongsch{\textit{dem Arzt das Bild}} && \orongsch{\textit{verkauft}} && \gruen{\textit{das er selber gemalt hatte}} \\
    \end{tabular}
  }
\end{frame}


\begin{frame}
  {Felderanalyse mit Konnektorfeld}
  \centering 
    \begin{tabular}{cp{0.1em}cp{0.1em}cp{0.1em}cp{0.1em}c}
    \textbf{Kf} && \textbf{Vf} && \textbf{LSK} && \textbf{Mf} && \textbf{RSK} \\
    \cmidrule{1-1}\cmidrule{3-3}\cmidrule{5-5}\cmidrule{7-7}\cmidrule{9-9}
        &&&&&&&&\\
    \tuerkis{\textit{denn}} && \orongsch{\textit{Ischariot}} && \orongsch{\textit{hat}} && \orongsch{\textit{ihm das Bild}} && \orongsch{\textit{verkauft}} \\
  \end{tabular}
\end{frame}


\begin{frame}
  {Felder | Zusammengefasst}
    \centering
  \resizebox{1\textwidth}{!}{
    \begin{tabular}{lp{0.3cm}llll}
    \lsptoprule
    \textbf{Satztyp} && \textbf{Vorfeld} & \textbf{LSK} & \textbf{Mittelfeld} & \textbf{RSK} \\
    \midrule
    \textbf{V2} && bel.\ Satzglied & finites Verb    & Rest der VP & infinite Verben \\
    \textbf{V1} && ---                  & finites Verb    & Rest der VP & infinite Verben \\
    \textbf{VL} && ---                  & Komplementierer & Rest der VP & Verbkomplex \\
    \lspbottomrule
  \end{tabular}
  }
\end{frame}

\begin{frame}
  {Felder und Konstituenten}
  \centering
  \begin{forest}
    [S, calign=child, calign child=2, l sep+=2em
      [AdvP\Sub{2}, tier=preterminal, name=Vfnode
        [\it wahrscheinlich, narroof, name=Vfterm]
      ]
      [\bf V\Sub{1}, tier=preterminal
        [\it hat, name=Lskterm]
      ]
      [VP, calign=last, s sep+=0.5em
        [NP, tier=preterminal
          [\it Ischariot, narroof, name=Mffirstterm]
        ]
        [\Tii, tier=preterminal]
        [NP, tier=preterminal
          [\it dem Arzt, narroof]
        ]
        [NP, tier=preterminal
          [\it das Bild, narroof]
        ]
        [AdvP, tier=preterminal
          [\it heimlich, narroof, name=Mflastterm]
        ]
        [\bf V, calign=last, name=Rsknode
          [\bf V, tier=preterminal
            [\it verkauft, name=Rskfirstterm]
          ]
          [\Ti, tier=preterminal, name=Rsklastterm]
        ]
      ]
      {\draw ($(Vfterm.west |- Vfnode.north) + (-0.2,0.3)$) -- ($(Vfterm.east |- Vfnode.north) + (0.15,0.3)$) -- ($(Vfterm.south east) + (0.15,0)$) -- node [midway, below] {Vf} ($(Vfterm.south west) + (-0.2,0)$) -- cycle;}
      {\draw ($(Lskterm.west |- Vfnode.north) + (0.05,0.3)$) -- ($(Lskterm.east |- Vfnode.north) + (-0.05,0.3)$) -- ($(Lskterm.east |- Vfterm.south) + (-0.05,0)$) -- node [midway, below] {LSK} ($(Lskterm.west |- Vfterm.south) + (0.05,0)$) -- cycle;}
      {\draw ($(Mffirstterm.west |- Vfnode.north) + (-0.15,0.3)$) -- ($(Mflastterm.east |- Vfnode.north) + (0.25,0.3)$) -- ($(Mflastterm.east |- Vfterm.south) + (0.25,0)$) -- node [midway, below] {Mf} ($(Mffirstterm.west |- Vfterm.south) + (-0.15,0)$) -- cycle;}
      {\draw ($(Rskfirstterm.west |- Rsknode.north) + (-0.025,0.3)$) -- ($(Rsklastterm.east |- Rsknode.north) + (0,0.3)$) -- ($(Rsklastterm.east |- Vfterm.south) + (0,0)$) -- node [midway, below] {RSK} ($(Rskfirstterm.west |- Vfterm.south) + (-0.025,0)$) -- cycle;}
    ]
  \end{forest}
  
\end{frame}

\ifdefined\TITLE
  \section{Zur nächsten Woche | Überblick}

  \begin{frame}
  {Deutsche Syntax | Plan}
  \rot{Alle} angegebenen Kapitel\slash Abschnitte aus \rot{\citet{Schaefer2018b}} sind \rot{Klausurstoff}!\\
  \Halbzeile
  \begin{enumerate}
    \item Grammatik und Grammatik im Lehramt \rot{(Kapitel 1 und 3)}
    \item Grundbegriffe \rot{(Kapitel 2)}
    \item Wortklassen \rot{(Kapitel 6)}
    \item Konstituenten und Satzglieder \rot{(Kapitel 11 und Abschnitt 12.1)}
    \item Nominalphrasen \rot{(Abschnitt 12.3)}
    \item Andere Phrasen \rot{(Abschnitte 12.2 und 12.4--12.7)}
    \item Verbphrasen und Verbkomplex \rot{(Abschnitte 12.8)}
    \item Sätze \rot{(Abschnitte 12.9 und 13.1--13.3)} 
    \item Nebensätze \rot{(Abschnitt 13.4)}
    \item \alert{Subjekte und Prädikate} \rot{(Abschnitte 14.1--14.3)}
    \item Passive und Objekte \rot{(14.4 und 14.5)}
    \item Syntax infiniter Verbformen \rot{(Abschnitte 14.7--14.9)}
  \end{enumerate}
  \Halbzeile
  \centering 
  \url{https://langsci-press.org/catalog/book/224}
  \end{frame}
\fi
