
\section{Überblick}


\begin{frame}
  {Infinitivsyntax}
  \begin{itemize}[<+->]
    \item morphologische vs.\ analytische Tempora
    \item Ersatzinfinitiv und Oberfeldumstellung
      \Halbzeile
    \item köhärente und inkohärente Infinitive
    \item Modalverben und Halbmodale
    \item Kontrollverben
    \Zeile
    \item \citet{Schaefer2018b}
  \end{itemize}
\end{frame}

\section{Analytische Tempora}

\begin{frame}
  {Beispiele für analytische Tempora}
  \begin{exe}
  \ex\label{ex:analytischetempora216}
  \begin{xlist}
    \ex{\label{ex:analytischetempora217} dass der Hufschmied das Pferd [behuft hat]}
    \ex{\label{ex:analytischetempora218} dass der Hufschmied das Pferd [behufen wird]}
  \end{xlist}
  \ex\label{ex:analytischetempora219}
  \begin{xlist}
    \ex{\label{ex:analytischetempora220} dass der Hufschmied das Pferd [behuft hatte]}
    \ex{\label{ex:analytischetempora221} dass der Hufschmied das Pferd [[behuft haben] wird]}
  \end{xlist}
  \end{exe}
\end{frame}

\begin{frame}
  {Die analytischen Tempora des Deutschen}
    \centering
  \begin{tabular}{lll}
    \lsptoprule
    & \textbf{Hilfsverb} & \textbf{regierter Status} \\
    \midrule
    \textbf{Futur} & \textit{werden} & 1 (Infinitiv) \\
    \textbf{Perfekt} & \textit{haben}\slash\textit{sein} & 3 (Partizip) \\
    \lspbottomrule
  \end{tabular}\\
  \raggedright
  \Zeile
  Finit sind nur Präsens, Präteritum und Futur!
  \Halbzeile
  \begin{itemize}[<+->]
    \item Perfekt = Präsensperfekt
    \item Plusquamperfekt = Präteritumsperfekt
    \item Futur II = Futurperfekt
  \end{itemize}
\end{frame}

\begin{frame}
  {Der Infinitiv des Perfekts}
  \begin{exe}
    \ex \alert{behuft haben}
    \ex \alert{geplatzt sein}
  \Zeile
  \ex\label{ex:analytischetempora222}
  \begin{xlist}
    \ex{\label{ex:analytischetempora223} dass der Hufschmied das Pferd [[behuft haben] will]}
    \ex{\label{ex:analytischetempora224} dass der Hufschmied das Pferd [[[behuft gehabt] haben] will]}
  \end{xlist}
  \end{exe}
\end{frame}

\begin{frame}
  {Analyse als Verbkomplex}
  \centering
  \begin{forest}
    [\textbf{V}, calign=last
      [\textbf{V}, calign=last
        [\textbf{V}, calign=last
          [\textbf{V}, tier=preterminal
            [\textit{behuft}]
          ]
          [\textbf{V}, tier=preterminal
            [\textit{gehabt}]
            {\draw [->, bend left=45] (.south) to node [below, midway] {\footnotesize\textsc{Status} (3)} (!uu11.south);}
          ]
        ]
        [\textbf{V}, tier=preterminal
          [\textit{haben}]
          {\draw [->, bend left=45] (.south) to node [below, midway] {\footnotesize\textsc{Status} (3)} (!uu121.south);}
        ]
      ]
      [\textbf{V}, tier=preterminal
        [\textit{will}]
        {\draw [->, bend left=45] (.south) to node [below, midway] {\footnotesize\textsc{Status} (1)} (!uu121.south);}
      ]
    ]
  \end{forest}
\end{frame}


\begin{frame}
  {Unterschiede zwischen Präteritum und Präsensperfekt}
  Stilistische Unterschiede\\
  \Halbzeile
  \begin{exe}
  \ex\label{ex:analytischetempora226}
  \begin{xlist}
    \ex{\label{ex:analytischetempora227} Das Pferd lief im Kreis.}
    \ex{\label{ex:analytischetempora228} Das Pferd ist im Kreis gelaufen.}
  \end{xlist}
  \end{exe}
  \Zeile
  Semantische Unterschiede\\
  \Halbzeile
  \begin{exe}
  \ex\label{ex:analytischetempora229}
  \begin{xlist}
    \ex{\label{ex:analytischetempora230} Im Jahr 1993 hat der Kommerz den Techno erobert.}
    \ex{\label{ex:analytischetempora231} Im Jahr 1993 eroberte der Kommerz den Techno.}
  \end{xlist}
  \end{exe}
\end{frame}


\begin{frame}
  {Nichtkanonische Infinitivrektion}
  Die sogenannte Oberfeldumstellung mit Ersatzinfinitiv\\
  \Zeile
  \begin{exe}
    \ex{\label{ex:ersatzinfinitivundoberfeldumstellung238} dass der Junge [hat [[schwimmen] wollen]]}
  \end{exe}
  \Zeile
    \centering
  \begin{forest}
    [\textbf{V}, calign=first
      [\textbf{V}, tier=preterminal
        [\textit{hat}\\1]
      ]
      [\textbf{V}, calign=last
        [\textbf{V}, tier=preterminal
          [\textit{schwimmen}\\3]
        ]
        [\textbf{V}, tier=preterminal
          [\textit{wollen}\\2]
          {\draw [->, bend left=20] (.south) to node [below, near end] {\footnotesize\textsc{Status} (1)} (!uu11.south);}
          {\draw [<-, bend left=60] (.south) to node [below, midway] {\footnotesize\textsc{Status} (1)} (!uuu11.south);}
        ]
      ]
    ]
  \end{forest}
\end{frame}


\section{Infinitivsyntax}

\begin{frame}
  {Kommas bei \textit{Infinitvkonstruktionen}}
  \begin{exe}
    \ex[*]{Nadezhda \rot{scheint}, die Kontrolle über die Hantel zu verlieren.}
    \ex[*]{Nadezhda \rot{will}, die Weltmeisterschaft gewinnen.}
    \ex[ ]{Nadezhda \alert{beschließt}, keine Steroide mehr einzunehmen.}
    \ex[?]{Nadezhda \alert{beschließt}\orongsch{,} zu trainieren.}
  \end{exe}
  \onslide<+->
  \Zeile
  \begin{itemize}[<+->]
    \item \alert{Infinitivsyntax} ist der Schlüssel
    \item Komma nur bei \alert{inkohärenten Infinitiven}
  \end{itemize}
\end{frame}

\begin{frame}
  {(In)kohärente Infinitive}
  Kohärente und inkohärente Infinitivkonstruktionen\\
  \onslide<+->
  \Zeile
  \centering
  \scalebox{0.7}{%
  \begin{forest}
    [VP\Sub{1+2}, calign=last
      [NP, tier=preterminal
        [\textit{Vanessa}, narroof]
      ]
      [NP, tier=preterminal
        [\textit{die Pferde}, narroof]
      ]
      [\textbf{V\Sub{2+1}}, calign=last
        [\textbf{V\Sub{2}}, tier=preterminal
          [\textit{behufen}]
        ]
        [\textbf{V\Sub{1}}, tier=preterminal
          [\textit{will}]
        ]
      ]
    ]
  \end{forest}}~\hspace{2em}~%
  \scalebox{0.7}{%
  \begin{forest}
    l sep+=3em, s sep+=2em
    [VP\Sub{1}, calign=last
      [NP, tier=preterminal
        [\textit{Vanessa}, narroof]
      ]
      [VP\Sub{2}, calign=last
        [NP, tier=preterminal
          [\textit{die Pferde}, narroof]
        ]
        [\textbf{V\Sub{2}}, tier=preterminal
          [\textit{zu behufen}]
        ]
      ]
      [\textbf{V\Sub{1}}, tier=preterminal
        [\textit{wünscht}]
      ]
    ]
  \end{forest}}
\end{frame}


\begin{frame}
  {Test | Herausstellbarkeit}
  \begin{exe}
  \ex\label{ex:kohaerenz243}
  \begin{xlist}
    \ex[ ]{\label{ex:kohaerenz244} Oma glaubt, dass Vanessa \Ti\ wünscht, [die Pferde zu behufen]\ORi.}
    \ex[*]{\label{ex:kohaerenz245} Oma glaubt, dass Vanessa \Ti\ will, [die Pferde behufen]\ORi.}
  \end{xlist}
\end{exe}
\end{frame}


\begin{frame}
  {Halbmnodale}
  Scheinbar gleich strukturiert\\
  \Halbzeile
  \begin{exe}
  \ex\label{ex:modalverbenundhalbmodalverben247}
  \begin{xlist}
    \ex{\label{ex:modalverbenundhalbmodalverben248} dass der Hufschmied das Pferd behufen will.}
    \ex{\label{ex:modalverbenundhalbmodalverben249} dass der Hufschmied das Pferd zu behufen scheint.}
    \ex{\label{ex:modalverbenundhalbmodalverben250} dass der Hufschmied das Pferd zu behufen beschließt.}
  \end{xlist}
\end{exe}
\Zeile
  Aber Abweichung bei der Extrahierbarkeit\\
  \Halbzeile
  \begin{exe}
  \ex\label{ex:modalverbenundhalbmodalverben251}
  \begin{xlist}
    \ex[*]{\label{ex:modalverbenundhalbmodalverben252} dass der Hufschmied will, das Pferd behufen}
    \ex[*]{\label{ex:modalverbenundhalbmodalverben253} dass der Hufschmied scheint, das Pferd zu behufen}
    \ex[ ]{\label{ex:modalverbenundhalbmodalverben254} dass der Hufschmied beschließt, das Pferd zu behufen}
  \end{xlist}
\end{exe}
\end{frame}

\begin{frame}
  {Halbmodale | \textit{scheinen} ohne Subjektrolle}
  Subjekt von \textit{scheinen} nicht erfragbar\\
  \Halbzeile
\begin{exe}
  \ex\label{ex:modalverbenundhalbmodalverben255}
  \begin{xlist}
    \ex[ ]{\label{ex:modalverbenundhalbmodalverben256} Frage: Wer will das Pferd behufen?\\
      Antwort: Der Hufschmied will das.}
    \ex[*]{\label{ex:modalverbenundhalbmodalverben257} Frage: Wer scheint das Pferd zu behufen?\\
      Antwort: Der Hufschmied scheint das.}
    \ex[ ]{\label{ex:modalverbenundhalbmodalverben258} Frage: Wer beschließt, das Pferd zu behufen?\\
      Antwort: Der Hufschmied beschließt das.}
  \end{xlist}
\end{exe}
\Zeile
Und \textit{scheinen} kann kein subjektloses Verb einbetten\\
\Halbzeile
\begin{exe}
  \ex\label{ex:modalverbenundhalbmodalverben259}
  \begin{xlist}
    \ex[*]{\label{ex:modalverbenundhalbmodalverben260} Dem Hufschmied will grauen.}
    \ex[ ]{\label{ex:modalverbenundhalbmodalverben261} Dem Hufschmied scheint zu grauen}
    \ex[*]{\label{ex:modalverbenundhalbmodalverben262} Dem Hufschmied beschließt zu grauen.}
  \end{xlist}
\end{exe}
\end{frame}


\begin{frame}
  {(In)kohärente Infinitve}
  \onslide<+->
  \onslide<+->
    \resizebox{1\textwidth}{!}{
    \begin{tabular}{lcllll}
      \lsptoprule
      & \multirow{2}{*}{\textbf{Status}} & \multirow{2}{*}{\textbf{Kohärenz}} & \textbf{eigenes} & \textbf{Subjekts-} \\
      & & & \textbf{Subjekt} & \textbf{Rolle} & \textbf{Beispiel}\\
      \midrule
      \textbf{Modalverben} & 1 & obl.\ kohärent & ja & Identität & \textit{wollen} \\
      \textbf{Halbmodalverben} & 2 & obl.\ kohärent & nein & nein & \textit{scheinen} \\
      \textbf{Kontrollverben} & 2 & \rot{opt.\ inkohärent} & ja & Kontrolle & \textit{beschließen} \\
      \lspbottomrule
    \end{tabular}
  }\\
  \Zeile
  \begin{itemize}[<+->]
    \item Nur \alert{inkohärente nachgestellte Infinitive} werden kommatiert!
    \item Sie gelten als satzwertig, aber \rot{Inkohärenz leider nur optional}.
    \item Es kommen also nur \alert{Abhängige von Halbmodalen} infrage.
  \end{itemize}
  \onslide<+->
  \Viertelzeile
  \begin{exe}
    \ex[*]{Nadezhda \rot{scheint}, die Kontrolle über die Hantel zu verlieren.}
    \ex[*]{Nedezhda \rot{will}, die Weltmeisterschaft gewinnen.}
  \end{exe}
\end{frame}

\begin{frame}
  {(In)kohärente Infinitve}
  \onslide<+->
  \onslide<+->
  Was ist jetzt hiermit?\\
  \Halbzeile
  \onslide<+->
  \begin{exe}
    \ex[ ]{Nadezhda \alert{beschließt}, keine Steroide mehr einzunehmen.}
    \ex[?]{Nadezhda \alert{beschließt}\orongsch{,} zu trainieren.}
  \end{exe}
  \onslide<+->
  \Halbzeile
  \alert{Eindeutig inkohärent} | hinter die RSK versetzte Infinitive\\
  \Viertelzeile
  \onslide<+->
  \begin{exe}
    \ex \rot{\textbf{Inkohärent}}
    \begin{xlist}
      \ex[ ]{\ldots dass Nadezhda beschließt, keine Steroide mehr zu nehmen.}
      \ex[*]{\ldots dass Nadezhda keine Steroide mehr zu nehmen beschließt.}
    \end{xlist}
    \ex \alert{\textbf{Kohärent oder inkohärent}}
    \begin{xlist}
      \ex[ ]{\ldots dass Nadezhda zu trainieren beschließt.}
      \ex[ ]{\ldots dass Nadezhda beschließt zu trainieren.}
    \end{xlist}
  \end{exe}
\end{frame}


\begin{frame}
  {(In)kohärente Infinitve}
  Es liegt also an der syntaktischen Struktur.\\
  \Zeile
  \onslide<+->
  \begin{exe}
    \ex
    \begin{xlist}
      \ex[ ]{[Nadezhda]\Sub{2} \alert{[beschließt]\Sub{1}} [t\Sub{2} \gruen{t\Sub{3}} \alert{[t\Sub{1}]\Sub{VK}}]\ \Sub{VP}\ \orongsch{,}\\
        {\hspace{1em}}\gruen{[keine Steroide mehr einzunehmen]\Sub{3}}.}
        \Viertelzeile
      \ex[*]{[Nadezhda]\Sub{2} \rot{[beschließt]\Sub{1}}\\
        {\hspace{1em}}[t\Sub{2} [keine Steroide] [mehr] \rot{[einzunehmen t\Sub{1}]\Sub{VK}}\ ]\Sub{VP}.}
    \end{xlist}
    \Halbzeile
    \ex
    \begin{xlist}
      \ex[ ]{[Nadezhda]\Sub{2} \alert{[beschließt]\Sub{1}}\ \orongsch{,} [t\Sub{2} \gruen{t\Sub{3}} \alert{[t\Sub{1}]\Sub{VK}}\ ]\Sub{VP} \gruen{[zu trainieren]\Sub{3}}.}
      \Viertelzeile
      \ex[ ]{[Nadezhda]\Sub{2} \tuerkis{[beschließt]\Sub{1}} [t\Sub{2} \tuerkis{[zu trainieren t\Sub{1}]\Sub{VK}}\ ]\Sub{VP}}
    \end{xlist}
  \end{exe}
  \Halbzeile
  \onslide<+->
  Füllen Sie den VK durch Hinzufügen von Hilfsverben auf,\\
  um das Phänomen noch deutlicher zu sehen.
\end{frame}


\section{Kontrollinfinitive}

\begin{frame}
  {\textit{zu}-Infinitive als Subjekte und Objekte}
  \begin{exe}
  \ex\label{ex:infinitivkontrolle264}
  \begin{xlist}
    \ex{\label{ex:infinitivkontrolle265} [Das Geschirr zu spülen] nervt Matthias. (Objektkontrolle)}
    \ex{\label{ex:infinitivkontrolle266} Doro wagt, [die Küche zu betreten]. (Subjektkontrolle)}
  \end{xlist}
\end{exe}
\Zeile
Auch mit Korrelat\\
\Halbzeile
\begin{exe}
  \ex\label{ex:infinitivkontrolle267}
  \begin{xlist}
    \ex{\label{ex:infinitivkontrolle268} Es nervt Matthias, [das Geschirr zu spülen].}
    \ex{\label{ex:infinitivkontrolle269} Doro wagt es, [die Küche zu betreten].}
  \end{xlist}
\end{exe}
\end{frame}

\begin{frame}
  {Erhalt der Kontrolle bei Passivierung}
\begin{exe}
  \ex\label{ex:infinitivkontrolle270}
  \begin{xlist}
    \ex{\label{ex:infinitivkontrolle271} Der Installateur hat gestern versucht, die Küche zu betreten.}
    \ex{\label{ex:infinitivkontrolle272} Gestern wurde versucht, die Küche zu betreten.}
  \end{xlist}
\end{exe}
\end{frame}

\begin{frame}
  {Kontrolle}
  \begin{block}
    {Infinitivkontrolle}
Die \textit{Kontrollrelation} besteht zwischen einer nominalen Valenzstelle eines Verbs und einem von diesem Verb abhängigen (subjektlosen) \textit{zu}"=Infinitiv.
Die Bedeutung des nicht ausgedrückten Subjekts des abhängigen \textit{zu}"=Infinitivs wird dabei durch die mit der nominalen Valenzstelle verbundene Bedeutung beigesteuert.
  \end{block}
\end{frame}


\begin{frame}
  {Subjektinfinitive meist mit Objektkontrolle}
  \begin{exe}
  \ex\label{ex:infinitivkontrolle274}
  \begin{xlist}
    \ex{\label{ex:infinitivkontrolle275} Das Geschirr zu spülen, nervt ihn.}
    \ex{\label{ex:infinitivkontrolle276} Das Geschirr zu spülen, fällt ihm leicht.}
    \ex{\label{ex:infinitivkontrolle277} Das Geschirr zu spülen, beschert ihm einen zufriedenen Mitbewohner.}
    \ex{\label{ex:infinitivkontrolle278} Sich für Hilfe zu bedanken, freut ihn immer besonders.}
  \end{xlist}
\end{exe}
\end{frame}

\begin{frame}
  {Objektinfinitive | präferiert Objektkontrolle}
  \begin{exe}
  \ex\label{ex:infinitivkontrolle279}
  \begin{xlist}
    \ex{\label{ex:infinitivkontrolle280} Er wagt, die Küche zu betreten.}
    \ex{\label{ex:infinitivkontrolle281} Er bittet seinen Mitbewohner, das Geschirr zu spülen.}
    \ex{\label{ex:infinitivkontrolle282} Doro erlaubt Matthias, sich den Wagen zu leihen.}
  \end{xlist}
\end{exe}
\end{frame}

\begin{frame}
  {Infinitivangaben | Subjektkontrolle}
  \begin{exe}
  \ex\label{ex:infinitivkontrolle283}
  \begin{xlist}
    \ex{\label{ex:infinitivkontrolle284} Matthias arbeitet, um Geld zu verdienen.}
    \ex{\label{ex:infinitivkontrolle285} Matthias begrüßt Doro, ohne aus der Rolle zu fallen.}
    \ex{\label{ex:infinitivkontrolle286} Matthias hilft Doro, anstatt untätig daneben zu stehen.}
    \ex{\label{ex:infinitivkontrolle287} Matthias bringt Doro den Wagen zurück, ohne den Lackschaden \\zu erwähnen.}
  \end{xlist}
\end{exe}
\end{frame}
