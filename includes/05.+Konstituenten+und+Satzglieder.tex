

\section{Überblick}

\begin{frame}
  {Überblick: Konstituenten und Phrasen}
  \pause
  \begin{itemize}[<+->]
    \item Warum und wie syntaktische Analyse?
    \item syntaktische Generalisierungen formulieren
    \item größere und kleinere Teilstrukturen (Konstituenten) identifizieren
  \end{itemize}
\end{frame}


\section{Konstituenten}


\begin{frame}
  {Generalisierungen anhand von Wortklassen in der Syntax}
  \pause
  Denkbare Abstraktion für einen Satzbauplan anhand von Wortklassen:\\
  \Zeile
  \pause
  \begin{center}
    \begin{forest}
      [Satz
        [\it Ein]
        [\it Snookerball]
        [\it ist]
        [\it eine]
        [\it Kugel]
        [\it aus]
        [\it Kunststoff]
      ]
    \end{forest}\\
    \pause
    \Halbzeile
    \begin{center}
      →
    \end{center}
    \Halbzeile
    \begin{forest}
      [Satz
        [Art]
        [Subst]
        [Kopula-Verb]
        [Art]
        [Subst]
        [Prp]
        [Subst]
      ]
    \end{forest}        
  \end{center}
\end{frame}


\begin{frame}
  {"`Flache Beschreibungen"'}
  \pause
  \rot{Solche flachen Strukturbeschreibungen sind extrem ineffizient!}\\
  \Zeile
  \pause
  Aus Korpus mit \alert{über 1 Mrd.\ Wörtern} (DeReKo) \alert{alle Sätze} mit der Struktur\\
  von der vorherigen Folie (Art Subst Kopula Art Subst Prp Subst):\\
  \pause
  \Zeile
  \begin{exe}
    \ex
    \begin{xlist}
      \ex{Die Verlierer sind die Schulkinder in Weyerbusch.}
      \pause
      \ex{Die Vienne ist ein Fluss in Frankreich.}
      \pause
      \ex{Ein Baustein ist die Begegnung beim Spiel.}
      \pause
      \ex{Das Problem ist die Ortsdurchfahrt in Großsachsen.}
    \end{xlist}
  \end{exe}
\end{frame}

\begin{frame}
  {Viele ähnliche Strukturen auf einmal beschreiben}
  \pause
  Strukturen, die ähnlich, aber \alert{nicht genau} \\
  \alert{[Art Subst Kopula Art Subst Prp Subst]} sind:\\
  \pause
  \Zeile
  \begin{exe}
    \ex\label{ex:syntaktischestruktur013}
    \begin{xlist}
      \ex{\label{ex:syntaktischestruktur014} [Dieses Endspiel] ist [eine spannende Partie].}
      \pause
      \ex{\label{ex:syntaktischestruktur015} [Eine Hose] war [eine Hose].}
      \pause
      \ex{\label{ex:syntaktischestruktur016} [Sieger] wurde [ein Teilnehmer aus dem Vereinigten Königreich].}
      \pause
      \ex{\label{ex:syntaktischestruktur017} [Lemmy] ist [Ian Kilmister].}
    \end{xlist}
  \end{exe}
  \pause
  \Halbzeile
  \begin{itemize}[<+->]
    \item Diese Sätze sie sind \alert{gleich aufgebaut}.
    \item Sie haben jeweils \alert{drei Konstituenten} (= Bestandteile).
    \item Die Konstituenten haben intern teilweise abweichende Strukturen.
    \item Aber ihre unterschiedlich aufgebauten Konstituenten (Nominalphrasen)\\
      verhalten sich in diesen Sätzen jeweils gleich. 
  \end{itemize}
\end{frame}


\begin{frame}
  {Bauplan und Analyse}
  \pause
  Bauplan "`Kopula-Satz"' (vorläufig):\\
  \pause
  \Halbzeile
  \begin{center}
    \begin{forest}
      [Satz
        [NP]
        [Kopula-Verb]
        [NP]
      ]
    \end{forest}\\
    \pause
    \Zeile
    \raggedright
    Analyse auf Basis dieses Plans (vorläufig):\\
    \pause
    \Halbzeile
    \centering
    \begin{forest}
      [Satz
        [NP
          [\it Dieses Endspiel, narroof]
        ]
        [Kopula-Verb
          [\it ist]
        ]
        [NP
          [\it eine spannende Partie, narroof]
        ]
      ]
    \end{forest}
  \end{center}
\end{frame}


\begin{frame}
  {Konstituenten und Konstituententests}
  \pause
  {\Large \alert{Konstituententests sollen uns helfen, herauszufinden,\\
  wie wir Sätze in Konstituenten unterteilen wollen.}}\\
  \Zeile
  \pause
  \rot{Achtung!}
  \pause
  \Halbzeile
  \begin{itemize}[<+->]
    \item \rot{Konstituententests sind heuristisch!}
    \item unerwünschte Ergebnisse in beide Richtungen
    \item keine "`wahre Konstituentenstruktur"'
    \item theorieabhängig bzw.\ abhängig von gewählten Tests
    \Zeile
    \item Ziel: kompakte Beschreibung aller möglichen Strukturen
    \item gewiss: möglichst "`natürliche"' Analyse erwünscht
  \end{itemize}
\end{frame}

\begin{frame}
  {Pronominalisierungstest}
  \pause
  \begin{exe}
    \ex Mausi isst \alert<3->{den leckeren Marmorkuchen}.\\
    \pause
      \KTArr{PronTest} Mausi isst \alert{ihn}.
    \pause
    \ex{\label{ex:konstituententests025} \rot<5->{Mausi isst} den Marmorkuchen.\\
    \pause
      \KTArr{PronTest} \Ast \rot{Sie} den Marmorkuchen.}
    \pause
    \ex{\label{ex:konstituententests026} Mausi isst \alert<7->{den Marmorkuchen und das Eis mit Multebeeren}.\\
    \pause
    \KTArr{PronTest} Mausi isst \alert{sie}.}
  \end{exe}
  \pause
  \Halbzeile
  Pronominalausdrücke i.\,w.\,S.:
  \begin{exe}
    \ex{\label{ex:konstituententests027} Ich treffe euch \alert<9->{am Montag} \gruen<10->{in der Mensa}.\\
    \pause
    \KTArr{PronTest} Ich treffe euch \alert{dann} \gruen<10->{dort}.}
      \pause
      \pause
      \ex{\label{ex:konstituententests028} Er liest den Text \alert<12->{auf eine Art, die ich nicht ausstehen kann}.\\
      \pause
      \KTArr{PronTest} Er liest den Text \alert{so}.}
  \end{exe}
\end{frame}

\begin{frame}
  {Vorfeldtest\slash Bewegungstest}
  \pause
  \begin{exe}
    \ex
    \begin{xlist}
      \ex{Sarah sieht den Kuchen \alert<3->{durch das Fenster}.\\
        \pause
        \KTArr{VfTest} \alert{Durch das Fenster} sieht Sarah den Kuchen.}
      \pause
      \ex{Er versucht \alert{zu essen}.\\
        \pause
        \KTArr{VfTest} \alert<5->{Zu essen} versucht er.}
      \pause
      \ex{Sarah möchte gerne \alert{einen Kuchen backen}.\\
        \pause
        \KTArr{VfTest} \alert<7->{Einen Kuchen backen} möchte Sarah gerne.}
      \pause
      \ex{Sarah möchte \rot<9->{gerne einen} Kuchen backen.\\
        \pause
        \KTArr{VfTest} \Ast \rot{Gerne einen} möchte Sarah Kuchen backen.}
    \end{xlist}
  \end{exe}
  \pause
  \Halbzeile
  verallgemeinerter "`Bewegungstest"':\\
  \begin{exe}
    \ex\label{ex:konstituententests037}
    \begin{xlist}
      \ex{\label{ex:konstituententests038} Gestern hat \alert<11->{Elena} \gruen<11->{im Turmspringen} \orongsch<11->{eine Medaille} gewonnen.}
      \pause
      \ex{\label{ex:konstituententests039} Gestern hat \gruen{im Turmspringen} \alert{Elena} \orongsch{eine Medaille} gewonnen.}
      \pause
      \ex{\label{ex:konstituententests040} Gestern hat \gruen{im Turmspringen} \orongsch{eine Medaille} \alert{Elena} gewonnen.}
    \end{xlist}
  \end{exe}
\end{frame}

\begin{frame}
  {Koordinationstest}
  \pause
  \begin{exe}
    \ex\label{ex:konstituententests041}
    \begin{xlist}
      \ex Wir essen \alert<3->{einen Kuchen}.\\
      \pause
        \KTArr{KoorTest} Wir essen \alert{einen Kuchen} \gruen{und} \alert{ein Eis}.
      \pause
      \ex Wir \alert<5->{essen einen Kuchen}.\\
      \pause
        \KTArr{KoorTest} Wir \alert{essen einen Kuchen} \gruen{und} \alert{lesen ein Buch}.
      \pause
      \ex Sarah hat versucht, \alert<7->{einen Kuchen zu backen}.\\
      \pause
        \KTArr{KoorTest} Sarah hat versucht, \alert{einen Kuchen zu backen} \gruen{und} \\{}\alert{heimlich das Eis aufzuessen}.
      \pause
      \ex Wir sehen, dass \alert<9->{die Sonne scheint}.\\
      \pause
        \KTArr{KoorTest} Wir sehen, dass \alert{die Sonne scheint} \gruen{und} \\{}\alert{Mausi den Rasen mäht}.
    \end{xlist}
  \end{exe}
  \pause
  \begin{exe}
    \ex{\label{ex:konstituententests047} Der Kellner notiert, dass \rot<11->{meine Kollegin einen Salat} möchte.\\
    \pause
    \KTArr{KoorTest} Der Kellner notiert, dass \rot{meine Kollegin einen Salat}\\
    \gruen{und} \rot{mein Kollege einen Sojaburger} möchte.}
    \end{exe}
\end{frame}



\section{Satzglieder}

\begin{frame}
  {Satzglieder?}
  \pause
  \begin{exe}
    \ex
    \begin{xlist}
      \ex Sarah riecht den Kuchen \alert<3->{mit ihrer Nase}.\\
      \pause
        \KTArr{VfTest} \alert{Mit ihrer Nase} riecht Sarah den Kuchen.
        \pause
      \ex \KTArr{KoorTest} Sarah riecht den Kuchen\\
      {}\alert{mit ihrer Nase} und \alert{trotz des Durchzugs}.
    \end{xlist}
    \pause
    \ex
    \begin{xlist}
      \ex Sarah riecht den Kuchen \gruen<6->{mit der Sahne}.\\
      \pause
        \KTArr{VfTest} \Ast \rot{Mit der Sahne} riecht Sarah den Kuchen.
        \pause
      \ex \KTArr{KoorTest} Sarah riecht den Kuchen\\
      {}\alert{mit der Sahne} und \alert{mit den leckeren Rosinen}.
    \end{xlist}
  \end{exe}
  \pause
  \resizebox{0.9\textwidth}{!}{
    \begin{forest}
      [Satz
        [\it Sarah]
        [\it riecht]
        [\it den Kuchen]
        [\it mit ihrer Nase]
      ]
    \end{forest}\pause\begin{forest}
      [Satz
        [\it Sarah, tier=term]
        [\it riecht, tier=term]
        [Konstituente X
          [\it den Kuchen, tier=term]
          [\it mit der Sahne, tier=term]
        ]
      ]
    \end{forest}
  }
\end{frame}

\begin{frame}
  {Satzglieder als "`vorfeldfähige Konstituenten"'}
  \pause
  Ganz so einfach ist das nicht\ldots\\
  \Zeile
  \pause
  \begin{exe}
    \ex \rot{[Kaufen können]} möchte Alma die Wolldecke.
    \pause
    \ex \rot{[Über Syntax]} hat Sarah sich \alert{ein Buch} ausgeliehen.
  \end{exe}
  \Zeile
  \pause
  \alert{Wozu überhaupt den begriff des Satzglieds?}
  \begin{itemize}[<+->]
    \item in der Linguistik kaum von Interesse
    \item Sammelbegriff für "`Objekte und Adverbiale"'? -- \rot{Wozu?}
    \item Vorfeldfähigkeit? -- Wohl kaum, denn das wäre \rot{zirkulär} (und s.\,o.).
    \item Desambiguierung von Sätzen (s.\ Kuchen-Nase)? --\\
      \rot{Dabei hilft aber der Begriff "`Satzglied"' nicht.}
    \item Außerdem: \alert{Fördert das die Sprachkompetenz, oder kann das weg?}
  \end{itemize}
\end{frame}

\begin{frame}
  {Strukturelle Ambiguitäten und Kompositionalität}
  \pause
  \begin{exe}
    \ex{\label{ex:strukturelleambiguitaet060} Scully sieht den Außerirdischen mit dem Teleskop.}
  \end{exe}
  \pause
  \Halbzeile
  \begin{block}{Erinnerung: Kompositionalität}
    Die syntaktische Struktur ist die Basis für die Interpretation des Satzes (bzw.\ jedes syntaktisch komplexen Ausdrucks).
  \end{block}
  \pause
  \Halbzeile
  \begin{exe}
    \ex
    \begin{xlist}
      \ex Scully sieht \gruen<5->{[den Außerirdischen]} \orongsch<6->{[mit dem Teleskop]}.
      \pause
      \pause
      \pause
      \ex Scully sieht \alert<8->{[den Außerirdischen [mit dem Teleskop]]}.
    \end{xlist}
  \end{exe}
\end{frame}



\begin{frame}
  {Repräsentationsformat: Phrasenschemata}
  \pause
  \begin{itemize}[<+->]
    \item \alert{Grammatikalität = Konformität zu einer spezifischen Grammatik}
    \item Strukturen ohne spezifizierte Struktur: \rot{ungrammatisch}
      \Halbzeile
    \item Phrasenschemata = \alert{Baupläne} für zulässige Strukturen
    \item Strukturen = \alert{Bäume}
    \item Bei einer konkreten Analyse muss für jede Verzweigung im Baum\\
      ein Phrasenschema vorliegen, \rot{sonst ist die Analyse nicht zulässig}.
  \end{itemize}
  \pause
  \Halbzeile
  \centering
  \begin{multicols}{2}
    \footnotesize Das Schema:~\scalebox{0.6}{%
      \begin{forest}
      phrasenschema, baseline
      [NP, Ephr, calign=last
        [Artikel, Eopt, Emult
          [Pronomen, Eopt]
        ]
        [A, Eoptrec]
        [N, Ehd]
      ]
    \end{forest}
    \hspace{4em}
    }
    \onslide<8->{\footnotesize erlaubt~die~Analyse:~\scalebox{0.6}{%
      \begin{forest}
        [NP, calign=last, baseline
          [Artikel
            [\it ein]
          ]
          [A
            [\it leckerer]
          ]
          [A
            [\it geräucherter]
          ]
          [\textbf{N}
            [\it Tofu]
          ]
        ]
      \end{forest}
    }
  }
  \end{multicols}
\end{frame}





\section{Ausblick}

\begin{frame}
  {Überblick: Konstituenten und Phrasen}
  \pause
  \begin{itemize}[<+->]
    \item Phrasen und Köpfe
    \item Strukur der deutschen \alert{Nominalphrase}
    \item (regierte) Attribute
      \Zeile
    \item \citet[Abschnitt~12.3]{Schaefer2018b}
  \end{itemize}
\end{frame}
