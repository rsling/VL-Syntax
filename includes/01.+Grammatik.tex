\section{Wiederholungsstoff}

\begin{frame}
  {Wiederholen Sie die erste Vorlesung \orongsch{Morphologie}}
  \Zeile
  \centering 
  Bitte schauen Sie sich die erste Woche aus meiner Vorlesung\\
  zur Morphologie (ggf.\ nochmals) an.\\
  \Zeile
  Deren Inhalte sind auch in der Syntax Klausurstoff.\\
\end{frame}

\section{Zur nächsten Woche | Überblick}

\begin{frame}
  {Deutsche Syntax | Plan}
  \rot{Alle} angegebenen Kapitel\slash Abschnitte aus \rot{\citet{Schaefer2018b}} sind \rot{Klausurstoff}!\\
  \Halbzeile
  \begin{enumerate}
    \item Grammatik und Grammatik im Lehramt \rot{(Kapitel 1 und 3)}
    \item \alert{Grundbegriffe} \rot{(Kapitel 2)}
    \item Wortklassen \rot{(Kapitel 6)}
    \item Konstituenten und Satzglieder \rot{(Kapitel 11 und Abschnitt 12.1)}
    \item Nominalphrasen \rot{(Abschnitt 12.3)}
    \item Andere Phrasen \rot{(Abschnitte 12.2 und 12.4--12.7)}
    \item Verbphrasen und Verbkomplex \rot{(Abschnitte 12.8)}
    \item Sätze \rot{(Abschnitte 12.9 und 13.1--13.3)} 
    \item Nebensätze \rot{(Abschnitt 13.4)}
    \item Subjekte und Prädikate \rot{(Abschnitte 14.1--14.3)}
    \item Passive und Objekte \rot{(14.4 und 14.5)}
    \item Syntax infiniter Verbformen \rot{(Abschnitte 14.7--14.9)}
  \end{enumerate}
  \Halbzeile
  \centering 
  \url{https://langsci-press.org/catalog/book/224}
\end{frame}

