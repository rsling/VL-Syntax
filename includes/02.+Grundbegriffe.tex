\section{Überblick}

\begin{frame}
  {Überblick}
  \onslide<+->
  \onslide<+->
  \begin{itemize}[<+->]
    \item \alert{Strukturbildung} | große Einheiten aus kleinen Einheiten
    \Zeile
    \item \alert{Relationen} | Kongruenz und Valenz
    \Zeile
    \item \alert{Valenz} | Verbklassen und Ereignisbeschreibung
  \end{itemize}
\end{frame}

\section{Struktur}

\begin{frame}
  {Auf allen Ebenen | Struktur}
  \onslide<+->
  \onslide<+->
  \alert{Struktur} | Einheiten sind aus Einheiten zusammengesetzt.
  \begin{exe}
    \ex\label{ex:strukturbildung021}
    \begin{xlist}
      \ex \textbf{Sätze} \\
      {\alert{[}Alexandra schießt den Ball ins gegnerische Tor.\alert{]}\ }
      \onslide<+->
      \ex \textbf{Satzteile} \\
      {\alert{[}Alexandra\alert{]}\  \alert{[}schießt\alert{]}\  \alert{[}den Ball\alert{]}\  \alert{[}ins gegnerische Tor\alert{]}\ }
      \onslide<+->
      \ex \textbf{Wörter} \\
      {\alert{[}Alexandra\alert{]}\  \alert{[}schießt\alert{]}\  \alert{[}den\alert{]}\  \alert{[}Ball\alert{]}\  \alert{[}ins\alert{]}\  \alert{[}gegnerische\alert{]}\  \alert{[}Tor\alert{]}\ }
      \onslide<+->
      \ex \textbf{Wortteile} \\
      {\alert{[}Alexandra\alert{]}\  \alert{[}schieß\alert{]}\ \alert{[}t\alert{]}\  \alert{[}den\alert{]}\  \alert{[}Ball\alert{]}\  \alert{[}ins\alert{]}\  \alert{[}gegner\alert{]}\ \alert{[}isch\alert{]}\ \alert{[}e\alert{]}\  \alert{[}Tor\alert{]}\ }
      \onslide<+-> 
      \ex \textbf{Laute} \\
      {\alert{[}A\alert{]}\ \alert{[}l\alert{]}\ \alert{[}e\alert{]}\ \alert{[}k\alert{]}\ \alert{[}s\alert{]}\ \alert{[}a\alert{]}\ \alert{[}n\alert{]}\ \alert{[}d\alert{]}\ \alert{[}r\alert{]}\ \alert{[}a\alert{]}\  \ldots \\}
    \end{xlist}
  \end{exe}
\end{frame}

\begin{frame}
  {Struktur in der Syntax}
  \onslide<+->
  \onslide<+->
  Durch mehrfache \alert{Aneinanderfügung} ergeben sich \alert{hirarchische Strukturen}.\\
  \Zeile
  \onslide<+->
  \centering 
    \begin{forest}
    [Alexandra schießt den Ball ins gegnerische Tor
      [Alexandra [Alexandra]]
      [schießt [schießt]]
      [den Ball
        [den]
        [Ball]
      ]
      [ins gegenerische Tor
        [ins]
        [gegnerische]
        [Tor]
      ]
    ]
  \end{forest}
\end{frame}

\begin{frame}
  {Struktur in der Morphologie}
  \onslide<+->
  \onslide<+->
  Auch innerhalb von Wörtern gibt es solche Strukturen.\\
  \Zeile
  \onslide<+->
  \centering 
    \begin{forest}
    [gegnerische
      [generisch
        [gegner]
        [isch, tier=terminal]
      ]
      [e, tier=terminal]
    ]
  \end{forest}
\end{frame}

\begin{frame}
  {Konstituenten}
  \onslide<+->
  \onslide<+->
  \centering 
  \begin{block}
    {Konstituenten einer Struktur}
    \textit{Konstituenten} einer Einheit sind die (meistens kleineren und höchstens genauso großen) Einheiten, aus denen eine Struktur besteht.    
  \end{block}
\end{frame}

\section{Rektion und Kongruenz}

\begin{frame}
  {Was sind Relationen?}
  \onslide<+->
  \onslide<+->
  \begin{exe}
    \ex\label{ex:rektionundkongruenz024}
    \begin{xlist}
      \ex{\label{ex:rektionundkongruenz025}[Dzsenifer] \alert{[schießt] } \orongsch{[ein Tor]}.}
      \ex{\label{ex:rektionundkongruenz026}[Kim] \alert{[läuft]} \gruen{[schnell]}.}
    \end{xlist}
  \end{exe}
  \onslide<+->
  \Zeile
  \begin{itemize}[<+->]
    \item \orongsch{\textit{ein Tor}} ist ein \orongsch{Objekt} zu \textit{schießt}.
    \item \gruen{\textit{schnell}} ist eine \gruen{adverbiale Bestimmung} zu \textit{läuft}.
      \Zeile
    \item Es gibt kein Objekt und keine adverbiale Bestimmung\\
      ohne ein Verb im Satzkontext \ldots
    \item die Begriffe \orongsch{Objekt} und \gruen{adverbiale Bestimmung} sind also \alert{relational}.
  \end{itemize}
\end{frame}

\begin{frame}
  \onslide<+->
  \onslide<+->
  {Rektion}
  Wörter (oder andere Einheiten) \alert{bestimmen}\\
  Eigenschaften anderer Wörter.\\
  \onslide<+->
  \Zeile
  \begin{exe}
    \ex[ ]{\gruen{Der Torwart} \alert{gedenkt} \orongsch{der Niederlage.}}
    \ex[ ]{\gruen{Der FCR Duisburg} \alert{besiegt} \orongsch{den FFC Frankfurt}.}
  \end{exe}
  \onslide<+->
  \Halbzeile
  \begin{exe}
    \ex
    \begin{xlist}
      \ex[*]{\rot{Den Torwart} \alert{gedenkt} \rot{die Niederlage}.}
    \ex{\rot{Des FCR Duisburgs} \alert{besiegt} \rot{dem FFC Frankfurt}.}
    \end{xlist}
  \end{exe}
\end{frame}

\begin{frame}
  {Kongruenz}
  \onslide<+->
  \onslide<+->
  Zwei oder mehr Wörter (bzw.\ Einheiten)\\
  \alert{stimmen in bestimmten Eigenschaften überein}\\
  \onslide<+->
  \Zeile
  \begin{exe}
    \ex
    \begin{xlist}
      \ex{Der FCR besiegt \gruen{den} \alert{gegnerischen Verein}.}
      \ex{Der FCR besiegt \alert{alle} \gruen{gegnerischen} \alert{Vereine}.}
    \end{xlist}
  \end{exe}
  \onslide<+->
  \Halbzeile
  \begin{exe}
    \ex
    \begin{xlist}
      \ex[*]{Der FCR besiegt \rot{die} \alert{gegnerischen Verein}.}
      \ex[*]{Der FCR besiegt \alert{alle} \rot{gegnerischer} \alert{Verein}.}
    \end{xlist}
  \end{exe}
\end{frame}

\begin{frame}
  {Wichtige Kongruenzrelationen im Deutschen}
  \onslide<+->
  \onslide<+->
  \alert{Subjekt-Verb-Kongruenz}\\
  \Viertelzeile
  Das Subjekt und das finite Verb eines Satzes stimmen in \alert{Person} und \alert{Numerus} überein.\\
  \Zeile
  \onslide<+->
  \Zeile
  \alert{Kongruenz in der Nominalgruppe}\\
  \Viertelzeile
  Zusammenstehende und zusammengehörige Artikelwörter, Adjektive und Substantive (in dieser Reihenfolge) stimmen in \alert{Numerus}, \alert{Genus} und \alert{Kasus} überein.
\end{frame}

\section{Valenz}

\begin{frame}
  {Traditionelle Verbtypen}
  \pause
  \begin{itemize}[<+->]
    \item traditionelle Termini für Verbtypen (s.\ Kapitel 14 für Neuordnung)
      \Halbzeile
      \begin{itemize}[<+->]
        \item \alert{intransitiv}: regiert nur einen Nominativ (\textit{leben}, \textit{schlafen})
          \Viertelzeile
        \item \alert{transitiv}: regiert einen Nominativ und einen Akkusativ (\textit{sehen}, \textit{lesen})
          \Viertelzeile
        \item \alert{ditransitiv}: regiert zusätzlich einen Dativ (\textit{geben}, \textit{schicken})
          \Viertelzeile
        \item \alert{präpositional transitiv}: regiert Nom und PP (\textit{leiden +unter})
          \Viertelzeile
        \item \alert{präpositional ditransitiv}: regiert Nom, Akk, PP (\textit{schreiben +an})
          \Viertelzeile
        \item \ldots
          \Zeile
      \end{itemize}
    \item nur Abkürzungen für einige (von sehr viel mehr) \alert{Valenztypen}
  \end{itemize}
\end{frame}

\begin{frame}
  {Ergänzungen und Angaben}
  \pause
  Siehe auch: Kapitel~2, Abschnitt~2.4 (S.~40--48)!\\
  \pause\Halbzeile
  \begin{exe}
    \ex\label{ex:valenz034}
    \begin{xlist}
      \ex{Gabriele malt \alert{[ein Bild]}.}
      \pause
      \ex{Gabriele malt \gruen{[gerne]}.}
      \pause
      \ex{Gabriele malt \gruen{[den ganzen Tag]}.}
      \pause
      \ex{Gabriele malt \gruen{[ihrem Mann]} \rot{[zu figürlich]}.}
    \end{xlist}
  \end{exe}
  \pause\Halbzeile
  \begin{itemize}[<+->]
    \item \alert{[ein Bild]} mit besonderer Relation zum Verb
    \item "`Weglassbarkeit"' (Optionalität) nicht entscheidend
  \end{itemize}
\end{frame}

\begin{frame}
  {Lizenzierung}
  \pause
  \begin{exe}
    \ex 
    \begin{xlist}
      \ex[ ]{Gabriele isst \gruen{[den ganzen Tag]} Walnüsse.}
    \pause
      \ex[ ]{Gabriele läuft \gruen{[den ganzen Tag]}.}
      \pause
      \ex[ ]{Gabriele backt ihrer Schwester \gruen{[den ganzen Tag]} Stollen.}
      \pause
      \ex[ ]{Gabriele litt \gruen{[den ganzen Tag]} unter Sonnenbrand.}
    \end{xlist}
    \pause\Halbzeile
    \ex 
    \begin{xlist}
      \ex[*]{Gabriele isst \alert{[ein Bild]} Walnüsse.}
      \pause
      \ex[*]{Gabriele läuft \alert{[ein Bild]}.}
      \pause
      \ex[*]{Gabriele backt ihrer Schwester \alert{[ein Bild]} Stollen.}
      \pause
      \ex[*]{Gabriele litt \alert{[ein Bild]} unter Sonnenbrand. }
      \pause
    \end{xlist}
  \end{exe}
  \pause\Halbzeile
  \begin{itemize}[<+->]
    \item \gruen{Angaben} sind verb-unspezifisch lizenziert
    \item \alert{Ergänzungen} sind verb(klassen)spezifisch \alert{genau einmal} lizenziert
    \item \rot{Valenz = Liste der Ergänzungen eines lexikalischen Worts}
  \end{itemize}
\end{frame}


\begin{frame}
  {Iterierbarkeit | Angaben sind beliebig stapelbar}
  \onslide<+->
  \onslide<+->
  \begin{exe}
    \ex[ ]{Wir müssen den Wagen\\
      \gruen{[jetzt]}\\
      \gruen{[mit aller Kraft]}\\
      \gruen{[vorsichtig]} anschieben.}
    \onslide<+->
    \ex[ ]{Wir essen \gruen{[schnell]}\\
    \gruen{[mit Appetit]}\\
    \gruen{[an einem Tisch]}\\
    \gruen{[mit der Gabel]}\\
    \alert{[einen Salat]}.}
    \onslide<+->
    \ex[*]{Wir essen \gruen{[schnell]}\\
    \rot{[ein Tofugericht]}\\
    \gruen{[mit Appetit]}\\
    \gruen{[an einem Tisch]}\\
    \gruen{[mit der Gabel]}\\
    \alert{[einen Salat]}.}
  \end{exe}
\end{frame}

\begin{frame}
  {Ergänzungen | Schnittstelle von Syntax und Semantik}
  \onslide<+->
  \onslide<+->
  Verbsemantik | Welche \alert{Rolle} spielen die von den Satzgliedern bezeichneten Dinge in der vom Verb beschriebenen Situation?\\
  \Zeile
  \onslide<+->
  Semantik von \alert{Ergänzungen} | \alert{abhängig} vom Verb\\
  \onslide<+->
  \Viertelzeile
  Semantik von \gruen{Angaben} | \gruen{unabhängig} vom Verb\\
  \Halbzeile
  \pause
  \begin{exe}
    \ex\label{ex:valenz071}
    \begin{xlist}
      \ex{\label{ex:valenz072}Ich lösche \alert{[den Ordner]} \gruen{[während der Hausdurchsuchung]}.}
      \pause
      \ex{\label{ex:valenz073}Ich mähe \alert{[den Rasen]} \gruen{[während der Ferien]}.}
      \pause
      \ex{\label{ex:valenz074}Ich fürchte \alert{[den Sturm]} \gruen{[während des Sommers]}.}
    \end{xlist}
  \end{exe}
\end{frame}

\begin{frame}
  {Valenz | Zusammenfassung}
  \onslide<+->
  \onslide<+->
  \alert{Angaben} sind grammatisch immer lizenziert\\
  und bringen ihre eigene semantische Rolle mit.\\
  \grau{Sie können aber semantisch\slash pragmatisch inkompatibel sein.}\\
  \Zeile
  \onslide<+->
  \Zeile
  \onslide<+->
  \gruen{Ergänzungen} werden spezifisch vom Verb lizenziert\\
  und in ihrer semantischen Rolle vom Verb festgelegt.\\
  Jede dieser Rollen kann nur einmal vergeben werden.
\end{frame}

\section{Ausblick}

\begin{frame}
  {Nächste Woche | Wortklassen}
  \begin{itemize}[<+->]
    \item Möglichkeiten, Wortklassen zu definieren
      \Halbzeile
    \item syntaktisch definierte Wortklassen
      \Halbzeile
    \item \citet[Kapitel~6]{Schaefer2018b}
  \end{itemize}
\end{frame}
