\section{Überblick}

\begin{frame}
  {Überblick}
  \onslide<+->
  \begin{itemize}[<+->]
    \item \alert{Strukturbildung} | große Einheiten aus kleinen Einheiten
    \Zeile
    \item \alert{Relationen} | Kongruenz und Valenz
    \Zeile
    \item \alert{Valenz} | Verbklassen und Ereignisbeschreibung
  \end{itemize}
\end{frame}

\section{Struktur}

\begin{frame}
  {Sprachliche Einheiten und ihre Bestandteile}
  \onslide<+->
  \onslide<+->
  Wichtig vor allem für die Syntax | \alert{Strukturbildung}\\
  \Zeile
  \begin{itemize}[<+->]
    \item\footnotesize \alert{Satz} \\
      {Nadezhda reißt die Hantel souveräner als andere Gewichtheberinnen.}
      \Halbzeile

    \item\footnotesize \alert{Satzteile} \\
      {Nadezhda | reißt | die Hantel | souveräner als andere Gewichtheberinnen}
      \Halbzeile

    \item\footnotesize \alert{Wörter} \\
      {Nadezhda | reißt | die | Hantel | souveräner | als | andere | Gewichtheberinnen}
      \Halbzeile

    \item\footnotesize \alert{Wortteile} \\
      {Nadezhda | reiß | t | d | ie | Hantel | souverän | er | als | ander | e | Gewicht | heb | er | inn | en}
      \Halbzeile

    \item\footnotesize \alert{Laute\slash Buchstaben} \\
      {N | a | d | e | z | h | d | a \ldots}
  \end{itemize}
\end{frame}


\begin{frame}
  {Syntaktische Strukturen}
  \onslide<+->
  \onslide<+->
  \begin{center}
  \resizebox{0.8\textwidth}{!}{\begin{forest}
    [Nadezhda reißt die Hantel souveräner als andere Gewichtheberinnen
      [Nadezhda]
      [reißt]
      [die Hantel
        [die]
        [Hantel]
      ]
      [souveräner als andere Gewichtheberinnen
        [souveräner]
        [als andere Gewichtheberinnen
          [als]
          [andere Gewichtheberinnen
            [andere]
            [Gewichtheberinnen]
          ]
        ]
      ]
    ]
  \end{forest}}
  \end{center}
\end{frame}


\begin{frame}
  {Struktur in der Morphologie}
  \onslide<+->
  \onslide<+->
  Auch innerhalb von Wörtern gibt es solche Strukturen.\\
  \Zeile
  \onslide<+->
  \centering 
    \begin{forest}
    [gegnerische
      [generisch
        [gegner]
        [isch, tier=terminal]
      ]
      [e, tier=terminal]
    ]
  \end{forest}
\end{frame}

\begin{frame}
  {Konstituenten}
  \onslide<+->
  \onslide<+->
  \centering 
  \begin{block}
    {Konstituenten einer Struktur}
    \textit{Konstituenten} einer Einheit sind die (meistens kleineren und höchstens genauso großen) Einheiten, aus denen eine Struktur besteht.    
  \end{block}
\end{frame}

\section{Rektion und Kongruenz}

\begin{frame}
  {Was sind Relationen?}
  \onslide<+->
  \onslide<+->
  \begin{exe}
    \ex\label{ex:rektionundkongruenz024}
    \begin{xlist}
      \ex{\label{ex:rektionundkongruenz025}[Martin] \alert{[zeigt] } \orongsch{[einen Schraubensprung]}.}
      \ex{\label{ex:rektionundkongruenz026}[Tina] \alert{[springt]} \gruen{[kraftvoll]}.}
    \end{xlist}
  \end{exe}
  \Zeile
  \begin{itemize}[<+->]
    \item \orongsch{\textit{einen Schraubensprung}} ist ein \orongsch{Objekt} zu \textit{zeigt}.
    \item \gruen{\textit{kraftvoll}} ist eine \gruen{adverbiale Bestimmung} zu \textit{springt}.
      \Zeile
    \item Es gibt kein Objekt und keine adverbiale Bestimmung\\
      ohne ein Verb im Satzkontext \ldots
    \item die Begriffe \orongsch{Objekt} und \gruen{adverbiale Bestimmung} sind also \alert{relational}.
  \end{itemize}
\end{frame}

\begin{frame}
  {Syntaktische Strukturen und morphologische Merkmale}
  \onslide<+->
  \onslide<+->
  \begin{center}
  \resizebox{0.8\textwidth}{!}{\begin{forest}
    [Nadezhda reißt die Hantel souveräner als andere Gewichtheberinnen
      [Nadezhda]
      [reißt]
      [die Hantel, alt=<3->{orongsch}{}
        [die, alt=<4->{orongsch}{}]
        [Hantel, alt=<5->{orongsch}{}]
      ]
      [souveräner als andere Gewichtheberinnen
        [souveräner]
        [als andere Gewichtheberinnen
          [als]
          [andere Gewichtheberinnen, alt=<6->{gruen}{}
            [andere, alt=<7->{gruen}{}]
            [Gewichtheberinnen, alt=<8->{gruen}{}]
          ]
        ]
      ]
    ]
  \end{forest}}
  \end{center}

  \Zeile
  \onslide<9->{Übereinstimmung von Merkmalen in syntaktischen Gruppen\\}
  \onslide<10->{\orongsch{Akkusativ Femininum Singular}} \onslide<11->{| \gruen{Nominativ Plural}}
\end{frame}

\begin{frame}
  {Kongruenz | NPs}
  \onslide<+->
  \onslide<+->
  \alert{Kongruenz} | Merkmalübereinstimmung in Nominalphrasen\\
  \Zeile
  \Zeile
  \centering 
  \onslide<+->
  \begin{tikzpicture}[node distance=1.5cm, auto]
    \node (context) {Wir möchten};
    \node [right=of context] (diesen) {\alert<5->{diesen}};
    \node[right=of diesen] (schönen) {\alert<4->{schönen}};
    \node[right=of schönen] (Sportwagen) {\alert<4->{Sportwagen}};
    \onslide<4->{\path[<->, trueblue, draw, bend left=30] (schönen) edge node {\scriptsize Akk Mask Sg} (Sportwagen);}
    \onslide<5->{\path[<->, trueblue, draw, bend left=60] (diesen) edge node[above] {\scriptsize Akk Mask Sg} (Sportwagen);}
    \onslide<6->{\path[<->, trueblue, draw, bend left=30] (diesen) edge node[above] {\scriptsize Akk Mask Sg} (schönen);}
  \end{tikzpicture}
\end{frame}

\begin{frame}
  {Kongruenz | Subjekt und finites Verb}
  \onslide<+->
  \onslide<+->
  \alert{Kongruenz} | Merkmalübereinstimmung zwischen Subjekt und finitem Verb\\
  \Zeile
  \Zeile
  \centering 
  \onslide<+->
  \begin{tikzpicture}[node distance=1cm, auto]
   \node                      (context)    {Ich glaube, dass};
   \node[right=of context]    (ihr)        {\alert<4->{ihr}};
   \node[right=of ihr]        (den)        {den};
   \node[right=of den]        (Wagen)      {Wagen};
   \node[right=of Wagen]      (anschieben) {anschieben};
   \node[right=of anschieben] (müsst)      {\alert<4->{müsst}};
   \onslide<4->{\path[<->, trueblue, draw, bend left=30]  (ihr) edge node {\small 2.~Per Pl} (müsst);}
  \end{tikzpicture}
\end{frame}

\begin{frame}
  {Rektion | Präpositionen}
  \onslide<+->
  \onslide<+->
  \gruen{Rektion} | Präpositionen bestimmen den Kasus von ganzen \alert{Nominalphrasen}\\
  \Zeile
  \Zeile
  \centering 
  \onslide<+->
  \begin{tikzpicture}[node distance=1cm, auto]
   \node                      (context)    {Wir fahren};
   \node[right=of context]    (mit)        {\gruen<4->{mit}};
   \node[right=of mit]        (dem)        {\alert<6->{dem}};
   \node[right=of dem]        (neuen)      {\alert<5->{neuen}};
   \node[right=of neuen]      (Wagen)      {\alert<4->{Wagen}};
   \node[right=of Wagen]      (rest)       {nach hause};
   \onslide<4->{\path[->, gruen, draw, bend right=30]  (mit) edge node[below] {Dat} (Wagen);}
   \onslide<5->{\path[<->, trueblue, draw, bend left=30]  (neuen) edge node {\footnotesize Dat Mask Sg} (Wagen);}
   \onslide<6->{\path[<->, trueblue, draw, bend left=30]  (dem) edge node {\footnotesize Dat Mask Sg} (neuen);}
  \end{tikzpicture}
\end{frame}

\begin{frame}
  {Rektion | Verben}
  \onslide<+->
  \onslide<+->
  \gruen{Rektion} | Verben bestimmen den Kasus von ganzen \alert{Nominalphrasen}\\
  \Zeile
  \Zeile
  \centering 
  \onslide<+->
  \begin{tikzpicture}[node distance=1cm, auto]
    \node                      (Nom)        {\alert<4->{Ich}};
    \node[right=of Nom]        (V)          {\gruen<4->{gab}};
    \node[right=of V]          (Dat)        {\alert<5->{dem netten Kollegen}};
    \node[right=of Dat]        (Akk)        {\alert<6->{den Stift}};
    \node[right=of Akk]        (rest)       {zurück};
    \onslide<4->{\path[->, gruen, draw, bend right=-30]  (V) edge node[below] {Nom} (Nom);}
    \onslide<5->{\path[->, gruen, draw, bend right=30]  (V) edge node[below] {Dat} (Dat);}
    \onslide<6->{\path[->, gruen, draw, bend right=60]  (V) edge node[below] {Akk} (Akk);}
  \end{tikzpicture}
\end{frame}


\section{Valenz}

\begin{frame}
  {Traditionelle Verbtypen}
  \pause
  \begin{itemize}[<+->]
    \item traditionelle Termini für Verbtypen (s.\ Kapitel 14 für Neuordnung)
      \Halbzeile
      \begin{itemize}[<+->]
        \item \alert{intransitiv}: regiert nur einen Nominativ (\textit{leben}, \textit{schlafen})
          \Viertelzeile
        \item \alert{transitiv}: regiert einen Nominativ und einen Akkusativ (\textit{sehen}, \textit{lesen})
          \Viertelzeile
        \item \alert{ditransitiv}: regiert zusätzlich einen Dativ (\textit{geben}, \textit{schicken})
          \Viertelzeile
        \item \alert{präpositional transitiv}: regiert Nom und PP (\textit{leiden +unter})
          \Viertelzeile
        \item \alert{präpositional ditransitiv}: regiert Nom, Akk, PP (\textit{schreiben +an})
          \Viertelzeile
        \item \ldots
          \Zeile
      \end{itemize}
    \item nur Abkürzungen für einige (von sehr viel mehr) \alert{Valenztypen}
  \end{itemize}
\end{frame}

\begin{frame}
  {Ergänzungen und Angaben}
  \pause
  Wo wollen wir denn hin?\\
  \pause\Halbzeile
  \begin{exe}
    \ex\label{ex:valenz034}
    \begin{xlist}
      \ex{Gabriele malt \alert{[ein Bild]}.}
      \pause
      \ex{Gabriele malt \gruen{[gerne]}.}
      \pause
      \ex{Gabriele malt \gruen{[den ganzen Tag]}.}
      \pause
      \ex{Gabriele malt \gruen{[ihrem Mann]} \rot{[zu figürlich]}.}
    \end{xlist}
  \end{exe}
  \pause\Halbzeile
  \begin{itemize}[<+->]
    \item \alert{[ein Bild]} mit besonderer Relation zum Verb
    \item "`Weglassbarkeit"' (Optionalität) nicht entscheidend
  \end{itemize}
\end{frame}

\begin{frame}
  {Lizenzierung}
  \pause
  \begin{exe}
    \ex 
    \begin{xlist}
      \ex[ ]{Gabriele isst \gruen{[den ganzen Tag]} Walnüsse.}
    \pause
      \ex[ ]{Gabriele läuft \gruen{[den ganzen Tag]}.}
      \pause
      \ex[ ]{Gabriele backt ihrer Schwester \gruen{[den ganzen Tag]} Stollen.}
      \pause
      \ex[ ]{Gabriele litt \gruen{[den ganzen Tag]} unter Sonnenbrand.}
    \end{xlist}
    \pause\Halbzeile
    \ex 
    \begin{xlist}
      \ex[*]{Gabriele isst \alert{[ein Bild]} Walnüsse.}
      \pause
      \ex[*]{Gabriele läuft \alert{[ein Bild]}.}
      \pause
      \ex[*]{Gabriele backt ihrer Schwester \alert{[ein Bild]} Stollen.}
      \pause
      \ex[*]{Gabriele litt \alert{[ein Bild]} unter Sonnenbrand. }
      \pause
    \end{xlist}
  \end{exe}
  \Halbzeile
  \begin{itemize}[<+->]
    \item \gruen{Angaben} sind verb-unspezifisch lizenziert
    \item \alert{Ergänzungen} sind verb(klassen)spezifisch \alert{genau einmal} lizenziert
    \item \rot{Valenz = Liste der Ergänzungen eines lexikalischen Worts}
  \end{itemize}
\end{frame}


\begin{frame}
  {Iterierbarkeit | Angaben sind beliebig stapelbar}
  \onslide<+->
  \onslide<+->
  \begin{exe}
    \ex[ ]{Wir müssen den Wagen\\
      \gruen{[jetzt]}\\
      \gruen{[mit aller Kraft]}\\
      \gruen{[vorsichtig]} anschieben.}
    \onslide<+->
    \ex[ ]{Wir essen \gruen{[schnell]}\\
    \gruen{[mit Appetit]}\\
    \gruen{[an einem Tisch]}\\
    \gruen{[mit der Gabel]}\\
    \alert{[einen Salat]}.}
    \onslide<+->
    \ex[*]{Wir essen \gruen{[schnell]}\\
    \rot{[ein Tofugericht]}\\
    \gruen{[mit Appetit]}\\
    \gruen{[an einem Tisch]}\\
    \gruen{[mit der Gabel]}\\
    \alert{[einen Salat]}.}
  \end{exe}
\end{frame}

\begin{frame}
  {Ergänzungen | Schnittstelle von Syntax und Semantik}
  \onslide<+->
  \onslide<+->
  Verbsemantik | Welche \alert{Rolle} spielen die von den Satzgliedern bezeichneten Dinge in der vom Verb beschriebenen Situation?\\
  \Zeile
  \onslide<+->
  Semantik von \alert{Ergänzungen} | \alert{abhängig} vom Verb\\
  \onslide<+->
  \Viertelzeile
  Semantik von \gruen{Angaben} | \gruen{unabhängig} vom Verb\\
  \Halbzeile
  \onslide<+->
  \begin{exe}
    \ex\label{ex:valenz071}
    \begin{xlist}
      \ex{\label{ex:valenz072}Ich lösche \alert{[den Ordner]} \gruen{[während der Hausdurchsuchung]}.}
      \onslide<+->
      \ex{\label{ex:valenz073}Ich mähe \alert{[den Rasen]} \gruen{[während der Ferien]}.}
      \onslide<+->
      \ex{\label{ex:valenz074}Ich fürchte \alert{[den Sturm]} \gruen{[während des Sommers]}.}
    \end{xlist}
  \end{exe}
\end{frame}


\begin{frame}
  {Valenz | Zusammenfassung}
  \onslide<+->
  \onslide<+->
  \begin{block}{Angaben}
    \alert{Angaben} sind grammatisch immer lizenziert und bringen\\
    ihre eigene semantische Rolle mit.\\
    \Halbzeile
    \grau{Sie können aber semantisch\slash pragmatisch inkompatibel sein.}
  \end{block}
  \Zeile
  \onslide<+->
  \begin{block}{Ergänzungen}
    \gruen{Ergänzungen} werden spezifisch vom Verb lizenziert und in ihrer semantischen Rolle\\
    vom Verb festgelegt. Jede dieser Rollen kann nur einmal vergeben werden.
  \end{block}
\end{frame}




\ifdefined\TITLE
  \section{Zur nächsten Woche | Überblick}

  \begin{frame}
    {Deutsche Syntax | Plan}
    \rot{Alle} angegebenen Kapitel\slash Abschnitte aus \rot{\citet{Schaefer2018b}} sind \rot{Klausurstoff}!\\
    \Halbzeile
    \begin{enumerate}
      \item Grammatik und Grammatik im Lehramt \rot{(Kapitel 1 und 3)}
      \item Grundbegriffe \rot{(Kapitel 2)}
      \item \alert{Wortklassen} \rot{(Kapitel 6)}
      \item Konstituenten und Satzglieder \rot{(Kapitel 11 und Abschnitt 12.1)}
      \item Nominalphrasen \rot{(Abschnitt 12.3)}
      \item Andere Phrasen \rot{(Abschnitte 12.2 und 12.4--12.7)}
      \item Verbphrasen und Verbkomplex \rot{(Abschnitte 12.8)}
      \item Sätze \rot{(Abschnitte 12.9 und 13.1--13.3)} 
      \item Nebensätze \rot{(Abschnitt 13.4)}
      \item Subjekte und Prädikate \rot{(Abschnitte 14.1--14.3)}
      \item Passive und Objekte \rot{(14.4 und 14.5)}
      \item Syntax infiniter Verbformen \rot{(Abschnitte 14.7--14.9)}
    \end{enumerate}
    \Halbzeile
    \centering 
    \url{https://langsci-press.org/catalog/book/224}
  \end{frame}
\fi
