
\section{Überblick}


\begin{frame}
  {Infinitivsyntax}
  \begin{itemize}[<+->]
    \item morphologische vs.\ analytische Tempora
    \item Ersatzinfinitiv und Oberfeldumstellung
      \Halbzeile
    \item kohärente und inkohärente Infinitive
    \item Modalverben und Halbmodale
    \item Kontrollverben
  \end{itemize}
\end{frame}

\section{Analytische Tempora}

\begin{frame}
  {Weitere Arten von Verben}
  \onslide<+->
  \onslide<+->
  Hilfs- und Modalverben mit besonderer Syntax und besonderer Formenbildung
  \onslide<+->
  \Halbzeile
  \begin{exe}
    \ex\label{ex:unterklassen072}
    \begin{xlist}
      \ex{\label{ex:unterklassen073} Frida \alert<9->{isst} den Marmorkuchen.}
      \onslide<+->
      \ex{\label{ex:unterklassen074} Frida \orongsch<10->{hat} den Marmorkuchen \alert<9->{gegessen}.}
      \onslide<+->
      \ex{\label{ex:unterklassen075} Der Marmorkuchen \orongsch<10->{wird} \alert<9->{gegessen}.}
      \onslide<+->
      \ex{\label{ex:unterklassen076} Frida \rot<11->{soll} den Marmorkuchen \alert<9->{essen}.}
      \onslide<+->
      \ex{\label{ex:unterklassen077} Dies hier \gruen<12->{ist} der leckere Marmorkuchen.}
      \onslide<+->
      \ex{\label{ex:unterklassen078} Der Marmorkuchen \gruen<12->{wird} lecker.}
    \end{xlist}
  \end{exe}
  \onslide<+->
  \Halbzeile
  \centering 
  \onslide<9->{\alert{Vollverben\slash lexikalische Verben}}\onslide<10->{, \orongsch{Hilfsverben}}\onslide<11->{, \rot{Modalverben}}\onslide<12->{, \gruen{Kopulaverben}}
\end{frame}

\begin{frame}
  {Welche Tempora hat das Deutsche?}
  \onslide<+->
  \onslide<+->
  Die Schulgrammatik lehrt \alert{sechs Tempusformen}, wir nur \rot{zwei}.\\
  \onslide<+->
  \Zeile
  \begin{center}
    \begin{tabular}[h]{lll}
      \textbf{Präsens}         & \textit{es \alert{geht}}                                     & \onslide<4->{\alert{synthetisch }} \\
      \textbf{Präteritum}      & \textit{es \alert{ging}}                                     & \onslide<4->{\alert{synthetisch }} \\
      && \\
      \textbf{Futur}         & \textit{es \orongsch{wird} \alert{gehen}}                    & \onslide<5->{\orongsch{analytisch }} \\
      && \\
      \textbf{Perfekt}         & \textit{es \orongsch{ist} \alert{gegangen}}                  & \onslide<5->{\orongsch{analytisch }} \\
      \textbf{Plusquamperfekt} & \textit{es \orongsch{war} \alert{gegangen}}                  & \onslide<5->{\orongsch{analytisch }} \\
      \textbf{Futurperfekt}         & \textit{es \orongsch{wird} \alert{gegangen} \orongsch{sein}} & \onslide<5->{\orongsch{analytisch }} \\
    \end{tabular}
  \end{center}
  \Zeile
  \begin{itemize}[<+->]
    \item Nur zwei werden als Form (\alert{synthetisch}) gebildet.
    \item Der Rest wird mit \orongsch{Hilfsverben} und \alert{infiniten Verbformen} (\orongsch{analytisch}) gebildet.
  \end{itemize}
\end{frame}

\begin{frame}
  {Präsens, Präteritum, Futur}
  \onslide<+->
  \begin{itemize}[<+->]
    \item Präsens
      \begin{itemize}[<+->]
        \item kein spezifischer Zeitbezug
        \item synthetische finite Form
      \end{itemize}
      \Viertelzeile
    \item Präteritum
      \begin{itemize}[<+->]
        \item Vergangenheitsbezug
        \item synthetische finite Form
      \end{itemize}
     \Viertelzeile 
    \item Futur
      \begin{itemize}[<+->]
        \item Zukunftsbezug oder Absichtserklärung
        \item analytische Form mit \rot{stets finitem} Hilfsverb
      \end{itemize}
  \end{itemize}
  \onslide<+->
  \Halbzeile
  \hspace{3em}\scalebox{0.8}{\begin{minipage}{\textwidth}
    \begin{exe}
      \onslide<11->{\ex[ ]{\ldots\ dass ich \alert{gehen werde}.}}
      \onslide<12->{\ex[*]{\ldots\ dass ich \rot{gehen werden} möchte.}}
      \onslide<13->{\ex[*]{\ldots\ dass ich \rot{gehen geworden} habe\slash bin.}}
      \onslide<14->{\ex[*]{\ldots\ dass ich \rot{gehen zu werden} habe.}}
    \end{exe}
  \end{minipage}}
\end{frame}

\begin{frame}
  {Perfekt}
  \onslide<+->
  \onslide<+->
  \rot{Das Perfekt ist nicht intrinsisch finit!}\\
  \onslide<+->
  \Viertelzeile
  Es kann daher im Infinitiv und in den drei finiten Tempora stehen.\\
  \Zeile
  \begin{itemize}[<+->]
    \item Hilfsverb \orongsch{sein} oder \orongsch{haben} + \alert{Partizip} des anderen Verbs
      \Halbzeile
    \item Infinitiv des Perfekts | \alert{gegangen} (Partizip) \orongsch{sein} (Inf des HVs)
    \item Präsens des Perfekts | \alert{gegangen} (Partizip) \orongsch{bin\slash bist\slash ist\slash\ldots} (Präs des HVs)
    \item Präteritum des Perfekts | \alert{gegangen} (Partizip) \orongsch{war\slash warst\slash\ldots} (Prät des HVs)
    \item Futur des Perfekts | \alert{gegangen} (Partizip) \orongsch{sein werde\slash wirst\slash wird\slash\ldots} (Futur des HVs)
  \end{itemize}
\end{frame}

\begin{frame}
  {Unterschiede zwischen Präteritum und Präsensperfekt}
  Stilistische Unterschiede\\
  \Halbzeile
  \begin{exe}
  \ex\label{ex:analytischetempora226}
  \begin{xlist}
    \ex{\label{ex:analytischetempora227} Das Pferd \alert{lief} im Kreis.}
    \ex{\label{ex:analytischetempora228} Das Pferd \orongsch{ist} im Kreis \alert{gelaufen}.}
  \end{xlist}
  \end{exe}
  \Zeile
  Semantische Unterschiede\\
  \Halbzeile
  \begin{exe}
  \ex\label{ex:analytischetempora229}
  \begin{xlist}
    \ex{\label{ex:analytischetempora230} Im Jahr 1993 \orongsch{hat} der Kommerz den Techno \alert{erobert}.}
    \ex{\label{ex:analytischetempora231} Im Jahr 1993 \alert{eroberte} der Kommerz den Techno.}
  \end{xlist}
  \onslide<+->
  \centering 
  Nicht alle Sprecher können die Lesarten differenzieren.
  \end{exe}
\end{frame}

\begin{frame}
  {Zusammenfassung | Finite Tempora und Perfekt}
  \onslide<+->
  \onslide<+->
  Klare Beziehungen zwischen den finiten Tempora und dem Perfekt\\
  \Zeile
  \begin{itemize}[<+->]
    \item Finite Tempora
      \begin{itemize}[<+->]
        \item Präsens | finite synthetische Form
        \item Präteritum | finite synthetische Form
        \item Futur (= Futur 1) | analytisch mit stets finitem Hilfsverb
      \end{itemize}
     \Zeile 
    \item \alert{Perfekta mit finiten Tempusformen des Hilfsverbs}
      \begin{itemize}[<+->]
        \item Präsensperfekt (= Perfekt) | Präsensform des Perfekts
        \item Präteritumsperfekt (= Plusquamperfekt) | Präteritalform des Perfekts
        \item Futurperfekt (= Futur 2) | Futur des Perfekts
      \end{itemize}
  \end{itemize}
\end{frame}


\begin{frame}
  {Analysen als Verbkomplex}
  \onslide<+->
  \onslide<+->
  Hilfsverben\slash Modalverben | \alert{Rektion des Status des anderen Verbs}\\
  \Halbzeile
  \centering
  \scalebox{0.85}{\begin{forest}
    [\textbf{V}, calign=last
      [\textbf{V}, calign=last
        [\textbf{V}, calign=last
          [\textbf{V}, tier=preterminal
            [\textit{behuft}]
          ]
          [\textbf{V}, tier=preterminal
            [\textit{gehabt}]
            {\draw [->, bend left=45] (.south) to node [below, midway] {\footnotesize\textsc{Status} (3)} (!uu11.south);}
          ]
        ]
        [\textbf{V}, tier=preterminal
          [\textit{haben}]
          {\draw [->, bend left=45] (.south) to node [below, midway] {\footnotesize\textsc{Status} (3)} (!uu121.south);}
        ]
      ]
      [\textbf{V}, tier=preterminal
        [\textit{will}]
        {\draw [->, bend left=45] (.south) to node [below, midway] {\footnotesize\textsc{Status} (1)} (!uu121.south);}
      ]
    ]
  \end{forest}}
\end{frame}




\begin{frame}
  {Nichtkanonische Infinitivrektion}
  \onslide<+->
  \onslide<+->
  Die sogenannte \alert{Oberfeldumstellung mit Ersatzinfinitiv}\\
  \Halbzeile
  \onslide<+->
  \begin{exe}
    \ex{\label{ex:ersatzinfinitivundoberfeldumstellung238} dass der Junge [\rot{hat} [[schwimmen] \rot{wollen}]]}
  \end{exe}
  \Zeile
  \onslide<+->
  \centering
  \scalebox{0.85}{\begin{forest}
    [\textbf{V}, calign=first
      [\textbf{V}, tier=preterminal
        [\textit{hat}\\1]
      ]
      [\textbf{V}, calign=last
        [\textbf{V}, tier=preterminal
          [\textit{schwimmen}\\3]
        ]
        [\textbf{V}, tier=preterminal
          [\textit{wollen}\\2]
          {\draw [->, bend left=20] (.south) to node [below, near end] {\footnotesize\textsc{Status} (1)} (!uu11.south);}
          {\draw [<-, bend left=60] (.south) to node [below, midway] {\footnotesize\textsc{Status} (1)} (!uuu11.south);}
        ]
      ]
    ]
  \end{forest}}
\end{frame}


\section{Infinitivsyntax}


\begin{frame}
  {Syntaktische Katgeorie von Infinitivphrasen}
  \onslide<+->
  \onslide<+->
  \alert{Infinitivphrasen mit Ergänzungen und Angaben} (\ref{ex:infvp}) vs.\ \orongsch{reine Infinitive} (\ref{ex:infv})\\
  \onslide<+->
  \Viertelzeile
  \begin{exe}
    \ex{\ldots\ dass Vanessa \alert{[das Pferd zu reiten]} scheint\label{ex:infvp}}
    \onslide<+->
    \ex{\ldots\ dass Vanessa \orongsch{[zu reiten]} scheint\label{ex:infv}}
  \end{exe}
  \onslide<+->
  \Halbzeile
  Da Infinitive kein Subjekt regieren, sind es VPs ohne Subjekt\\
  \Viertelzeile
  \centering 
  \onslide<+->
  \begin{forest}
    [VP, calign=last
      [NP
        [das Pferd, narroof]
      ]
      [V
        [\it zu reiten]
      ]
    ]
  \end{forest}
\end{frame}


\begin{frame}
  {Kommas bei \textit{Infinitvkonstruktionen}}
  \onslide<+->
  \onslide<+->
  Komma oder nich?
  \onslide<+->
  \begin{exe}
    \ex[*]{Nadezhda \rot{scheint}, die Kontrolle über die Hantel zu verlieren.}
    \ex[*]{Nadezhda \rot{will}, die Weltmeisterschaft gewinnen.}
    \ex[ ]{Nadezhda \alert{beschließt}, keine Steroide mehr einzunehmen.}
    \ex[?]{Nadezhda \alert{beschließt}\orongsch{,} zu trainieren.}
  \end{exe}
  \Zeile
  \begin{itemize}[<+->]
    \item \alert{Infinitivsyntax} ist der Schlüssel
    \item Komma nur bei \alert{inkohärenten Infinitiven}
  \end{itemize}
\end{frame}

\begin{frame}
  {(In)kohärente Infinitive}
  Kohärente und inkohärente Infinitivkonstruktionen\\
  \onslide<+->
  \Zeile
  \centering
  \scalebox{0.7}{\begin{minipage}{0.4\textwidth}
    \vspace{1.15cm}
    \begin{forest}
      [VP\Sub{1+2}, calign=last
        [NP, tier=preterminal
          [\textit{Vanessa}, narroof]
        ]
        [NP, tier=preterminal
          [\textit{die Pferde}, narroof]
        ]
        [\textbf{V\Sub{2+1}}, calign=last
          [\textbf{V\Sub{2}}, tier=preterminal
            [\textit{behufen}]
          ]
          [\textbf{V\Sub{1}}, tier=preterminal
            [\textit{will}]
          ]
        ]
      ]
    \end{forest}
  \end{minipage}}\hspace{0.1\textwidth}\scalebox{0.7}{\begin{minipage}{0.4\textwidth}
    \begin{forest}
      l sep+=3em, s sep+=2em
      [VP\Sub{1}, calign=last
        [NP, tier=preterminal
          [\textit{Vanessa}, narroof]
        ]
        [VP\Sub{2}, calign=last
          [NP, tier=preterminal
            [\textit{die Pferde}, narroof]
          ]
          [\textbf{V\Sub{2}}, tier=preterminal
            [\textit{zu behufen}]
          ]
        ]
        [\textbf{V\Sub{1}}, tier=preterminal
          [\textit{wünscht}]
        ]
      ]
    \end{forest}
  \end{minipage}}
\end{frame}


\begin{frame}
  {Test | Herausstellbarkeit}
  \onslide<+->
  \onslide<+->
  In der \rot{kohärenten} Konstruktion bildet der Infinitiv mit seinen Ergänzungen und Angaben keine Konstituente, also kann diese auf nicht nach rechts herausgestellt werden.\\
  \Zeile
  \onslide<+->
  \begin{exe}
    \ex[*]{Oma glaubt, dass Vanessa \rot{\Ti}\ will, \rot{[die Pferde behufen]\ORi}.}
  \end{exe}
  \onslide<+->
  \Zeile
  In der \gruen{inkohärenten} Konstruktion bildet der Infinitiv eine solche Konstituente.\\
  \Zeile
  \onslide<+->  
  \begin{exe}
    \ex[ ]{Oma glaubt, dass Vanessa \gruen{\Ti}\ wünscht, \gruen{[die Pferde zu behufen]\ORi}.}
  \end{exe}
\end{frame}


\begin{frame}
  {Halbmodale}
  \onslide<+->
  \onslide<+->
  Scheinbar gleich strukturiert | \gruen{wollen}, \orongsch{scheinen}, \alert{beschließen}\\
  \Halbzeile
  \onslide<+->
  \begin{exe}
  \ex
  \begin{xlist}
    \ex{dass der Hufschmied \gruen{das Pferd behufen will}.}
    \ex{dass der Hufschmied \orongsch{das Pferd zu behufen scheint}.}
    \ex{dass der Hufschmied \alert{das Pferd zu behufen beschließt}.}
  \end{xlist}
  \end{exe}
  \onslide<+->
  \Zeile
  Aber Abweichung bei der Extrahierbarkeit\\
  \Halbzeile
  \onslide<+->
  \begin{exe}
  \ex
  \begin{xlist}
    \ex[*]{dass der Hufschmied \gruen{\Ti}\ will, \gruen{[das Pferd behufen]\ORi}.}
    \ex[*]{dass der Hufschmied \orongsch{\Ti}\ scheint, \orongsch{[das Pferd zu behufen]\ORi}.}
    \ex[ ]{dass der Hufschmied \alert{\Ti}\ beschließt, \alert{[das Pferd zu behufen]\ORi}.}
  \end{xlist}
\end{exe}
\end{frame}

\begin{frame}
  {Halbmodale | \textit{scheinen} ohne Subjektrolle}
  \onslide<+->
  \onslide<+->
  Subjekt von \textit{scheinen} nicht erfragbar\\
  \onslide<+->
  \Halbzeile
  \begin{exe}
    \ex
    \begin{xlist}
      \ex[ ]{Frage: Wer \gruen{will} das Pferd behufen?\\
      Antwort: \gruen{Der Hufschmied will} das.}
      \onslide<+->
      \ex[*]{Frage: Wer \orongsch{scheint} das Pferd zu behufen?\\
      Antwort: \orongsch{Der Hufschmied scheint} das.}
      \onslide<+->
      \ex[ ]{Frage: Wer \alert{beschließt}, das Pferd zu behufen?\\
      Antwort: \alert{Der Hufschmied beschließt} das.}
    \end{xlist}
  \end{exe}
  \Zeile
  \onslide<+->
  Und \textit{scheinen} kann kein subjektloses Verb einbetten\\
  \Halbzeile
  \onslide<+->
  \begin{exe}
    \ex
    \begin{xlist}
      \ex[*]{Dem Hufschmied \gruen{will} grauen.}
      \onslide<+->
      \ex[ ]{Dem Hufschmied \orongsch{scheint} zu grauen}
      \onslide<+->
      \ex[*]{Dem Hufschmied \alert{beschließt} zu grauen.}
    \end{xlist}
  \end{exe}
\end{frame}


\begin{frame}
  {(In)kohärente Infinitve}
  \onslide<+->
  \onslide<+->
    \resizebox{1\textwidth}{!}{
    \begin{tabular}{lcllll}
      \lsptoprule
      & \multirow{2}{*}{\textbf{Status}} & \multirow{2}{*}{\textbf{Kohärenz}} & \textbf{eigenes} & \textbf{Subjekts-} \\
      & & & \textbf{Subjekt} & \textbf{Rolle} & \textbf{Beispiel}\\
      \midrule
      \textbf{Modalverben} & 1 & obl.\ kohärent & ja & Identität & \textit{wollen} \\
      \textbf{Halbmodalverben} & 2 & obl.\ kohärent & nein & nein & \textit{scheinen} \\
      \textbf{Kontrollverben} & 2 & \rot{opt.\ inkohärent} & ja & Kontrolle & \textit{beschließen} \\
      \lspbottomrule
    \end{tabular}
  }\\
  \Zeile
  \begin{itemize}[<+->]
    \item Nur \alert{inkohärente nachgestellte Infinitive} werden kommatiert!
    \item Sie gelten als satzwertig, aber die \rot{Inkohärenz ist leider nur optional}.
    \item Es kommen also nur \alert{Abhängige von Kontrollverben} infrage.
  \end{itemize}
  \onslide<+->
  \Viertelzeile
  \begin{exe}
    \ex[*]{Nadezhda \rot{scheint}, die Kontrolle über die Hantel zu verlieren.}
    \ex[*]{Nadezhda \rot{will}, die Weltmeisterschaft gewinnen.}
  \end{exe}
\end{frame}

\begin{frame}
  {(In)kohärente Infinitve}
  \onslide<+->
  \onslide<+->
  Was ist jetzt hiermit?\\
  \Halbzeile
  \onslide<+->
  \begin{exe}
    \ex[ ]{Nadezhda \alert{beschließt}, keine Steroide mehr einzunehmen.}
    \ex[?]{Nadezhda \alert{beschließt}\orongsch{,} zu trainieren.}
  \end{exe}
  \onslide<+->
  \Halbzeile
  \alert{Eindeutig inkohärent} | hinter die RSK versetzte Infinitive\\
  \Viertelzeile
  \onslide<+->
  \begin{exe}
    \ex \rot{\textbf{Inkohärent}}
    \begin{xlist}
      \ex[ ]{\ldots dass Nadezhda beschließt, keine Steroide mehr zu nehmen.}
      \ex[?]{\ldots dass Nadezhda keine Steroide mehr zu nehmen beschließt.}
    \end{xlist}
    \ex \alert{\textbf{Kohärent oder inkohärent}}
    \begin{xlist}
      \ex[ ]{\ldots dass Nadezhda zu trainieren beschließt.}
      \ex[ ]{\ldots dass Nadezhda beschließt zu trainieren.}
    \end{xlist}
  \end{exe}
\end{frame}


\begin{frame}
  {(In)kohärente Infinitve}
  Es liegt also an der syntaktischen Struktur.\\
  \Zeile
  \onslide<+->
  \begin{exe}
    \ex
    \begin{xlist}
      \ex[ ]{[Nadezhda]\Sub{2} \alert{[beschließt]\Sub{1}} [[t\Sub{2} \gruen{t\Sub{3}} \alert{[t\Sub{1}]\Sub{VK}}]\ \Sub{VP}\ \orongsch{,}\\
      {\hspace{1em}}\gruen{[keine Steroide mehr einzunehmen]\Sub{3}}]\Sub{VP}.}
        \Viertelzeile
      \ex[*]{[Nadezhda]\Sub{2} \rot{[beschließt]\Sub{1}}\\
      {\hspace{1em}}[t\Sub{2} [keine Steroide] [mehr] \rot{[einzunehmen t\Sub{1}]\Sub{VK}}\ ]\Sub{VP}.\label{ex:ohweia}}
    \end{xlist}
    \Halbzeile
    \ex
    \begin{xlist}
      \ex[ ]{[Nadezhda]\Sub{2} \alert{[beschließt]\Sub{1}}\ \orongsch{,} [[t\Sub{2} \gruen{t\Sub{3}} \alert{[t\Sub{1}]\Sub{VK}}\ ]\Sub{VP} \gruen{[zu trainieren]\Sub{3}}]\Sub{VP}.}
      \Viertelzeile
      \ex[ ]{[Nadezhda]\Sub{2} \tuerkis{[beschließt]\Sub{1}} [t\Sub{2} \tuerkis{[zu trainieren t\Sub{1}]\Sub{VK}}\ ]\Sub{VP}}
    \end{xlist}
  \end{exe}
  \Halbzeile
  \onslide<+->
  Füllen Sie den VK durch Hinzufügen von Hilfsverben auf,\\
  um das Phänomen noch deutlicher zu sehen.
\end{frame}

\begin{frame}
  {Bäume | Inkohärent}
  \onslide<+->
  \onslide<+->
  \rot{Inkohärent konstruiert}\\
  \Zeile
  \centering 
  \begin{forest}
    [S, calign=child, calign child=2
      [NP\Sub{2}, tier=pt
        [\it Nadezhda, narroof, tier=t]
      ]
      [V\Sub{1}, tier=pt
        [\it beschließt, tier=t]
      ]
      [VP, calign=child, calign child=1
        [VP, calign=child, calign child=3
          [\Tii, tier=t]
          [\rot{\Tiii}, tier=t]
          [\Ti, tier=t]
        ]
        [VP\Sub{3}, tier=pt, rottree
          [\it keine Steroide mehr einzunehmen, narroof, tier=t]
        ]
      ]
    ]
  \end{forest}
\end{frame}


\begin{frame}
  {Bäume | Inkohärent mit Hilfsverb}
  \onslide<+->
  \onslide<+->
  Dank des Verbs im Verbkomplex \rot{sieht man die Extraktion}\\
  \Zeile
  \centering 
  \begin{forest}
    [S, calign=child, calign child=2
      [NP\Sub{2}, tier=pt, tier=pt
        [\it Nadezhda, tier=t, narroof]
      ]
      [V\Sub{1}, tier=pt
        [\it hat, tier=t]
      ]
      [VP, , calign=child, calign child=1
        [VP, calign=last
          [\Tii, tier=t, forky]
          [\rot{\Tiii}, tier=t, forky]
          [V, calign=last
            [V, tier=pt
              [\it beschlossen, tier=t]
            ]
            [\Ti, tier=t]
          ]
        ]
        [VP\Sub{3}, tier=pt, rottree
          [\it keine Steroide mehr einzunehmen, narroof, tier=t]
        ]
      ]
    ]
  \end{forest}
\end{frame}


\begin{frame}
  {Bäume | Kohärent mit Hilfsverb}
  \onslide<+->
  \onslide<+->
  \rot{So gut wie ungrammatisch!}\\
  \Zeile
  \centering 
  \begin{forest}
    [S, calign=child, calign child=2
      [NP\Sub{2}, tier=pt
        [\it Nadezhda, tier=t, narroof]
      ]
      [V\Sub{1}, tier=pt
        [\it hat, tier=t]
      ]
      [VP, calign=last
        [\Tii, tier=t, forky]
        [NP, tier=pt
          [\it keine Steroide, narroof, tier=t]
        ]
        [AdvP, tier=pt
          [\it mehr, narroof, tier=t]
        ]
        [V, calign=last
          [V, calign=last
            [V
              [\it einzunehmen]
            ]
            [V, tier=pt
              [\it beschlossen, tier=t]
            ]
          ]
          [\Ti, tier=t]
        ]
      ]
    ]
  \end{forest}
\end{frame}

\begin{frame}
  {Bäume | Kohärent ohne Hilfsverb}
  \onslide<+->
  \onslide<+->
  Man kann daher davon ausgehen, dass diese Struktur auch nicht grammatisch ist.\\
  \onslide<+->
  \Viertelzeile
  Sie entspricht (\ref{ex:ohweia}), also der nicht kommatierten Version.\\
  \onslide<+->
  \Zeile
  \centering
  \begin{forest}
    [S, calign=child, calign child=2
      [NP\Sub{2}, tier=pt
        [\it Nadezhda, tier=t, narroof]
      ]
      [V\Sub{1}, tier=pt
        [\it beschließt, tier=t]
      ]
      [VP, calign=last
        [\Tii, tier=t, forky]
        [VP\Sub{3}, tier=pt, rottree
          [\it keine Steroide, narroof, tier=t]
        ]
        [AdvP, tier=pt
          [\it mehr, narroof]
        ]
        [V, calign=last
          [V, tier=pt
            [\it einzunehmen]
          ]
          [\Ti, tier=t]
        ]
      ]
    ]
  \end{forest}
\end{frame}

\section{Kontrollinfinitive}

\begin{frame}
  {\textit{zu}-Infinitive als Subjekte und Objekte}
  \onslide<+->
  \onslide<+->
  \alert{Controller} | Logisches Argument des Verbs, das die Bedeutung\\
  des fehlenden Subjekts des Infinitivs beisteuert\\
  \onslide<+->
  \Halbzeile
  \begin{exe}
  \ex\label{ex:infinitivkontrolle264}
  \begin{xlist}
    \ex{\label{ex:infinitivkontrolle265} [Das Geschirr \gruen{zu spülen}] \alert{nervt} Matthias. (Objektkontrolle)}\\
    \onslide<+->
    \Viertelzeile
    Matthias | der \alert{Genervte} (Objekt) und der \gruen{Spülende}\\
    \Halbzeile
    \onslide<+->
    \ex{\label{ex:infinitivkontrolle266} Doro wagt, [die Küche \gruen{zu betreten}]. (Subjektkontrolle)}\\
    \onslide<+->
    \Viertelzeile
    Doro | die \alert{Wagende} (Subjekt) und die \gruen{Betrende}
  \end{xlist}
\end{exe}
\Zeile
\onslide<+->
Auch mit Korrelat\\
\Halbzeile
\begin{exe}
  \ex\label{ex:infinitivkontrolle267}
  \begin{xlist}
    \ex{\label{ex:infinitivkontrolle268} Es nervt Matthias, [das Geschirr zu spülen].}
    \ex{\label{ex:infinitivkontrolle269} Doro wagt es, [die Küche zu betreten].}
  \end{xlist}
\end{exe}
\end{frame}

\begin{frame}
  {Kontrolle im Passiv}
  \onslide<+->
  \onslide<+->
  Kontrolle bleibt im Passiv erhalten | \alert{logische Valenz}, nicht Syntax\\
  \Halbzeile
  \onslide<+->
  \begin{exe}
  \ex\label{ex:infinitivkontrolle270}
  \begin{xlist}
    \ex{\label{ex:infinitivkontrolle271} Der Installateur hat gestern \alert{versucht}, die Küche \gruen{zu betreten}.}\\
    \onslide<+->
    \Viertelzeile
    der Installateur | der \alert{Versuchende} (Subjekt) und der \gruen{Betrende}\\
    \onslide<+->
    \Halbzeile
    \ex{\label{ex:infinitivkontrolle272} Gestern wurde (vom Installateur) versucht, die Küche zu betreten.}\\
    \Viertelzeile
    \onslide<+->
    der Installateur | der \alert{Versuchende} (Subjekt des Aktivs) und der \gruen{Betrende}\\
  \end{xlist}
\end{exe}
\end{frame}

\begin{frame}
  {Kontrolle}
  \begin{block}
    {Infinitivkontrolle}
Die \textit{Kontrollrelation} besteht zwischen einer nominalen Valenzstelle eines Verbs und einem von diesem Verb abhängigen (subjektlosen) \textit{zu}"=Infinitiv.
Die Bedeutung des nicht ausgedrückten Subjekts des abhängigen \textit{zu}"=Infinitivs wird dabei durch die mit der nominalen Valenzstelle verbundene Bedeutung beigesteuert.
  \end{block}
\end{frame}


\begin{frame}
  {Subjektinfinitive}
  \onslide<+->
  \onslide<+->
  Objektkontrolle präferiert\\
  \onslide<+->
  \Halbzeile
  \begin{exe}
  \ex\label{ex:infinitivkontrolle274}
  \begin{xlist}
    \ex{\label{ex:infinitivkontrolle275} Das Geschirr zu spülen, nervt \gruen{ihn}.\\
    Controller | \gruen{Akkusativobjekt}}
    \onslide<+->
    \Viertelzeile
    \ex{\label{ex:infinitivkontrolle276} Das Geschirr zu spülen, fällt \gruen{ihm} leicht.\\
    Controller | \gruen{Dativobjekt}}
    \onslide<+->
    \Viertelzeile
    \ex{\label{ex:infinitivkontrolle277} Das Geschirr zu spülen, beschert \gruen{ihm} einen zufriedenen Mitbewohner.\\
    Controller | \gruen{Dativobjekt}}
    \onslide<+->
    \Viertelzeile
    \ex{\label{ex:infinitivkontrolle278} Sich für Hilfe zu bedanken, freut \gruen{ihn} immer besonders.\\
    Controller | \gruen{Akkusativobjekt}}
  \end{xlist}
\end{exe}
\end{frame}

\begin{frame}
  {Objektinfinitive}
  \onslide<+->
  \onslide<+->
  Objektkontrolle präferiert, falls Objekte vorhanden\\
  \onslide<+->
  \Halbzeile
  \begin{exe}
  \ex\label{ex:infinitivkontrolle279}
  \begin{xlist}
    \ex{\label{ex:infinitivkontrolle280} \gruen{Er} wagt, die Küche zu betreten.\\
    Controller | \gruen{Subjekt}}
    \onslide<+->
    \Viertelzeile
    \ex{\label{ex:infinitivkontrolle281} Er bittet \gruen{seinen Mitbewohner}, das Geschirr zu spülen.\\
    Controller | \gruen{Akkusativobjekt}}
    \onslide<+->
    \Viertelzeile
    \ex{\label{ex:infinitivkontrolle282} Doro erlaubt \gruen{Matthias}, sich den Wagen zu leihen.\\
    Controller | \gruen{Dativobjekt}}
  \end{xlist}
\end{exe}
\end{frame}

\begin{frame}
  {Infinitivangaben}
  \onslide<+->
  \onslide<+->
  Immer Subjektkontrolle
  \begin{exe}
  \ex\label{ex:infinitivkontrolle283}
  \begin{xlist}
    \ex{\label{ex:infinitivkontrolle284} \gruen{Matthias} arbeitet, um Geld zu verdienen.\\
    Controller | \gruen{Subjekt}}
    \onslide<+->
    \Viertelzeile
    \ex{\label{ex:infinitivkontrolle285} \gruen{Matthias} begrüßt Doro, ohne aus der Rolle zu fallen.\\
    Controller | \gruen{Subjekt}}
    \onslide<+->
    \Viertelzeile
    \ex{\label{ex:infinitivkontrolle286} \gruen{Matthias} hilft Doro, anstatt untätig daneben zu stehen.\\
    Controller | \gruen{Subjekt}}
    \onslide<+->
    \Viertelzeile
    \ex{\label{ex:infinitivkontrolle287} \gruen{Matthias} bringt Doro den Wagen zurück, ohne den Lackschaden \\zu erwähnen.\\
    Controller | \gruen{Subjekt}}
  \end{xlist}
\end{exe}
\end{frame}


\section{Vor der Klausur | Überblick}

\begin{frame}
  {Deutsche Syntax | Plan}
  \rot{Alle} angegebenen Kapitel\slash Abschnitte aus \rot{\citet{Schaefer2018b}} sind \rot{Klausurstoff}!\\
  \Halbzeile
  \begin{enumerate}
    \item Grammatik und Grammatik im Lehramt \rot{(Kapitel 1 und 3)}
    \item Grundbegriffe \rot{(Kapitel 2)}
    \item Wortklassen \rot{(Kapitel 6)}
    \item Konstituenten und Satzglieder \rot{(Kapitel 11 und Abschnitt 12.1)}
    \item Nominalphrasen \rot{(Abschnitt 12.3)}
    \item Andere Phrasen \rot{(Abschnitte 12.2 und 12.4--12.7)}
    \item Verbphrasen und Verbkomplex \rot{(Abschnitte 12.8)}
    \item Sätze \rot{(Abschnitte 12.9 und 13.1--13.3)} 
    \item Nebensätze \rot{(Abschnitt 13.4)}
    \item Subjekte und Prädikate \rot{(Abschnitte 14.1--14.3)}
    \item Passive und Objekte \rot{(14.4 und 14.5)}
    \item Syntax infiniter Verbformen \rot{(Abschnitte 14.7--14.9)}
  \end{enumerate}
  \Halbzeile
  \centering 
  \url{https://langsci-press.org/catalog/book/224}
\end{frame}
