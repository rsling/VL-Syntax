\documentclass[12pt,a4paper,twoside]{article}

\usepackage[margin=2cm]{geometry}

\usepackage[ngerman]{babel}

\usepackage{setspace}
\usepackage{booktabs}
\usepackage{array,graphics}
\usepackage{color}
\usepackage{soul}
\usepackage[linecolor=gray,backgroundcolor=yellow!50,textsize=tiny]{todonotes}
\usepackage[linguistics]{forest}
\usepackage{multirow}
\usepackage{pifont}
\usepackage{wasysym}
\usepackage{langsci-gb4e}
\usepackage{soul}
\usepackage{enumitem}
\usepackage{marginnote}

\usepackage[maxbibnames=99,
  maxcitenames=2,
  uniquelist=false,
  backend=biber,
  doi=false,
  url=false,
  isbn=false,
  bibstyle=biblatex-sp-unified,
  citestyle=sp-authoryear-comp]{biblatex}

\definecolor{rot}{rgb}{0.7,0.2,0.0}
\newcommand{\rot}[1]{\textcolor{rot}{#1}}
\definecolor{blau}{rgb}{0.1,0.2,0.7}
\newcommand{\blau}[1]{\textcolor{blau}{#1}}
\definecolor{gruen}{rgb}{0.0,0.7,0.2}
\newcommand{\gruen}[1]{\textcolor{gruen}{#1}}
\definecolor{grau}{rgb}{0.6,0.6,0.6}
\newcommand{\grau}[1]{\textcolor{grau}{#1}}
\definecolor{orongsch}{RGB}{255,165,0}
\newcommand{\orongsch}[1]{\textcolor{orongsch}{#1}}
\definecolor{tuerkis}{RGB}{63,136,143}
\definecolor{braun}{RGB}{108,71,65}
\newcommand{\tuerkis}[1]{\textcolor{tuerkis}{#1}}
\newcommand{\braun}[1]{\textcolor{braun}{#1}}

\newcommand*\Rot{\rotatebox{75}}

\newcommand{\zB}{z.\,B.\ }
\newcommand{\ZB}{Z.\,B.\ }
\newcommand{\Sub}[1]{\ensuremath{_{\text{#1}}}}
\newcommand{\Up}[1]{\ensuremath{^{\text{#1}}}}
\newcommand{\UpSub}[2]{\ensuremath{^{\text{#1}}_{\text{#2}}}}
\newcommand{\Doppelzeile}{\vspace{2\baselineskip}}
\newcommand{\Zeile}{\vspace{\baselineskip}}
\newcommand{\Halbzeile}{\vspace{0.5\baselineskip}}
\newcommand{\Viertelzeile}{\vspace{0.25\baselineskip}}

\newcommand{\whyte}[1]{\textcolor{white}{#1}}

\newcommand{\Spur}[1]{t\Sub{#1}}
\newcommand{\Ti}{\Spur{1}}
\newcommand{\Tii}{\Spur{2}}
\newcommand{\Tiii}{\Spur{3}}
\newcommand{\Tiv}{\Spur{4}}
\newcommand*{\mybox}[1]{\framebox{#1}}
\newcommand\ol[1]{{\setul{-0.9em}{}\ul{#1}}}

\newenvironment{nohyphens}{%
  \par
  \hyphenpenalty=10000
  \exhyphenpenalty=10000
  \sloppy
}{\par}

\newcommand{\Lf}{
  \setlength{\itemsep}{1pt}
  \setlength{\parskip}{0pt}
  \setlength{\parsep}{0pt}
}

\forestset{
  Ephr/.style={draw, ellipse, thick, inner sep=2pt},
  Eobl/.style={draw, rounded corners, inner sep=5pt},
  Eopt/.style={draw, rounded corners, densely dashed, inner sep=5pt},
  Erec/.style={draw, rounded corners, double, inner sep=5pt},
  Eoptrec/.style={draw, rounded corners, densely dashed, double, inner sep=5pt},
  Ehd/.style={rounded corners, fill=gray, inner sep=5pt,
    delay={content=\whyte{##1}}
  },
  Emult/.style={for children={no edge}, for tree={l sep=0pt}},
  phrasenschema/.style={for tree={l sep=2em, s sep=2em}},
  sake/.style={tier=preterminal},
  ake/.style={
    tier=preterminal
    },
}

\forestset{
  decide/.style={draw, chamfered rectangle, inner sep=2pt},
  finall/.style={rounded corners, fill=gray, text=white},
  intrme/.style={draw, rounded corners},
  yes/.style={edge label={node[near end, above, sloped, font=\scriptsize]{Ja}}},
  no/.style={edge label={node[near end, above, sloped, font=\scriptsize]{Nein}}}
}

\usepackage{tikz}
\usetikzlibrary{arrows,positioning} 


\author{Prof.\ Dr.\ Roland Schäfer | Germanistische Linguistik FSU Jena}
\title{Syntax | 04 | Phrasen und Satzglieder}
\date{Version Sommer 2023 (\today)}


\usepackage{fontspec}
\defaultfontfeatures{Ligatures=TeX,Numbers=OldStyle, Scale=MatchLowercase}
\setmainfont{Linux Libertine O}
\setsansfont{Linux Biolinum O}

\setlength{\parindent}{0pt}

\usepackage[headings]{fancyhdr}
\fancyhead[E,O]{}
\fancyfoot[E,O]{}
\renewcommand{\headrulewidth}{0pt}
\pagestyle{fancy}
\setlength{\headsep}{50pt}
\setlength{\textheight}{\textheight-25pt}



\begin{document}

\maketitle

\section*{Voraussetzungen}

Sie sollten für diese Übungen folgende Themen aus dem Buch bzw.\ der Vorlesung wiederholen, falls sie Ihnen noch nicht klar sind:

\begin{enumerate}\Lf
  \item Was sind \textbf{Phrasen}?
  \item Welche Phrasen sind die sogenannten \textbf{Satzglieder}?
  \item Wie funkionieren die \textbf{Konstituententests}
    \begin{enumerate}\Lf
      \item \textbf{Pronominalisierungstest} bzw.\ \textbf{Ein-Wort-Ersetzungstext}
      \item \textbf{Bewegungstest} und \textbf{Vorfeldtest}
      \item \textbf{Koordinationstest}\\
        Achtung! Dieser Test produziert sehr viele falschpositive Ergebnisse!
    \end{enumerate}
  \item \textbf{Strukturelle Ambiguität}
\end{enumerate}

\section{Konstituenten und Tests}

(a)~Wenden Sie jeweils alle Tests auf die in eckigen Klammern gesetzten potentiellen Konstituenten an und entscheiden Sie, ob Sie sie als Phrase behandeln möchten.
Erwarten Sie wie gesagt nicht, dass das perfekt und immer eindeutig funktioniert.
Durch die Anwendung der Tests sollen Sie vor allem ihre Intuition für Satzstrukturen entwickeln.
Erst durch die wohldefinierten Phrasenschemata können Sie dann in ein paar Wochen sicher entscheiden, was wir als Phrase auffassen wollen.
(b)~Wenn eine Konstituente eine Phrase ist, ist sie dann auch ein Satzglied?
(c)~Überlegen Sie, welches Wort der Kopf der Phrase sein könnte.
Teilaufgabe (c) ist ausdrücklich eine Entlastungsaufgabe.
Sie müssen das zum jetzigen Zeitpunkt noch nicht perfekt können.

\begin{exe}
  \ex{} [Unter dem Tisch] steht der [Papierkorb].
  \ex{} Wir [waren alle] etwas neben [der Spur].
  \ex{} Johan weiß [nur] zu genau, dass die Frage, [die gestellt wurde], nicht [beantwortbar] ist.
  \ex{} Früher entschied [in der BRD] [die FDP] [meistens] [mit nur 5 \%] [die Kanzlerfrage]. 
  \ex{} Ich sehe in [letzter Zeit] [ein blaues] Auto [immer] wieder die Straße entlangfahren.
  \ex{} Ob Syntax wirklich [schwerer als] Morphologie ist, [wage ich] [zu bezweifeln].
  \ex{} Die Universität, [deren Kassen] leer sind, [streicht vakante Professuren].
  \ex{} Das ergibt dann [eben] in der gegebenen Situation keinen Sinn.
\end{exe}


\section{Strukturelle Ambiguität}

\subsection{Ambige Phrasengrenzen}

Strukturelle Ambiguität kommt in der Regel so zustande, dass sich in der einen Interpretation relativ zur anderen eine Phrasengrenze verschiebt.
Stellen Sie fest, inwiefern die folgenden Sätze strukturell ambig sind und stellen Sie fest, welche Konstituenten sich in den beiden Lesarten unterschiedlich zusammensetzen.

\begin{exe}
  \ex Sandy hat einen Mann ohne Schuhe fotografiert.
  \ex Auf der A9 waren wieder große LKWs und Autos unterwegs.
  \ex Letztes Jahr hat Rahel sich eine Olive im Martini bestellt.
  \ex Sie hat den Tathergang in Begleitung ihres Anwalts niedergeschrieben.
  \ex Wir haben das Problem mit dem Taschenrechner gelöst.
\end{exe}

\subsection{Strukturelle Ambiguität und Satzglieder}

Vier der Beispiele kann man durch eine Umstellung desambiguieren (= eindeutig machen)?
Welche sind das, und was ist die Umstellung?

\end{document}
