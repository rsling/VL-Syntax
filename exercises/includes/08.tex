\section*{Voraussetzungen}

Sie sollten für diese Übungen mit dem Buch und der Vorlesung folgende Themen wiederholen, wenn sie Ihnen nicht mehr oder noch nicht präsent sind:

\begin{enumerate}\Lf
  \item Hauptsatz und Nebensatz
  \item V1-Satz und V2-Satz
  \item Verbbewegung
  \item Vorfeldbesetzung als XP-Bewegung
\end{enumerate}

\Zeile

\section{V1-Sätze und V2-Sätze aus VPs bilden}\label{sec:bilden}

Im folgenden kurzen Text sind vier in KPs eingebettete VPs enthalten.
Finden Sie diese, klammern Sie sie ein, und bilden Sie dann jeweils einen V1-Satz (Entscheidungsfrage) und einen V2-Satz (unabhängiger Aussagesatz) aus der VP.
Lassen Sie nichts weg und fügen Sie nichts hinzu.
Um die Aufgabe interessanter zu machen, dürfen Sie in den V2-Sätzen alles vor das finite Verb bewegen (= ins Vorfeld stellen) \textbf{außer dem Subjekt}.

\begin{quote}
  Das Lehramt im Fach Deutsch ist besonders schwierig, weil \Solmark{[es neben fundiertem Wissen verschiedenste Kompetenzen erfordert]}.
  Lehrpersonen sollten von vornherein wissen, dass \Solmark{[sie sich ein überaus anspruchsvolles Fach ausgesucht haben]}.
  Obwohl \Solmark{[im Studium eigentlich keine Zeit dafür ist]}, sollten zum Beispiel weitreichende Fähigkeiten in grammatischer Analyse erworben werden.
  Dies ist unerlässlich, denn die Bewertung sprachlicher Leistungen und die Förderung bildungssprachlicher Kompetenzen sind ohne solche Fähigkeiten nicht möglich.
  Es wäre aus Sicht der Linguistik durchaus denkbar, andere Studieninhalte zu reduzieren, damit \Solmark{[Studierende nach dem Studium mit der nötigen Souveränität in den Deutschunterricht gehen können]}.
\end{quote}

\Halbzeile

\Sol{\textbf{Beispiele für Umstellungen}
\begin{enumerate}
  \item \begin{enumerate}
      \item \ldots\ es neben fundiertem Wissen verschiedenste Kompetenzen erfordert 
      \item \blau{Erfordert} es neben fundiertem Wissen verschiedenste Kompetenzen \blau{\_}?
      \item \orongsch{Neben fundiertem Wissen} \blau{erfordert} es \orongsch{\_} verschiedenste Kompetenzen \blau{\_}.
    \end{enumerate}
  \item \begin{enumerate}
      \item \ldots\ sie sich ein überaus anspruchsvolles Fach ausgesucht haben
      \item \blau{Haben} sie sich ein überaus anspruchsvolles Fach ausgesucht \blau{\_}?
      \item \orongsch{Ein überaus anspruchsvolles Fach} \blau{haben} sie sich \orongsch{\_} ausgesucht \blau{\_}.
    \end{enumerate}
  \item \begin{enumerate}
      \item \ldots\ im Studium eigentlich keine Zeit dafür ist
      \item \blau{Ist} im Studium eigentlich keine Zeit dafür \blau{\_}?
      \item \orongsch{Im Studium} \blau{ist} \orongsch{\_} eigentlich keine Zeit dafür \blau{\_}.
    \end{enumerate}
  \item \begin{enumerate}
      \item \ldots\ Studierende nach dem Studium mit der nötigen Souveränität in den Deutschunterricht gehen können
      \item \blau{Können} Studierende nach dem Studium mit der nötigen Souveränität in den Deutschunterricht gehen \blau{\_}?
      \item \orongsch{Nach dem Studium} \blau{können} Studierende \orongsch{\_} mit der nötigen Souveränität in den Deutschunterricht gehen \blau{\_}?
    \end{enumerate}
\end{enumerate}
}

\Zeile

\section{Konstituentenanalysen von VPs, V1-Sätzen und V2-Sätzen}\label{sec:analysieren}

Zeichnen Sie Konstituentenstrukturdiagramme (Bäume) für die VPs aus Aufgabe~\ref{sec:bilden} sowie für die zugehörigen V1- und V2-Sätze.
Kürzen Sie alle Phrasen außer der VP und dem Verbkomplex mit Dreiecken ab.
Vergessen Sie nicht für jede bewegte Konstituente die Spur in der VP!
Zeichnen Sie zur Verdeutlichung Bewegungspfeile von den Spuren zu den jeweiligen bewegten Konstituenten.

\Sol{\begin{enumerate}
  \item \begin{enumerate}
      \item\begin{forest}
          [VP, calign=last
            [NP
              [\it es, narroof]
            ]
            [PP
              [\it neben fund.\ Wissen, narroof]
            ]
            [NP
              [\it versch. Komp., narroof]
            ]
            [V
              [\it erfordert]
            ]
          ]
        \end{forest}
      \item\begin{forest}
          [FS, calign=first
            [V\Sub{1}, ake
              [\it erfordert]
            ]
            [VP, calign=last
              [NP, ake
                [\it es, narroof]
              ]
              [PP, ake
                [\it neben fund.\ Wissen, narroof]
              ]
              [NP, ake
                [\it versch. Komp., narroof]
              ]
              [\Ti, ake]
            ]
          ]
        \end{forest}
      \item\begin{forest}
          [S, calign=child, calign child=2
            [PP\Sub{2}, ake
                [\it neben fund.\ Wissen, narroof]
            ]
            [V\Sub{1}, ake
              [\it erfordert]
            ]
            [VP, calign=last
              [NP, ake
                [\it es, narroof]
              ]
              [\Tii, ake]
              [NP, ake
                [\it versch. Komp., narroof]
              ]
              [\Ti]
            ]
          ]
        \end{forest}
    \end{enumerate}
  \item \begin{enumerate}
      \item\begin{forest}
          [VP, calign=last
            [NP, ake
              [\it sie, narroof]
            ]
            [NP, ake
              [\it sich, narroof]
            ]
            [NP, ake
              [\it ein überaus anspr.\ Fach, narroof]
            ]
            [V, calign=last
              [V, ake
                [\it ausgesucht]
              ]
              [V, ake
                [\it haben]
              ]
            ]
          ]
        \end{forest}
      \item\begin{forest}
          [FS, calign=first
            [V\Sub{1}, ake
              [\it haben]
            ]
            [VP, calign=last
              [NP, ake
                [\it sie, narroof]
              ]
              [NP, ake
                [\it sich, narroof]
              ]
              [NP, ake
                [\it ein überaus anspr.\ Fach, narroof]
              ]
              [V, calign=last
                [V, ake
                  [\it ausgesucht]
                ]
                [\Ti]
              ]
            ]
          ]
        \end{forest}
      \item\begin{forest}
          [S, calign=child, calign child=2
            [NP\Sub{2}, ake
              [\it ein überaus anspr.\ Fach, narroof]
            ]
            [V\Sub{1}, ake
              [\it haben]
            ]
            [VP, calign=last
              [NP, ake
                [\it sie, narroof]
              ]
              [NP, ake
                [\it sich, narroof]
              ]
              [\Tii, ake]
              [V, calign=last
                [V, ake
                  [\it ausgesucht]
                ]
                [\Ti]
              ]
            ]
          ]
        \end{forest}
    \end{enumerate}
  \item \begin{enumerate}
      \item\begin{forest}
          [VP, calign=last
            [PP, ake
              [\it im Studium, narroof]
            ]
            [AdvP, ake
              [\it eigentlich, narroof]
            ]
            [NP, ake
              [\it keine Zeit, narroof]
            ]
            [AdvP, ake
              [\it dafür, narroof]
            ]
            [V
              [\it ist]
            ]
          ]
        \end{forest}
      \item\begin{forest}
          [FS , calign=first
            [V\Sub{1}, ake
              [\it ist]
            ]
            [VP, calign=last
              [PP, ake
                [\it im Studium, narroof]
              ]
              [AdvP, ake
                [\it eigentlich, narroof]
              ]
              [NP, ake
                [\it keine Zeit, narroof]
              ]
              [AdvP, ake
                [\it dafür, narroof]
              ]
              [\Ti]
            ]
          ]
        \end{forest}
      \item\begin{forest}
          [S, calign=child, calign child=2
            [PP\Sub{2}, ake
              [\it im Studium, narroof]
            ]
            [V\Sub{1}, ake
              [\it ist]
            ]
            [VP, calign=last
              [\Tii, ake, forky]
              [AdvP, ake
                [\it eigentlich, narroof]
              ]
              [NP, ake
                [\it keine Zeit, narroof]
              ]
              [AdvP, ake
                [\it dafür, narroof]
              ]
              [\Ti]
            ]
          ]
        \end{forest}
    \end{enumerate}
  \item \begin{enumerate}
      \item\begin{forest}
          [VP, calign=last
            [NP, ake
              [\it Studierende, narroof]
            ]
            [PP, ake
              [\it nach dem St., narroof]
            ]
            [PP, ake
              [\it mit der nöt.\ Souv., narroof]
            ]
            [PP, ake
              [\it in den Deutschunt., narroof]
            ]
            [V, calign=last
              [V, ake
                [\it gehen]
              ]
              [V, ake
                [\it können]
              ]
            ]
          ]
        \end{forest}
      \item\begin{forest}
          [FS , calign=first
            [V\Sub{1}, ake
              [\it können]
            ]
            [VP, calign=last
              [NP, ake
                [\it Studierende, narroof]
              ]
              [PP, ake
                [\it nach dem St., narroof]
              ]
              [PP, ake
                [\it mit der nöt.\ Souv., narroof]
              ]
              [PP, ake
                [\it in den Deutschunt., narroof]
              ]
              [V, calign=last
                [V, ake
                  [\it gehen]
                ]
                [\Ti]
              ]
            ]
          ]
        \end{forest}
      \item\begin{forest}
          [S, calign=child, calign child=2
            [PP\Sub{2}, ake
              [\it nach dem St., narroof]
            ]
            [V\Sub{1}, ake
              [\it können]
            ]
            [VP, calign=last
              [NP, ake
                [\it Studierende, narroof]
              ]
              [\Tii, ake]
              [PP, ake
                [\it mit der nöt.\ Souv., narroof]
              ]
              [PP, ake
                [\it in den Deutschunt., narroof]
              ]
              [V, calign=last
                [V, ake
                  [\it gehen]
                ]
                [\Ti]
              ]
            ]
          ]
        \end{forest}
    \end{enumerate}
\end{enumerate}}


\section{Fehlerhafte Phrasenstrukturdiagramme erkennen}\label{sec:fehlersuche}

Sie sehen im Folgenden jeweils zwei Fragmente aus Phrasenstrukturanalysen nebeneinander.
Eins der Fragment wird von unserer Grammatik beschrieben, das andere nicht.
Es spielt jeweils keine Rolle, welche Wörter bzw.\ weiteren Strukturen man einsetzt:
Die eine Struktur folgt den Regeln der Grammatik aus EGBD3, die andere folgt diesen Regeln nicht.
Kreuzen Sie die von der Grammatik im Buch \slash\ in der Vorlesung beschriebene und damit korrekte Struktur an.

\Zeile

(1)~\hspace{4em}~\Solalt{\XBox}{\Square}~\begin{forest}
  [S, calign=child, calign child=2
    [AdvP\Sub{2}]
    [V\Sub{1}]
    [VP]
  ]
\end{forest}~\hspace{4em}~\Solalt{\Square}{\Square}~\begin{forest}
  [S, calign=child, calign child=2
    [AdvP\Sub{2}]
    [V\Sub{1}]
    [PP]
  ]
\end{forest}

\Zeile

(2)~\hspace{4em}~\Solalt{\Square}{\Square}~\begin{forest}
  [PP, calign=first
    [P]
    [NP]
    [VP]
  ]
\end{forest}~\hspace{4em}~\Solalt{\XBox}{\Square}~\begin{forest}
  [PP, calign=child, calign child=2
    [NP]
    [P]
    [NP]
  ]
\end{forest}

\Zeile

(3)~\hspace{4em}~\Solalt{\Square}{\Square}~\begin{forest}
  [S, calign=child, calign child=2
    []
    []
    []
    []
  ]
\end{forest}~\hspace{4em}~\Solalt{\XBox}{\Square}~\begin{forest}
  [S, calign=child, calign child=2
    []
    []
    []
  ]
\end{forest}

\Zeile

(4)~\hspace{4em}~\Solalt{\Square}{\Square}~\begin{forest}
  [VP, calign=first
    [V]
    [PP]
  ]
\end{forest}~\hspace{4em}~\Solalt{\XBox}{\Square}~\begin{forest}
  [VP, calign=last
    [PP]
    [V]
  ]
\end{forest}

\Zeile

(5)~\hspace{4em}~\Solalt{\XBox}{\Square}~\begin{forest}
  [NP, calign=first
    [N]
    [NP]
    [PP]
    [RS]
  ]
\end{forest}~\hspace{4em}~\Solalt{\Square}{\Square}~\begin{forest}
  [NP, calign=child, calign child=3
    [Art]
    [NP]
    [N]
    [RS]
  ]
\end{forest}

\Halbzeile

\Sol{\textbf{Erläuterungen zu den falschen Strukturen:}
\begin{enumerate}\Lf
  \item Die rechte Tochter eines Satzknotens ist immer ein VP-Knoten, niemals eine PP.
  \item Es gibt in der PP keine Position ganz rechts für eine VP.
  \item Sätze im Deutschen sind immer dreigliedrig.
  \item In einer VP stehen alle anderen Konstituenten immer links vom Verbkomplex, niemals rechts davon.
  \item In einer NP kann links vom Kopf zwar eine NP (im Genitiv) stehen. Dies ist allerdings ausgeschlossen, wenn ein Artikel links vom Kopf steht.
\end{enumerate}
}

\Sol{\newpage\ \\}

