\section*{Voraussetzungen}

Sie sollten für diese Übungen mit dem Buch und der Vorlesung folgende Themen wiederholen, wenn sie Ihnen nicht mehr oder noch nicht präsent sind:

\begin{enumerate}\Lf
  \item Hauptsatz und Nebensatz
  \item V1-Satz und V2-Satz
  \item Verbbewegung
  \item Vorfeldbesetzung als XP-Bewegung
\end{enumerate}

\Zeile

\section{V1-Sätze und V2-Sätze aus VPs bilden}\label{sec:bilden}

Im folgenden kurzen Text sind vier in KPs eingebettete VPs enthalten.
Finden Sie diese und bilden Sie jeweils einen V1-Satz (Entscheidungsfrage) und einen V2-Satz (unabhängiger Aussagesatz) aus der VP.
Lassen Sie nichts weg und fügen Sie nichts hinzu.
In den V2-Sätzen dürfen Sie alles vor das finite Verb bewegen (= ins Vorfeld stellen) \textbf{außer dem Subjekt}.

\begin{quote}
  Das Lehramt im Fach Deutsch ist besonders schwierig, weil es neben fundiertem Wissen verschiedenste Kompetenzen erfordert.
  Lehrpersonen sollten von vornherein wissen, dass sie sich ein überaus anspruchsvolles Fach ausgesucht haben.
  Obwohl im Studium eigentlich keine Zeit dafür ist, sollten zum Beispiel weitreichende Fähigkeiten in grammatischer Analyse erworben werden.
  Dies ist unerlässlich, denn die Bewertung sprachlicher Leistungen und die Förderung bildungssprachlicher Kompetenzen sind ohne solche Fähigkeiten nicht möglich.
  Es wäre aus Sicht der Linguistik durchaus denkbar, andere Studieninhalte zu reduzieren, damit Studierende nach dem Studium mit der nötigen Souveränität in den Deutschunterricht gehen können.
\end{quote}

\Zeile

\section{Konstituentenanalysen von VPs, V1-Sätzen und V2-Sätzen}\label{sec:analysieren}

Zeichnen Sie Konstituentenstrukturdiagramme (Bäume) für die VPs aus Aufgabe~\ref{sec:bilden} sowie für die zugehörigen V1- und V2-Sätze.
Kürzen Sie alle Phrasen außer der VP und dem Verbkomplex mit Dreiecken ab.
Vergessen Sie nicht für jede bewegte Konstituente die Spur in der VP!
Zeichnen Sie zur Verdeutlichung Bewegungspfeile von den Spuren zu den jeweiligen bewegten Konstituenten.

\newpage

\section{Fehlerhafte Phrasenstrukturdiagramme erkennen}\label{sec:fehlersuche}

Sie sehen im Folgenden jeweils zwei Fragmente aus Phrasenstrukturanalysen nebeneinander.
Eins der Fragment wird von unserer Grammatik beschrieben, das andere nicht.
Es spielt jeweils keine Rolle, welche Wörter bzw.\ weiteren Strukturen man einsetzt:
Die eine Struktur folgt den Regeln der Grammatik aus EGBD3, die andere folgt diesen Regeln nicht.
Streichen Sie die von der Grammatik im Buch \slash\ in der Vorlesung ausgeschlossene (= nicht beschriebene) Struktur durch.

\Zeile

(1)~\hspace{4em}~\begin{forest}
  [S, calign=child, calign child=2
    [AdvP\Sub{2}]
    [V\Sub{1}]
    [VP]
  ]
\end{forest}~\hspace{4em}~\begin{forest}
  [S, calign=child, calign child=2
    [AdvP\Sub{2}]
    [V\Sub{1}]
    [PP]
  ]
\end{forest}

\Zeile

(2)~\hspace{4em}~\begin{forest}
  [PP, calign=first
    [P]
    [NP]
    [VP]
  ]
\end{forest}~\hspace{4em}~\begin{forest}
  [PP, calign=child, calign child=2
    [NP]
    [P]
    [NP]
  ]
\end{forest}

\Zeile

(3)~\hspace{4em}~\begin{forest}
  [S, calign=child, calign child=2
    []
    []
    []
    []
  ]
\end{forest}~\hspace{4em}~\begin{forest}
  [S, calign=child, calign child=2
    []
    []
    []
  ]
\end{forest}

\Zeile

(4)~\hspace{4em}~\begin{forest}
  [VP, calign=first
    [V]
    [PP]
  ]
\end{forest}~\hspace{4em}~\begin{forest}
  [VP, calign=last
    [PP]
    [V]
  ]
\end{forest}

\Zeile

(5)~\hspace{4em}~\begin{forest}
  [NP, calign=first
    [N]
    [NP]
    [PP]
    [RS]
  ]
\end{forest}~\hspace{4em}~\begin{forest}
  [NP, calign=child, calign child=3
    [Art]
    [NP]
    [N]
    [RS]
  ]
\end{forest}

\newpage
\ \\
\newpage
\ \\
\newpage
\ \\

