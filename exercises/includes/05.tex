\section*{Voraussetzungen}

Sie sollten für diese Übungen folgende Themen aus dem Buch bzw.\ der Vorlesung wiederholen, falls sie Ihnen noch nicht klar sind:

\begin{enumerate}\Lf
  \item Wie können Nominalphrasen (NPs) im Deutschen strukturiert sein? 
  \item Was für Wörter kommen als \textbf{Kopf der NP} infrage?
  \item Was kann\slash muss \textbf{links vom Kopf} stehen?
  \item Was kann\slash muss \textbf{rechts vom Kopf} stehen?
%  \item Was sind \textbf{regierte Attribute}? Im Besonderen
%    \begin{itemize}
%      \item Subjektsgenitive
%      \item Objektsgenitive
%      \item regierte satzförmige Attribute
%    \end{itemize}
  \item Wo stehen \textbf{innere Rechtsattribute} im Gegensatz zu \textbf{satzförmigen Attributen} typischerweise?
  \item Wann verwendet man in syntaktischen Analysen Dreiecke und wann einfache Kanten?
\end{enumerate}

\section{NPs identifizieren}\label{sec:erkennen}

Klammern Sie im folgenden Textausschnitt \textit{Ausnahmebehandlung} alle NPs ein.
Die erste ist beispielhaft eingeklammert.
Falls NPs in NPs eingebettet sind, klammern Sie nur die äußere NP ein.

\begin{quote}\onehalfspacing
   \textbf{Ausnahmebehandlung (Ausschnitt, bearbeitet)}\\
   {\footnotesize\url{https://de.wikipedia.org/wiki/Ausnahmebehandlung}}\\
  \begin{linenumbers}
  \newdimen\origiwspc%
  \newdimen\origiwstr%
  \origiwspc=\fontdimen2\font
  \origiwstr=\fontdimen3\font
  \fontdimen2\font=1em
  {[Eine Ausnahme oder Ausnahmesituation]} bezeichnet in \Solbrack{der Computertechnik} \Solbrack{ein Verfahren, mit dem Informationen über bestimmte Programmzustände an andere Programmebenen zur Weiterbehandlung weitergereicht werden} .
  Kann in \Solbrack{einem Programm} beispielsweise \Solbrack{einer Speicheranforderung} nicht stattgegeben werden, wird \Solbrack{eine Speicheranforderungsausnahme} ausgelöst .
  \Solbrack{Ein Computerprogramm} kann zur \Solbrack{Behandlung dieses Problems} \Solbrack{dafür definierte Algorithmen} abarbeiten, \Solbrack{die} \Solbrack{den Fehler} beheben oder anzeigen .
  \Solbrack{Exceptions} haben in \Solbrack{weiten Teilen} \Solbrack{die Behandlung von Fehlern mittels Fehlercodes oder Sprunganweisungen} abgelöst und stellen im \Solbrack{technischen Sinne} \Solbrack{einen zweiten, optionalen Rückgabewert einer Methode bzw.\ Funktion} dar .
  \fontdimen2\font=\origiwspc
  \fontdimen3\font=\origiwstr
  \end{linenumbers}
\end{quote}

\Halbzeile

\Sol{\textbf{Hinweis:} nach \textit{zur} und \textit{im} folgen NPs ohne Artikelwort, weil der Artikel im Lauf der Sprachgeschichte mit der Präposition verschmolzen ist.}


\section{Abgekürzte und volle Analysen}

Entscheiden Sie in den folgenden Analysen, ob ein Dreieck oder eine einfache Kante eingezeichnet werden muss, und vervollständigen Sie die Bäume entsprechend.
Zur Erinnerung: Sobald eine Struktur abgekürzt wurde, markieren wir das mit einem Dreieck.
Eine Abkürzung liegt selbst dann vor, wenn wir eine Phrase, die nur aus einem einzelnen Wort besteht, direkt als Phrase analysieren.
In der vollen Analyse gäbe es zunächst eine Analyseebene für die Wortklasse, und über dieser käme dann die Phrase.
Hinweis: Das Symbol KP steht für Komplementiererphrase, eine Art von Nebensatz.

\Zeile

(1)~\Solalt{%
\begin{forest}
  [NP
    [\it Orangensaft, narroof]
  ]
\end{forest}
}{\begin{forest}
  [NP, whitearc
    [\it Orangensaft]
  ]
\end{forest}}

\Doppelzeile

(2)~\Solalt{%
\begin{forest}
  [NP, calign=child, calign child=2
    [Art
      [\it jeden]
    ]
    [N 
      [\it Winter]
    ]
  ]
\end{forest}%
}{%
\begin{forest}
  [NP, calign=child, calign child=2
    [Art, whitearc
      [\it jeden]
    ]
    [N, whitearc
      [\it Winter]
    ]
  ]
\end{forest}}

\Doppelzeile

(3)~\Solalt{%
\begin{forest}
  [NP, calign=child, calign child=1
    [N
      [\it Blütenstaub]
    ]
    [PP 
      [\it auf dem Auto, narroof]
    ]
  ]
\end{forest}%
}{%
\begin{forest}
  [NP, calign=child, calign child=1
    [N, whitearc
      [\it Blütenstaub]
    ]
    [PP, whitearc
      [\it auf dem Auto]
    ]
  ]
\end{forest}%
}

\Doppelzeile

(4)~\Solalt{%
\begin{forest}
  [NP, calign=child, calign child=3
    [Art
      [\it der]
    ]
    [AP
      [\it festen, narroof]
    ]
    [N
      [\it Überzeugung]
    ]
    [KP
      [\it dass man das lernen kann, narroof]
    ]
  ]
\end{forest}%
}{%
\begin{forest}
  [NP, calign=child, calign child=3
    [Art, whitearc
      [\it der]
    ]
    [AP, whitearc
      [\it festen]
    ]
    [N, whitearc
      [\it Überzeugung]
    ]
    [KP, whitearc
      [\it dass man das lernen kann]
    ]
  ]
\end{forest}%
}

\Doppelzeile

(5)~\Solalt{%
\begin{forest}
  [NP
    [\it dieses unsympathische Lachen, narroof]
  ]
\end{forest}%
}{%
\begin{forest}
  [NP, whitearc
    [\it dieses unsympathische Lachen]
  ]
\end{forest}%
}

\newpage

\section{Phrasenstruktur von NPs}

Zeichnen Sie für die unterstrichenen NPs aus Aufgabe~\ref{sec:erkennen} Phrasenstrukturdiagramme.
Kürzen Sie dabei alle Konstituenten unterhalb der Ebene der NP ungeachtet ihrer internen Komplexität durch Dreiecke ab.

\Zeile

(0)~\begin{center}
  \begin{forest}
    [NP, calign=child, calign child=2
      [Art
        [\it eine]
      ]
      [N
        [\it Ausnahme oder Ausnahmesituation, narroof]
      ]
    ]
  \end{forest}
\end{center}


\Sol{\begin{enumerate}
  \item \begin{forest}
      [NP
        [Art
          [\it der]
        ]
        [N
          [\it Computertechnik]
        ]
      ]
    \end{forest}\\
    Ebenso: \textit{einem Programm}, \textit{einer Speicheranforderung}, \textit{eine Speicheranforderungsausnahme}, \textit{ein Computerprogramm}, \textit{den Fehler}
  \item \begin{forest}
      [NP, calign=child, calign child=2
        [Art
          [\it ein]
        ]
        [N
          [\it Verfahren]
        ]
        [RS
          [\it mit \ldots\ werden, narroof]
        ]
      ]
    \end{forest}
  \item \begin{forest}
      [NP, calign=first
        [N
          [\it Behandlung]
        ]
        [NP
          [\it dieses Problems, narroof]
        ]
      ]
    \end{forest}
  \item \begin{forest}
      [NP, calign=last
        [AP
          [\it dafür definierte, narroof]
        ]
        [N
          [\it Algorithmen]
        ]
      ]
    \end{forest}
  \item \begin{forest}
      [NP
        [N
          [\it die]
        ]
      ]
    \end{forest}
  \item \begin{forest}
      [NP
        [N
          [\it Exceptions]
        ]
      ]
    \end{forest}
  \item \begin{forest}
      [NP, calign=last
        [AP
          [\it weiten, narroof]
        ]
        [N
          [\it Teilen]
        ]
      ]
    \end{forest}\\
    Ebenso: \textit{technischen Sinne}
  \item \begin{forest}
      [NP, calign=child, calign child=2
        [Art
          [\it die]
        ]
        [N
          [\it Behandlung]
        ]
        [PP
          [\it von Fehlern, narroof]
        ]
        [PP
          [\it mittels Fehlercodes oder Sprunganweisungen, narroof]
        ]
      ]
    \end{forest}
  \item \begin{forest}
      [NP, calign=child, calign child=3
        [Art
          [\it einen]
        ]
        [AP
          [\it {zweiten, optionalen}, narroof]
        ]
        [N
          [\it Rückgabewert]
        ]
        [NP
          [\it einer \ldots\ Funktion, narroof]
        ]
      ]
    \end{forest}
\end{enumerate}}



\section{Teile von NPs identifizieren}

Unterstreichen Sie in den NPs in der zweiten Spalte der unten stehenden Tabelle die in der ersten Spalte genannten Teile.


\begin{center}
  \begin{tabular}[h]{clp{0.6\textwidth}}
    \toprule
    & \textbf{Zu unterstreichen} & \textbf{NP} \\
    \midrule
    (1) & eine AP & die \Solulmark{sehr angenehme} Kälte \\
   && \\
   (2) & alle inneren Rechtsattribute & \doublespacing Orangensaft \Solulmark{ohne Zusätze in einer Flasche aus Glas}, der in der Region hergestellt wurde\\
   && \\
   (3) & einen Nebensatz & mit der festen Überzeugung, \Solulmark{dass man das lernen kann} \\
   && \\
   (4) & alle inneren Rechtsattribute & \doublespacing Orangensaft \Solulmark{ohne Zusätze in einer Flasche aus Glas, die in der Region hergestellt wurde}\\
   && \\
   (5) & den Kopf & \doublespacing Karins überaus blasses und doch lebendiges \Solulmark{Angesicht} im Spiegel, an das sich Ingmar sein leben lang erinnern würde \\
   && \\
   (6) & alle inneren Rechtsattribute & \doublespacing Orangensaft \Solulmark{ohne Zusätze in einer Flasche aus Glas, das in der Region hergestellt wurde}\\
   && \\
   \bottomrule
  \end{tabular}
\end{center}

\newpage\hspace{1em}

