\section*{Voraussetzungen}

\begin{enumerate}\Lf
  \item Definition der Subjekts
  \item Satzprädikat als die Gesamtheit der Verben des Verbkomplexes
  \item Prädikativergänzungen bei der Kopula
  \item \ldots\ in Abgrenzung zu Pseudo-Prädikativergänzungen bei Verben wie \textit{halten} (\textit{für})
  \item Resultativprädikate
  \item die vier (bzw.\ fünf) Arten von \textit{es} im Nominativ
\end{enumerate}

\section{Subjekte identifizieren}\label{sec:subjekte}

Unterstreichen Sie in den folgenden Sätzen die Subjekte.
Wenn Nebensätze in den Sätzen enthalten sind, unterstreichen Sie nur das Subjekt des gesamten Satzes und nicht auch die Subjekte der Nebensätze.

\begin{exe}
  \ex \textit{Morgens bringt Herr Oelschlägel Herrn Uhl die Zeitung mit hoch.} \\
  \Viertelzeile
  \ex \textit{Auf der sicheren Seite ist, wer kein Risiko eingeht.} \\
  \Viertelzeile
  \ex \textit{Der neue Kollege ist unser Buchhalter.} \\
  \Viertelzeile
  \ex \textit{Wen alles nervt, nervt, dass die Sonne morgens aufgeht.} \\
  \Viertelzeile
  \ex \textit{Es wird jetzt getanzt!} \\
  \Viertelzeile
  \ex \textit{Dass die Sonne irgendwann erlischt, kann man kaum glauben.} \\
  \Viertelzeile
  \ex \textit{Wo ist Mirjam?} \\
  \Viertelzeile
  \ex \textit{Mit dem anderen kann es es nicht vereinen.} \\
  \Viertelzeile
\end{exe}

\section{Arten von \textit{es} unterscheiden}\label{sec:es}

Führen Sie für die Vorkommen von \textit{es} in den folgenden Sätzen die Tests durch, die es Ihnen erlauben, die genaue Kategorie des Vorkommens zu bestimmen.
Korrelate werden hier nicht berücksichtigt.
Die Tests sind:

\begin{itemize}\Lf
  \item Test auf Ersetzbarkeit durch Vollpronomen (VP)
  \item Test auf Verdrängbarkeit aus dem Vorfeld (VF)
  \item Weglasstest (WL)
\end{itemize}

Vergessen Sie nicht den Asterisk bei ungrammatischen Ergebnissen der Testanwendung.
Ordnen Sie es daraufhin in eine der folgenden Kategorien ein:

\begin{itemize}\Lf
  \item normales Subjektspronomen (SP)
  \item Vorfeld-\textit{es} (VF)
  \item fakultative Ergänzung (FE)
  \item obligatorische Ergänzung (OE)
\end{itemize}

\begin{center}
  \begin{longtable}[h]{clp{0.8\textwidth}}
    (1) & \textbf{Satz} & \textit{Es wird heute endlich das Boot repariert!} \\
    &&\\
    & \textbf{Test VP} & \\\cline{3-3}
    &&\\
    & \textbf{Test VF} & \\\cline{3-3}
    &&\\
    & \textbf{Test WL} & \\\cline{3-3}
    &&\\
    & \textbf{Kategorie} & \\\cline{3-3}
    &&\\
    (2) & \textbf{Satz} & \textit{Es starten vierzehn Autos.} \\
    &&\\
    & \textbf{Test VP} & \\\cline{3-3}
    &&\\
    & \textbf{Test VF} & \\\cline{3-3}
    &&\\
    & \textbf{Test WL} & \\\cline{3-3}
    &&\\
    & \textbf{Kategorie} & \\\cline{3-3}
    &&\\
    (3) & \textbf{Satz} & \textit{Es dürstet sie.} \\
    &&\\
    & \textbf{Test VP} & \\\cline{3-3}
    &&\\
    & \textbf{Test VF} & \\\cline{3-3}
    &&\\
    & \textbf{Test WL} & \\\cline{3-3}
    &&\\
    & \textbf{Kategorie} & \\\cline{3-3}
    &&\\
 \newpage
    (4) & \textbf{Satz} & \textit{Es kracht gleich!} \\
    &&\\
    & \textbf{Test VP} & \\\cline{3-3}
    &&\\
    & \textbf{Test VF} & \\\cline{3-3}
    &&\\
    & \textbf{Test WL} & \\\cline{3-3}
    &&\\
    & \textbf{Kategorie} & \\\cline{3-3}
    &&\\
    (5) & \textbf{Satz} & \textit{Es zerschellte auf dem Mars.} \\
    &&\\
    & \textbf{Test VP} & \\\cline{3-3}
    &&\\
    & \textbf{Test VF} & \\\cline{3-3}
    &&\\
    & \textbf{Test WL} & \\\cline{3-3}
    &&\\
    & \textbf{Kategorie} & \\\cline{3-3}
  \end{longtable}
\end{center}


\section{Prädikate}

Entscheiden Sie für die unterstrichenen Konstituenten in den unten stehenden Sätzen, ob sie ein sogenanntes Satzprädikat, eine Prädikativergäzung bei einer Kopula, eine Pseudo-Prädikativergänzung bei Verben wie \textit{halten für} oder ein Resultativprädikat darstellen.

% sich etwas X vorstellen
% etwas X nennen
% halten für X
% behandeln wie X
% etwas als X bezeichnen

\begin{center}
  \begin{longtable}{cl}
    (1) & \textit{Manche Menschen behandeln Tiere \ul{wie Gegenstände}}. \\
    & \Square~Satzprädikat\ \ \ \Square~Prädikativergäzung\ \ \ \Square~Pseudo-Präd.erg.\ \ \ \Square~Resultativprädikat \\
    &\\
    (2) & \textit{Das wird \ul{nichts}.} \\
    & \Square~Satzprädikat\ \ \ \Square~Prädikativergäzung\ \ \ \Square~Pseudo-Präd.erg.\ \ \ \Square~Resultativprädikat \\
    &\\
    (3) & \textit{Petra \ul{hat} den Rasen \ul{mähen wollen}.} \\
    & \Square~Satzprädikat\ \ \ \Square~Prädikativergäzung\ \ \ \Square~Pseudo-Präd.erg.\ \ \ \Square~Resultativprädikat \\
    &\\
    (4) & \textit{Das Wetter bleibt nicht, \ul{was es immer war}!} \\
    & \Square~Satzprädikat\ \ \ \Square~Prädikativergäzung\ \ \ \Square~Pseudo-Präd.erg.\ \ \ \Square~Resultativprädikat \\
    &\\
    (5) & \textit{Es \ul{tropft}.} \\
    & \Square~Satzprädikat\ \ \ \Square~Prädikativergäzung\ \ \ \Square~Pseudo-Präd.erg.\ \ \ \Square~Resultativprädikat \\
    &\\
    (6) & \textit{Michelle stampft die Walnüsse \ul{kurz und klein}.} \\
    & \Square~Satzprädikat\ \ \ \Square~Prädikativergäzung\ \ \ \Square~Pseudo-Präd.erg.\ \ \ \Square~Resultativprädikat \\
    &\\
    (7) & \textit{Sei nicht \ul{so naiv}!} \\
    & \Square~Satzprädikat\ \ \ \Square~Prädikativergäzung\ \ \ \Square~Pseudo-Präd.erg.\ \ \ \Square~Resultativprädikat \\
    &\\
    (8) & \textit{Vlado bügelt die Hemden \ul{platt}.} \\
    & \Square~Satzprädikat\ \ \ \Square~Prädikativergäzung\ \ \ \Square~Pseudo-Präd.erg.\ \ \ \Square~Resultativprädikat \\
  \end{longtable}
\end{center}

\newpage\hspace{1em}
