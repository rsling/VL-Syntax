\section*{Voraussetzungen}

\begin{enumerate}\Lf
  \item Relativsätze
  \item Genus-Numerus-Kongruenz des Relativelements
  \item Kasus bzw.\ präpositionale Form des Relativelements
  \item Rektion von Subjekt- und Objektsätzen (Komplementsätzen\slash Ergänzungssätzen) 
  \item Stellung von Ergänzungssätzen und Angabensätzen (Adverbialsätzen) in der Matrix
  \item Korrelate
  \item Feldermodell für V1-, V2- und VL-Sätze (inkl. Nachfeld)
\end{enumerate}

\section{Relativsätze analysieren}\label{sec:relativ}

(i) Erstellen Sie Phrasenstrukturdiagramme (Bäume) für die Relativsätze in den folgenden Sätzen.
Die Matrix (den Rest des Satzes) analysieren Sie bitte nicht, aber unterstreichen Sie unten in der Liste mit den Sätzen jeweils das Bezugssubstantiv.
Kürzen Sie innerhalb des Relativsatz alle in der VP enthaltenen Phrasen ab.
Nicht abkürzen dürfen Sie: den eigentlichen Relativsatz (RS), den Verbkomplex, die extrahierte Phrase im RS (die das Relativelement enthält).
(ii) Benennen Sie die Funktion des Relativelements.
Es kommen infrage:

\begin{itemize}\Lf
  \item Ergänzung
    \begin{itemize}\Lf
      \item Subjekt im Relativsatz (Nominativ) | \textbf{Subj}
      \item nominales Objekt im Relativsatz (Akkusativ oder Dativ) | \textbf{Akk-Obj}, \textbf{Dat-Obj}
      \item präpositionales Objekt im Relativsatz | \textbf{P-Obj}
    \end{itemize}
  \item Angabe\slash Adverbial im Relativsatz | \textbf{Adv}
  \item Genitivattribut in einer NP im Relativsatz | \textbf{Gen}
\end{itemize}

\Halbzeile

Hier die Sätze, die die Relativsätze enthalten:

\begin{exe}
\setcounter{xnumi}{0}
  \ex Ich kann so ziemlich jedes \Solulmark{Buch}, das ich in der Schule lesen musste, nicht ausstehen.
  \ex Der \Solulmark{Stock}, mit dem mein Vater nach dem Unfall immer spazieren gegangen ist, stammt noch von meinem Opa.
  \ex Das Auto der Kollegin mit den \Solulmark{Kindern}, die auf der Geburtstagsfeier offensichtlich stören wollten, fährt wieder.
  \ex \Solulmark{Denjenigen}, mit dessen Auto wir zurückfahren, losen wir vorher immer aus.
\end{exe}

\Halbzeile

\Sol{%
\begin{enumerate}
  \item Funktion: Akkusativergänzung (Objekt)\\
    \begin{forest}
      [RS, calign=first
        [NP\Sub{1}, tier=prepre
          [N, ake, narroof
            [\it \blau{das}]
          ]
        ]
        [VP, calign=last
          [NP, ake
            [\it ich, narroof]
          ]
          [\Ti, ake]
          [PP, ake
            [\it in der Schule, narroof]
          ]
          [V, calign=last, tier=prepre
            [V, ake
              [\it lesen]
            ]
            [V, ake
              [\it musste]
            ]
          ]
        ]
      ]
    \end{forest}
  \item Funktion: Angabe (PP) \\
    \begin{forest}
      [RS, calign=first
        [PP\Sub{1}, calign=first, tier=prepre
          [P, ake
            [\it mit]
          ]
          [NP
            [N, ake
              [\it \blau{dem}]
            ]
          ]
        ]
        [VP, calign=last
          [NP, ake
            [\it mein Vater, narroof]
          ]
          [PP, ake
            [\it nach dem Unfall, narroof]
          ]
          [\Ti, ake]
          [AdvP, ake
            [\it immer, narroof]
          ]
          [V, calign=last, tier=prepre
            [V, calign=last
              [V, ake
                [\it spazieren]
              ]
              [V, ake
                [\it gegangen]
              ]
            ]
            [V, ake
              [\it ist]
            ]
          ]
        ]
      ]
    \end{forest}
  \item Funktion: Nominativergänzung (Subjekt)\\
    \begin{forest}
      [RS, calign=first
        [NP\Sub{1}, tier=prepre
          [N, ake
            [\it \blau{die}]
          ]
        ]
        [VP, calign=last
          [\Ti, ake]
          [PP, ake
            [\it auf der Geburtstagsf., narroof]
          ]
          [AdvP, ake
            [\it offensichtlich, narroof]
          ]
          [V, calign=last, tier=prepre
            [V, ake
              [\it stören]
            ]
            [V, ake
              [\it wollten]
            ]
          ]
        ]
      ]
    \end{forest}
  \item Funktion: attributiver Genitiv\\
    \begin{forest}
      [RS, calign=first
        [PP\Sub{1}, calign=first
          [P, ake
            [\it mit]
          ]
          [NP, calign=last
            [NP
              [N, ake
                [\it \blau{dessen}]
              ]
            ]
            [N, ake
              [\it Auto]
            ]
          ]
        ]
        [VP, calign=last
          [NP, ake
            [\it wir, narroof]
          ]
          [\Ti, ake]
          [V, ake
            [\it zurückfahren]
          ]
        ]
      ]
    \end{forest}
\end{enumerate}
}


\section{Rektionsmuster von Objektsätzen}\label{sec:nebensaetze}

Bestimmen Sie für die unten genannten Verben, welches der folgenden Rektionsmuster für Objektsätze sie haben.

\begin{itemize}\Lf
  \item nur \textit{dass}-Sätze
  \item nur Fragesätze (\textit{w}-Sätze)
  \item \textit{dass}-Sätze und Fragesätze
\end{itemize}

Bilden Sie dazu für jedes der Verben einen Satz mit einem \textit{dass}-Objektsatz und einem Fragesatz in Objektfunktion und entscheiden Sie, ob die Sätze grammatisch sind.

\begin{exe}
\setcounter{xnumi}{0}
\ex (\textit{sich}) \textit{überlegen} % ob
  \ex \textit{vergessen} % beides
  \ex \textit{vorwerfen} % dass
  \ex \textit{ahnen} % beides
\end{exe}

\Halbzeile

\Sol{%
  \begin{enumerate}
    \item Nur \textit{w}-Sätze, aber bei manchen Sprechern auch \textit{dass}-Sätze:
      \begin{enumerate}
        \item ? Ich überlege mir, dass ich zum Konzert fahre. 
        \item Ich überlege mir, ob ich zum Konzert fahre.
      \end{enumerate}
    \item \textit{dass}- und \textit{w}-Sätze:
      \begin{enumerate}
        \item Ich vergaß, dass Sonntag war.
        \item Ich vergaß, ob Sonntag war.
      \end{enumerate}
    \item Nur \textit{dass}-Sätze:
      \begin{enumerate}
        \item Ich werfe mir vor, dass ich das Buch vergessen habe.
        \item * Ich werfe mir vor, ob ich das Buch vergessen habe.
      \end{enumerate}
    \item \textit{dass}- und \textit{w}-Sätze:
      \begin{enumerate}
        \item Ich ahne, dass  es bald regnet.
        \item Ich ahne, wer gleich nassgeregnet wird.
      \end{enumerate}
  \end{enumerate}
}


\section{Korrelate}\label{sec:korrelate}

(i) Bestimmen Sie für die folgenden \textit{dass}-Sätze, ob es Subjekt- oder Objektsätze sind.
(ii) Fügen Sie ein Korrelat des \textit{dass}-Satzes hinzu und stellen Sie die Sätze -- falls nötig -- so um, dass sie mit Korrelat grammatisch sind.


\begin{exe}
\setcounter{xnumi}{0}
  \ex Chloe teilte uns mit, dass der Regen stärker als sonst ausfiel.
  \ex Dass wir alle im Regen nass geworden sind, ist mir für heute genug Erfrischung.
  \ex Dass der Wagen an jedem Morgen voll Blütenstaub ist, gefällt mir nicht.
  \ex Ich behaupte einfach mal, dass Chloe weiß, dass die Regenschirme im Keller sind.
\end{exe}

\Sol{%
  \begin{exe}
  \setcounter{xnumi}{0}
    \ex Chloe teilte \orongsch{es} uns mit, dass der Regen stärker als sonst ausfiel.\\
    \tuerkis{(Objektsatz)}
    \ex \orongsch{Es} ist mir für heute genug Erfrischung, dass wir alle im Regen nass geworden sind.\\
    \tuerkis{(Subjektsatz)}
    \ex \orongsch{Es} gefällt mir nicht, dass der Wagen an jedem Morgen voll Blütenstaub ist.\\
    \tuerkis{(Subjektsatz)}
    \ex Ich behaupte \orongsch{es} einfach mal, dass Chloe \blau{es} weiß, dass die Regenschirme im Keller sind.\\
    \tuerkis{(Objektsatz, Objektsatz)}
  \end{exe}
}

\section{Analysen im Feldermodell}\label{sec:felder}

Analysieren Sie die folgenden Sätze im Feldermodell wie im Beispiel dargestellt.
Wenn eins der Felder einen Nebensatz enthält, fertigen Sie für den Nebensatz eine zusätzliche unabhängige Felderanalyse an.

\begin{exe}
\setcounter{xnumi}{0}
  \ex Wir würden viel geben für den Frieden.
  \ex Wer nicht hilfsbereit ist, kann meistens auch nicht auf die Hilfe anderer hoffen.
  \ex Willst du dem Typ etwa wirklich glauben, dass die Syntaxklausur schwer ist?
  \ex Der Vogel singt.
  \ex Der Vogel hat gesungen, bevor der Regen kam.
  \ex Dass ich den Vogel, der vor meinem Fenster saß, singen hören konnte, stimmt.
\end{exe}

Beispielanalyse:

\Zeile

\begin{tabular}[h]{lp{1em}l}
  \textbf{Vf}  && \textit{Wir} \\
  \textbf{LSK} && \textit{würden} \\
  \textbf{Mf}  && \textit{viel} \\
  \textbf{RSK} && \textit{geben} \\
  \textbf{Nf}  && \textit{für den Frieden} \\
\end{tabular}

\Sol{%
  \begin{enumerate}
    \item
      \begin{enumerate}
      \item 
        \begin{tabular}[h]{lp{1em}l}
          \textbf{Vf}  && \textit{Wer nicht hilfsbereit ist} \\
          \textbf{LSK} && \textit{kann} \\
          \textbf{Mf}  && \textit{meistens \ldots\ anderer} \\
          \textbf{RSK} && \textit{hoffen} \\
          \textbf{Nf}  && \textit{---} \\
        \end{tabular}
      \item 
        \blau{\begin{tabular}[h]{lp{1em}l}
          \textbf{Vf}  && \textit{wer} \\
          \textbf{LSK} && \textit{---} \\
          \textbf{Mf}  && \textit{nicht hilfsbereit} \\
          \textbf{RSK} && \textit{ist} \\
          \textbf{Nf}  && \textit{---} \\
        \end{tabular}}
      \end{enumerate}
    \item
      \begin{enumerate}
      \item 
        \begin{tabular}[h]{lp{1em}l}
          \textbf{Vf}  && \textit{---} \\
          \textbf{LSK} && \textit{Willst} \\
          \textbf{Mf}  && \textit{du dem Typ wirklich} \\
          \textbf{RSK} && \textit{glauben} \\
          \textbf{Nf}  && \textit{dass \ldots\ ist} \\
        \end{tabular}
      \item 
        \blau{\begin{tabular}[h]{lp{1em}l}
          \textbf{Vf}  && \textit{---} \\
          \textbf{LSK} && \textit{dass} \\
          \textbf{Mf}  && \textit{die Syntaxklausur schwer} \\
          \textbf{RSK} && \textit{ist} \\
          \textbf{Nf}  && \textit{---} \\
        \end{tabular}}
      \end{enumerate}
    \item 
      \begin{tabular}[h]{lp{1em}l}
        \textbf{Vf}  && \textit{Der Vogel} \\
        \textbf{LSK} && \textit{singt} \\
        \textbf{Mf}  && \textit{---} \\
        \textbf{RSK} && \textit{---} \\
        \textbf{Nf}  && \textit{---} \\
      \end{tabular}
    \item
      \begin{enumerate}
      \item 
        \begin{tabular}[h]{lp{1em}l}
          \textbf{Vf}  && \textit{Der Vogel} \\
          \textbf{LSK} && \textit{hat} \\
          \textbf{Mf}  && \textit{---} \\
          \textbf{RSK} && \textit{gesungen} \\
          \textbf{Nf}  && \textit{bevor \ldots} \\
        \end{tabular}
      \item 
        \blau{\begin{tabular}[h]{lp{1em}l}
          \textbf{Vf}  && \textit{---} \\
          \textbf{LSK} && \textit{bevor} \\
          \textbf{Mf}  && \textit{der Regen} \\
          \textbf{RSK} && \textit{kam} \\
          \textbf{Nf}  && \textit{---} \\
        \end{tabular}}
      \end{enumerate}
    \item
      \begin{enumerate}
      \item 
        \begin{tabular}[h]{lp{1em}l}
          \textbf{Vf}  && \textit{Dass \ldots\ konnte} \\
          \textbf{LSK} && \textit{stimmt} \\
          \textbf{Mf}  && \textit{---} \\
          \textbf{RSK} && \textit{---} \\
          \textbf{Nf}  && \textit{---} \\
        \end{tabular}
      \item 
        \blau{\begin{tabular}[h]{lp{1em}l}
          \textbf{Vf}  && \textit{---} \\
          \textbf{LSK} && \textit{Dass} \\
          \textbf{Mf}  && \textit{ich \ldots\ saß} \\
          \textbf{RSK} && \textit{singen hören konnte} \\
          \textbf{Nf}  && \textit{---} \\
        \end{tabular}}
      \item 
        \orongsch{\begin{tabular}[h]{lp{1em}l}
          \textbf{Vf}  && \textit{der} \\
          \textbf{LSK} && \textit{---} \\
          \textbf{Mf}  && \textit{vor meinem Fenster} \\
          \textbf{RSK} && \textit{saß} \\
          \textbf{Nf}  && \textit{---} \\
        \end{tabular}}
      \end{enumerate}
  \end{enumerate}
}

\section{Bonusaufgabe}

Erstellen Sie Phrasenstrukturdiagramme ohne Abkürzungen für die Sätze aus Aufgabe \ref{sec:felder}.

\Sol{\newpage\ \\}

