\section{Kongruenz}

Finden Sie im Textausschnitt \textit{Gang durch das Ried} fünf Fälle von Subjekt-Verb-Kongruenz und fünf Nominalphrasen (Substantiv mit mindestens einem vorausgehenden Artikel oder einem vorausgehenden Adjektiv), in denen Kongruenz herrscht.

\textbf{Hinweise:} Kongruenz von Relativpronomina soll hier noch hicht erfasst werden. Und achten Sie darauf, dass in Koordinationsstrukturen mit Kommata und \textit{und} oder \textit{oder} Kasuskongruenz herrscht!

\section{Verbvalenz}

Entscheiden Sie für die numerierten und in eckige Klammern gesetzten Phrasen im Textausschnitt \textit{Gang durch das Ried}, ob sie Ergänzungen oder Angaben sind.
Finden Sie dafür zunächst das Verb, von dem sie abhängen, und entscheiden Sie dann, ob es sich um Ergänzungen und Angaben handelt.
Im Fall von Ergänzungen geben Sie an, welches Merkmal \slash\ welche Form das Verb regiert.
Rechnen Sie damit, dass es in einigen Fällen schwer entscheidbar ist, ob es sich um eine Ergänzung oder eine Angabe handelt.
Wir kommen im letzten drittel der Vorlesung aber nochmal darauf zurück, um etwas mehr Klarheit zu schaffen.

\textbf{Hinweis:} Die Phrasenstruktur ist teilweise komplexer, als die Klammerung andeutet.
Teilweise wurden Relativsätze nicht mit eingeklammert, um das Ganze für Sie übersichtlicher zu gestalten.

Zur Wiederholung: Die wichtigsten Unterschiede zwischen valenzgebundenen Ergänzungen und Angaben sind die folgenden:

\Zeile

\begin{center}
  \begin{tabular}[h]{lll}
    \toprule
    & \textbf{Ergänzungen} & \textbf{Angaben} \\
    \midrule
    \textbf{Semantik} & verbgebunden & verbunabhängig \\
    \textbf{Weglassbarkeit} & manchmal\slash oft obligatorisch & immer fakultativ \\
    \textbf{Kasus\slash Präposition\slash\ldots} & regiert & frei \\
    \textbf{Lizenzierung} & einmalig & iterierbar \\
    \midrule
    \textbf{Schultermini} & Subjekt, Objekte & adverbiale Bestimmung \\
    \bottomrule
  \end{tabular}
\end{center}

\newpage

\begin{center}
  \begin{tabular}[h]{ccp{0.2\textwidth}lp{0.3\textwidth}}
    \toprule
    & \textbf{Z.} & \textbf{Verb} & \textbf{Status} & \textbf{regiertes Merkmal} \\
    \midrule
    &&& \\ (1)  & \lineref{lne:phrase1}  & \Sol{ging}           & \Solalt{\Square}{\Square}~Ergänzung\ \ \ \Solalt{\XBox}{\Square}~Angabe & \Sol{} \\ \cline{3-3}\cline{5-5} 
    &&& \\ (2)  & \lineref{lne:phrase2}  & \Sol{nieselte}       & \Solalt{\XBox}{\Square}~Ergänzung\ \ \ \Solalt{\Square}{\Square}~Angabe & \Sol{die Form \textit{es} (Wetterverb)} \\ \cline{3-3}\cline{5-5} 
    &&& \\ (3)  & \lineref{lne:phrase3}  & \Sol{nieselte}       & \Solalt{\Square}{\Square}~Ergänzung\ \ \ \Solalt{\XBox}{\Square}~Angabe & \Sol{} \\ \cline{3-3}\cline{5-5} 
    &&& \\ (4)  & \lineref{lne:phrase4}  & \Sol{schlug}         & \Solalt{\XBox}{\Square}~Ergänzung\ \ \ \Solalt{\Square}{\Square}~Angabe & \Sol{Akk (Objekt)} \\ \cline{3-3}\cline{5-5} 
    &&& \\ (5)  & \lineref{lne:phrase5}  & \Sol{wuchs}          & \Solalt{\Square}{\Square}~Ergänzung\ \ \ \Solalt{\XBox}{\Square}~Angabe & \Sol{} \\ \cline{3-3}\cline{5-5} 
    &&& \\ (6)  & \lineref{lne:phrase6}  & \Sol{vorbeiführten}  & \Solalt{\Square}{\Square}~Ergänzung\ \ \ \Solalt{\XBox}{\Square}~Angabe & \Sol{} \\ \cline{3-3}\cline{5-5} 
    &&& \\ (7)  & \lineref{lne:phrase7}  & \Sol{vorbeiführten}  & \Solalt{\XBox}{\Square}~Ergänzung\ \ \ \Solalt{\Square}{\Square}~Angabe & \Sol{Präp \textit{an}} \\ \cline{3-3}\cline{5-5} 
    &&& \\ (8)  & \lineref{lne:phrase8}  & \Sol{gehört}         & \Solalt{(?)}{\Square}~Ergänzung\ \ \ \Solalt{(?)}{\Square}~Angabe       & \Sol{Präp \textit{von} (Passiv)} \\ \cline{3-3}\cline{5-5} 
    &&& \\ (9)  & \lineref{lne:phrase9}  & \Sol{erfüllt}        & \Solalt{\XBox}{\Square}~Ergänzung\ \ \ \Solalt{\Square}{\Square}~Angabe & \Sol{Nom (Subjekt)} \\ \cline{3-3}\cline{5-5} 
    &&& \\ (10) & \lineref{lne:phrase10} & \Sol{erfüllt}        & \Solalt{\XBox}{\Square}~Ergänzung\ \ \ \Solalt{\Square}{\Square}~Angabe & \Sol{Präp \textit{von}} \\ \cline{3-3}\cline{5-5} 
    &&& \\ (11) & \lineref{lne:phrase11} & \Sol{spürbar waren}  & \Solalt{\Square}{\Square}~Ergänzung\ \ \ \Solalt{\XBox}{\Square}~Angabe & \Sol{} \\ \cline{3-3}\cline{5-5} 
    &&& \\ (12) & \lineref{lne:phrase12} & \Sol{herrschte}      & \Solalt{\XBox}{\Square}~Ergänzung\ \ \ \Solalt{\Square}{\Square}~Angabe & \Sol{Nom (Subjekt)} \\ \cline{3-3}\cline{5-5} 
    &&& \\ (13) & \lineref{lne:phrase13} & \Sol{erinnert}       & \Solalt{\XBox}{\Square}~Ergänzung\ \ \ \Solalt{\Square}{\Square}~Angabe & \Sol{Gen (Objekt)} \\ \cline{3-3}\cline{5-5} 
    &&& \\ (14) & \lineref{lne:phrase14} & \Sol{enthoben}       & \Solalt{\XBox}{\Square}~Ergänzung\ \ \ \Solalt{\Square}{\Square}~Angabe & \Sol{Dat\slash Gen (Objekt)} \\ \cline{3-3}\cline{5-5} 
    &&& \\ (15) & \lineref{lne:phrase15} & \Sol{wucherte}       & \Solalt{\XBox}{\Square}~Ergänzung\ \ \ \Solalt{\Square}{\Square}~Angabe & \Sol{Nom (Subjekt)} \\ \cline{3-3}\cline{5-5} 
    &&& \\ (16) & \lineref{lne:phrase16} & \Sol{abfing}         & \Solalt{\XBox}{\Square}~Ergänzung\ \ \ \Solalt{\Square}{\Square}~Angabe & \Sol{Nom (Subjekt)} \\ \cline{3-3}\cline{5-5} 
    &&& \\ (17) & \lineref{lne:phrase17} & \Sol{standen}        & \Solalt{(?)}{\Square}~Ergänzung\ \ \ \Solalt{(?)}{\Square}~Angabe       & \Sol{s.\ Anm.} \\ \cline{3-3}\cline{5-5} 
    &&& \\ (18) & \lineref{lne:phrase18} & \Sol{angelaufen}     & \Solalt{\Square}{\Square}~Ergänzung\ \ \ \Solalt{\XBox}{\Square}~Angabe & \Sol{} \\ \cline{3-3}\cline{5-5} 
    &&& \\ (19) & \lineref{lne:phrase19} & \Sol{übriggeblieben} & \Solalt{\XBox}{\Square}~Ergänzung\ \ \ \Solalt{\Square}{\Square}~Angabe & \Sol{Präp \textit{von}} \\ \cline{3-3}\cline{5-5} 
    &&& \\ (20) & \lineref{lne:phrase20} & \Sol{verschüttet}    & \Solalt{(?)}{\Square}~Ergänzung\ \ \ \Solalt{(?)}{\Square}~Angabe       & \Sol{Präp \textit{von} (Passiv)} \\ \cline{3-3}\cline{5-5} 
    &&& \\ (21) & \lineref{lne:phrase21} & \Sol{stand}          & \Solalt{(?)}{\Square}~Ergänzung\ \ \ \Solalt{(?)}{\Square}~Angabe       & \Sol{s.\ Anm.} \\ \cline{3-3}\cline{5-5} 
    &&& \\ (22) & \lineref{lne:phrase22} & \Sol{tätowiert}      & \Solalt{\XBox}{\Square}~Ergänzung\ \ \ \Solalt{\Square}{\Square}~Angabe & \Sol{Nom (Subjekt)} \\ \cline{3-3}\cline{5-5} 
    &&& \\ (23) & \lineref{lne:phrase23} & \Sol{entschließen}   & \Solalt{\XBox}{\Square}~Ergänzung\ \ \ \Solalt{\Square}{\Square}~Angabe & \Sol{Infinitiv mit \textit{zu}} \\ \cline{3-3}\cline{5-5} 
  \end{tabular}
\end{center}

\Sol{\textbf{Anmerkungen:} Die Phrasen in (17) und (21) stellen Grenzfälle dar, da sie obligatorisch sind, aber ihre Form nicht regiert wird. Manche behandeln sie als (pragmatisch konditionierte) "`obligatorische Angaben"', was für mich ein Widerspruch in sich ist. Ab EGBD4 werden sie bei mir konsequent als valenzgebundene Ergänzungen behandelt. In der dritten Auflage gibt es dazu Vertiefung~2.2 auf S.~46. Weiterhin werden die PPs mit \textit{von} beim Passiv bald als Ergänzung, bald als Angabe analysiert. Ab EGBD4 sind auch sie für mich Ergänzungen, aber selbstverständlich fakultative.
}

\section{Text}

\newcommand{\Kong}[1]{\Solalt{(#1)}{#1}}

\newcommand{\Nokong}[1]{\Solalt{\orongsch{(#1)}}{#1}}

\newcommand{\Relkong}[1]{\Solalt{\blau{(#1)}}{#1}}

\Sol{\textbf{Hinweise:} Die Phrasenstruktur wird hier insgesamt vereinfacht analysiert, um nur die Kongruenzphänomene zu verdeutlichen. Alle kongruierenden NPs sind grün markiert und eingeklammert. \orongsch{Einwort-NPs ohne Kongruenz sind orange markiert und eingeklammert.} \blau{Die blau markierten und eingeklammerten Wörter sind Relativpronomina.} Diese kongruieren in Genus und Numerus mit ihrem Bezugswort (direkt links vor dem Relativsatz, zumindest wenn der Relativsatz nicht rechtsversetzt wurde). In einer oberflächlichen syntaktischen Analyse wie hier lässt sich das nicht besser darstellen.}

\Sol{\textbf{Zusatzfrage}: Betrachten Sie den Unterschied zwischen \textit{deren Holz} (Z.~\lineref{lne:deren}; ohne Kongruenz zwischen \textit{deren} und \textit{Holz}) und \textit{seine leere Schale} (Z.~\lineref{lne:seine}; mit Kongruenz zwischen \textit{seine} und \textit{Schale}). Sie benötigen dazu auch Ihr Wissen aus der Morphologie. Welche Wortart haben die vier relevanten Wörter? In welchen Kasus stehen die Wörter jeweils? In diesen NPs lauern zwei typische Verständnisfehler.}

\begin{nohyphens}
  \begin{quote}
    \textbf{Elisabeth Langgässer: Gang durch das Ried (Anfang)}\\
    Verlag Jakob Hegner 1936, S. 1\\[0.5\baselineskip]
    \begin{linenumbers}
      Im \Nokong{Spätherbst} \Kong{des Jahres 1930} \linelabel{lne:ging} ging \Kong{ein Mann} \linelabel{lne:phrase1}1[über \Kong{das verlassene französische Lager}, das früher \Kong{ein deutsches} gewesen war und \Nokong{sich} zwischen \Kong{der hessischen Hauptstadt, umschließenden Tannen- und Birkenwäldern und dem großen Sande} dahinzieht]\Sub{(1)}.
      \linelabel{lne:phrase2}2[Es]\Sub{2} nieselte langsam \linelabel{lne:phrase3}3[\Kong{vom Himmel} herunter]\Sub{3}, \Kong{der Mann} \linelabel{lne:schlug} schlug \linelabel{lne:phrase4}4[\Kong{den Kragen} \Kong{der Jacke}] hoch und rückte \Kong{das Kappenschild} noch tiefer in \Kong{die Stirne}.
      \linelabel{lne:phrase5}5[Auf \Kong{den breiten Kasernenstraßen}]\Sub{5}, \Relkong{die} \linelabel{lne:phrase6}6[durch \Kong{leere Barackenreihen}]\Sub{6}, \linelabel{lne:phrase7}7[an \Kong{Stallungen, Vorratshäusern und Kantinen}]\Sub{7} vorüberführten, \linelabel{lne:wuchs} wuchs \Kong{dichtes, grünbraunes Gras}, \Relkong{das} \Kong{jeden Schritt} verschluckte und \Kong{den Wandernden} wesenlos wie \Kong{eine Traumgestalt} machte, \Relkong{die}, wenn \Nokong{sie} auch rufen würde, \linelabel{lne:phrase8}8[von \Nokong{niemand}]\Sub{8} gehört werden könnte.
      Noch vor kurzem \linelabel{lne:hatten} hatten hier \Nokong{Feuerwerker} aus \Kong{Koblenz und Ludwigshafen} \Kong{den Übungsplatz} abgesucht und \Kong{die Blindgänger, Handgranaten und letzten Depots} gesprengt – \Kong{die Erde} \linelabel{lne:war} war damals zerstampft und \linelabel{lne:phrase9}9[\Kong{der Himmel}]\Sub{9} \linelabel{lne:phrase10}10[von \Kong{dem Echo} \Kong{jener dumpfen Schläge}]\Sub{10} erfüllt gewesen, \Relkong{die} \linelabel{lne:phrase11}11[bis in \Kong{das Ried} hinein[\Sub{11} und noch weiterhin spürbar waren.
      Jetzt aber herrschte \linelabel{lne:phrase12}12[\Nokong{Stille}]\Sub{12}, \Kong{eine blöde Taubheit} gleichsam, \Relkong{die} sich wohl noch \linelabel{lne:phrase13}13[\Kong{der Töne}]\Sub{13} erinnert, doch so sehr mit \Nokong{ihnen} gesättigt ist, daß \Nokong{sie} nichts mehr vernehmen kann.
      \Kong{Manche Fensterscheibe} war da und dort durch \Kong{die Erschütterung} eingefallen und starrte gezackt wie \Kong{ein schwarzer Stern} aus \Kong{der bröckelnden Mauerfüllung}; \linelabel{lne:phrase14}14[\Kong{der rostigen Angel}]\Sub{14} enthoben, hing \Kong{eine morsche Tür} schief zu \Kong{der eigenen Achse}; \Kong{eine andere} schlug unaufhörlich, von \Kong{dem Zugwind} angetrieben, bis \Kong{zur Hälfte} \Kong{der Schwelle} vor, wo \linelabel{lne:phrase15}15[\Kong{ein üppiges Mooskissen}]\Sub{15} wucherte, \linelabel{lne:phrase16}16[\Relkong{das}]\Sub{16} \Nokong{sie} geräuschlos abfing.
      Auch \Kong{ein paar fetzige Wellblechbaracken} standen \linelabel{lne:phrase17}17[neben \Kong{den Backsteinbauten}]\Sub{17}; \Nokong{sie} waren \linelabel{lne:phrase18}18[rötlichgelb]\Sub{18} angelaufen und \linelabel{lne:phrase19}19[von \Kong{der großen Versteigerung}]\Sub{19} vor \Nokong{Wochen} übriggeblieben.

      Was \Kong{diese Versteigerung} anging, so konnte \Nokong{man} damals glauben, in \Kong{einer Stadt} zu sein, die \linelabel{lne:phrase20}20[von \Nokong{Erdkatastrophen}]\Sub{20} verschüttet gewesen und dann wieder ausgegraben und aufgebaut worden war: \linelabel{lne:phrase21}21[unter \Kong{freiem Himmel}]\Sub{21} stand, abgenutzt, \Kong{das Inventar} \Kong{der Kasernen} – \Kong{alte Schränke}, \Relkong{die} jammervoll quietschten, \Kong{verwanzte Betten und Öfen}, \Relkong{welche} glatt auseinanderfielen, \Kong{ein paar Schemel} mit \Kong{starrenden Beinen}, \Kong{befleckte Bänke und Tische}, \linelabel{lne:deren}\linelabel{lne:phrase22}22[\Relkong{deren} \Nokong{Holz}]\Sub{22}, wo es irgend anging, unzüchtig tätowiert war, \Kong{nutzlose Eisenteile}, \Relkong{die} von \Nokong{Lumpensammlern} hinausgefahren und auf \Kong{halbem Weg} wieder verloren wurden.
      Nur \Kong{einige Feldbettstellen} waren ungefragt hiergeblieben und \Kong{jene Wellblechbaracken}, \Relkong{die}, zerrissen, als ob \Kong{eine Schere} \Nokong{sie} geschlitzt und geschnitten hätte, ja, teilweise schon zusammengebrochen, in \Kong{dem weiten Gelände} ruhten und \Kong{den Eindruck} \Kong{riesiger Raupen oder Fabeltiere} machten, \Relkong{welche} rasselnd niedergesunken, doch immer noch gefährlich und voll \Kong{tückischer Drohung} sind.
      \Kong{Der Krieg} hatte, wie \Nokong{ihm} gemäß ist, wenn \Nokong{er} irgendwo \Nokong{Abschied} nimmt, \linelabel{lne:seine}\Kong{seine leere Schale} zurückgelassen, \Kong{diese armselig rohen Kasernen}, in \Relkong{denen} \Nokong{er} noch immer so gegenwärtig war, daß selbst \Kong{die Allerärmsten} \Nokong{sich} nicht entschließen konnten, \linelabel{lne:phrase23}23[in \Kong{den verlassenen Höhlen} \Kong{einen Unterschlupf} zu suchen]\Sub{23}.
    \end{linenumbers}
  \end{quote}
\end{nohyphens}

\Sol{\textbf{Beispiele für Subjekt-Verb-Kongruenz}
\begin{enumerate}\Lf
  \item \textit{ging} --- \textit{ein Mann} (Z.~\lineref{lne:ging})
  \item \textit{der Mann} --- \textit{schlug} (Z.~\lineref{lne:schlug})
  \item \textit{wuchs} --- \textit{dichtes, grünbraunes Gras} (Z.~\lineref{lne:wuchs})
  \item \textit{hatten} --- \textit{Feuerwerker} (Z.~\lineref{lne:hatten})
  \item \textit{die Erde} --- \textit{war} (Z.~\lineref{lne:war})
\end{enumerate}
}

\newpage\hspace{1em}
