\section*{Voraussetzungen}

Sie sollten für diese Übungen folgende Themen aus dem Buch bzw.\ der Vorlesung wiederholen, falls sie Ihnen noch nicht klar sind:

\begin{enumerate}\Lf
  \item Was sind \textbf{Phrasen}?
  \item Welche Phrasen sind die sogenannten \textbf{Satzglieder}?
  \item Wie funkionieren die \textbf{Konstituententests}
    \begin{enumerate}\Lf
      \item \textbf{Pronominalisierungstest} bzw.\ \textbf{Ein-Wort-Ersetzungstext}
      \item \textbf{Bewegungstest} und \textbf{Vorfeldtest}
      \item \textbf{Koordinationstest}\\
        Achtung! Dieser Test produziert sehr viele falschpositive Ergebnisse!
    \end{enumerate}
  \item \textbf{Strukturelle Ambiguität}
\end{enumerate}

\section{Konstituenten und Tests}

(a)~Wenden Sie jeweils alle Tests auf die in eckigen Klammern gesetzten potentiellen Konstituenten an und entscheiden Sie, ob Sie sie als Phrase behandeln möchten.
Erwarten Sie wie gesagt nicht, dass das perfekt und immer eindeutig funktioniert.
Durch die Anwendung der Tests sollen Sie vor allem ihre Intuition für Satzstrukturen entwickeln.
Erst durch die wohldefinierten Phrasenschemata können Sie dann in ein paar Wochen sicher entscheiden, was wir als Phrase auffassen wollen.
(b)~Wenn eine Konstituente eine Phrase ist, ist sie dann auch ein Satzglied?
(c)~Überlegen Sie, welches Wort der Kopf der Phrase sein könnte.
Teilaufgabe (c) ist ausdrücklich eine Entlastungsaufgabe.
Sie müssen das zum jetzigen Zeitpunkt noch nicht perfekt können.

\Halbzeile

\Sol{\textbf{Hinweise:} Phrasen wurden grün markiert, \blau{Satzglieder blau}. Die \orongsch{orange markierten Konstituenten} werden im Laufe der Vorlesung noch genauer besprochen. Die Köpfe sind \uline{unterstrichen} außer in diesen rot markierten Phrasen. Das Wort \textit{eben} ist etwas schwierig. Es bildet zwar eine Konstituente, als Partikel aber keine eigene Phrase. Daher scheitern auch die Tests.}

\begin{exe}
  \ex{} \Solmark[blau]{[\Solul{Unter} dem Tisch]} steht der [Papierkorb].
  \ex{} Wir [waren alle] etwas neben \Solmark{[der \Solul{Spur}]}.
  \ex{} Johan weiß [nur] zu genau, dass die Frage, \Solmark[orongsch]{[die gestellt wurde]}, nicht \Solmark[blau]{[\Solul{beantwortbar}]} ist.
  \ex{} Früher entschied \Solmark[blau]{[\Solul{in} der BRD]} \Solmark[blau]{[die \Solul{FDP}]} \Solmark[blau]{[\Solul{meistens}]} \Solmark[blau]{[\Solul{mit} nur 5 \%]} \Solmark[blau]{[die \Solul{Kanzlerfrage}]}.
  \ex{} Ich sehe in \Solmark[blau]{[letzter \Solul{Zeit}]} [ein blaues] Auto [immer] wieder die Straße entlangfahren.
  \ex{} Ob Syntax wirklich [schwerer als] Morphologie ist, [wage ich] [zu bezweifeln].
  \ex{} Die Universität, \Solmark{[deren \Solul{Kassen}]} leer sind, \Solmark[orongsch]{[streicht vakante Professuren]}.
  \ex{} Das ergibt dann [eben] in der gegebenen Situation keinen Sinn.
\end{exe}


\section{Strukturelle Ambiguität}

\subsection{Ambige Phrasengrenzen}

Strukturelle Ambiguität kommt in der Regel so zustande, dass sich in der einen Interpretation relativ zur anderen eine Phrasengrenze verschiebt.
Stellen Sie fest, inwiefern die folgenden Sätze strukturell ambig sind und stellen Sie fest, welche Konstituenten sich in den beiden Lesarten unterschiedlich zusammensetzen.

\begin{exe}
  \ex Sandy hat einen Mann ohne Schuhe fotografiert.\label{ex:schuhe}
  \ex Auf der A9 waren wieder große LKWs und Autos unterwegs.\label{ex:a9}
  \ex Letztes Jahr hat Rahel sich eine Olive im Martini bestellt.\label{ex:rahel}
  \ex Sie hat den Tathergang in Begleitung ihres Anwalts niedergeschrieben.\label{ex:tathergang}
  \ex Wir haben das Problem mit dem Taschenrechner gelöst.\label{ex:problem}
\end{exe}

\Halbzeile

\Sol{\textbf{Lösung:} Es gibt jeweils eine Lesart mit \blau{einer großen Konstituente (blau)} und eine mit \orongsch{zwei kleineren Konstituenten (orange)}.
\begin{itemize}
  \item \blau{einen Mann ohne Schuhe}  ;  \orongsch{einen Mann}  ;  \orongsch{ohne Schuhe}
  \item \blau{große LKWs und Autos}  ;  \orongsch{große LKWs}  ;  \orongsch{Autos}
  \item \blau{eine Olive im Martini}  ;  \orongsch{eine Olive}  ;  \orongsch{im Martini}
  \item \blau{den Tathergang in Begleitung ihres Anwalts}  ;  \orongsch{den Tathergang}  ;  \orongsch{in Begleitung ihres Anwalts}
  \item \blau{das Problem mit dem Taschenrechner}  ;  \orongsch{das Problem}  ;  \orongsch{mit dem Taschenrechner}
\end{itemize}
}

\Halbzeile

\Sol{\textbf{Hinweise zu den Lesarten:}
\begin{itemize}
  \item[(\ref{ex:schuhe})] Entweder trägt Sandy keine Schuhe, oder der Mann trägt keine.
  \item[(\ref{ex:a9})] Es werden entweder die LKWs und die Autos oder nur die LKWs als groß bezeichnet.
  \item[(\ref{ex:rahel})] Entweder befand sich Rahel im Martini bei der Bestellung der Olive, oder sie hat einen Martini mit Olive bestellt.
  \item[(\ref{ex:tathergang})] Entweder fand die Niederschrift oder der Tathergang in Begleitung des Anwalts statt.
  \item[(\ref{ex:problem})] Entweder war der Taschenrechner das Mittel zur Problemlösung, oder er hat das Problem verursacht.
\end{itemize}
}

\subsection{Strukturelle Ambiguität und Satzglieder}

Man kann die gegebenen Beispiele durch Umstellungen desambiguieren (= eindeutig machen)?
Welche Umstellungen sind das?

\Halbzeile

\Sol{\textbf{Anmerkungen:} In (\ref{ex:schuhe}) und (\ref{ex:rahel})--(\ref{ex:problem}) sind die blau unterstrichenen Konstituenten Satzglieder. Wenn man sie insgesamt ins Vorfeld stellt, gibt es nur noch die blaue Lesart. Für (\ref{ex:a9}) gilt: Wenn wir die koordinierten kleineren Konstituenten umdrehen, entfällt die blaue Lesart, bei der nur die N-Köpfe koordiniert wurden: \textit{Auf der A9 waren wieder Autos und große LKWs unterwegs.}}

\Sol{\newpage\hspace{1em}}

