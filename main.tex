\def\GRAPHPATH{localgraphics}

\ifdefined\HANDOUT
  \documentclass[handout,aspectratio=1610,dvipsnames]{beamer}
  \def\GRAPHPATH{graphics}
\else
  \documentclass[aspectratio=1610,dvipsnames]{beamer}
\fi

\usepackage[ngerman]{babel}
\usepackage{ifthen}
\usepackage{color}
\usepackage{colortbl}
\usepackage{textcomp}
\usepackage{multirow}
\usepackage{nicefrac}
\usepackage{multicol}
\usepackage{langsci-gb4e}
\usepackage{verbatim}
\usepackage{cancel}
\usepackage{graphicx}
\usepackage{hyperref}
\usepackage{verbatim}
\usepackage{boxedminipage}
\usepackage{adjustbox}
\usepackage{rotating}
\usepackage{booktabs}
\usepackage{bbding}
\usepackage{pifont}
\usepackage{multicol}
\usepackage{stmaryrd}
\usepackage{FiraSans}
\usepackage{soul}
\usepackage{tikz}
\usepackage{array}
\usepackage{xstring}

\usetikzlibrary{calc,decorations.pathmorphing,tikzmark,positioning,chains,trees,graphs,shapes,shadows,arrows}

\renewcommand\tikzmark[2]{%
  \tikz[remember picture,baseline=(chain-1.base),start chain] \node[on chain,inner sep=2pt,outer sep=0] (#1){#2};%
}

\newcommand\link[2]{%
  \begin{tikzpicture}[remember picture, overlay, >=stealth, shift={(0,0)}]
    \draw[-] (#1) --++(0,-12pt) -| (#2);
   \end{tikzpicture}%
}


\makeatletter
\g@addto@macro{\endtabular}{\rowfont{}}% Clear row font
\makeatother
\newcommand{\rowfonttype}{}% Current row font
\newcommand{\rowfont}[1]{% Set current row font
   \gdef\rowfonttype{#1}#1%
}
\newcolumntype{L}{>{\rowfonttype}l}


\usepackage{tikz-qtree}
\usepackage[linguistics]{forest}
\usepackage[maxbibnames=99,
  maxcitenames=2,
  uniquelist=false,
  backend=biber,
  doi=false,
  url=false,
  isbn=false,
  bibstyle=biblatex-sp-unified,
  citestyle=sp-authoryear-comp]{biblatex}

% Biblatex ============================================================

\addbibresource{rs.bib}

% Colors ==============================================================

% \ifdefined\HANDOUT
  \definecolor{grau}{rgb}{0.5,0.5,0.5}
  \definecolor{lg}{rgb}{0.8,0.8,0.8}
  \definecolor{trueblue}{rgb}{0.3,0.3,1}
  \definecolor{ltb}{rgb}{0.8,0.8,1}
  \definecolor{lgr}{rgb}{0.5,1,0.5}
  \definecolor{orongsch}{RGB}{255,165,0}
  \definecolor{gruen}{rgb}{0,0.4,0}
  \definecolor{rot}{rgb}{0.7,0.2,0.0}
  \definecolor{tuerkis}{RGB}{63,136,143}
  \definecolor{braun}{RGB}{108,71,65}
  \definecolor{blaw}{rgb}{0,0,0.9}
% \else
%   \definecolor{grau}{rgb}{0.7,0.7,0.7}
%   \definecolor{lg}{rgb}{0.9,0.9,0.9}
%   \definecolor{trueblue}{rgb}{0.8,0.8,1}
%   \definecolor{ltb}{rgb}{0.9,0.9,1}
%   \definecolor{lgr}{rgb}{0.7,1,0.7}
%   \definecolor{orongsch}{RGB}{255,200,100}
%   \definecolor{gruen}{RGB}{0,230,0}
%   \definecolor{rot}{RGB}{255,155,100}
%   \definecolor{tuerkis}{RGB}{150,205,205}
%   \definecolor{braun}{RGB}{140,120,115}
%   \definecolor{blaw}{rgb}{0,0,0.9}
% \fi

\newcommand{\gruen}[1]{\textcolor{gruen}{#1}}
\newcommand{\blaw}[1]{\textcolor{blaw}{#1}}
\newcommand{\rot}[1]{\textcolor{rot}{#1}}
\newcommand{\blau}[1]{\textcolor{trueblue}{#1}}
\newcommand{\orongsch}[1]{\textcolor{orongsch}{#1}}
\newcommand{\grau}[1]{\textcolor{grau}{#1}}
\newcommand{\whyte}[1]{\textcolor{white}{#1}}
\newcommand{\tuerkis}[1]{\textcolor{tuerkis}{#1}}
\newcommand{\braun}[1]{\textcolor{braun}{#1}}

% Newcommands =========================================================

\newcommand{\Dim}{\cellcolor{lg}}
\newcommand{\Dimblue}{\cellcolor{ltb}}
\newcommand{\Dimgreen}{\cellcolor{lgr}}
\newcommand{\Sub}[1]{\ensuremath{_{\text{#1}}}}
\newcommand{\Up}[1]{\ensuremath{^{\text{#1}}}}
\newcommand{\UpSub}[2]{\ensuremath{^{\text{#1}}_{\text{#2}}}}
\newcommand{\Spur}[1]{t\Sub{#1}}
\newcommand{\Ti}{\Spur{1}}
\newcommand{\Tii}{\Spur{2}}
\newcommand{\Tiii}{\Spur{3}}
\newcommand{\Tiv}{\Spur{4}}
\newcommand{\Ck}{\CheckmarkBold}
\newcommand{\Fl}{\XSolidBrush}
\newcommand{\xxx}{\hspaceThis{[}}
\newcommand{\zB}{z.\,B.\ }
\newcommand{\down}[1]{\ensuremath{\mathrm{#1}}}
\newcommand{\Doppelzeile}{\vspace{2\baselineskip}}
\newcommand{\Zeile}{\vspace{\baselineskip}}
\newcommand{\Halbzeile}{\vspace{0.5\baselineskip}}
\newcommand{\Viertelzeile}{\vspace{0.25\baselineskip}}
\newcommand{\KTArr}[1]{\ding{222}~\fbox{#1}~\ding{222}}
\newcommand{\Ast}{*}
\newcommand{\SL}{\ensuremath{\llbracket}}
\newcommand{\SR}{\ensuremath{\rrbracket}}
\def\lspbottomrule{\bottomrule}
\def\lsptoprule{\toprule}
\newcommand{\Sw}[1]{\begin{sideways}#1\end{sideways}}
\newcommand{\Lab}{\ensuremath{\langle}}
\newcommand{\Rab}{\ensuremath{\rangle}}
\newcommand{\AbUmlautBreaker}{}
\ifdefined\HANDOUT
  \renewcommand{\AbUmlautBreaker}{\ /}
\fi
\newcommand{\LocStrutGrph}{\hspace{0.1\textwidth}}
\newcommand{\Nono}{---}

\newcommand{\Bewegtes}[1]{\ensuremath{_{\textrm{#1}}}}
\newcommand{\ORi}{\Bewegtes{1}}
\newcommand{\ORii}{\Bewegtes{2}}
\newcommand{\ORiii}{\Bewegtes{3}}
\newcommand{\ORiv}{\Bewegtes{4}}
\newcommand{\ORv}{\Bewegtes{5}}


% Beamer ==============================================================

\usetheme[hideothersubsections]{Boadilla}

% \ifdefined\HANDOUT
  \usecolortheme{whale}
% \else
%   \usecolortheme{magpie}
% \fi

\renewcommand<>{\rot}[1]{%
  \alt#2{\beameroriginal{\rot}{#1}}{#1}%
}
\renewcommand<>{\blau}[1]{%
  \alt#2{\beameroriginal{\blau}{#1}}{#1}%
}
\renewcommand<>{\orongsch}[1]{%
  \alt#2{\beameroriginal{\orongsch}{#1}}{#1}%
}
\renewcommand<>{\gruen}[1]{%
  \alt#2{\beameroriginal{\gruen}{#1}}{#1}%
}

\setbeamercolor{alerted text}{fg=trueblue}

\newcounter{lastpagemainpart}

\resetcounteronoverlays{exx}

\AtBeginSection[]{
  \begingroup
  \setbeamertemplate{navigation symbols}{}
  \begin{frame}[noframenumbering,plain]
  \vfill
  \centering
  \begin{beamercolorbox}[sep=8pt,center,shadow=true,rounded=true]{title}
    \usebeamerfont{title}\insertsectionhead\par%
  \end{beamercolorbox}
  \vfill
  \end{frame}
  \endgroup
}

\setbeamertemplate{itemize item}[circle]
\setbeamertemplate{enumerate item}[square]


\makeatother
\setbeamertemplate{footline}
{
  \leavevmode%
  \hbox{%
  \begin{beamercolorbox}[wd=.4\paperwidth,ht=2.25ex,dp=1ex,center]{author in head/foot}%
    \usebeamerfont{author in head/foot}\insertshortauthor
  \end{beamercolorbox}%
  \begin{beamercolorbox}[wd=.6\paperwidth,ht=2.25ex,dp=1ex,center]{title in head/foot}%
    \usebeamerfont{title in head/foot}\insertshorttitle\hspace*{3em}
    \insertframenumber{} / \inserttotalframenumber\hspace*{1ex}
  \end{beamercolorbox}}%
  \vskip0pt%
}
\makeatletter
\setbeamertemplate{navigation symbols}{}


% Tikz ================================================================

\usetikzlibrary{positioning,arrows,cd}
\tikzset{>=latex}

% Forest

\forestset{
  Ephr/.style={draw, ellipse, thick, inner sep=2pt},
  Eobl/.style={draw, rounded corners, inner sep=5pt},
  Eopt/.style={draw, rounded corners, densely dashed, inner sep=5pt},
  Erec/.style={draw, rounded corners, double, inner sep=5pt},
  Eoptrec/.style={draw, rounded corners, densely dashed, double, inner sep=5pt},
  Ehd/.style={rounded corners, fill=gray, inner sep=5pt,
    delay={content=\whyte{##1}}
  },
  Emult/.style={for children={no edge}, for tree={l sep=0pt}},
  phrasenschema/.style={for tree={l sep=2em, s sep=2em}},
  decide/.style={draw, chamfered rectangle, inner sep=2pt},
  finall/.style={rounded corners, fill=gray, text=white},
  intrme/.style={draw, rounded corners},
  yes/.style={edge label={node[near end, above, sloped, font=\scriptsize]{Ja}}},
  no/.style={edge label={node[near end, above, sloped, font=\scriptsize]{Nein}}},
  sake/.style={tier=preterminal},
  ake/.style={
    tier=preterminal
    },
}

\tikzset{
    invisible/.style={opacity=0,text opacity=0},
    visible on/.style={alt=#1{}{invisible}},
    alt/.code args={<#1>#2#3}{%
      \alt<#1>{\pgfkeysalso{#2}}{\pgfkeysalso{#3}}
    },
}

\forestset{
  visible on/.style={
    for tree={
      /tikz/visible on={#1},
      edge+={/tikz/visible on={#1}}}}}

\useforestlibrary{edges}

\forestset{
  narroof/.style={roof, inner xsep=-0.25em, rounded corners},
  forky/.style={forked edge, fork sep-=7.5pt},
  bluetree/.style={for tree={trueblue}, for children={edge=trueblue}},
  orongschtree/.style={for tree={orongsch}, for children={edge=orongsch}},
  rottree/.style={for tree={rot}, for children={edge=rot}},
  gruentree/.style={for tree={gruen}, for children={edge=gruen}},
  tuerkistree/.style={for tree={tuerkis}, for children={edge=tuerkis}},
  brauntree/.style={for tree={braun}, for children={edge=braun}},
  grautree/.style={for tree={grau}, for children={edge=grau}}, 
  gruennode/.style={gruen, edge=gruen},
  graunode/.style={grau, edge=grau},
}


% Drawing sonority diagrams =========================================== 

\makeatletter

\long\def\ifnodedefined#1#2#3{%
  \@ifundefined{pgf@sh@ns@#1}{#3}{#2}}

\newcommand\aeundefinenode[1]{%
  \expandafter\ifx\csname pgf@sh@ns@#1\endcsname\relax
  \else
    \typeout{Undefining node "#1"}%
    \global\expandafter\let\csname pgf@sh@ns@#1\endcsname\relax
  \fi
}

\newcommand\aeundefinethesenodes[1]{%
  \foreach \myn  in {#1}
    {%
      \ifnodedefined{\myn}{%
      \expandafter\aeundefinenode\expandafter{\myn}%
    }{}
    }%
}

\newcommand\aeundefinenumericnodes{%
  \foreach \myn in {1,2,...,50}
    {%
      \ifnodedefined{\myn}{%
      \expandafter\aeundefinenode\expandafter{\myn}%
    }{}
    }%
}
\makeatother

\newcommand{\plo}{0}
\newcommand{\fri}{0.5}
\newcommand{\nas}{1}
\newcommand{\liq}{1.5}
\newcommand{\vok}{2}

% Save text.
\newcommand{\lastsaved}{}
\newcommand{\textsave}[1]{\gdef\lastsaved{#1}#1}

\newcommand{\SonDiag}[2][0]{%
  \begin{tikzpicture}
    \textsave{.}
    \tikzset{
      normalseg/.style={fill=white},
      extrasyll/.style={circle, draw, fill=white},
      sylljoint/.style={diamond, draw, fill=white}
    }
    \node at (0,\plo) {P};
    \node at (0,\fri) {F};
    \node at (0,\nas) {N};
    \node at (0,\liq) {L};
    \node at (0,\vok) {V};

    % Draw the helper lines if required.
    \ifthenelse{\equal{#1}{0}}{}{%
      \foreach \y in {\plo, \fri, \nas, \liq,\vok} {%
	\draw [dotted, |-|] (0.25, \y) -- (#1.75, \y);
      }
    }

    \foreach [count=\x from 1, remember=\x as \lastx] \p / \y / \g in #2 {
      \ifthenelse{\equal{\y}{-1}}{\textsave{.}}{%

	% Draw the node, either plain, as Silbenbgelenk, or as extrasyllabic.
        \ifthenelse{\equal{\g}{1}}{%
	  \node (\x) [sylljoint] at (\x, \y) {\p};
	}{%
	  \ifthenelse{\equal{\g}{2}}{%
	    \node (\x) [extrasyll] at (\x, \y) {\p};
	  }{%
	    \node (\x) [normalseg] at (\x, \y) {\p};
	  }
	}

	% Draw the connection unless the previous node was not or was empty.
	\ifthenelse{\NOT\equal{\lastsaved}{.}}{%
	  \draw [->] (\lastx) to (\x);
	}{}
	\textsave{1}
      }
    }
    \aeundefinenumericnodes
  \end{tikzpicture}
}


\setbeamertemplate{navigation symbols}{}
\setbeamertemplate{section in toc}[circle]
\setbeamertemplate{subsection in toc}[square]
\setbeamertemplate{subsubsection in toc}[triangle]

\setbeamerfont{section in toc}{size=\tiny}
\setbeamerfont{subsection in toc}{size=\tiny}
\setbeamerfont{subsubsection in toc}{size=\tiny}




\ifdefined\TITLE
  \title[Syntax | \StrSubstitute{\TITLE}{+}{ }]{Deutsche Syntax\\\StrSubstitute{\TITLE}{+}{ }}
\else
  \title[Deutsche Syntax]{Deutsche Syntax}
\fi

\author{Roland Schäfer}
\institute[FSU Jena]{Institut für Germanistische Sprachwissenschaft\\Friedrich-Schiller-Universität Jena}
\date[EGBD3]{\grau{stets aktuelle Fassungen: \url{https://github.com/rsling/VL-Deutsche-Syntax}}}

\begin{document}

\begingroup
  \setbeamertemplate{navigation symbols}{}
  \begin{frame}[noframenumbering,plain]
   \titlepage
  \end{frame}

  \ifdefined\TITLE
    \begin{frame}[noframenumbering,plain]
      \centering 
      \begin{minipage}[c]{0.975\textwidth}
      \begin{block}
        {\rot{Hinweise für diejenigen, die die Klausur bestehen möchten}}
        \begin{enumerate}
          \item Folien sind niemals selbsterklärend und nicht zum Selbststudium geeignet.\\
            Sie müssen sich die Videos ansehen und regelmäßig das Seminar besuchen.
          \item Ohne eine gründliche Lektüre der angegebenen Abschnitte des Buchs\\
            bestehen Sie die Klausur nicht.
            Das Buch definiert den Klausurstoff.
          \item Arbeiten Sie die entsprechenden Übungen im Buch durch.
            Nichts hilft\\
            Ihnen besser, um sich auf die Klausur vorzubereiten.
          \item \rot{Beginnen Sie spätestens jetzt mit dem Lernen.}
            \Zeile
          \item \rot{Langjähriger Erfahrungswert:
            Wenn Sie diese Hinweise nicht berücksichtigen, bestehen Sie die Klausur wahrscheinlich nicht.}
        \end{enumerate}
      \end{block}
      \end{minipage}
    \end{frame}
  \else
  \begin{frame}{Inhalt}
    \centering 
    \scalebox{0.9}{\begin{minipage}{\textwidth}
      \begin{multicols}{2}
        \tableofcontents
      \end{multicols}
    \end{minipage}}
    \end{frame}
  \fi
\endgroup

\ifdefined\TITLE
  \input{includes/\TITLE}
\else
  \section[Grammatik]{Grammarik}
  \let\woopsi\section\let\section\subsection\let\subsection\subsubsection
  \section{Wiederholungsstoff}

\begin{frame}
  {Wiederholen Sie die erste Vorlesung \orongsch{Morphologie}}
  \Zeile
  \centering 
  Bitte schauen Sie sich die erste Woche aus meiner Vorlesung\\
  zur Morphologie nochmal an.\\
  \Zeile
  Die Inhalte sind auch in der Syntax Klausurstoff.\\
\end{frame}

\section{Zur nächsten Woche | Überblick}

\begin{frame}
  {Deutsche Syntax | Plan}
  \rot{Alle} angegebenen Kapitel\slash Abschnitte aus \rot{\citet{Schaefer2018b}} sind \rot{Klausurstoff}!\\
  \Halbzeile
  \begin{enumerate}
    \item Grammatik und Grammatik im Lehramt \rot{(Kapitel 1 und 3)}
    \item Grundbegriffe \rot{(Kapitel 2)}
    \item Wortklassen \rot{(Kapitel 6)}
    \item Konstituenten und Satzglieder \rot{(Kapitel 11 und Abschnitt 12.1)}
    \item Nominalphrasen \rot{(Abschnitt 12.3)}
    \item Andere Phrasen \rot{(Abschnitte 12.2 und 12.4--12.7)}
    \item Verbphrasen und Verbkomplex \rot{(Abschnitte 12.8)}
    \item Sätze \rot{(Abschnitte 12.9 und 13.1--13.3)} 
    \item Nebensätze \rot{(Abschnitt 13.4)}
    \item Subjekte und Prädikate \rot{(Abschnitte 14.1--14.3)}
    \item Passive und Objekte \rot{(14.4 und 14.5)}
    \item Syntax infiniter Verbformen \rot{(Abschnitte 14.7--14.9)}
  \end{enumerate}
  \Halbzeile
  \centering 
  \url{https://langsci-press.org/catalog/book/224}
\end{frame}


  \let\subsection\section\let\section\woopsi

  \section[Grundbegriffe]{Grundbegriffe}
  \let\woopsi\section\let\section\subsection\let\subsection\subsubsection
  \section{Überblick}

\begin{frame}
  {Überblick}
  \onslide<+->
  \onslide<+->
  \begin{itemize}[<+->]
    \item \alert{Strukturbildung} | große Einheiten aus kleinen Einheiten
    \Zeile
    \item \alert{Relationen} | Kongruenz und Valenz
    \Zeile
    \item \alert{Valenz} | Verbklassen und Ereignisbeschreibung
  \end{itemize}
\end{frame}

\section{Struktur}

\begin{frame}
  {Auf allen Ebenen | Struktur}
  \onslide<+->
  \onslide<+->
  \alert{Struktur} | Einheiten sind aus Einheiten zusammengesetzt.
  \begin{exe}
    \ex\label{ex:strukturbildung021}
    \begin{xlist}
      \ex \textbf{Sätze} \\
      {\alert{[}Alexandra schießt den Ball ins gegnerische Tor.\alert{]}\ }
      \onslide<+->
      \ex \textbf{Satzteile} \\
      {\alert{[}Alexandra\alert{]}\  \alert{[}schießt\alert{]}\  \alert{[}den Ball\alert{]}\  \alert{[}ins gegnerische Tor\alert{]}\ }
      \onslide<+->
      \ex \textbf{Wörter} \\
      {\alert{[}Alexandra\alert{]}\  \alert{[}schießt\alert{]}\  \alert{[}den\alert{]}\  \alert{[}Ball\alert{]}\  \alert{[}ins\alert{]}\  \alert{[}gegnerische\alert{]}\  \alert{[}Tor\alert{]}\ }
      \onslide<+->
      \ex \textbf{Wortteile} \\
      {\alert{[}Alexandra\alert{]}\  \alert{[}schieß\alert{]}\ \alert{[}t\alert{]}\  \alert{[}den\alert{]}\  \alert{[}Ball\alert{]}\  \alert{[}ins\alert{]}\  \alert{[}gegner\alert{]}\ \alert{[}isch\alert{]}\ \alert{[}e\alert{]}\  \alert{[}Tor\alert{]}\ }
      \onslide<+-> 
      \ex \textbf{Laute} \\
      {\alert{[}A\alert{]}\ \alert{[}l\alert{]}\ \alert{[}e\alert{]}\ \alert{[}k\alert{]}\ \alert{[}s\alert{]}\ \alert{[}a\alert{]}\ \alert{[}n\alert{]}\ \alert{[}d\alert{]}\ \alert{[}r\alert{]}\ \alert{[}a\alert{]}\  \ldots \\}
    \end{xlist}
  \end{exe}
\end{frame}

\begin{frame}
  {Struktur in der Syntax}
  \onslide<+->
  \onslide<+->
  Durch mehrfache \alert{Aneinanderfügung} ergeben sich \alert{hirarchische Strukturen}.\\
  \Zeile
  \onslide<+->
  \centering 
    \begin{forest}
    [Alexandra schießt den Ball ins gegnerische Tor
      [Alexandra [Alexandra]]
      [schießt [schießt]]
      [den Ball
        [den]
        [Ball]
      ]
      [ins gegenerische Tor
        [ins]
        [gegnerische]
        [Tor]
      ]
    ]
  \end{forest}
\end{frame}

\begin{frame}
  {Struktur in der Morphologie}
  \onslide<+->
  \onslide<+->
  Auch innerhalb von Wörtern gibt es solche Strukturen.\\
  \Zeile
  \onslide<+->
  \centering 
    \begin{forest}
    [gegnerische
      [generisch
        [gegner]
        [isch, tier=terminal]
      ]
      [e, tier=terminal]
    ]
  \end{forest}
\end{frame}

\begin{frame}
  {Konstituenten}
  \onslide<+->
  \onslide<+->
  \centering 
  \begin{block}
    {Konstituenten einer Struktur}
    \textit{Konstituenten} einer Einheit sind die (meistens kleineren und höchstens genauso großen) Einheiten, aus denen eine Struktur besteht.    
  \end{block}
\end{frame}

\section{Rektion und Kongruenz}

\begin{frame}
  {Was sind Relationen?}
  \onslide<+->
  \onslide<+->
  \begin{exe}
    \ex\label{ex:rektionundkongruenz024}
    \begin{xlist}
      \ex{\label{ex:rektionundkongruenz025}[Dzsenifer] \alert{[schießt] } \orongsch{[ein Tor]}.}
      \ex{\label{ex:rektionundkongruenz026}[Kim] \alert{[läuft]} \gruen{[schnell]}.}
    \end{xlist}
  \end{exe}
  \onslide<+->
  \Zeile
  \begin{itemize}[<+->]
    \item \orongsch{\textit{ein Tor}} ist ein \orongsch{Objekt} zu \textit{schießt}.
    \item \gruen{\textit{schnell}} ist eine \gruen{adverbiale Bestimmung} zu \textit{läuft}.
      \Zeile
    \item Es gibt kein Objekt und keine adverbiale Bestimmung\\
      ohne ein Verb im Satzkontext \ldots
    \item die Begriffe \orongsch{Objekt} und \gruen{adverbiale Bestimmung} sind also \alert{relational}.
  \end{itemize}
\end{frame}

\begin{frame}
  \onslide<+->
  \onslide<+->
  {Rektion}
  Wörter (oder andere Einheiten) \alert{bestimmen}\\
  Eigenschaften anderer Wörter.\\
  \onslide<+->
  \Zeile
  \begin{exe}
    \ex[ ]{\gruen{Der Torwart} \alert{gedenkt} \orongsch{der Niederlage.}}
    \ex[ ]{\gruen{Der FCR Duisburg} \alert{besiegt} \orongsch{den FFC Frankfurt}.}
  \end{exe}
  \onslide<+->
  \Halbzeile
  \begin{exe}
    \ex
    \begin{xlist}
      \ex[*]{\rot{Den Torwart} \alert{gedenkt} \rot{die Niederlage}.}
    \ex{\rot{Des FCR Duisburgs} \alert{besiegt} \rot{dem FFC Frankfurt}.}
    \end{xlist}
  \end{exe}
\end{frame}

\begin{frame}
  {Kongruenz}
  \onslide<+->
  \onslide<+->
  Zwei oder mehr Wörter (bzw.\ Einheiten)\\
  \alert{stimmen in bestimmten Eigenschaften überein}\\
  \onslide<+->
  \Zeile
  \begin{exe}
    \ex
    \begin{xlist}
      \ex{Der FCR besiegt \gruen{den} \alert{gegnerischen Verein}.}
      \ex{Der FCR besiegt \alert{alle} \gruen{gegnerischen} \alert{Vereine}.}
    \end{xlist}
  \end{exe}
  \onslide<+->
  \Halbzeile
  \begin{exe}
    \ex
    \begin{xlist}
      \ex[*]{Der FCR besiegt \rot{die} \alert{gegnerischen Verein}.}
      \ex[*]{Der FCR besiegt \alert{alle} \rot{gegnerischer} \alert{Verein}.}
    \end{xlist}
  \end{exe}
\end{frame}

\begin{frame}
  {Wichtige Kongruenzrelationen im Deutschen}
  \onslide<+->
  \onslide<+->
  \alert{Subjekt-Verb-Kongruenz}\\
  \Viertelzeile
  Das Subjekt und das finite Verb eines Satzes stimmen in \alert{Person} und \alert{Numerus} überein.\\
  \Zeile
  \onslide<+->
  \Zeile
  \alert{Kongruenz in der Nominalgruppe}\\
  \Viertelzeile
  Zusammenstehende und zusammengehörige Artikelwörter, Adjektive und Substantive (in dieser Reihenfolge) stimmen in \alert{Numerus}, \alert{Genus} und \alert{Kasus} überein.
\end{frame}

\section{Valenz}

\begin{frame}
  {Traditionelle Verbtypen}
  \pause
  \begin{itemize}[<+->]
    \item traditionelle Termini für Verbtypen (s.\ Kapitel 14 für Neuordnung)
      \Halbzeile
      \begin{itemize}[<+->]
        \item \alert{intransitiv}: regiert nur einen Nominativ (\textit{leben}, \textit{schlafen})
          \Viertelzeile
        \item \alert{transitiv}: regiert einen Nominativ und einen Akkusativ (\textit{sehen}, \textit{lesen})
          \Viertelzeile
        \item \alert{ditransitiv}: regiert zusätzlich einen Dativ (\textit{geben}, \textit{schicken})
          \Viertelzeile
        \item \alert{präpositional transitiv}: regiert Nom und PP (\textit{leiden +unter})
          \Viertelzeile
        \item \alert{präpositional ditransitiv}: regiert Nom, Akk, PP (\textit{schreiben +an})
          \Viertelzeile
        \item \ldots
          \Zeile
      \end{itemize}
    \item nur Abkürzungen für einige (von sehr viel mehr) \alert{Valenztypen}
  \end{itemize}
\end{frame}

\begin{frame}
  {Ergänzungen und Angaben}
  \pause
  Siehe auch: Kapitel~2, Abschnitt~2.4 (S.~40--48)!\\
  \pause\Halbzeile
  \begin{exe}
    \ex\label{ex:valenz034}
    \begin{xlist}
      \ex{Gabriele malt \alert{[ein Bild]}.}
      \pause
      \ex{Gabriele malt \gruen{[gerne]}.}
      \pause
      \ex{Gabriele malt \gruen{[den ganzen Tag]}.}
      \pause
      \ex{Gabriele malt \gruen{[ihrem Mann]} \rot{[zu figürlich]}.}
    \end{xlist}
  \end{exe}
  \pause\Halbzeile
  \begin{itemize}[<+->]
    \item \alert{[ein Bild]} mit besonderer Relation zum Verb
    \item "`Weglassbarkeit"' (Optionalität) nicht entscheidend
  \end{itemize}
\end{frame}

\begin{frame}
  {Lizenzierung}
  \pause
  \begin{exe}
    \ex 
    \begin{xlist}
      \ex[ ]{Gabriele isst \gruen{[den ganzen Tag]} Walnüsse.}
    \pause
      \ex[ ]{Gabriele läuft \gruen{[den ganzen Tag]}.}
      \pause
      \ex[ ]{Gabriele backt ihrer Schwester \gruen{[den ganzen Tag]} Stollen.}
      \pause
      \ex[ ]{Gabriele litt \gruen{[den ganzen Tag]} unter Sonnenbrand.}
    \end{xlist}
    \pause\Halbzeile
    \ex 
    \begin{xlist}
      \ex[*]{Gabriele isst \alert{[ein Bild]} Walnüsse.}
      \pause
      \ex[*]{Gabriele läuft \alert{[ein Bild]}.}
      \pause
      \ex[*]{Gabriele backt ihrer Schwester \alert{[ein Bild]} Stollen.}
      \pause
      \ex[*]{Gabriele litt \alert{[ein Bild]} unter Sonnenbrand. }
      \pause
    \end{xlist}
  \end{exe}
  \pause\Halbzeile
  \begin{itemize}[<+->]
    \item \gruen{Angaben} sind verb-unspezifisch lizenziert
    \item \alert{Ergänzungen} sind verb(klassen)spezifisch \alert{genau einmal} lizenziert
    \item \rot{Valenz = Liste der Ergänzungen eines lexikalischen Worts}
  \end{itemize}
\end{frame}


\begin{frame}
  {Iterierbarkeit | Angaben sind beliebig stapelbar}
  \onslide<+->
  \onslide<+->
  \begin{exe}
    \ex[ ]{Wir müssen den Wagen\\
      \gruen{[jetzt]}\\
      \gruen{[mit aller Kraft]}\\
      \gruen{[vorsichtig]} anschieben.}
    \onslide<+->
    \ex[ ]{Wir essen \gruen{[schnell]}\\
    \gruen{[mit Appetit]}\\
    \gruen{[an einem Tisch]}\\
    \gruen{[mit der Gabel]}\\
    \alert{[einen Salat]}.}
    \onslide<+->
    \ex[*]{Wir essen \gruen{[schnell]}\\
    \rot{[ein Tofugericht]}\\
    \gruen{[mit Appetit]}\\
    \gruen{[an einem Tisch]}\\
    \gruen{[mit der Gabel]}\\
    \alert{[einen Salat]}.}
  \end{exe}
\end{frame}

\begin{frame}
  {Ergänzungen | Schnittstelle von Syntax und Semantik}
  \onslide<+->
  \onslide<+->
  Verbsemantik | Welche \alert{Rolle} spielen die von den Satzgliedern bezeichneten Dinge in der vom Verb beschriebenen Situation?\\
  \Zeile
  \onslide<+->
  Semantik von \alert{Ergänzungen} | \alert{abhängig} vom Verb\\
  \onslide<+->
  \Viertelzeile
  Semantik von \gruen{Angaben} | \gruen{unabhängig} vom Verb\\
  \Halbzeile
  \pause
  \begin{exe}
    \ex\label{ex:valenz071}
    \begin{xlist}
      \ex{\label{ex:valenz072}Ich lösche \alert{[den Ordner]} \gruen{[während der Hausdurchsuchung]}.}
      \pause
      \ex{\label{ex:valenz073}Ich mähe \alert{[den Rasen]} \gruen{[während der Ferien]}.}
      \pause
      \ex{\label{ex:valenz074}Ich fürchte \alert{[den Sturm]} \gruen{[während des Sommers]}.}
    \end{xlist}
  \end{exe}
\end{frame}

\begin{frame}
  {Valenz | Zusammenfassung}
  \onslide<+->
  \onslide<+->
  \alert{Angaben} sind grammatisch immer lizenziert\\
  und bringen ihre eigene semantische Rolle mit.\\
  \grau{Sie können aber semantisch\slash pragmatisch inkompatibel sein.}\\
  \Zeile
  \onslide<+->
  \Zeile
  \onslide<+->
  \gruen{Ergänzungen} werden spezifisch vom Verb lizenziert\\
  und in ihrer semantischen Rolle vom Verb festgelegt.\\
  Jede dieser Rollen kann nur einmal vergeben werden.
\end{frame}

\section{Ausblick}

\begin{frame}
  {Nächste Woche | Wortklassen}
  \begin{itemize}[<+->]
    \item Möglichkeiten, Wortklassen zu definieren
      \Halbzeile
    \item syntaktisch definierte Wortklassen
      \Halbzeile
    \item \citet[Kapitel~6]{Schaefer2018b}
  \end{itemize}
\end{frame}

  \let\subsection\section\let\section\woopsi

  \section[Wortklassen]{Wortklassen}
  \let\woopsi\section\let\section\subsection\let\subsection\subsubsection
  \section{Überblick}

\begin{frame}
  {Nächste Woche | Wortklassen}
  \onslide<+->
  \begin{itemize}[<+->]
    \item Was sind Wörter?
      \Halbzeile 
    \item Möglichkeiten, Wortklassen zu definieren
      \Halbzeile
    \item syntaktisch definierte Wortklassen
  \end{itemize}
\end{frame}

\section{Wörter}


\begin{frame}
  {Ebenen und Einheiten}
  \pause
  Kombinatorik von Wortbestandteilen und von Wörtern:
  \pause
  \Zeile
%  \begin{itemize}[<+->]
%    \item Wortakzent: \textit{\alert{\textbf{Sie}}ges\alert{säu}le}\\
%      $\rightarrow$ phonologisches\slash prosodisches Wort
%      \Zeile
%    \item Eigenschaften von Wörtern jenseits der Phonologie?
%  \end{itemize}
%  \Zeile
%  \pause
  \begin{exe}
    \ex
    \begin{xlist}
      \ex[]{Staat-es}
      \pause
      \ex[*]{Tür-es}
    \end{xlist}
    \pause
    \Zeile
    \ex
    \begin{xlist}
      \ex[]{Der Satz ist eine grammatische Einheit.}
      \pause
      \ex[*]{Die Satz ist eine grammatische Einheit.}
    \end{xlist}
  \end{exe}
\end{frame}

\begin{frame}
  {Wörter haben eine Bedeutung?}
  \pause
  \begin{exe}
    \ex \alert{Es} \alert{wird} schon wieder früh dunkel.
    \pause
    \ex Kristine denkt, \alert{dass} \alert{es} bald regnen \alert{wird}.
    \pause
    \ex Adrianna \alert{hat} gestern \alert{den} Keller inspiziert.
    \pause
    \ex Camilla \alert{und} Emma sehen \alert{sich} \alert{die} Fotos \alert{an}.
  \end{exe}
  \Zeile
  \pause
  \large
  Bedeutungstragende Wörter und \alert{Funktionswörter}
\end{frame}

\begin{frame}
  {Morphologie und Syntax}
  \pause
  \begin{itemize}[<+->]
    \item Kombinatorik für \alert{Wortbestandteile}: Morphologie
      \begin{itemize}[<+->]
        \item Wortbestandteile \zB mit \alert{Umlaut}: \textit{rot} -- \textit{röter}
        \item oder \alert{Ablaut}: \textit{heben} -- \textit{hob}
      \end{itemize}
    \item Kombinatorik für \alert{Wörter}: Syntax
      \Zeile
    \item \alert{Zirkuläre oder leere Definitionen?}
    \item \rot{Nein!} Prinzip: eigene Regularität → eigene Struktur
      \Zeile
    \item Wortbestandteile \alert{nicht trennbar}:
      \begin{itemize}
        \item \textit{heb-t}\\
          *\textit{heb mit Mühe t}
        \item \textit{Ge-hob-en-heit} \\
          *\textit{Gehoben anspruchsvolle heit}
        \item \textit{Sie geht schnell heim.}\\
          \textit{Schnell geht sie heim.}
      \end{itemize}
  \end{itemize}
\end{frame}



\section{Syntaktische Wörter}

\begin{frame}
  {Wort und Wortform I}
  \pause
  \begin{exe}
    \ex
    \begin{xlist}
      \ex (der) Tisch
      \pause
      \ex (den) Tisch
      \pause
      \ex (dem) Tisch\alert{e}
      \pause
      \ex (des) Tisch\alert{es}
      \pause
      \ex (die) Tisch\alert{e}
      \pause
      \ex (den) Tisch\alert{en}
    \end{xlist}
  \end{exe}
  \pause
  \begin{exe}
    \ex
    \begin{xlist}
      \ex Der \_\_\_\ ist voll hässlich.
      \pause
      \ex Ich kaufe den \_\_\_ nicht.
      \pause
      \ex Wir speisten am \_\_\_\ des Bundespräsidenten.
      \pause
      \ex Der Preis des \_\_\_\ ist eine Unverschämtheit.
      \pause
      \ex Die \_\_\_\ kosten nur noch die Hälfte.
      \pause
      \ex Mit den \_\_\_\ können wir nichts mehr anfangen.
    \end{xlist}
  \end{exe}
\end{frame}

\begin{frame}
  {Wort und Wortform II}
  \pause
  \begin{block}{Wortform}
    Eine \alert{Wortform} ist eine in syntaktischen Strukturen auftretende und in diesen Strukturen nicht weiter zu unterteilende Einheit.
    [\ldots]
  \end{block}
  \Zeile
  \pause
  \begin{block}{Lexikalisches Wort}
    Das (\alert{lexikalische}) \alert{Wort} ist eine Repräsentation von lexikalisch (bedeutungsmäßig) zusammengehörigen Wortformen.
    [\ldots]
  \end{block}
\end{frame}

\begin{frame}
  {Syntaktisches Wort}
  \onslide<+->
  \onslide<+->
  Ein \alert{syntaktisches Wort} ist eine \alert{Wortform} im syntaktischen Kontext.\\
  \Zeile
  \onslide<+->
  Ein syntaktisches Wort ist immer \alert{für alle Merkmale spezifiziert},\\
  auch wenn man ihm (morphologisch) nicht die volle Spezifikation ansieht.\\
  \Zeile
  \onslide<+->
  \begin{exe}
    \ex \alert{Ein [Mitglied]\Sub{Nom Sg Neut}} widersprach dem Beschluss.
    \onslide<+->
    \ex Wir überzeugten \alert{ein [Mitglied]\Sub{Akk Sg Neut}}, dem Beschluss zuzustimmen.
  \end{exe}
\end{frame}

\section{Methode}

\begin{frame}
  {Klassische Grundschul-Wortarten}
  \onslide<+->
    \begin{itemize}[<+->]
      \item Dingwort
      \item Tuwort, Tätigkeitswort
      \item Wiewort, Eigenschaftswort
      \item Umstandswort
    \end{itemize}
    \onslide<+->
    \Zeile
    Überwiegend \alert{bedeutungsbasiert}!
\end{frame}

\begin{frame}
  {Ein paar neue Wortarten nach Bedeutungen I}
  \pause
  \begin{itemize}[<+->]
    \item \alert{Bewegungsverben}: \textit{laufen}, \textit{springen}, \textit{fahren}, \dots
    \item \alert{Zustandsverben}: \textit{duften}, \textit{wohnen}, \textit{liegen}, \dots
      \Halbzeile
    \item \alert{Konkreta}: \textit{Haus}, \textit{Buch}, \textit{Blume}, \textit{Stier}, \dots
    \item \alert{Abstrakta}: \textit{Konzept}, \textit{Glaube}, \textit{Wunder}, \textit{Kausalität}, \dots
      \Halbzeile
    \item \alert{Zählsubstantive}: \textit{Keks}, \textit{Student}, \textit{Mikrobe}, \textit{Kneipe}, \dots
    \item \alert{Stoffsubstantive}: \textit{Wasser}, \textit{Wein}, \textit{Zement}, \textit{Mehl}, \dots
  \end{itemize}
\end{frame}

\begin{frame}
  {Ein paar neue Wortarten nach Bedeutungen II}
  \pause
  Aber Moment mal\dots\\
  \pause
  \Zeile
  \begin{exe}
   \ex
   \begin{xlist}
     \ex[ ]{\alert{Wein} kann lecker sein.}
     \ex[ ]{\alert{Ein Keks kann} lecker sein.}
     \ex[*]{\rot{Keks} kann lecker sein.}
     \ex[ ]{\alert{Kekse können} lecker sein.}
   \end{xlist}
    \pause
    \ex
    \begin{xlist}
      \ex Johanna hätte gerne \alert{einen Keks}.
      \ex Johanna hätte gerne \alert{einen Wein}.
    \end{xlist}
  \end{exe}
  \pause
  \Zeile
  Es gibt hier durchaus auch \alert{formale} Unterschiede.
\end{frame}


\begin{frame}
  {Syntaktische Klassifikation}
  \pause
  \begin{exe}
    \ex
    \begin{xlist}
      \ex[]{Ronnie spielt schnell \alert{und} präzise.}
      \pause
      \ex[*]{Ronnie spielt schnell \alert{obwohl} präzise.}
      \pause
      \ex[]{Ronnie \alert{und} Mark spielen eine gute Saison.}
      \pause
      \ex[*]{Ronnie \alert{obwohl} Mark spielen eine gute Saison.}
    \end{xlist}
    \pause
    \Zeile
    \ex
    \begin{xlist}
      \ex[]{Ronnie spielt herausragend,\\
        \alert{obwohl} der Leistungsdruck hoch ist.}
      \pause
      \ex[*]{Ronnie spielt herausragend,\\
        \alert{und} der Leistungsdruck hoch ist.}
    \end{xlist}
  \end{exe}
    \pause
    \Zeile
    \centering 
    Alles nur Bedeutung?
\end{frame}

\begin{frame}
  {Syntaktische Klassifikation}
  \pause
  \begin{center}
    \Large Wörter lassen sich in Kategorien einordnen, je nachdem\\
    \alert{in welchen syntaktischen Kontexten sie auftreten}.
  \end{center}
  \Zeile
  \pause
  \begin{itemize}[<+->]
    \item Konjunktionen: zwischen zwei gleichartigen Satzteilen
    \item Komplementierer: am Anfang bestimmter Nebensätze
  \end{itemize}
\end{frame}



\begin{frame}[fragile]
  {Filter}
  \onslide<+->
  \onslide<+->
  Mittels syntaktischer Klassifikation können wir den rechten Arm des Wortklassenbaums aufbauen (nicht-flektierbare Wörter).\\
  \Zeile
  \centering 
  \hspace{0.25\textwidth}\scalebox{0.6}{
    \begin{minipage}{0.5\textwidth}  
      \centering 
    \begin{forest}
      /tikz/every node/.append style={font=\footnotesize},
      for tree={l sep=2em, s sep=2.5em},
      [\textit{Wort}, intrme, {visible on=<3->}, for children={visible on=<4->}
        [{Hat  Numerus?}, decide, for children={visible on=<5->}
          [\textit{flektierbar}, intrme, yes, {visible on=<5->}, for children={visible on=<7->}
            [{Ist finit  flektierbar?}, decide, {visible on=<7->}, for children={visible on=<8->}
              [\textbf{Verb}, finall, yes, {visible on=<8->}]
              [\textit{Nomen}, intrme, no, {visible on=<9->}]
            ]
          ]
          [\textit{nicht flektierbar}, intrme, no, {visible on=<6->}, for children={visible on=<10->}
            [{Hat Valenz-\slash  Kasusrektion?}, decide, {visible on=<10->}, for children={visible on=<11->}
              [\textbf{Präposition}, finall, yes, {visible on=<11->}]
              [\textit{andere}, intrme, no, {visible on=<12->}]
            ]
          ]
        ]
      ]
    \end{forest}
   \end{minipage}
   }
\end{frame}


\section{Wortklassen}

\begin{frame}
  {Präpositionen flektieren nicht und regieren Kasus}
  \pause
  \begin{exe}
    \ex
    \begin{xlist}
      \ex{\alert<3->{Mit} \rot<4->{dem kaputten Rasen} ist nichts mehr anzufangen.}
      \pause
      \pause
      \pause
      \ex{\alert<6->{Angesichts} \rot<7->{des kaputten Rasens} wurde das Spiel abgesagt.}
    \end{xlist}
  \end{exe}
  \pause
  \pause
  \pause
  \Zeile
  \begin{block}{Rektion}
    In einer Rektionsrelation werden durch die regierende Einheit (das \alert{Regens}) Werte für bestimmte Merkmale\slash Werte (und damit ggf.\ auch die Form) beim regierten Element (dem \alert{Rectum}) verlangt.\\
  \end{block}
  \Zeile
  \pause
  \begin{block}{Präposition}
    Präpositionen kasusregieren eine obligatorische Nominalphrase.
  \end{block}
\end{frame}

\begin{frame}
  {Komplementierer}
  \pause
  \begin{exe}
    \ex
    \begin{xlist}
      \ex[]{Ich glaube, [\alert<3->{dass} dieser Nebensatz ein Verb \alert<4->{enthält}].}
      \ex[]{[\alert<5->{Während} die Spielzeit \alert<6->{läuft}], zählt jedes Tor.}
      \ex[]{Es fällt ihnen schwer [\rot<7->{zu laufen}].}
      \ex[\rot<10->{*}]{[\alert<8->{Obwohl} kein Tor \alert<9->{fiel}].}
    \end{xlist}
  \end{exe}
  \Zeile
  \onslide<11->
  \begin{block}{Komplementierer}
    Komplementierer leiten Nebensätze ein.\\
    Die Rede von der \textit{unterordnenden Konjunktion} ist ungeschickt.
  \end{block}
\end{frame}

\begin{frame}
  {Nicht-flektierbare Wörter im "`Vorfeld"'}
  \pause
  Was steht im unabhängigen Aussagesatz am Satzanfang?\\
  \pause
  {\rot{Antworten Sie nie mehr mit "`das Subjekt"'!}}
  \pause
  \begin{exe}
    \ex\label{ex:adverbenadkopulasundpartikeln038}
    \begin{xlist}
      \ex[ ]{\alert<5->{Gestern} hat Ronnie gewonnen.}
      \pause
      \pause
      \ex[ ]{\alert<7->{Erfreulicherweise} hat Ronnie gestern gewonnen.}
      \pause
      \pause
      \ex[ ]{\alert<9->{Oben} finden wir andere Beispiele.}
      \pause
      \pause
      \ex[*]{\alert<11->{Doch} ist das aber nicht das Ende der Saison.}
      \pause
      \pause
      \ex[*]{\alert<13->{Und} ist die Saison zuende.}
      \pause
      \pause
    \end{xlist}
    \ex\label{ex:adverbenadkopulasundpartikeln044} Das ist aber \alert{doch} nicht das Ende der Saison.
  \end{exe}
  \pause
  \Viertelzeile
  \begin{block}{Adverb}
    Adverben sind die übriggebliebenen nicht-flektierbaren Wörter,\\
    die im Vorfeld stehen können.
  \end{block}
\end{frame}


\begin{frame}
  {Konjunktionen}
  \onslide<+->
  \onslide<+->
  \begin{exe}
    \ex
    \begin{xlist}
      \ex Wir \alert{laufen} \rot{und} \alert{springen}.
      \ex Ich bin allergisch gegen \alert{Haselnüsse} \rot{und} \alert{Bananen}.
      \ex \alert{Kommst du jetzt} \rot{oder} \alert{sollen wir schon gehen}?
      \ex \alert{Erschöpft}, \rot{aber} \alert{zufrieden} lief sie über die Ziellinie.
    \end{xlist}
  \end{exe}
  \Zeile
  \onslide<+->
  \begin{block}{Kunjunktion}
    Eine Konjunktion (\textit{und}, \textit{oder}, \textit{aber}, \textit{sondern}, \ldots) verbindet zwei Konstituenten A und B, die sich syntaktisch gleich verhalten. Die Gesamtheit [A Konjunktion B] verhält sich ebenso.
  \end{block}
\end{frame}


\section{Zur nächsten Woche | Überblick}

\begin{frame}
  {Deutsche Syntax | Plan}
  \rot{Alle} angegebenen Kapitel\slash Abschnitte aus \rot{\citet{Schaefer2018b}} sind \rot{Klausurstoff}!\\
  \Halbzeile
  \begin{enumerate}
    \item Grammatik und Grammatik im Lehramt \rot{(Kapitel 1 und 3)}
    \item Grundbegriffe \rot{(Kapitel 2)}
    \item Wortklassen \rot{(Kapitel 6)}
    \item \alert{Konstituenten und Satzglieder} \rot{(Kapitel 11 und Abschnitt 12.1)}
    \item Nominalphrasen \rot{(Abschnitt 12.3)}
    \item Andere Phrasen \rot{(Abschnitte 12.2 und 12.4--12.7)}
    \item Verbphrasen und Verbkomplex \rot{(Abschnitte 12.8)}
    \item Sätze \rot{(Abschnitte 12.9 und 13.1--13.3)} 
    \item Nebensätze \rot{(Abschnitt 13.4)}
    \item Subjekte und Prädikate \rot{(Abschnitte 14.1--14.3)}
    \item Passive und Objekte \rot{(14.4 und 14.5)}
    \item Syntax infiniter Verbformen \rot{(Abschnitte 14.7--14.9)}
  \end{enumerate}
  \Halbzeile
  \centering 
  \url{https://langsci-press.org/catalog/book/224}
\end{frame}

  \let\subsection\section\let\section\woopsi

  \section[Konstituenten]{Konstituen und Satzglieder}
  \let\woopsi\section\let\section\subsection\let\subsection\subsubsection
  

\section{Überblick}

\begin{frame}
  {Überblick: Konstituenten und Phrasen}
  \pause
  \begin{itemize}[<+->]
    \item Warum und wie syntaktische Analyse?
    \item syntaktische Generalisierungen formulieren
    \item größere und kleinere Teilstrukturen (Konstituenten) identifizieren
  \end{itemize}
\end{frame}


\section{Konstituenten}


\begin{frame}
  {Generalisierungen anhand von Wortklassen in der Syntax}
  \pause
  Denkbare Abstraktion für einen Satzbauplan anhand von Wortklassen:\\
  \Zeile
  \pause
  \begin{center}
    \begin{forest}
      [Satz
        [\it Ein]
        [\it Snookerball]
        [\it ist]
        [\it eine]
        [\it Kugel]
        [\it aus]
        [\it Kunststoff]
      ]
    \end{forest}\\
    \pause
    \Halbzeile
    \begin{center}
      →
    \end{center}
    \Halbzeile
    \begin{forest}
      [Satz
        [Art]
        [Subst]
        [Kopula-Verb]
        [Art]
        [Subst]
        [Prp]
        [Subst]
      ]
    \end{forest}        
  \end{center}
\end{frame}


\begin{frame}
  {"`Flache Beschreibungen"'}
  \pause
  \rot{Solche flachen Strukturbeschreibungen sind extrem ineffizient!}\\
  \Zeile
  \pause
  Aus Korpus mit \alert{über 1 Mrd.\ Wörtern} (DeReKo) \alert{alle Sätze} mit der Struktur\\
  von der vorherigen Folie (Art Subst Kopula Art Subst Prp Subst):\\
  \pause
  \Zeile
  \begin{exe}
    \ex
    \begin{xlist}
      \ex{Die Verlierer sind die Schulkinder in Weyerbusch.}
      \pause
      \ex{Die Vienne ist ein Fluss in Frankreich.}
      \pause
      \ex{Ein Baustein ist die Begegnung beim Spiel.}
      \pause
      \ex{Das Problem ist die Ortsdurchfahrt in Großsachsen.}
    \end{xlist}
  \end{exe}
\end{frame}

\begin{frame}
  {Viele ähnliche Strukturen auf einmal beschreiben}
  \pause
  Strukturen, die ähnlich, aber \alert{nicht genau} \\
  \alert{[Art Subst Kopula Art Subst Prp Subst]} sind:\\
  \pause
  \Zeile
  \begin{exe}
    \ex\label{ex:syntaktischestruktur013}
    \begin{xlist}
      \ex{\label{ex:syntaktischestruktur014} [Dieses Endspiel] ist [eine spannende Partie].}
      \pause
      \ex{\label{ex:syntaktischestruktur015} [Eine Hose] war [eine Hose].}
      \pause
      \ex{\label{ex:syntaktischestruktur016} [Sieger] wurde [ein Teilnehmer aus dem Vereinigten Königreich].}
      \pause
      \ex{\label{ex:syntaktischestruktur017} [Lemmy] ist [Ian Kilmister].}
    \end{xlist}
  \end{exe}
  \pause
  \Halbzeile
  \begin{itemize}[<+->]
    \item Diese Sätze sie sind \alert{gleich aufgebaut}.
    \item Sie haben jeweils \alert{drei Konstituenten} (= Bestandteile).
    \item Die Konstituenten haben intern teilweise abweichende Strukturen.
    \item Aber ihre unterschiedlich aufgebauten Konstituenten (Nominalphrasen)\\
      verhalten sich in diesen Sätzen jeweils gleich. 
  \end{itemize}
\end{frame}


\begin{frame}
  {Bauplan und Analyse}
  \pause
  Bauplan "`Kopula-Satz"' (vorläufig):\\
  \pause
  \Halbzeile
  \begin{center}
    \begin{forest}
      [Satz
        [NP]
        [Kopula-Verb]
        [NP]
      ]
    \end{forest}\\
    \pause
    \Zeile
    \raggedright
    Analyse auf Basis dieses Plans (vorläufig):\\
    \pause
    \Halbzeile
    \centering
    \begin{forest}
      [Satz
        [NP
          [\it Dieses Endspiel, narroof]
        ]
        [Kopula-Verb
          [\it ist]
        ]
        [NP
          [\it eine spannende Partie, narroof]
        ]
      ]
    \end{forest}
  \end{center}
\end{frame}


\begin{frame}
  {Konstituenten und Konstituententests}
  \pause
  {\Large \alert{Konstituententests sollen uns helfen, herauszufinden,\\
  wie wir Sätze in Konstituenten unterteilen wollen.}}\\
  \Zeile
  \pause
  \rot{Achtung!}
  \pause
  \Halbzeile
  \begin{itemize}[<+->]
    \item \rot{Konstituententests sind heuristisch!}
    \item unerwünschte Ergebnisse in beide Richtungen
    \item keine "`wahre Konstituentenstruktur"'
    \item theorieabhängig bzw.\ abhängig von gewählten Tests
    \Zeile
    \item Ziel: kompakte Beschreibung aller möglichen Strukturen
    \item gewiss: möglichst "`natürliche"' Analyse erwünscht
  \end{itemize}
\end{frame}

\begin{frame}
  {Pronominalisierungstest}
  \pause
  \begin{exe}
    \ex Mausi isst \alert<3->{den leckeren Marmorkuchen}.\\
    \pause
      \KTArr{PronTest} Mausi isst \alert{ihn}.
    \pause
    \ex{\label{ex:konstituententests025} \rot<5->{Mausi isst} den Marmorkuchen.\\
    \pause
      \KTArr{PronTest} \Ast \rot{Sie} den Marmorkuchen.}
    \pause
    \ex{\label{ex:konstituententests026} Mausi isst \alert<7->{den Marmorkuchen und das Eis mit Multebeeren}.\\
    \pause
    \KTArr{PronTest} Mausi isst \alert{sie}.}
  \end{exe}
  \pause
  \Halbzeile
  Pronominalausdrücke i.\,w.\,S.:
  \begin{exe}
    \ex{\label{ex:konstituententests027} Ich treffe euch \alert<9->{am Montag} \gruen<10->{in der Mensa}.\\
    \pause
    \KTArr{PronTest} Ich treffe euch \alert{dann} \gruen<10->{dort}.}
      \pause
      \pause
      \ex{\label{ex:konstituententests028} Er liest den Text \alert<12->{auf eine Art, die ich nicht ausstehen kann}.\\
      \pause
      \KTArr{PronTest} Er liest den Text \alert{so}.}
  \end{exe}
\end{frame}

\begin{frame}
  {Vorfeldtest\slash Bewegungstest}
  \pause
  \begin{exe}
    \ex
    \begin{xlist}
      \ex{Sarah sieht den Kuchen \alert<3->{durch das Fenster}.\\
        \pause
        \KTArr{VfTest} \alert{Durch das Fenster} sieht Sarah den Kuchen.}
      \pause
      \ex{Er versucht \alert{zu essen}.\\
        \pause
        \KTArr{VfTest} \alert<5->{Zu essen} versucht er.}
      \pause
      \ex{Sarah möchte gerne \alert{einen Kuchen backen}.\\
        \pause
        \KTArr{VfTest} \alert<7->{Einen Kuchen backen} möchte Sarah gerne.}
      \pause
      \ex{Sarah möchte \rot<9->{gerne einen} Kuchen backen.\\
        \pause
        \KTArr{VfTest} \Ast \rot{Gerne einen} möchte Sarah Kuchen backen.}
    \end{xlist}
  \end{exe}
  \pause
  \Halbzeile
  verallgemeinerter "`Bewegungstest"':\\
  \begin{exe}
    \ex\label{ex:konstituententests037}
    \begin{xlist}
      \ex{\label{ex:konstituententests038} Gestern hat \alert<11->{Elena} \gruen<11->{im Turmspringen} \orongsch<11->{eine Medaille} gewonnen.}
      \pause
      \ex{\label{ex:konstituententests039} Gestern hat \gruen{im Turmspringen} \alert{Elena} \orongsch{eine Medaille} gewonnen.}
      \pause
      \ex{\label{ex:konstituententests040} Gestern hat \gruen{im Turmspringen} \orongsch{eine Medaille} \alert{Elena} gewonnen.}
    \end{xlist}
  \end{exe}
\end{frame}

\begin{frame}
  {Koordinationstest}
  \pause
  \begin{exe}
    \ex\label{ex:konstituententests041}
    \begin{xlist}
      \ex Wir essen \alert<3->{einen Kuchen}.\\
      \pause
        \KTArr{KoorTest} Wir essen \alert{einen Kuchen} \gruen{und} \alert{ein Eis}.
      \pause
      \ex Wir \alert<5->{essen einen Kuchen}.\\
      \pause
        \KTArr{KoorTest} Wir \alert{essen einen Kuchen} \gruen{und} \alert{lesen ein Buch}.
      \pause
      \ex Sarah hat versucht, \alert<7->{einen Kuchen zu backen}.\\
      \pause
        \KTArr{KoorTest} Sarah hat versucht, \alert{einen Kuchen zu backen} \gruen{und} \\{}\alert{heimlich das Eis aufzuessen}.
      \pause
      \ex Wir sehen, dass \alert<9->{die Sonne scheint}.\\
      \pause
        \KTArr{KoorTest} Wir sehen, dass \alert{die Sonne scheint} \gruen{und} \\{}\alert{Mausi den Rasen mäht}.
    \end{xlist}
  \end{exe}
  \pause
  \begin{exe}
    \ex{\label{ex:konstituententests047} Der Kellner notiert, dass \rot<11->{meine Kollegin einen Salat} möchte.\\
    \pause
    \KTArr{KoorTest} Der Kellner notiert, dass \rot{meine Kollegin einen Salat}\\
    \gruen{und} \rot{mein Kollege einen Sojaburger} möchte.}
    \end{exe}
\end{frame}



\section{Satzglieder}

\begin{frame}
  {Satzglieder?}
  \pause
  \begin{exe}
    \ex
    \begin{xlist}
      \ex Sarah riecht den Kuchen \alert<3->{mit ihrer Nase}.\\
      \pause
        \KTArr{VfTest} \alert{Mit ihrer Nase} riecht Sarah den Kuchen.
        \pause
      \ex \KTArr{KoorTest} Sarah riecht den Kuchen\\
      {}\alert{mit ihrer Nase} und \alert{trotz des Durchzugs}.
    \end{xlist}
    \pause
    \ex
    \begin{xlist}
      \ex Sarah riecht den Kuchen \gruen<6->{mit der Sahne}.\\
      \pause
        \KTArr{VfTest} \Ast \rot{Mit der Sahne} riecht Sarah den Kuchen.
        \pause
      \ex \KTArr{KoorTest} Sarah riecht den Kuchen\\
      {}\alert{mit der Sahne} und \alert{mit den leckeren Rosinen}.
    \end{xlist}
  \end{exe}
  \pause
  \resizebox{0.9\textwidth}{!}{
    \begin{forest}
      [Satz
        [\it Sarah]
        [\it riecht]
        [\it den Kuchen]
        [\it mit ihrer Nase]
      ]
    \end{forest}\pause\begin{forest}
      [Satz
        [\it Sarah, tier=term]
        [\it riecht, tier=term]
        [Konstituente X
          [\it den Kuchen, tier=term]
          [\it mit der Sahne, tier=term]
        ]
      ]
    \end{forest}
  }
\end{frame}

\begin{frame}
  {Satzglieder als "`vorfeldfähige Konstituenten"'}
  \pause
  Ganz so einfach ist das nicht\ldots\\
  \Zeile
  \pause
  \begin{exe}
    \ex \rot{[Kaufen können]} möchte Alma die Wolldecke.
    \pause
    \ex \rot{[Über Syntax]} hat Sarah sich \alert{ein Buch} ausgeliehen.
  \end{exe}
  \Zeile
  \pause
  \alert{Wozu überhaupt den begriff des Satzglieds?}
  \begin{itemize}[<+->]
    \item in der Linguistik kaum von Interesse
    \item Sammelbegriff für "`Objekte und Adverbiale"'? -- \rot{Wozu?}
    \item Vorfeldfähigkeit? -- Wohl kaum, denn das wäre \rot{zirkulär} (und s.\,o.).
    \item Desambiguierung von Sätzen (s.\ Kuchen-Nase)? --\\
      \rot{Dabei hilft aber der Begriff "`Satzglied"' nicht.}
    \item Außerdem: \alert{Fördert das die Sprachkompetenz, oder kann das weg?}
  \end{itemize}
\end{frame}

\begin{frame}
  {Strukturelle Ambiguitäten und Kompositionalität}
  \pause
  \begin{exe}
    \ex{\label{ex:strukturelleambiguitaet060} Scully sieht den Außerirdischen mit dem Teleskop.}
  \end{exe}
  \pause
  \Halbzeile
  \begin{block}{Erinnerung: Kompositionalität}
    Die syntaktische Struktur ist die Basis für die Interpretation des Satzes (bzw.\ jedes syntaktisch komplexen Ausdrucks).
  \end{block}
  \pause
  \Halbzeile
  \begin{exe}
    \ex
    \begin{xlist}
      \ex Scully sieht \gruen<5->{[den Außerirdischen]} \orongsch<6->{[mit dem Teleskop]}.
      \pause
      \pause
      \pause
      \ex Scully sieht \alert<8->{[den Außerirdischen [mit dem Teleskop]]}.
    \end{xlist}
  \end{exe}
\end{frame}



\begin{frame}
  {Repräsentationsformat: Phrasenschemata}
  \pause
  \begin{itemize}[<+->]
    \item \alert{Grammatikalität = Konformität zu einer spezifischen Grammatik}
    \item Strukturen ohne spezifizierte Struktur: \rot{ungrammatisch}
      \Halbzeile
    \item Phrasenschemata = \alert{Baupläne} für zulässige Strukturen
    \item Strukturen = \alert{Bäume}
    \item Bei einer konkreten Analyse muss für jede Verzweigung im Baum\\
      ein Phrasenschema vorliegen, \rot{sonst ist die Analyse nicht zulässig}.
  \end{itemize}
  \pause
  \Halbzeile
  \centering
  \begin{multicols}{2}
    \footnotesize Das Schema:~\scalebox{0.6}{%
      \begin{forest}
      phrasenschema, baseline
      [NP, Ephr, calign=last
        [Artikel, Eopt, Emult
          [Pronomen, Eopt]
        ]
        [A, Eoptrec]
        [N, Ehd]
      ]
    \end{forest}
    \hspace{4em}
    }
    \onslide<8->{\footnotesize erlaubt~die~Analyse:~\scalebox{0.6}{%
      \begin{forest}
        [NP, calign=last, baseline
          [Artikel
            [\it ein]
          ]
          [A
            [\it leckerer]
          ]
          [A
            [\it geräucherter]
          ]
          [\textbf{N}
            [\it Tofu]
          ]
        ]
      \end{forest}
    }
  }
  \end{multicols}
\end{frame}

\section{Zur nächsten Woche | Überblick}

\begin{frame}
  {Deutsche Syntax | Plan}
  \rot{Alle} angegebenen Kapitel\slash Abschnitte aus \rot{\citet{Schaefer2018b}} sind \rot{Klausurstoff}!\\
  \Halbzeile
  \begin{enumerate}
    \item Grammatik und Grammatik im Lehramt \rot{(Kapitel 1 und 3)}
    \item Grundbegriffe \rot{(Kapitel 2)}
    \item Wortklassen \rot{(Kapitel 6)}
    \item Konstituenten und Satzglieder \rot{(Kapitel 11 und Abschnitt 12.1)}
    \item \alert{Nominalphrasen} \rot{(Abschnitt 12.3)}
    \item Andere Phrasen \rot{(Abschnitte 12.2 und 12.4--12.7)}
    \item Verbphrasen und Verbkomplex \rot{(Abschnitte 12.8)}
    \item Sätze \rot{(Abschnitte 12.9 und 13.1--13.3)} 
    \item Nebensätze \rot{(Abschnitt 13.4)}
    \item Subjekte und Prädikate \rot{(Abschnitte 14.1--14.3)}
    \item Passive und Objekte \rot{(14.4 und 14.5)}
    \item Syntax infiniter Verbformen \rot{(Abschnitte 14.7--14.9)}
  \end{enumerate}
  \Halbzeile
  \centering 
  \url{https://langsci-press.org/catalog/book/224}
\end{frame}




  \let\subsection\section\let\section\woopsi

  \section[Nominalphrasen]{Nominalphrasen}
  \let\woopsi\section\let\section\subsection\let\subsection\subsubsection
  

\section{Überblick}

\begin{frame}
  {Überblick: Konstituenten und Phrasen}
  \pause
  \begin{itemize}[<+->]
    \item Phrasen und Köpfe
    \item Strukur der deutschen \alert{Nominalphrase}
    \item (regierte) Attribute
  \end{itemize}
\end{frame}

\begin{frame}
  {Syntax und (bildungssprachliche) Funktion}
  \pause
  \begin{itemize}[<+->]
    \item \alert{hohe Komplexität} des syntaktischen Systems
    \item \rot{Regularitätensystem kaum vollständig explizit lernbar}
    \item überall \alert{starke Interaktion mit Semantik, Pragmatik usw.}
    \item \alert{Kompositionalität}
      \Zeile
    \item Der Versuch, Funktionen zu beschreiben, ohne Formsystem zu kennen,\\
      wäre in der Syntax völlig absurd.
      \Zeile
    \item reduzierte Syntax = erhebliche Einschränkung des Ausdrucks
    \item komplexe schriftsprachliche Syntax, ggf.\ \rot{Rezeptionsprobleme}
  \end{itemize}
\end{frame}


\section{Phrasentypen}

\begin{frame}
  {Jede Phrase hat genau einen Kopf}
  \pause
  \resizebox{\textwidth}{!}{
    \begin{tabular}{lll}
      \toprule
      \textbf{Kopf} & \textbf{Phrase} & \textbf{Beispiel} \\
      \midrule
      Nomen (Substantiv, Pronomen) & Nominalphrase (NP) & \textit{die tolle \alert{Auf"|führung}} \\
      Adjektiv & Adjektivphrase (AP) & \textit{sehr \alert{schön}} \\
      Präposition & Präpositionalphrase (PP) & \textit{\alert{in} der Uni} \\
      Adverb & Adverbphrase (AdvP) & \textit{total \alert{offensichtlich}} \\
      Verb & Verbphrase (VP) & \textit{Sarah den Kuchen gebacken \alert{hat}} \\
      Komplementierer & Komplementiererphrase (KP) & \textit{\alert{dass} es läuft} \\
      \bottomrule
    \end{tabular}
  } 
  \pause
  \Halbzeile
  \begin{itemize}[<+->]
    \item Der Kopf bestimmt den \alert{internen Aufbau} der Phrase.
    \item Der Kopf bestimmt die \alert{externen kategorialen Merkmale} der Phrase\\
      und so das syntaktische Verhalten der Phrase (Parallele: \alert{Kompositum}).
  \end{itemize}
\end{frame}


\begin{frame}
  {Wieviele Wortklassen? Wieviele Phrasentypen?}
  \pause
  \begin{itemize}[<+->]
    \item \alert{Phrasentyp: passend zur Wortklasse des Kopfes}
    \item maximal so viele Phrasentypen wie Wortklassen
    \item aber: nicht alle Wortklassen kopffähig (\alert{Funktionswörter})
      \Zeile
    \item heute nur der wahrscheinlich komplexeste nicht-satzförmige Phrasentyp:
      \begin{itemize}[<+->]
        \item Nominalphrase
      \end{itemize}
  \end{itemize}
\end{frame}

\section{Nominalphrasen}

\begin{frame}
  {Ziemlich volle NP-Struktur mit Substantiv-Kopf}
  \pause
  \centering
  \begin{forest}
    [NP, calign=child, calign child=3
      [Art
        [\it die]
      ]
      [AP
        [\it antiken, narroof]
      ]
      [\textbf{N}, tier=preterminal
        [\it Zahnbürsten]
      ]
      [NP, tier=preterminal
        [\it des Königs, narroof
        ]
      ]
      [RS
        [\it die nicht benutzt wurden, narroof]
      ]
    ]
  \end{forest}
  \pause
  \Zeile
  \begin{itemize}[<+->]
    \item \textit{die antiken Zahnbürsten}: \alert{Kongruenz}
    \item Baum über dem \alert{genusfesten} Kopf aufgebaut
    \item \alert{inneres Rechtsattribut} \textit{des Königs}
    \item \alert{Relativsatz} \textit{die nicht benutzt wurden}
  \end{itemize}
\end{frame}


\begin{frame}
  {Struktur mit pronominalem Kopf}
  \pause
  \centering
  \begin{forest}
    [NP, calign=child, calign child=1
      [\textbf{N}, tier=preterminal
        [\it einige]
      ]
      [NP, tier=preterminal
        [\it des Königs, narroof
        ]
      ]
      [RS
        [\it die geklaut wurden, narroof]
      ]
    ]
  \end{forest}
  \pause
  \Zeile
  \begin{itemize}[<+->]
    \item links vom Kopf: \rot{nichts}
    \item Determinierung erfolgt beim Pronomen \alert{im Kopf}.
    \item Determinierung schließt NP nach links ab.
    \item → \alert{Also kann links vom Pron-Kopf nichts stehen!}
  \end{itemize}
\end{frame}


\begin{frame}
  {Nominalphrase allgemein (Schema)}
  \pause
  \centering
  \begin{forest}
    phrasenschema
    [NP, Ephr
      [Art, Eopt, Emult, [NP\Sub{Genitiv}, Eopt]]
      [AP, Eopt, Erec]
      [N, Ehd, name=Nkopf]
      [innere Rechtsattribute, Eopt, Erec]
      {\draw [bend left=45, dashed,<-] (.south) to (Nkopf.south);}
      [RS, Eopt, Erec]
    ]
  \end{forest}
\end{frame}


\begin{frame}
  {Nochmal einige typische Muster von NPs}
  \onslide<+->
  \onslide<+->
 \Halbzeile 
  \begin{center}
    \scalebox{0.8}{\begin{tabular}[h]{lllll}
      \toprule
      \grau{Artikel oder}    & \grau{AP}         & \alert{nominaler}     & \grau{PPs, Adverben}   & \grau{Relativsätze und}  \\
      \grau{Genitiv-NP}      & \grau{} & \alert{Kopf}          & \grau{usw.} & \grau{Komplementsätze}   \\
      \midrule
      &&&& \\
      \textit{die}             & \textit{drei}       & \alert{\textit{Tische}}\Sub{Subst} & \textit{vor der Tafel}    & \textit{die heute fehlen}              \\
      &&&& \\
      \textit{Otjes}           & \textit{intelligente} & \alert{\textit{Kinder}}\Sub{Subst} & & \\
      &&&& \\
      && \alert{\textit{Orangensaft}}\Sub{Subst} && \\
      &&&& \\
      \Dim                     & \Dim                & \alert{\textit{Lemmy}}\Sub{Name} & \textit{von Motörhead}      &                               \\
      &&&& \\
      \Dim                     & \Dim               & \alert{\textit{jener}}\Sub{Pro}  & \textit{dort drüben} & \\
      &&&& \\
      \Dim                     & \Dim               & \alert{\textit{alle}}\Sub{Pro}   & & \textit{die einen Kaffe möchten} \\
    \end{tabular}}
  \end{center}
\end{frame}



\begin{frame}
  {Regierte Rechtsattribute}
  \pause
  \begin{exe}
    \ex die \gruen{Beachtung} \alert{[ihrer Lyrik]}
    \pause
    \ex mein \gruen{Wissen} \alert{[um die Bedeutung der komplexen Zahlen]}
    \pause
    \ex die \gruen{Überzeugung}, \alert{[dass die Quantenfeldtheorie \\
    die Welt korrekt beschreibt]}
    \pause
    \ex die \gruen{Frage}, \alert{[ob sich die Luftdruckanomalie von 2018 wiederholen wird]}
    \pause
    \ex die \gruen{Frage} \alert{[nach der möglichen Wiederholung der Luftdruckanomalie]}
  \end{exe}
  \pause
  \Halbzeile
  \begin{itemize}[<+->]
    \item typisch: postnominale Genitive, PPs, satzförmige Recta
  \end{itemize}
\end{frame}


\begin{frame}
  {Korrespondenzen zwischen Verben und Nomina(lisierungen)}
  \pause
  Viele Substantive entsprechen einem Verb mit bestimmten Rektionsanforderungen.\\
  \pause
  \Zeile
  \begin{exe}
    \ex\label{ex:rektionundvalenzindernp031}
    \begin{xlist}
      \ex{\label{ex:rektionundvalenzindernp032} \orongsch{Sarah} \alert{verziert} \gruen{[den Kuchen]}.}
      \pause
      \ex{\label{ex:rektionundvalenzindernp033} [Die \alert{Verzierung} \gruen{[des Kuchens]} \orongsch{[durch Sarah]}]}
      \pause
      \ex{\label{ex:rektionundvalenzindernp034} [Die \alert{Verzierung} \gruen{[von dem Kuchen]} \orongsch{[durch Sarah]}]}
    \end{xlist}
  \end{exe}
  \pause
  \Zeile
  \begin{itemize}[<+->]
    \item \gruen{Akkusativ} beim transitiven Verb $\Leftrightarrow$ \gruen{Genitiv}\slash\gruen{von-PP} beim Substantiv
    \item \orongsch{Nominativ} beim transitiven Verb $\Leftrightarrow$ \orongsch{durch-PP} beim Substantiv
    \item Beim nominalen Kopf: alle Ergänzungen optional
  \end{itemize}
\end{frame}


\begin{frame}
  {Alternative Korrespondenzen für Nominative}
  \pause
  \begin{exe}
    \ex\label{ex:rektionundvalenzindernp035}
    \begin{xlist}
      \ex{\label{ex:rektionundvalenzindernp036} \orongsch{[Sarah]} rettet [den Kuchen] [vor dem Anbrennen].}
      \pause
      \ex{\label{ex:rektionundvalenzindernp037} [\orongsch{[Sarahs]} Rettung [des Kuchens] [vor dem Anbrennen]]}
    \end{xlist}
  \end{exe}
  \pause
  \begin{itemize}[<+->]
    \item \orongsch{Nominativ} beim transitiven Verb $\Leftrightarrow$\\
      \orongsch{pränominaler Genitiv} beim Substantiv
  \end{itemize}
  \Halbzeile
  \begin{exe}
    \ex[ ]{\gruen{[Die Schokolade]} wirkt gemütsaufhellend.}
    \pause
    \ex[ ]{[Die Wirkung \gruen{[der Schokolade]}] ist gemütsaufhellend.}
    \pause
    \ex[?]{[Die Wirkung \gruen{[von der Schokolade]}] ist gemütsaufhellend.}
    \pause
    \ex[*]{[\gruen{[Der Schokolade]} Wirkung] ist gemütsaufhellend.}
  \end{exe}
  \pause
  \begin{itemize}[<+->]
    \item \gruen{Nominativ} beim intransitiven Verb $\Leftrightarrow$\\
      \gruen{prä-\slash postnominaler Genitiv}\slash\gruen{von-PP} beim Substantiv
  \end{itemize}
\end{frame}


\begin{frame}
  {Komplexität der NP | Sätze und NPs}
  \onslide<+->
  \onslide<+->
  Die NP erreicht eine außergewöhnliche Komplexität,\\
  weil sich ganze Sätze als NP verpacken lassen.\\
  \onslide<+->
  \Zeile
  \begin{exe}
    \ex{ } \grau{Martinas Freundin ist wieder zuhause.}\\
      \rot{Martina} \alert{teilt} \gruen{ihr} \alert{mit}, \orongsch{dass die Pferde bereits gefüttert wurden}.
    \onslide<+->
    \Zeile
    \ex{ } [\rot{[Martinas]} \alert{Mitteilung} [\gruen{an ihre Freundin}, \grau{[die wieder zuhause ist]}],\\
      { }\orongsch{[dass die Pferde bereits gefüttert wurden]}],\\
      (kam gerade noch rechtzeitig.)
  \end{exe}
\end{frame}


\begin{frame}
  {Baum für die NP}
  \onslide<+->
  \onslide<+->
  \centering
  \begin{forest}
    [NP, calign=child, calign child=2, tier=root
      [NP, tier=subroot, rottree
        [\it Martinas, narroof, tier=terminal]
      ]
      [\textbf{N}, tier=subroot, bluetree
        [\it Mitteilung, tier=terminal]
      ]
      [PP, tier=subroot, gruen
        [P, calign=child, calign child=1, gruennode
          [\it an, tier=terminal, gruennode]
          [NP, calign=child, calign child=2, gruennode
            [Art, tier=preterminal, gruennode
              [\it ihre, tier=terminal, gruennode]
            ]
            [N, tier=preterminal, gruennode
              [\it Freundin, gruennode]
            ]
            [RS, tier=preterminal, grautree
              [\it die \ldots\ ist, narroof]
            ]
          ]
        ]
      ]
      [KP, tier=subroot, orongschtree
        [\it dass \ldots\ wurden, narroof, tier=terminal]
      ]
    ]
  \end{forest}

\end{frame}


\ifdefined\TITLE
  \section{Zur nächsten Woche | Überblick}

  \begin{frame}
    {Deutsche Syntax | Plan}
    \rot{Alle} angegebenen Kapitel\slash Abschnitte aus \rot{\citet{Schaefer2018b}} sind \rot{Klausurstoff}!\\
    \Halbzeile
    \begin{enumerate}
      \item Grammatik und Grammatik im Lehramt \rot{(Kapitel 1 und 3)}
      \item Grundbegriffe \rot{(Kapitel 2)}
      \item Wortklassen \rot{(Kapitel 6)}
      \item Konstituenten und Satzglieder \rot{(Kapitel 11 und Abschnitt 12.1)}
      \item Nominalphrasen \rot{(Abschnitt 12.3)}
      \item \alert{Andere Phrasen} \rot{(Abschnitte 12.2 und 12.4--12.7)}
      \item Verbphrasen und Verbkomplex \rot{(Abschnitte 12.8)}
      \item Sätze \rot{(Abschnitte 12.9 und 13.1--13.3)} 
      \item Nebensätze \rot{(Abschnitt 13.4)}
      \item Subjekte und Prädikate \rot{(Abschnitte 14.1--14.3)}
      \item Passive und Objekte \rot{(14.4 und 14.5)}
      \item Syntax infiniter Verbformen \rot{(Abschnitte 14.7--14.9)}
    \end{enumerate}
    \Halbzeile
    \centering 
    \url{https://langsci-press.org/catalog/book/224}
  \end{frame}
\fi

  \let\subsection\section\let\section\woopsi

  \section[Phrasen]{Andere Phrasen}
  \let\woopsi\section\let\section\subsection\let\subsection\subsubsection
  
\section{Überblick}

\begin{frame}
  {Andere Phrasentypen}
  \onslide<+->
  \begin{itemize}[<+->]
    \item Adjektivphrasen
    \item Präpositionalphrasen
    \item Adverbphrasen
    \item Koordination
    \item Komplementiererphrase
  \end{itemize}
\end{frame}

\section[AP]{Adjektivphrasen}

\begin{frame}
  {Gradierungselemente vor dem Adjektiv}
  \onslide<+->
  \onslide<+->
  \begin{exe}
    \ex[]{\label{ex:adjektivphrase081} die [\gruen{sehr} \alert{angenehme}] Stimmung}
    \ex[]{\label{ex:adjektivphrase082} die [\gruen{ziemlich} \alert{angenehme}] Stimmung}
    \ex[]{\label{ex:adjektivphrase083} die [\gruen{wenig} \alert{angenehme}] Stimmung}
      \onslide<+->
      \Zeile
      \ex[]{\label{ex:adjektivphrase085} die [\gruen{[über alle Maßen]} \alert{angenehme}] Stimmung}
      \ex[]{\label{ex:adjektivphrase086} die [\gruen{[ja mal wieder so rein gar nicht]} \alert{angenehme}] Stimmung}
  \end{exe}
\end{frame}

\begin{frame}
  {Modifizierer | noch vor Gradierungselementen}
  \onslide<+->
  \onslide<+->
  \begin{exe}
    \ex\label{ex:adjektivphrase087}
    \begin{xlist}
      \ex{\label{ex:adjektivphrase088} die [\braun{[seit gestern]} \gruen{sehr} \alert{angenehme}] Stimmung}
      \ex{\label{ex:adjektivphrase089} das [\braun{[in Hessen]} \gruen{überaus} \alert{beliebte}] Getränk}
    \end{xlist}
    \onslide<+->
    \Zeile
    \ex[*]{\label{ex:adjektivphrase090}die [\gruen{sehr} \braun{[seit gestern]} \alert{angenehme}] Stimmung}
  \end{exe}
\end{frame}

\begin{frame}
  {Adjektivphrase | Baumbeispiel}
  \onslide<+->
  \onslide<+->
  \centering
  \begin{forest}
    [AP, calign=last
      [PP, tier=preterminal
        [\it seit gestern, narroof]
      ]
      [Ptkl, tier=preterminal
        [\it sehr]
      ]
      [\bf A, tier=preterminal
        [\it angenehme]
      ]
    ]
  \end{forest}
\end{frame}

\begin{frame}
  {Ergänzungen in der AP}
  \onslide<+->
  \onslide<+->\begin{exe}
  \ex\label{ex:adjektivphrase092}
  \begin{xlist}
    \ex[]{\label{ex:adjektivphrase093} die [\tuerkis{[auf ihre Tochter]} \alert{stolze}] Frau}
    \ex[*]{die [\alert{stolze} \tuerkis{[auf ihre Tochter]}] Frau}
    \Zeile
    \onslide<+->
    \ex[]{die [\tuerkis{[über ihre Tochter]} \alert{verwunderte}] Frau}
    \ex[*]{die [\alert{verwunderte} \tuerkis{[über ihre Tochter]}] Frau}
    \onslide<+->
    \Zeile
    \ex[]{die [\tuerkis{[ihres Lieblingseises]} \alert{überdrüssige}] Frau}
    \ex[*]{die [\alert{überdrüssige} \tuerkis{[ihres Lieblingseises]}] Frau}
  \end{xlist}
\end{exe}
\end{frame}

\begin{frame}
  {Ziemlich volle AP}
  \onslide<+->
  \onslide<+->
  \centering
  \begin{forest}
    [AP, calign=last
      [PP, tier=preterminal
        [\it seit gestern, narroof]
      ]
      [PP, tier=preterminal
        [\it auf ihre Tochter, narroof, name=AufIhreTochter]
      ]
      [Ptkl, tier=preterminal
        [\it sehr]
      ]
      [\bf A, tier=preterminal
        [\it stolze]
        {\draw [->, bend left=30] (.south) to (AufIhreTochter);}
      ]
    ]
  \end{forest}
\end{frame}

\begin{frame}
  {Adjektivphrase | Schema}
  \onslide<+->
  \onslide<+->
  \centering 
  \begin{forest}
    phrasenschema
    [AP, Ephr, calign=child, calign child=2
      [Modifizierer, Eopt, Erec, Emult [Ergänzungen, Eopt, Erec, name=Apergaenzi]]
      [Gradierungselement, Eopt]
      [A, Ehd]
      {\draw [->, bend left=45] (.south) to (Apergaenzi.south);}
    ]
  \end{forest}
\end{frame}

\section[PP]{Präpositionalphrasen}

\begin{frame}
  {Präpositionalphrasen | Beispiele}
  \onslide<+->
  \onslide<+->
  Erinnerung | \alert{Präpositionen haben eine einstellige Valenz.}\\
  \onslide<+->
  \Zeile
  \begin{exe}
    \ex\label{ex:normalepp096}
    \begin{xlist}
      \ex{[\alert{Auf} \orongsch{[dem Tisch]}] steht Ischariots Skulptur.}
      \ex{[\gruen{[Einen Meter]} \alert{unter} \orongsch{[der Erde]}] ist die Skulptur versteckt.}
    \end{xlist}
    \onslide<+->
    \Zeile
    \ex{\label{ex:normalepp097} Seit der EM springt Christina [\gruen{weit} \alert{über} \orongsch{[ihrem früheren Niveau]}].}
  \end{exe}{}
\end{frame}

\begin{frame}
  {Baumbeispiel | PP mit Maßangabe}
  \onslide<+->
  \onslide<+->
  \centering
  \begin{forest}
    [PP, calign=child, calign child=2
      [NP, tier=preterminal
        [\it einen Meter, narroof]
      ]
      [\bf P, tier=preterminal
        [\it unter]
      ]
      [NP, tier=preterminal
        [\it der Erde, narroof]
      ]
    ]
  \end{forest}
\end{frame}


\begin{frame}
  {Präpositionalphrase | Schema}
  \onslide<+->
  \onslide<+->
  \centering
  \begin{forest}
    phrasenschema
    [PP, Ephr, calign=child, calign child=2
      [Modifizierer, Eopt]
      [P, Ehd, name=Ppkopf]
      [NP, Eobl]
      {\draw [<-, bend left=45] (.south) to (Ppkopf.south);}
    ]
  \end{forest}
\end{frame}


\section[AdvP]{Adverbphrasen}

\begin{frame}
  {Adverbphrasen}
  \onslide<+->
  \onslide<+->
  \alert{Adverben} | Präpositionen mit \alert{nullstelliger Valenz}.\\
  \Zeile
  \onslide<+->
  \begin{exe}
    \ex{\label{ex:adverbphrase106} Ischariot malt [\orongsch{sehr} \alert{oft}].}
      \Halbzeile
      \ex{\label{ex:adverbphrase107} Ischariot schwimmt [\orongsch{weit} \alert{draußen}].}
      \Halbzeile
      \ex{\label{ex:adverbphrase108} Ischariot verreist [\orongsch{sehr} \alert{wahrscheinlich}].}
  \end{exe}
\end{frame}


\begin{frame}
  {Baumbeispiel | AdvP mit Modifizierer}
  \onslide<+->
  \onslide<+->
  \centering
  \begin{forest}
    [AdvP, calign=last
      [Ptkl, tier=preterminal
        [\it sehr]
      ]
      [\bf Adv, tier=preterminal
        [\it oft]
      ]
    ]
  \end{forest}
\end{frame}


\begin{frame}
  {Adverbphrase | Schema}
  \onslide<+->
  \onslide<+->
  \centering
  \begin{forest}
    phrasenschema
    [AdvP, Ephr, calign=last
      [Modifizierer, Eopt]
      [Adv, Ehd]
    ]
  \end{forest}
\end{frame}



\section{Koordination}

\begin{frame}
  {Koordination | Beispiele}
  \onslide<+->
  \onslide<+->
  \alert{Koordination} | Gleiches mit Gleichem zu Gleichem verbinden.\\
  \onslide<+->
  \Zeile
  \begin{exe}
    \ex\label{ex:koordination006}
    \begin{xlist}
      \ex{Ihre Freundin möchte [\orongsch{Kuchen} \alert{und} \orongsch{Sahne}].}
      \Halbzeile
      \onslide<+->
      \ex{[\orongsch{[Es ist Sonntag]} \alert{und} \orongsch{[die Zeit wird knapp]}].}
      \Halbzeile
      \onslide<+->
      \ex{Hast du das Teepulver [\orongsch{auf} \alert{oder} \orongsch{neben}]\\
    den Tatami-Matten verstreut?}
    \end{xlist}
  \end{exe}
\end{frame}


\begin{frame}
  {Koordination von Substantiven (oder NPs?)}
  \centering
  \begin{forest}
    [\textbf{N}, calign=child, calign child=2
      [\textbf{N}, tier=preterminal
        [\it Kuchen]
      ]
      [Konj, tier=preterminal
        [\it und]
      ]
      [\textbf{N}, tier=preterminal
        [\it Sahne]
      ]
    ]
  \end{forest}
\end{frame}

\begin{frame}
  {Koordination von Sätzen}
  \centering
  \begin{forest}
    [S, calign=child, calign child=2
      [S, tier=preterminal
        [\it Es ist Sonntag, narroof]
      ]
      [Konj, tier=preterminal
        [\it und]
      ]
      [S, tier=preterminal
        [\it die Zeit wird knapp, narroof]
      ]
    ]
  \end{forest}
\end{frame}

\begin{frame}
  {Koordination von Präpositionen}
  \centering
  \begin{forest}
    [\textbf{P}, calign=child, calign child=2
      [\textbf{P}, tier=preterminal
        [\it auf]
      ]
      [Konj, tier=preterminal
        [\it oder]
      ]
      [\textbf{P}, tier=preterminal
        [\it neben]
      ]
    ]
  \end{forest}
\end{frame}

\begin{frame}
  {Koordination | Schema}
  \onslide<+->
  \onslide<+->
  Die Koordination selber ist kein Kopf!\\
  \onslide<+->
  \Zeile
  \centering
  \begin{forest}
    phrasenschema
    [$\kappa$, Ephr
      [$\kappa$, Eobl]
      [Konj, Eopt]
      [$\kappa$, Eobl]
    ]
  \end{forest}
\end{frame}


\section[KP]{Komplementiererphrase}

\begin{frame}
  {Komplementiererphrasen = eingeleitete Nebensätze}
  \pause
  \begin{exe}
    \ex\label{ex:komplementiererphrase111}
    \begin{xlist}
      \ex[]{\label{ex:komplementiererphrase112} Der Arzt möchte, [dass [der Privatpatient die Rechnung \alert{bezahlt}]].}
      \pause
      \ex[*]{\label{ex:komplementiererphrase113} Der Arzt möchte, [dass [der Privatpatient \rot{bezahlt} die Rechnung]].}
      \pause
      \ex[*]{\label{ex:komplementiererphrase114} Der Arzt möchte, [dass [\rot{bezahlt} der Privatpatient die Rechnung]].}
    \end{xlist}
  \end{exe}
  \pause
  \Halbzeile
  \centering
  \begin{forest}
    [KP, calign=first
      [\bf K, tier=preterminal
        [\it dass, name=Kpkopf]
      ]
      [\alert{VP}, tier=preterminal
        [\it der Kassenpatient \alert{geht}, narroof]
      ]
    ]
  \end{forest}\\
  \pause
  \Zeile
  \alert{Verb-Letzt-Stellung!}\\
\end{frame}



\begin{frame}
  {Komplementiererphrase | Schema}
  \begin{center}
    \begin{forest}
      phrasenschema
      [KP, Ephr, calign=first
        [K, Ehd, name=Kpkopf]
        [VP, Eobl]
        {\draw [bend left=45, <-] (.south) to (Kpkopf.south);}
      ]
    \end{forest}
  \end{center}
  \onslide<+->
  \Zeile
  \alert{Aber wie sieht die VP aus?}\\
  \Viertelzeile
  \onslide<+->
  \orongsch{Und was ist mit unabhängigen Sätzen?}
\end{frame}

\section{Zur nächsten Woche | Überblick}

\begin{frame}
  {Deutsche Syntax | Plan}
  \rot{Alle} angegebenen Kapitel\slash Abschnitte aus \rot{\citet{Schaefer2018b}} sind \rot{Klausurstoff}!\\
  \Halbzeile
  \begin{enumerate}
    \item Grammatik und Grammatik im Lehramt \rot{(Kapitel 1 und 3)}
    \item Grundbegriffe \rot{(Kapitel 2)}
    \item Wortklassen \rot{(Kapitel 6)}
    \item Konstituenten und Satzglieder \rot{(Kapitel 11 und Abschnitt 12.1)}
    \item Nominalphrasen \rot{(Abschnitt 12.3)}
    \item Andere Phrasen \rot{(Abschnitte 12.2 und 12.4--12.7)}
    \item \alert{Verbphrasen und Verbkomplex} \rot{(Abschnitte 12.8)}
    \item Sätze \rot{(Abschnitte 12.9 und 13.1--13.3)} 
    \item Nebensätze \rot{(Abschnitt 13.4)}
    \item Subjekte und Prädikate \rot{(Abschnitte 14.1--14.3)}
    \item Passive und Objekte \rot{(14.4 und 14.5)}
    \item Syntax infiniter Verbformen \rot{(Abschnitte 14.7--14.9)}
  \end{enumerate}
  \Halbzeile
  \centering 
  \url{https://langsci-press.org/catalog/book/224}
\end{frame}

  \let\subsection\section\let\section\woopsi

  \section[Verbphrasen]{Verbphrasen und Verbkomplexe}
  \let\woopsi\section\let\section\subsection\let\subsection\subsubsection
  \input{includes/07.+Verbphrasen+und+Verbkomplexe.tex}
  \let\subsection\section\let\section\woopsi

  \section[Sätze]{Sätze}
  \let\woopsi\section\let\section\subsection\let\subsection\subsubsection
  \input{includes/08.+Sätze.tex}
  \let\subsection\section\let\section\woopsi

  \section[Nebensätze]{Nebensätze}
  \let\woopsi\section\let\section\subsection\let\subsection\subsubsection
  \section{Überblick}

\begin{frame}
  {Nebensätze und unabhängige Sätze}
  \begin{itemize}[<+->]
    \item \alert{Relativsätze} | interne und externe Beziehungen des Relativelements
    \item \alert{Objektsätze} | Rektion und Stellung
    \Zeile
    \item \alert{Feldermodell} | alternative Beschreibung deutscher Saztsyntax
  \end{itemize}
\end{frame}


\section{Relativsätze}

\begin{frame}
  {Relativsätze als etwas andere VL-Sätze}
  \pause
  Das \tuerkis{Relativelement} wird nach links gestellt. Das \alert{Verb} bleibt rechts.\\
  \pause
  \begin{center}
    \adjustbox{max width=0.4\textwidth}{%
      \begin{minipage}{0.7\textwidth}
      \begin{forest}
        [NP, calign=child, calign child=2
          [Art, tier=preterminal
            [\it einen]
          ]
          [\bf N, tier=preterminal
            [\it Tofu]
          ]
          [RS, calign=first
            [NP\Sub{1}, tier=preterminal, tuerkistree
              [\it der, narroof, name=BeweDer]
            ]
            [VP, calign=last
              [NP, tier=preterminal, baseline
                [\it mir, narroof]
              ]
              [\Ti, tier=preterminal, tuerkistree]
              {\draw[dotted, tuerkis, thick, ->] (.south) |- ++(0,-4.5em) -| (BeweDer.south);}
              [Ptkl, tier=preterminal
                [nicht]
              ]
              [\bf V, calign=last
                [\bf V, tier=preterminal
                  [\it geschmeckt]
                ]
                [\bf V, tier=preterminal, bluetree
                  [\it hat]
                ]
              ]
            ]
          ]
        ]
      \end{forest}
    \end{minipage}
    }
    \pause
    \hspace{0.1\textwidth}\adjustbox{max width=0.35\textwidth}{%
      \begin{minipage}{0.35\textwidth}
      \begin{forest}
        [RS, Ephr
          [XP\UpSub{relativ}{1}, Eobl, baseline, tuerkis]
          [VP\\{[\ldots\tuerkis{\Ti}\ldots]}, Eobl]
        ]
      \end{forest}
      \end{minipage}
    }
  \end{center}
  \pause
  \Halbzeile
  \begin{itemize}[<+->]
    \item Relativelement
      \begin{itemize}[<+->]
        \item \alert{Bedeutung}: Bezugs-Substantiv
        \item \alert{Genus, Numerus}: Kongruenz mit Bezugs-Substantiv
        \item \alert{Kasus\slash PP-Form}: gemäß Status als Ergänzung\slash Angabe im RS
      \end{itemize}
  \end{itemize}
\end{frame}


\begin{frame}
  {Komplexe Einbettung des Relativelements}
  \pause
  Das \tuerkis{Relativelement} als pränominaler Genitiv nimmt die Matrix-NP mit.\\
  \pause
  \Halbzeile
  \centering
  \begin{forest}
    [NP, calign=child, calign child=2
      [Art, tier=preterminal
        [\it der]
      ]
      [\bf N, tier=preterminal
        [\it Tofu]
        {\draw [->, bend right=30] (.south) to node [below, near start] {\footnotesize\textsc{Genus,Numerus}} (RekDessen.south);}
      ]
      [RS, calign=first
        [NP\Sub{1}, calign=first, tuerkistree
          [NP, tier=preterminal
            [\it dessen, narroof, name=RekDessen]
          ]
          [\bf N, tier=preterminal
            [\it Geschmack, name=RekGeschmack]
            {\draw [->, bend left=25] (.south) to node [below, near start] {\footnotesize\textsc{Kasus}} (RekDessen.south);}
          ]
        ]
        [VP, calign=last
          [NP, tier=preterminal
            [\it ich, narroof]
          ]
          [\Ti, tuerkistree]
          [\bf V, tier=preterminal
            [\it mag]
            {\draw [->, bend left=15] (.south) to node [below, near start] {\footnotesize\textsc{Kasus}} (RekGeschmack.south);}
          ]
        ]
      ]
    ]
  \end{forest}
\end{frame}



\section{Objektsätze}

\begin{frame}
  {Objektsätze}
  \pause
  \begin{exe}
    \ex{\label{ex:komplementsaetze127} Michelle weiß, [\rot{dass} die Corvette nicht anspringen wird].}
    \pause
    \ex\label{ex:komplementsaetze128}
    \begin{xlist}
      \ex{\label{ex:komplementsaetze129} Michelle will wissen, [\rot{wer} die Corvette gewartet hat].}
      \pause
      \ex{\label{ex:komplementsaetze130} Michelle will wissen, [\rot{ob} die Corvette gewartet wurde].}
    \end{xlist}
  \end{exe}
  \pause
  \Halbzeile
  \alert{Achtung: \textit{ob} ist eigentlich nur ein w-Wort ohne w (vgl.\ engl.\ \textit{whether}).}\\
  \pause
  \Halbzeile
\end{frame}

\begin{frame}
  {Regierende Verben und Alternationen}
  \pause
  \alert{Drei primäre Muster}, welche Satz-Objekte Verben regieren.\\
  \pause\Halbzeile
  \begin{exe}
    \ex\label{ex:komplementsaetze131}
    \begin{xlist}
      \ex[]{\label{ex:komplementsaetze132} Michelle behauptet, \alert{dass} die Corvette nicht anspringt.}
      \pause
      \ex[*]{\label{ex:komplementsaetze133} Michelle behauptet, \rot{wie\slash ob} die Corvette nicht anspringt.}
    \end{xlist}
    \pause
    \ex\label{ex:komplementsaetze134}
    \begin{xlist}
      \ex[*]{\label{ex:komplementsaetze135} Michelle untersucht, \rot{dass} der Vergaser funktioniert.}
      \pause
      \ex[]{\label{ex:komplementsaetze136} Michelle untersucht, \alert{wie\slash ob} der Vergaser funktioniert.}
    \end{xlist}
    \pause
    \ex\label{ex:komplementsaetze137}
    \begin{xlist}
      \ex[]{\label{ex:komplementsaetze138} Michelle hört, \alert{dass} die Nockenwelle läuft.}
      \pause
      \ex[]{\label{ex:komplementsaetze139} Michelle hört, \alert{wie\slash ob} die Nockenwelle läuft.}
    \end{xlist}
  \end{exe}
  \pause\Halbzeile
  Außerdem: \textit{dass} alterniert oft mit \textit{zu}-Infinitiv.\\
  \pause
  \Halbzeile
  \begin{exe}
  \ex\label{ex:komplementsaetze140}
  \begin{xlist}
    \ex{\label{ex:komplementsaetze141} Michelle glaubt, [\alert{dass} sie das Geräusch erkennt].}
    \pause
    \ex{\label{ex:komplementsaetze142} Michelle glaubt, [das Geräusch \alert{zu} erkennen].}
  \end{xlist}
  \end{exe}
\end{frame}

\begin{frame}
  {Stellung von Adverbial- und Komplementsätzen}
  \pause
  \Halbzeile
  \begin{exe}
  \ex\label{ex:komplementsaetze146}
  \begin{xlist}
    \ex[]{\label{ex:komplementsaetze147} \alert{[Dass sie unseren Kuchen mag]}, hat Sarah uns eröffnet.}
    \pause
    \ex[]{\label{ex:komplementsaetze148} Sarah hat uns eröffnet, \alert{[dass sie unseren Kuchen mag]}.}
    \pause
    \ex[?]{\label{ex:komplementsaetze149} Sarah hat uns, \rot{[dass sie unseren Kuchen mag]}, eröffnet.}
  \end{xlist}
    \pause

  \ex\label{ex:komplementsaetze150}
  \begin{xlist}
    \ex[]{\label{ex:komplementsaetze151} \alert{[Ob Pavel unseren Kuchen mag]}, haben wir uns oft gefragt.}
    \pause
    \ex[]{\label{ex:komplementsaetze152} Wir haben uns oft gefragt, \alert{[ob Pavel unseren Kuchen mag]}.}
    \pause
    \ex[?]{\label{ex:komplementsaetze153} Wir haben uns, \rot{[ob Pavel unseren Kuchen mag]}, oft gefragt.}
  \end{xlist}
    \pause
  \ex\label{ex:komplementsaetze154}
  \begin{xlist}
    \ex[]{\label{ex:komplementsaetze155} \alert{[Wer die Rosinen geklaut hat]}, wollen wir endlich wissen.}
    \pause
    \ex[]{\label{ex:komplementsaetze156} Wir wollen endlich wissen, \alert{[wer die Rosinen geklaut hat]}.}
    \pause
    \ex[?]{\label{ex:komplementsaetze157} Wir wollen, \rot{[wer die Rosinen geklaut hat]}, endlich wissen.}
  \end{xlist}
  \end{exe}
  \pause
  \begin{itemize}[<+->]
    \item Fast immer Bewegung nach links oder Rechtsversetzung \alert{hinter VK}!\\
    \item \grau{Fehlendes Schema für Rechtsversetzung: Transferaufgabe im Buch.}
  \end{itemize}
  \pause
\end{frame}


\begin{frame}
  {Korrelate bei Komplementsätzen}
  \onslide<+->
  \onslide<+->
  Komplementsätze werden also meistens aus der VP herausbewegt.\\
  \Halbzeile
  Anstelle des Nebensatzes kann ein optionales \alert{Korrelat} stehen.\\
  \Zeile
  \onslide<+->
  \begin{exe}
    \ex\label{ex:komplementsaetze158}
    \begin{xlist}
      \ex{\label{ex:komplementsaetze159} Sarah hat \alert{es} uns eröffnet, [dass sie unseren Kuchen mag].}
      \ex{\label{ex:komplementsaetze160} Wir haben \alert{es} uns gefragt, [ob Pavel unseren Kuchen mag].}
      \ex{\label{ex:komplementsaetze161} Wir wollen \alert{es} wissen, [wer die Rosinen geklaut hat].}
    \end{xlist}
  \end{exe}
\end{frame}


\begin{frame}
  {Korrelate bei Subjektsätzen}
  \onslide<+->
  \onslide<+->
  Subjektskorrelate, immer \alert{vor} dem Subjektsatz.\\
  \Zeile
  \begin{exe}
  \ex\label{ex:komplementsaetze166}
  \begin{xlist}
    \ex[ ]{\label{ex:komplementsaetze167} \alert{Es} hat uns gefreut, [dass Sarah unseren Kuchen mochte].}
    \ex[ ]{\label{ex:komplementsaetze168} Uns hat \alert{es} gefreut, [dass Sarah unseren Kuchen mochte].}
    \ex[ ]{\label{ex:komplementsaetze169} Uns hat gefreut, [dass Sarah unseren Kuchen mochte].}
    \ex[*]{\label{ex:komplementsaetze170} [Dass Sarah unseren Kuchen mochte], hat \rot{es} uns gefreut.}
  \end{xlist}
\end{exe}
\end{frame}

\begin{frame}
  {Obligatorische Korrelate von Präpositionalobjekten}
  \onslide<+->
  \onslide<+->
  Objektsätze können auch Präpositionalobjekte vertreten.\\
  \onslide<+->
  \Zeile
  \begin{exe}
  \ex\label{ex:komplementsaetze162}
  \begin{xlist}
    \ex[]{\label{ex:komplementsaetze163} Ich weise [auf den leckeren Kuchen] hin.}
    \Halbzeile
    \ex[]{\label{ex:komplementsaetze164} Ich weise \alert{darauf} hin, [dass der Kuchen lecker ist].}
    \ex[*]{\label{ex:komplementsaetze165} Ich weise hin, [dass der Kuchen lecker ist].}
  \end{xlist}
  \end{exe}
  \onslide<+->
  \Zeile
  Vertritt der Objektsatz ein Präpositionalobjekt, ist das Korrelat manchmal obligatorisch.
\end{frame}

\section{Feldermodell}

\begin{frame}
  {Das topologische Satzmodell}
  \onslide<+->
  \onslide<+->
  \begin{itemize}[<+->]
    \item (Neben-)Sätze werden eingeteilt in \alert{Felder} und die \rot{Satzklammer} \\
      \Halbzeile
      \alert{Vorfeld} | \rot{linke Klammer} | \alert{Mittelfeld} | \rot{rechte Klammer} | \alert{Nachfeld}\\
      \Halbzeile
      \grau{\ldots\ und ggf.\ weitere Felder}
      \Zeile
    \item angeblich eine vereinfachte Analyse deutscher Syntax
    \item keine hierarchische Struktur, nur topologische Anordnung
    \item nicht ordentlich rekursiv
      \Zeile
    \item \rot{führt bei komplexeren Sätzen prinzipiell zu 0 Punkten in Klausuren}
    \item meines Erachtens überflüssig, aber populär in bestimmten Didaktiken
  \end{itemize}
\end{frame}

  
\begin{frame}
  {Felder im unabhängigen Aussagesatz}
  \centering 
    \resizebox{\textwidth}{!}{
      \begin{tabular}{cp{0.1em}cp{0.1em}cp{0.1em}c}
        \textbf{Vf} && \textbf{LSK} && \textbf{Mf} && \textbf{RSK} \\
        \cmidrule{1-1}\cmidrule{3-3}\cmidrule{5-5}\cmidrule{7-7}
        &&&&&&\\
        \alert{irgendeine Konstituente} && \alert{finites Verb} && \alert{(Rest)} && \alert{infinite Verben} \\
        &&&&&&\\
        \orongsch{\textit{das Bild}} && \orongsch{\textit{hat}} && \orongsch{\textit{Ischariot wahrscheinlich}} && \orongsch{\textit{verkauft}} \\
      \end{tabular}
    }
\end{frame}

\begin{frame}
  {Felder im eingeleiteten Nebensatz}
  \centering 
  \resizebox{\textwidth}{!}{
    \begin{tabular}{cp{0.1em}cp{0.1em}cp{0.1em}c}
      \textbf{Vf} && \textbf{LSK} && \textbf{Mf} && \textbf{RSK} \\
      \cmidrule{1-1}\cmidrule{3-3}\cmidrule{5-5}\cmidrule{7-7}
        &&&&&&\\
        \alert{(leer)} && \alert{Komplementierer} && \alert{(Rest)} && \alert{Verbkomplex} \\
        &&&&&&\\
        && \orongsch{\textit{dass}} && \orongsch{\textit{Ischariot das Bild wahrscheinlich}} && \orongsch{\textit{verkauft hat}} \\
    \end{tabular}
  }
\end{frame}

\begin{frame}
  {Felder im Ja\slash Nein-Fragesatz}
  \centering 
    \begin{tabular}{cp{0.1em}cp{0.1em}cp{0.1em}c}
      \textbf{Vf} && \textbf{LSK} && \textbf{Mf} && \textbf{RSK} \\
      \cmidrule{1-1}\cmidrule{3-3}\cmidrule{5-5}\cmidrule{7-7}
        &&&&&&\\
      \alert{(leer)} && \alert{finites Verb} && \alert{(Rest)} && \alert{infinite Verben} \\
        &&&&&&\\
      && \orongsch{\textit{hat}} && \orongsch{\textit{Ischariot das Bild}} && \orongsch{\textit{verkauft}} \\
    \end{tabular}
\end{frame}

\begin{frame}
  {Felder im Relativsatz}
  \centering 
  \resizebox{\textwidth}{!}{
    \begin{tabular}{cp{0.1em}cp{0.1em}cp{0.1em}c}
      \textbf{Vf} && \textbf{LSK} && \textbf{Mf} && \textbf{RSK} \\
      \cmidrule{1-1}\cmidrule{3-3}\cmidrule{5-5}\cmidrule{7-7}
        &&&&&&\\
      \alert{Relativpronomen} && \alert{(leer)} && \alert{(Rest)} && \alert{Verbkomplex} \\
        &&&&&&\\
      \orongsch{\textit{dem}} &&&& \orongsch{\textit{Ischariot das Bild wahrscheinlich}} && \orongsch{\textit{verkauft hat}} \\
    \end{tabular}
  }
\end{frame}

\begin{frame}
  {Felderanalyse mit Nachfeld} 
  \centering 
  \resizebox{\textwidth}{!}{
    \begin{tabular}{cp{0.1em}cp{0.1em}cp{0.1em}cp{0.1em}c}
      \textbf{Vf} && \textbf{LSK} && \textbf{Mf} && \textbf{RSK} && \textbf{Nf} \\
      \cmidrule{1-1}\cmidrule{3-3}\cmidrule{5-5}\cmidrule{7-7}\cmidrule{9-9}
        &&&&&&&&\\
      \orongsch{\textit{Ischariot}} && \orongsch{\textit{hat}} && \orongsch{\textit{dem Arzt das Bild}} && \orongsch{\textit{verkauft}} && \gruen{\textit{das er selber gemalt hatte}} \\
    \end{tabular}
  }
\end{frame}


\begin{frame}
  {Felderanalyse mit Konnektorfeld}
  \centering 
    \begin{tabular}{cp{0.1em}cp{0.1em}cp{0.1em}cp{0.1em}c}
    \textbf{Kf} && \textbf{Vf} && \textbf{LSK} && \textbf{Mf} && \textbf{RSK} \\
    \cmidrule{1-1}\cmidrule{3-3}\cmidrule{5-5}\cmidrule{7-7}\cmidrule{9-9}
        &&&&&&&&\\
    \tuerkis{\textit{denn}} && \orongsch{\textit{Ischariot}} && \orongsch{\textit{hat}} && \orongsch{\textit{ihm das Bild}} && \orongsch{\textit{verkauft}} \\
  \end{tabular}
\end{frame}


\begin{frame}
  {Felder | Zusammengefasst}
    \centering
  \resizebox{1\textwidth}{!}{
    \begin{tabular}{lp{0.3cm}llll}
    \lsptoprule
    \textbf{Satztyp} && \textbf{Vorfeld} & \textbf{LSK} & \textbf{Mittelfeld} & \textbf{RSK} \\
    \midrule
    \textbf{V2} && bel.\ Satzglied & finites Verb    & Rest der VP & infinite Verben \\
    \textbf{V1} && ---                  & finites Verb    & Rest der VP & infinite Verben \\
    \textbf{VL} && ---                  & Komplementierer & Rest der VP & Verbkomplex \\
    \lspbottomrule
  \end{tabular}
  }
\end{frame}

\begin{frame}
  {Felder und Konstituenten}
  \centering
  \begin{forest}
    [S, calign=child, calign child=2, l sep+=2em
      [AdvP\Sub{2}, tier=preterminal, name=Vfnode
        [\it wahrscheinlich, narroof, name=Vfterm]
      ]
      [\bf V\Sub{1}, tier=preterminal
        [\it hat, name=Lskterm]
      ]
      [VP, calign=last, s sep+=0.5em
        [NP, tier=preterminal
          [\it Ischariot, narroof, name=Mffirstterm]
        ]
        [\Tii, tier=preterminal]
        [NP, tier=preterminal
          [\it dem Arzt, narroof]
        ]
        [NP, tier=preterminal
          [\it das Bild, narroof]
        ]
        [AdvP, tier=preterminal
          [\it heimlich, narroof, name=Mflastterm]
        ]
        [\bf V, calign=last, name=Rsknode
          [\bf V, tier=preterminal
            [\it verkauft, name=Rskfirstterm]
          ]
          [\Ti, tier=preterminal, name=Rsklastterm]
        ]
      ]
      {\draw ($(Vfterm.west |- Vfnode.north) + (-0.2,0.3)$) -- ($(Vfterm.east |- Vfnode.north) + (0.15,0.3)$) -- ($(Vfterm.south east) + (0.15,0)$) -- node [midway, below] {Vf} ($(Vfterm.south west) + (-0.2,0)$) -- cycle;}
      {\draw ($(Lskterm.west |- Vfnode.north) + (0.05,0.3)$) -- ($(Lskterm.east |- Vfnode.north) + (-0.05,0.3)$) -- ($(Lskterm.east |- Vfterm.south) + (-0.05,0)$) -- node [midway, below] {LSK} ($(Lskterm.west |- Vfterm.south) + (0.05,0)$) -- cycle;}
      {\draw ($(Mffirstterm.west |- Vfnode.north) + (-0.15,0.3)$) -- ($(Mflastterm.east |- Vfnode.north) + (0.25,0.3)$) -- ($(Mflastterm.east |- Vfterm.south) + (0.25,0)$) -- node [midway, below] {Mf} ($(Mffirstterm.west |- Vfterm.south) + (-0.15,0)$) -- cycle;}
      {\draw ($(Rskfirstterm.west |- Rsknode.north) + (-0.025,0.3)$) -- ($(Rsklastterm.east |- Rsknode.north) + (0,0.3)$) -- ($(Rsklastterm.east |- Vfterm.south) + (0,0)$) -- node [midway, below] {RSK} ($(Rskfirstterm.west |- Vfterm.south) + (-0.025,0)$) -- cycle;}
    ]
  \end{forest}
  
\end{frame}

\section{Zur nächsten Woche | Überblick}

\begin{frame}
  {Deutsche Syntax | Plan}
  \rot{Alle} angegebenen Kapitel\slash Abschnitte aus \rot{\citet{Schaefer2018b}} sind \rot{Klausurstoff}!\\
  \Halbzeile
  \begin{enumerate}
    \item Grammatik und Grammatik im Lehramt \rot{(Kapitel 1 und 3)}
    \item Grundbegriffe \rot{(Kapitel 2)}
    \item Wortklassen \rot{(Kapitel 6)}
    \item Konstituenten und Satzglieder \rot{(Kapitel 11 und Abschnitt 12.1)}
    \item Nominalphrasen \rot{(Abschnitt 12.3)}
    \item Andere Phrasen \rot{(Abschnitte 12.2 und 12.4--12.7)}
    \item Verbphrasen und Verbkomplex \rot{(Abschnitte 12.8)}
    \item Sätze \rot{(Abschnitte 12.9 und 13.1--13.3)} 
    \item Nebensätze \rot{(Abschnitt 13.4)}
    \item \alert{Subjekte und Prädikate} \rot{(Abschnitte 14.1--14.3)}
    \item Passive und Objekte \rot{(14.4 und 14.5)}
    \item Syntax infiniter Verbformen \rot{(Abschnitte 14.7--14.9)}
  \end{enumerate}
  \Halbzeile
  \centering 
  \url{https://langsci-press.org/catalog/book/224}
\end{frame}

  \let\subsection\section\let\section\woopsi

  \section[Subjekte\slash Prädikate]{Subjekte und Prädikate}
  \let\woopsi\section\let\section\subsection\let\subsection\subsubsection
  \input{includes/10.+Subjekte+und+Prädikate.tex}
  \let\subsection\section\let\section\woopsi

  \section[Passive\slash Objekte]{Passive und Objekte}
  \let\woopsi\section\let\section\subsection\let\subsection\subsubsection
  \input{includes/11.+Passive+und+Objekte.tex}
  \let\subsection\section\let\section\woopsi

  \section[Infinitivsyntax]{Syntax infiniter Verbformen}
  \let\woopsi\section\let\section\subsection\let\subsection\subsubsection
  
\section{Überblick}


\begin{frame}
  {Infinitivsyntax}
  \begin{itemize}[<+->]
    \item morphologische vs.\ analytische Tempora
    \item Ersatzinfinitiv und Oberfeldumstellung
      \Halbzeile
    \item kohärente und inkohärente Infinitive
    \item Modalverben und Halbmodale
    \item Kontrollverben
  \end{itemize}
\end{frame}

\section{Analytische Tempora}

\begin{frame}
  {Weitere Arten von Verben}
  \onslide<+->
  \onslide<+->
  Hilfs- und Modalverben mit besonderer Syntax und besonderer Formenbildung
  \onslide<+->
  \Halbzeile
  \begin{exe}
    \ex\label{ex:unterklassen072}
    \begin{xlist}
      \ex{\label{ex:unterklassen073} Frida \alert<9->{isst} den Marmorkuchen.}
      \onslide<+->
      \ex{\label{ex:unterklassen074} Frida \orongsch<10->{hat} den Marmorkuchen \alert<9->{gegessen}.}
      \onslide<+->
      \ex{\label{ex:unterklassen075} Der Marmorkuchen \orongsch<10->{wird} \alert<9->{gegessen}.}
      \onslide<+->
      \ex{\label{ex:unterklassen076} Frida \rot<11->{soll} den Marmorkuchen \alert<9->{essen}.}
      \onslide<+->
      \ex{\label{ex:unterklassen077} Dies hier \gruen<12->{ist} der leckere Marmorkuchen.}
      \onslide<+->
      \ex{\label{ex:unterklassen078} Der Marmorkuchen \gruen<12->{wird} lecker.}
    \end{xlist}
  \end{exe}
  \onslide<+->
  \Halbzeile
  \centering 
  \onslide<9->{\alert{Vollverben\slash lexikalische Verben}}\onslide<10->{, \orongsch{Hilfsverben}}\onslide<11->{, \rot{Modalverben}}\onslide<12->{, \gruen{Kopulaverben}}
\end{frame}

\begin{frame}
  {Welche Tempora hat das Deutsche?}
  \onslide<+->
  \onslide<+->
  Die Schulgrammatik lehrt \alert{sechs Tempusformen}, wir nur \rot{zwei}.\\
  \onslide<+->
  \Zeile
  \begin{center}
    \begin{tabular}[h]{lll}
      \textbf{Präsens}         & \textit{es \alert{geht}}                                     & \onslide<4->{\alert{synthetisch }} \\
      \textbf{Präteritum}      & \textit{es \alert{ging}}                                     & \onslide<4->{\alert{synthetisch }} \\
      && \\
      \textbf{Futur}         & \textit{es \orongsch{wird} \alert{gehen}}                    & \onslide<5->{\orongsch{analytisch }} \\
      && \\
      \textbf{Perfekt}         & \textit{es \orongsch{ist} \alert{gegangen}}                  & \onslide<5->{\orongsch{analytisch }} \\
      \textbf{Plusquamperfekt} & \textit{es \orongsch{war} \alert{gegangen}}                  & \onslide<5->{\orongsch{analytisch }} \\
      \textbf{Futurperfekt}         & \textit{es \orongsch{wird} \alert{gegangen} \orongsch{sein}} & \onslide<5->{\orongsch{analytisch }} \\
    \end{tabular}
  \end{center}
  \Zeile
  \begin{itemize}[<+->]
    \item Nur zwei werden als Form (\alert{synthetisch}) gebildet.
    \item Der Rest wird mit \orongsch{Hilfsverben} und \alert{infiniten Verbformen} (\orongsch{analytisch}) gebildet.
  \end{itemize}
\end{frame}

\begin{frame}
  {Präsens, Präteritum, Futur}
  \onslide<+->
  \begin{itemize}[<+->]
    \item Präsens
      \begin{itemize}[<+->]
        \item kein spezifischer Zeitbezug
        \item synthetische finite Form
      \end{itemize}
      \Viertelzeile
    \item Präteritum
      \begin{itemize}[<+->]
        \item Vergangenheitsbezug
        \item synthetische finite Form
      \end{itemize}
     \Viertelzeile 
    \item Futur
      \begin{itemize}[<+->]
        \item Zukunftsbezug oder Absichtserklärung
        \item analytische Form mit \rot{stets finitem} Hilfsverb
      \end{itemize}
  \end{itemize}
  \onslide<+->
  \Halbzeile
  \hspace{3em}\scalebox{0.8}{\begin{minipage}{\textwidth}
    \begin{exe}
      \onslide<11->{\ex[ ]{\ldots\ dass ich \alert{gehen werde}.}}
      \onslide<12->{\ex[*]{\ldots\ dass ich \rot{gehen werden} möchte.}}
      \onslide<13->{\ex[*]{\ldots\ dass ich \rot{gehen geworden} habe\slash bin.}}
      \onslide<14->{\ex[*]{\ldots\ dass ich \rot{gehen zu werden} habe.}}
    \end{exe}
  \end{minipage}}
\end{frame}

\begin{frame}
  {Perfekt}
  \onslide<+->
  \onslide<+->
  \rot{Das Perfekt ist nicht intrinsisch finit!}\\
  \onslide<+->
  \Viertelzeile
  Es kann daher im Infinitiv und in den drei finiten Tempora stehen.\\
  \Zeile
  \begin{itemize}[<+->]
    \item Hilfsverb \orongsch{sein} oder \orongsch{haben} + \alert{Partizip} des anderen Verbs
      \Halbzeile
    \item Infinitiv des Perfekts | \alert{gegangen} (Partizip) \orongsch{sein} (Inf des HVs)
    \item Präsens des Perfekts | \alert{gegangen} (Partizip) \orongsch{bin\slash bist\slash ist\slash\ldots} (Präs des HVs)
    \item Präteritum des Perfekts | \alert{gegangen} (Partizip) \orongsch{war\slash warst\slash\ldots} (Prät des HVs)
    \item Futur des Perfekts | \alert{gegangen} (Partizip) \orongsch{sein werde\slash wirst\slash wird\slash\ldots} (Futur des HVs)
  \end{itemize}
\end{frame}

\begin{frame}
  {Unterschiede zwischen Präteritum und Präsensperfekt}
  Stilistische Unterschiede\\
  \Halbzeile
  \begin{exe}
  \ex\label{ex:analytischetempora226}
  \begin{xlist}
    \ex{\label{ex:analytischetempora227} Das Pferd \alert{lief} im Kreis.}
    \ex{\label{ex:analytischetempora228} Das Pferd \orongsch{ist} im Kreis \alert{gelaufen}.}
  \end{xlist}
  \end{exe}
  \Zeile
  Semantische Unterschiede\\
  \Halbzeile
  \begin{exe}
  \ex\label{ex:analytischetempora229}
  \begin{xlist}
    \ex[ ]{\label{ex:analytischetempora230} Ich \orongsch{habe} schonmal Rilke \alert{gelesen}.}
    \ex[?]{\label{ex:analytischetempora231} Ich \alert{las} schonmal Rilke.}
  \end{xlist}
  \end{exe}
\end{frame}

\begin{frame}
  {Zusammenfassung | Finite Tempora und Perfekt}
  \onslide<+->
  \onslide<+->
  Klare Beziehungen zwischen den finiten Tempora und dem Perfekt\\
  \Zeile
  \begin{itemize}[<+->]
    \item Finite Tempora
      \begin{itemize}[<+->]
        \item Präsens | finite synthetische Form
        \item Präteritum | finite synthetische Form
        \item Futur (= Futur 1) | analytisch mit stets finitem Hilfsverb
      \end{itemize}
     \Zeile 
    \item \alert{Perfekta mit finiten Tempusformen des Hilfsverbs}
      \begin{itemize}[<+->]
        \item Präsensperfekt (= Perfekt) | Präsensform des Perfekts
        \item Präteritumsperfekt (= Plusquamperfekt) | Präteritalform des Perfekts
        \item Futurperfekt (= Futur 2) | Futur des Perfekts
      \end{itemize}
  \end{itemize}
\end{frame}


\begin{frame}
  {Analysen als Verbkomplex}
  \onslide<+->
  \onslide<+->
  Hilfsverben\slash Modalverben | \alert{Rektion des Status des anderen Verbs}\\
  \Halbzeile
  \centering
  \scalebox{0.85}{\begin{forest}
    [\textbf{V}, calign=last
      [\textbf{V}, calign=last
        [\textbf{V}, calign=last
          [\textbf{V}, tier=preterminal
            [\textit{behuft}]
          ]
          [\textbf{V}, tier=preterminal
            [\textit{gehabt}]
            {\draw [->, bend left=45] (.south) to node [below, midway] {\footnotesize\textsc{Status} (3)} (!uu11.south);}
          ]
        ]
        [\textbf{V}, tier=preterminal
          [\textit{haben}]
          {\draw [->, bend left=45] (.south) to node [below, midway] {\footnotesize\textsc{Status} (3)} (!uu121.south);}
        ]
      ]
      [\textbf{V}, tier=preterminal
        [\textit{will}]
        {\draw [->, bend left=45] (.south) to node [below, midway] {\footnotesize\textsc{Status} (1)} (!uu121.south);}
      ]
    ]
  \end{forest}}
\end{frame}




\begin{frame}
  {Nichtkanonische Infinitivrektion}
  \onslide<+->
  \onslide<+->
  Die sogenannte \alert{Oberfeldumstellung mit Ersatzinfinitiv}\\
  \Halbzeile
  \onslide<+->
  \begin{exe}
    \ex{\label{ex:ersatzinfinitivundoberfeldumstellung238} dass der Junge [\rot{hat} [[schwimmen] \rot{wollen}]]}
  \end{exe}
  \Zeile
  \onslide<+->
  \centering
  \scalebox{0.85}{\begin{forest}
    [\textbf{V}, calign=first
      [\textbf{V}, tier=preterminal
        [\textit{hat}]
      ]
      [\textbf{V}, calign=last
        [\textbf{V}, tier=preterminal
          [\textit{schwimmen}]
        ]
        [\textbf{V}, tier=preterminal
          [\textit{wollen}]
          {\draw [->, bend left=20] (.south) to node [below, near end] {\footnotesize\textsc{Status} (1)} (!uu11.south);}
          {\draw [<-, bend left=60] (.south) to node [below, midway] {\footnotesize\textsc{Status} (1)} (!uuu11.south);}
        ]
      ]
    ]
  \end{forest}}
\end{frame}


\section{Infinitivsyntax}


\begin{frame}
  {Syntaktische Katgeorie von Infinitivphrasen}
  \onslide<+->
  \onslide<+->
  \alert{Infinitivphrasen mit Ergänzungen und Angaben} (\ref{ex:infvp}) vs.\ \orongsch{reine Infinitive} (\ref{ex:infv})\\
  \onslide<+->
  \Viertelzeile
  \begin{exe}
    \ex{\ldots\ dass Vanessa \alert{[das Pferd zu reiten]} scheint\label{ex:infvp}}
    \onslide<+->
    \ex{\ldots\ dass Vanessa \orongsch{[zu reiten]} scheint\label{ex:infv}}
  \end{exe}
  \onslide<+->
  \Halbzeile
  Da Infinitive kein Subjekt regieren, sind es VPs ohne Subjekt\\
  \Viertelzeile
  \centering 
  \onslide<+->
  \begin{forest}
    [VP, calign=last
      [NP
        [das Pferd, narroof]
      ]
      [V
        [\it zu reiten]
      ]
    ]
  \end{forest}
\end{frame}


\begin{frame}
  {Kommas bei \textit{Infinitvkonstruktionen}}
  \onslide<+->
  \onslide<+->
  Komma oder nicht?
  \onslide<+->
  \begin{exe}
    \ex[*]{Nadezhda \rot{scheint}, die Kontrolle über die Hantel zu verlieren.}
    \ex[*]{Nadezhda \rot{will}, die Weltmeisterschaft gewinnen.}
    \ex[ ]{Nadezhda \alert{beschließt}, keine Steroide mehr einzunehmen.}
    \ex[?]{Nadezhda \alert{beschließt}\orongsch{,} zu trainieren.}
  \end{exe}
  \Zeile
  \begin{itemize}[<+->]
    \item \alert{Infinitivsyntax} ist der Schlüssel
    \item Komma nur bei \alert{inkohärenten Infinitiven}
  \end{itemize}
\end{frame}

\begin{frame}
  {(In)kohärente Infinitive}
  Kohärente und inkohärente Infinitivkonstruktionen\\
  \onslide<+->
  \Zeile
  \centering
  \scalebox{0.7}{\begin{minipage}{0.4\textwidth}
    \vspace{1.15cm}
    \begin{forest}
      [VP\Sub{B+A}, calign=last
        [NP, tier=preterminal
          [\textit{Vanessa}, narroof]
        ]
        [NP, tier=preterminal
          [\textit{die Pferde}, narroof]
        ]
        [\textbf{V\Sub{B+A}}, calign=last
          [\textbf{V\Sub{B}}, tier=preterminal
            [\textit{behufen}]
          ]
          [\textbf{V\Sub{A}}, tier=preterminal
            [\textit{will}]
          ]
        ]
      ]
    \end{forest}
  \end{minipage}}\hspace{0.1\textwidth}\scalebox{0.7}{\begin{minipage}{0.4\textwidth}
    \begin{forest}
      l sep+=3em, s sep+=2em
      [VP\Sub{A}, calign=last
        [NP, tier=preterminal
          [\textit{Vanessa}, narroof]
        ]
        [VP\Sub{B}, calign=last
          [NP, tier=preterminal
            [\textit{die Pferde}, narroof]
          ]
          [\textbf{V\Sub{B}}, tier=preterminal
            [\textit{zu behufen}]
          ]
        ]
        [\textbf{V\Sub{A}}, tier=preterminal
          [\textit{wünscht}]
        ]
      ]
    \end{forest}
  \end{minipage}}
\end{frame}


\begin{frame}
  {Test | Herausstellbarkeit}
  \onslide<+->
  \onslide<+->
  In der \rot{kohärenten} Konstruktion bildet der Infinitiv mit seinen Ergänzungen und Angaben keine Konstituente, also kann diese auf nicht nach rechts herausgestellt werden.\\
  \Zeile
  \onslide<+->
  \begin{exe}
    \ex[*]{Oma glaubt, dass Vanessa \rot{\Ti}\ will, \rot{[die Pferde behufen]\ORi}.}
  \end{exe}
  \onslide<+->
  \Zeile
  In der \gruen{inkohärenten} Konstruktion bildet der Infinitiv eine solche Konstituente.\\
  \Zeile
  \onslide<+->  
  \begin{exe}
    \ex[ ]{Oma glaubt, dass Vanessa \gruen{\Ti}\ wünscht, \gruen{[die Pferde zu behufen]\ORi}.}
  \end{exe}
\end{frame}


\begin{frame}
  {Halbmodale}
  \onslide<+->
  \onslide<+->
  Scheinbar gleich strukturiert | \gruen{wollen}, \orongsch{scheinen}, \alert{beschließen}\\
  \Halbzeile
  \onslide<+->
  \begin{exe}
  \ex
  \begin{xlist}
    \ex{dass der Hufschmied \gruen{das Pferd behufen will}.}
    \ex{dass der Hufschmied \orongsch{das Pferd zu behufen scheint}.}
    \ex{dass der Hufschmied \alert{das Pferd zu behufen beschließt}.}
  \end{xlist}
  \end{exe}
  \onslide<+->
  \Zeile
  Aber Abweichung bei der Extrahierbarkeit\\
  \Halbzeile
  \onslide<+->
  \begin{exe}
  \ex
  \begin{xlist}
    \ex[*]{dass der Hufschmied \gruen{\Ti}\ will, \gruen{[das Pferd behufen]\ORi}.}
    \ex[*]{dass der Hufschmied \orongsch{\Ti}\ scheint, \orongsch{[das Pferd zu behufen]\ORi}.}
    \ex[ ]{dass der Hufschmied \alert{\Ti}\ beschließt, \alert{[das Pferd zu behufen]\ORi}.}
  \end{xlist}
\end{exe}
\end{frame}

\begin{frame}
  {Halbmodale | \textit{scheinen} ohne Subjektrolle}
  \onslide<+->
  \onslide<+->
  Subjekt von \textit{scheinen} nicht erfragbar\\
  \onslide<+->
  \Halbzeile
  \begin{exe}
    \ex
    \begin{xlist}
      \ex[ ]{Frage: Wer \gruen{will} das Pferd behufen?\\
      Antwort: \gruen{Der Hufschmied will} das.}
      \onslide<+->
      \ex[*]{Frage: Wer \orongsch{scheint} das Pferd zu behufen?\\
      Antwort: \orongsch{Der Hufschmied scheint} das.}
      \onslide<+->
      \ex[ ]{Frage: Wer \alert{beschließt}, das Pferd zu behufen?\\
      Antwort: \alert{Der Hufschmied beschließt} das.}
    \end{xlist}
  \end{exe}
  \Zeile
  \onslide<+->
  Und \textit{scheinen} kann ein subjektloses Verb einbetten!\\
  \Halbzeile
  \onslide<+->
  \begin{exe}
    \ex
    \begin{xlist}
      \ex[*]{Dem Hufschmied \gruen{will} grauen.}
      \onslide<+->
      \ex[ ]{Dem Hufschmied \orongsch{scheint} zu grauen}
      \onslide<+->
      \ex[*]{Dem Hufschmied \alert{beschließt} zu grauen.}
    \end{xlist}
  \end{exe}
\end{frame}


\begin{frame}
  {(In)kohärente Infinitive}
  \onslide<+->
  \onslide<+->
    \resizebox{1\textwidth}{!}{
    \begin{tabular}{lcllll}
      \lsptoprule
      & \multirow{2}{*}{\textbf{Status}} & \multirow{2}{*}{\textbf{Kohärenz}} & \textbf{eigenes} & \textbf{Subjekts-} \\
      & & & \textbf{Subjekt} & \textbf{Rolle} & \textbf{Beispiel}\\
      \midrule
      \textbf{Modalverben} & 1 & obl.\ kohärent & ja & Identität & \textit{wollen} \\
      \textbf{Halbmodalverben} & 2 & obl.\ kohärent & nein & nein & \textit{scheinen} \\
      \textbf{Kontrollverben} & 2 & \rot{opt.\ inkohärent} & ja & Kontrolle & \textit{beschließen} \\
      \lspbottomrule
    \end{tabular}
  }\\
  \Zeile
  \begin{itemize}[<+->]
    \item Nur \alert{inkohärente nachgestellte Infinitive} werden kommatiert!
    \item Sie gelten als satzwertig, aber die \rot{Inkohärenz ist leider nur optional}.
    \item Es kommen also nur \alert{Abhängige von Kontrollverben} infrage.
  \end{itemize}
  \onslide<+->
  \Viertelzeile
  \begin{exe}
    \ex[*]{Nadezhda \rot{scheint}, die Kontrolle über die Hantel zu verlieren.}
    \ex[*]{Nadezhda \rot{will}, die Weltmeisterschaft gewinnen.}
  \end{exe}
\end{frame}

\begin{frame}
  {(In)kohärente Infinitve}
  \onslide<+->
  \onslide<+->
  Was ist jetzt hiermit?\\
  \Halbzeile
  \onslide<+->
  \begin{exe}
    \ex[ ]{Nadezhda \alert{beschließt}, keine Steroide mehr einzunehmen.}
    \ex[?]{Nadezhda \alert{beschließt}\orongsch{,} zu trainieren.}
  \end{exe}
  \onslide<+->
  \Halbzeile
  \alert{Eindeutig inkohärent} | hinter die RSK versetzte Infinitive\\
  \Viertelzeile
  \onslide<+->
  \begin{exe}
    \ex \rot{\textbf{Inkohärent}}
    \begin{xlist}
      \ex[ ]{\ldots dass Nadezhda beschließt, keine Steroide mehr zu nehmen.}
      \ex[?]{\ldots dass Nadezhda keine Steroide mehr zu nehmen beschließt.}
    \end{xlist}
    \ex \alert{\textbf{Kohärent oder inkohärent}}
    \begin{xlist}
      \ex[ ]{\ldots dass Nadezhda zu trainieren beschließt.}
      \ex[ ]{\ldots dass Nadezhda beschließt zu trainieren.}
    \end{xlist}
  \end{exe}
\end{frame}


% \begin{frame}
%   {(In)kohärente Infinitve}
%   Es liegt also an der syntaktischen Struktur.\\
%   \Zeile
%   \onslide<+->
%   \begin{exe}
%     \ex
%     \begin{xlist}
%       \ex[ ]{[Nadezhda]\Sub{2} \alert{[beschließt]\Sub{1}} [[t\Sub{2} \gruen{t\Sub{3}} \alert{[t\Sub{1}]\Sub{VK}}]\ \Sub{VP}\ \orongsch{,}\\
%       {\hspace{1em}}\gruen{[keine Steroide mehr einzunehmen]\Sub{3}}]\Sub{VP}.}
%         \Viertelzeile
%       \ex[*]{[Nadezhda]\Sub{2} \rot{[beschließt]\Sub{1}}\\
%       {\hspace{1em}}[t\Sub{2} [keine Steroide] [mehr] \rot{[einzunehmen t\Sub{1}]\Sub{VK}}\ ]\Sub{VP}.\label{ex:ohweia}}
%     \end{xlist}
%     \Halbzeile
%     \ex
%     \begin{xlist}
%       \ex[ ]{[Nadezhda]\Sub{2} \alert{[beschließt]\Sub{1}}\ \orongsch{,} [[t\Sub{2} \gruen{t\Sub{3}} \alert{[t\Sub{1}]\Sub{VK}}\ ]\Sub{VP} \gruen{[zu trainieren]\Sub{3}}]\Sub{VP}.}
%       \Viertelzeile
%       \ex[ ]{[Nadezhda]\Sub{2} \tuerkis{[beschließt]\Sub{1}} [t\Sub{2} \tuerkis{[zu trainieren t\Sub{1}]\Sub{VK}}\ ]\Sub{VP}}
%     \end{xlist}
%   \end{exe}
%   \Halbzeile
%   \onslide<+->
%   Füllen Sie den VK durch Hinzufügen von Hilfsverben auf,\\
%   um das Phänomen noch deutlicher zu sehen.
% \end{frame}

\begin{frame}
  {Bäume | Inkohärent}
  \onslide<+->
  \onslide<+->
  \rot{Inkohärent konstruiert}\\
  \Zeile
  \centering 
  \begin{forest}
    [S, calign=child, calign child=2
      [NP\Sub{2}, tier=pt
        [\it Nadezhda, narroof, tier=t]
      ]
      [V\Sub{1}, tier=pt
        [\it beschließt, tier=t]
      ]
      [VP, calign=child, calign child=1
        [VP, calign=child, calign child=3
          [\Tii, tier=t]
          [\rot{\Tiii}, tier=t]
          [\Ti, tier=t]
        ]
        [VP\Sub{3}, tier=pt, rottree
          [\it keine Steroide mehr einzunehmen, narroof, tier=t]
        ]
      ]
    ]
  \end{forest}
\end{frame}


\begin{frame}
  {Bäume | Inkohärent mit Hilfsverb}
  \onslide<+->
  \onslide<+->
  Mit einem infiniten Verb im Verbkomplex \rot{sieht man die Extraktion}\\
  \Zeile
  \centering 
  \begin{forest}
    [S, calign=child, calign child=2
      [NP\Sub{2}, tier=pt, tier=pt
        [\it Nadezhda, tier=t, narroof]
      ]
      [V\Sub{1}, tier=pt
        [\it hat, tier=t]
      ]
      [VP, , calign=child, calign child=1
        [VP, calign=last
          [\Tii, tier=pt]
          [\rot{\Tiii}, tier=pt]
          [V, calign=last
            [V, tier=pt
              [\it beschlossen, tier=t]
            ]
            [\Ti, tier=t]
          ]
        ]
        [VP\Sub{3}, tier=pt, rottree
          [\it keine Steroide mehr einzunehmen, narroof, tier=t]
        ]
      ]
    ]
  \end{forest}
\end{frame}


\begin{frame}
  {Bäume | Kohärent mit Hilfsverb}
  \onslide<+->
  \onslide<+->
  \rot{So gut wie ungrammatisch!}\\
  \Zeile
  \centering 
  \begin{forest}
    [S, calign=child, calign child=2
      [NP\Sub{2}, tier=pt
        [\it Nadezhda, tier=t, narroof]
      ]
      [V\Sub{1}, tier=pt
        [\it hat, tier=t]
      ]
      [VP, calign=last
        [\Tii, tier=pt]
        [NP, tier=pt
          [\it keine Steroide, narroof, tier=t]
        ]
        [AdvP, tier=pt
          [\it mehr, narroof, tier=t]
        ]
        [V, calign=last
          [V, calign=last
            [V
              [\it einzunehmen]
            ]
            [V, tier=pt
              [\it beschlossen, tier=t]
            ]
          ]
          [\Ti, tier=t]
        ]
      ]
    ]
  \end{forest}
\end{frame}

\begin{frame}
  {Bäume | Kohärent ohne Hilfsverb}
  \onslide<+->
  \onslide<+->
  Man kann daher davon ausgehen, dass diese Struktur auch nicht grammatisch ist.\\
  \onslide<+->
  \Viertelzeile
  Sie entspricht der nicht kommatierten Version.\\
  \onslide<+->
  \Zeile
  \centering
  \begin{forest}
    [S, calign=child, calign child=2
      [NP\Sub{2}, tier=pt
        [\it Nadezhda, tier=t, narroof]
      ]
      [V\Sub{1}, tier=pt
        [\it beschließt, tier=t]
      ]
      [VP, calign=last
        [\Tii, tier=pt]
        [NP, tier=pt
          [\it keine Steroide, narroof, tier=t]
        ]
        [AdvP, tier=pt
          [\it mehr, narroof]
        ]
        [V, calign=last
          [V, tier=pt
            [\it einzunehmen]
          ]
          [\Ti, tier=t]
        ]
      ]
    ]
  \end{forest}
\end{frame}

\section{Kontrollinfinitive}

\begin{frame}
  {\textit{zu}-Infinitive als Subjekte und Objekte}
  \onslide<+->
  \onslide<+->
  \alert{Controller} | Logisches Argument des Verbs, das die Bedeutung\\
  des fehlenden Subjekts des Infinitivs beisteuert\\
  \onslide<+->
  \Halbzeile
  \begin{exe}
  \ex\label{ex:infinitivkontrolle264}
  \begin{xlist}
    \ex{\label{ex:infinitivkontrolle265} [Das Geschirr \gruen{zu spülen}] \alert{nervt} Matthias. (Objektkontrolle)}\\
    \onslide<+->
    \Viertelzeile
    Matthias | der \alert{Genervte} (Objekt) und der \gruen{Spülende}\\
    \Halbzeile
    \onslide<+->
    \ex{\label{ex:infinitivkontrolle266} Doro wagt, [die Küche \gruen{zu betreten}]. (Subjektkontrolle)}\\
    \onslide<+->
    \Viertelzeile
    Doro | die \alert{Wagende} (Subjekt) und die \gruen{Betrende}
  \end{xlist}
\end{exe}
\Zeile
\onslide<+->
Auch mit Korrelat\\
\Halbzeile
\begin{exe}
  \ex\label{ex:infinitivkontrolle267}
  \begin{xlist}
    \ex{\label{ex:infinitivkontrolle268} Es nervt Matthias, [das Geschirr zu spülen].}
    \ex{\label{ex:infinitivkontrolle269} Doro wagt es, [die Küche zu betreten].}
  \end{xlist}
\end{exe}
\end{frame}

\begin{frame}
  {Kontrolle im Passiv}
  \onslide<+->
  \onslide<+->
  Kontrolle bleibt im Passiv erhalten | \alert{logische Valenz}, nicht Syntax\\
  \Halbzeile
  \onslide<+->
  \begin{exe}
  \ex\label{ex:infinitivkontrolle270}
  \begin{xlist}
    \ex{\label{ex:infinitivkontrolle271} Der Installateur hat gestern \alert{versucht}, die Küche \gruen{zu betreten}.}\\
    \onslide<+->
    \Viertelzeile
    der Installateur | der \alert{Versuchende} (Subjekt) und der \gruen{Betrende}\\
    \onslide<+->
    \Halbzeile
    \ex{\label{ex:infinitivkontrolle272} Gestern wurde (vom Installateur) versucht, die Küche zu betreten.}\\
    \Viertelzeile
    \onslide<+->
    der Installateur | der \alert{Versuchende} (Subjekt des Aktivs) und der \gruen{Betrende}\\
  \end{xlist}
\end{exe}
\end{frame}

\begin{frame}
  {Kontrolle}
  \begin{block}
    {Infinitivkontrolle}
Die \textit{Kontrollrelation} besteht zwischen einer nominalen Valenzstelle eines Verbs und einem von diesem Verb abhängigen (subjektlosen) \textit{zu}"=Infinitiv.
Die Bedeutung des nicht ausgedrückten Subjekts des abhängigen \textit{zu}"=Infinitivs wird dabei durch die mit der nominalen Valenzstelle verbundene Bedeutung beigesteuert.
  \end{block}
\end{frame}


\begin{frame}
  {Subjektinfinitive}
  \onslide<+->
  \onslide<+->
  Objektkontrolle präferiert\\
  \onslide<+->
  \Halbzeile
  \begin{exe}
  \ex\label{ex:infinitivkontrolle274}
  \begin{xlist}
    \ex{\label{ex:infinitivkontrolle275} Das Geschirr zu spülen, nervt \gruen{ihn}.\\
    Controller | \gruen{Akkusativobjekt}}
    \onslide<+->
    \Viertelzeile
    \ex{\label{ex:infinitivkontrolle276} Das Geschirr zu spülen, fällt \gruen{ihm} leicht.\\
    Controller | \gruen{Dativobjekt}}
    \onslide<+->
    \Viertelzeile
    \ex{\label{ex:infinitivkontrolle277} Das Geschirr zu spülen, beschert \gruen{ihm} einen zufriedenen Mitbewohner.\\
    Controller | \gruen{Dativobjekt}}
    \onslide<+->
    \Viertelzeile
    \ex{\label{ex:infinitivkontrolle278} Sich für Hilfe zu bedanken, freut \gruen{ihn} immer besonders.\\
    Controller | \gruen{Akkusativobjekt}}
  \end{xlist}
\end{exe}
\end{frame}

\begin{frame}
  {Objektinfinitive}
  \onslide<+->
  \onslide<+->
  Objektkontrolle präferiert, falls Objekte vorhanden\\
  \onslide<+->
  \Halbzeile
  \begin{exe}
  \ex\label{ex:infinitivkontrolle279}
  \begin{xlist}
    \ex{\label{ex:infinitivkontrolle280} \gruen{Er} wagt, die Küche zu betreten.\\
    Controller | \gruen{Subjekt}}
    \onslide<+->
    \Viertelzeile
    \ex{\label{ex:infinitivkontrolle281} Er bittet \gruen{seinen Mitbewohner}, das Geschirr zu spülen.\\
    Controller | \gruen{Akkusativobjekt}}
    \onslide<+->
    \Viertelzeile
    \ex{\label{ex:infinitivkontrolle282} Doro erlaubt \gruen{Matthias}, sich den Wagen zu leihen.\\
    Controller | \gruen{Dativobjekt}}
  \end{xlist}
\end{exe}
\end{frame}

\begin{frame}
  {Infinitivangaben}
  \onslide<+->
  \onslide<+->
  Immer Subjektkontrolle
  \begin{exe}
  \ex\label{ex:infinitivkontrolle283}
  \begin{xlist}
    \ex{\label{ex:infinitivkontrolle284} \gruen{Matthias} arbeitet, um Geld zu verdienen.\\
    Controller | \gruen{Subjekt}}
    \onslide<+->
    \Viertelzeile
    \ex{\label{ex:infinitivkontrolle285} \gruen{Matthias} begrüßt Doro, ohne aus der Rolle zu fallen.\\
    Controller | \gruen{Subjekt}}
    \onslide<+->
    \Viertelzeile
    \ex{\label{ex:infinitivkontrolle286} \gruen{Matthias} hilft Doro, anstatt untätig daneben zu stehen.\\
    Controller | \gruen{Subjekt}}
    \onslide<+->
    \Viertelzeile
    \ex{\label{ex:infinitivkontrolle287} \gruen{Matthias} bringt Doro den Wagen zurück, ohne den Lackschaden \\zu erwähnen.\\
    Controller | \gruen{Subjekt}}
  \end{xlist}
\end{exe}
\end{frame}


\ifdefined\TITLE
  \section{Vor der Klausur | Überblick}

  \begin{frame}
    {Deutsche Syntax | Plan}
    \rot{Alle} angegebenen Kapitel\slash Abschnitte aus \rot{\citet{Schaefer2018b}} sind \rot{Klausurstoff}!\\
    \Halbzeile
    \begin{enumerate}
      \item Grammatik und Grammatik im Lehramt \rot{(Kapitel 1 und 3)}
      \item Grundbegriffe \rot{(Kapitel 2)}
      \item Wortklassen \rot{(Kapitel 6)}
      \item Konstituenten und Satzglieder \rot{(Kapitel 11 und Abschnitt 12.1)}
      \item Nominalphrasen \rot{(Abschnitt 12.3)}
      \item Andere Phrasen \rot{(Abschnitte 12.2 und 12.4--12.7)}
      \item Verbphrasen und Verbkomplex \rot{(Abschnitte 12.8)}
      \item Sätze \rot{(Abschnitte 12.9 und 13.1--13.3)} 
      \item Nebensätze \rot{(Abschnitt 13.4)}
      \item Subjekte und Prädikate \rot{(Abschnitte 14.1--14.3)}
      \item Passive und Objekte \rot{(14.4 und 14.5)}
      \item Syntax infiniter Verbformen \rot{(Abschnitte 14.7--14.9)}
    \end{enumerate}
    \Halbzeile
    \centering 
    \url{https://langsci-press.org/catalog/book/224}
  \end{frame}
\fi

  \let\subsection\section\let\section\woopsi

  \section{Vor der Klausur}

  \begin{frame}
    {Deutsche Syntax | Plan}
    \rot{Alle} angegebenen Kapitel\slash Abschnitte aus \rot{\citet{Schaefer2018b}} sind \rot{Klausurstoff}!\\
    \Halbzeile
    \begin{enumerate}
      \item Grammatik und Grammatik im Lehramt \rot{(Kapitel 1 und 3)}
      \item Grundbegriffe \rot{(Kapitel 2)}
      \item Wortklassen \rot{(Kapitel 6)}
      \item Konstituenten und Satzglieder \rot{(Kapitel 11 und Abschnitt 12.1)}
      \item Nominalphrasen \rot{(Abschnitt 12.3)}
      \item Andere Phrasen \rot{(Abschnitte 12.2 und 12.4--12.7)}
      \item Verbphrasen und Verbkomplex \rot{(Abschnitte 12.8)}
      \item Sätze \rot{(Abschnitte 12.9 und 13.1--13.3)} 
      \item Nebensätze \rot{(Abschnitt 13.4)}
      \item Subjekte und Prädikate \rot{(Abschnitte 14.1--14.3)}
      \item Passive und Objekte \rot{(14.4 und 14.5)}
      \item Syntax infiniter Verbformen \rot{(Abschnitte 14.7--14.9)}
    \end{enumerate}
    \Halbzeile
    \centering 
    \url{https://langsci-press.org/catalog/book/224}
  \end{frame}
\fi


\makeatletter
\setcounter{lastpagemainpart}{\the\c@framenumber}
\makeatother

\appendix

\begin{frame}[allowframebreaks]
  {Literatur}
  \renewcommand*{\bibfont}{\footnotesize}
  \setbeamertemplate{bibliography item}{}
  \printbibliography
\end{frame}

\begin{frame}
  {Autor}
  \begin{block}{Kontakt}
    Prof.\ Dr.\ Roland Schäfer\\
    Institut für Germanistische Sprachwissenschaft\\
    Friedrich-Schiller-Universität Jena\\
    Fürstengraben 30\\
    07743 Jena\\[\baselineskip]
    \url{https://rolandschaefer.net}\\
    \texttt{roland.schaefer@uni-jena.de}
  \end{block}
\end{frame}

\begin{frame}
  {Lizenz}
  \begin{block}{Creative Commons BY-SA-3.0-DE}
    Dieses Werk ist unter einer Creative Commons Lizenz vom Typ \textit{Namensnennung - Weitergabe unter gleichen Bedingungen 3.0 Deutschland} zugänglich.
    Um eine Kopie dieser Lizenz einzusehen, konsultieren Sie \url{http://creativecommons.org/licenses/by-sa/3.0/de/} oder wenden Sie sich brieflich an Creative Commons, Postfach 1866, Mountain View, California, 94042, USA.
  \end{block}
\end{frame}

\mode<beamer>{\setcounter{framenumber}{\thelastpagemainpart}}

\end{document}
