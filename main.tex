% Communication with make =============================================

\def\GRAPHPATH{localgraphics}

\ifdefined\HANDOUT
  \documentclass[handout,aspectratio=1610,dvipsnames]{beamer}
  \def\GRAPHPATH{graphics}
\else
  \documentclass[aspectratio=1610,dvipsnames]{beamer}
\fi

\ifdefined\TITLE
\else
  \def\TITLE{}
\fi

\usepackage[ngerman]{babel}
\usepackage{ifthen}
\usepackage{color}
\usepackage{colortbl}
\usepackage{textcomp}
\usepackage{multirow}
\usepackage{nicefrac}
\usepackage{multicol}
\usepackage{langsci-gb4e}
\usepackage{verbatim}
\usepackage{cancel}
\usepackage{graphicx}
\usepackage{hyperref}
\usepackage{verbatim}
\usepackage{boxedminipage}
\usepackage{adjustbox}
\usepackage{rotating}
\usepackage{booktabs}
\usepackage{bbding}
\usepackage{pifont}
\usepackage{multicol}
\usepackage{stmaryrd}
\usepackage{FiraSans}
\usepackage{soul}
\usepackage{tikz}
\usepackage[linguistics]{forest}
\usepackage[maxbibnames=99,
  maxcitenames=2,
  uniquelist=false,
  backend=biber,
  doi=false,
  url=false,
  isbn=false,
  bibstyle=biblatex-sp-unified,
  citestyle=sp-authoryear-comp]{biblatex}

% Biblatex ============================================================

\addbibresource{rs.bib}

% Colors ==============================================================

\definecolor{grau}{rgb}{0.5,0.5,0.5}
\definecolor{lg}{rgb}{0.8,0.8,0.8}
\definecolor{trueblue}{rgb}{0.3,0.3,1}
\definecolor{ltb}{rgb}{0.8,0.8,1}
\definecolor{lgr}{rgb}{0.5,1,0.5}
\definecolor{orongsch}{RGB}{255,165,0}
\definecolor{gruen}{rgb}{0,0.4,0}
\definecolor{rot}{rgb}{0.7,0.2,0.0}
\definecolor{tuerkis}{RGB}{63,136,143}
\definecolor{braun}{RGB}{108,71,65}
\definecolor{blaw}{rgb}{0,0,0.9}
\newcommand{\gruen}[1]{\textcolor{gruen}{#1}}
\newcommand{\blaw}[1]{\textcolor{blaw}{#1}}
\newcommand{\rot}[1]{\textcolor{rot}{#1}}
\newcommand{\blau}[1]{\textcolor{trueblue}{#1}}
\newcommand{\orongsch}[1]{\textcolor{orongsch}{#1}}
\newcommand{\grau}[1]{\textcolor{grau}{#1}}
\newcommand{\whyte}[1]{\textcolor{white}{#1}}
\newcommand{\tuerkis}[1]{\textcolor{tuerkis}{#1}}
\newcommand{\braun}[1]{\textcolor{braun}{#1}}

% Newcommands =========================================================

\newcommand{\Dim}{\cellcolor{lg}}
\newcommand{\Dimblue}{\cellcolor{ltb}}
\newcommand{\Dimgreen}{\cellcolor{lgr}}
\newcommand{\Sub}[1]{\ensuremath{_{\text{#1}}}}
\newcommand{\Up}[1]{\ensuremath{^{\text{#1}}}}
\newcommand{\UpSub}[2]{\ensuremath{^{\text{#1}}_{\text{#2}}}}
\newcommand{\Ti}{\Spur{1}}
\newcommand{\Tii}{\Spur{2}}
\newcommand{\Tiii}{\Spur{3}}
\newcommand{\Tiv}{\Spur{4}}
\newcommand{\Ck}{\CheckmarkBold}
\newcommand{\Fl}{\XSolidBrush}
\newcommand{\xxx}{\hspaceThis{[}}
\newcommand{\zB}{z.\,B.\ }
\newcommand{\down}[1]{\ensuremath{\mathrm{#1}}}
\newcommand{\Zeile}{\vspace{\baselineskip}}
\newcommand{\Halbzeile}{\vspace{0.5\baselineskip}}
\newcommand{\Viertelzeile}{\vspace{0.25\baselineskip}}
\newcommand{\KTArr}[1]{\ding{226}~\textit{#1}~\ding{226}}
\newcommand{\Ast}{*}
\newcommand{\SL}{\ensuremath{\llbracket}}
\newcommand{\SR}{\ensuremath{\rrbracket}}
\def\lspbottomrule{\bottomrule}
\def\lsptoprule{\toprule}
\newcommand{\Sw}[1]{\begin{sideways}#1\end{sideways}}
\newcommand{\Lab}{\ensuremath{\langle}}
\newcommand{\Rab}{\ensuremath{\rangle}}
\newcommand{\AbUmlautBreaker}{}
\ifdefined\HANDOUT
  \renewcommand{\AbUmlautBreaker}{\ /}
\fi
\newcommand{\LocStrutGrph}{\hspace{0.1\textwidth}}
\newcommand{\Nono}{---}

\newcommand{\Bewegtes}[1]{\ensuremath{_{\textrm{#1}}}}
\newcommand{\ORi}{\Bewegtes{1}}
\newcommand{\ORii}{\Bewegtes{2}}
\newcommand{\ORiii}{\Bewegtes{3}}
\newcommand{\ORiv}{\Bewegtes{4}}
\newcommand{\ORv}{\Bewegtes{5}}
\newcommand{\Spur}[1]{t\Sub{#1}}


% Beamer ==============================================================

\usetheme[hideothersubsections]{PaloAlto}

\renewcommand<>{\rot}[1]{%
  \alt#2{\beameroriginal{\rot}{#1}}{#1}%
}
\renewcommand<>{\blau}[1]{%
  \alt#2{\beameroriginal{\blau}{#1}}{#1}%
}
\renewcommand<>{\orongsch}[1]{%
  \alt#2{\beameroriginal{\orongsch}{#1}}{#1}%
}
\renewcommand<>{\gruen}[1]{%
  \alt#2{\beameroriginal{\gruen}{#1}}{#1}%
}

\setbeamercolor{alerted text}{fg=trueblue}

\addtobeamertemplate{navigation symbols}{}{%
    \usebeamerfont{footline}%
    \usebeamercolor[fg]{footline}%
    \hspace{1em}%
    \insertframenumber/\inserttotalframenumber
}

\newcounter{lastpagemainpart}

\resetcounteronoverlays{exx}

\AtBeginSection[]{
  \begingroup
  \setbeamertemplate{navigation symbols}{}
  \begin{frame}[noframenumbering]
  \vfill
  \centering
  \begin{beamercolorbox}[sep=8pt,center,shadow=true,rounded=true]{title}
    \usebeamerfont{title}\insertsectionhead\par%
  \end{beamercolorbox}
  \vfill
  \end{frame}
  \endgroup
}

\setbeamertemplate{navigation symbols}{\insertframenumber/\inserttotalframenumber\hspace{5em}}

% Tikz ================================================================

\usetikzlibrary{positioning,arrows,cd}
\tikzset{>=latex}

% Forest

\forestset{
  Ephr/.style={draw, ellipse, thick, inner sep=2pt},
  Eobl/.style={draw, rounded corners, inner sep=5pt},
  Eopt/.style={draw, rounded corners, densely dashed, inner sep=5pt},
  Erec/.style={draw, rounded corners, double, inner sep=5pt},
  Eoptrec/.style={draw, rounded corners, densely dashed, double, inner sep=5pt},
  Ehd/.style={rounded corners, fill=gray, inner sep=5pt,
    delay={content=\whyte{##1}}
  },
  Emult/.style={for children={no edge}, for tree={l sep=0pt}},
  phrasenschema/.style={for tree={l sep=2em, s sep=2em}},
  decide/.style={draw, chamfered rectangle, inner sep=2pt},
  finall/.style={rounded corners, fill=gray, text=white},
  intrme/.style={draw, rounded corners},
  yes/.style={edge label={node[near end, above, sloped, font=\scriptsize]{Ja}}},
  no/.style={edge label={node[near end, above, sloped, font=\scriptsize]{Nein}}},
  sake/.style={tier=preterminal},
  ake/.style={
    tier=preterminal
    },
}

\tikzset{
    invisible/.style={opacity=0,text opacity=0},
    visible on/.style={alt=#1{}{invisible}},
    alt/.code args={<#1>#2#3}{%
      \alt<#1>{\pgfkeysalso{#2}}{\pgfkeysalso{#3}}
    },
}

\forestset{
  visible on/.style={
    for tree={
      /tikz/visible on={#1},
      edge+={/tikz/visible on={#1}}}}}

\useforestlibrary{edges}

\forestset{
  narroof/.style={roof, inner xsep=-0.25em, rounded corners},
  forky/.style={forked edge, fork sep-=7.5pt},
  bluetree/.style={for tree={trueblue}, for children={edge=trueblue}},
  orongschtree/.style={for tree={orongsch}, for children={edge=orongsch}},
  rottree/.style={for tree={rot}, for children={edge=rot}},
  gruentree/.style={for tree={gruen}, for children={edge=gruen}},
  tuerkistree/.style={for tree={tuerkis}, for children={edge=tuerkis}},
  brauntree/.style={for tree={braun}, for children={edge=braun}}, 
  grautree/.style={for tree={grau}, for children={edge=grau}}, 
  gruennode/.style={gruen, edge=gruen},
  graunode/.style={grau, edge=grau},
}


% Drawing sonority diagrams =========================================== 

\makeatletter

\long\def\ifnodedefined#1#2#3{%
  \@ifundefined{pgf@sh@ns@#1}{#3}{#2}}

\newcommand\aeundefinenode[1]{%
  \expandafter\ifx\csname pgf@sh@ns@#1\endcsname\relax
  \else
    \typeout{Undefining node "#1"}%
    \global\expandafter\let\csname pgf@sh@ns@#1\endcsname\relax
  \fi
}

\newcommand\aeundefinethesenodes[1]{%
  \foreach \myn  in {#1}
    {%
      \ifnodedefined{\myn}{%
      \expandafter\aeundefinenode\expandafter{\myn}%
    }{}
    }%
}

\newcommand\aeundefinenumericnodes{%
  \foreach \myn in {1,2,...,50}
    {%
      \ifnodedefined{\myn}{%
      \expandafter\aeundefinenode\expandafter{\myn}%
    }{}
    }%
}
\makeatother

\newcommand{\plo}{0}
\newcommand{\fri}{0.5}
\newcommand{\nas}{1}
\newcommand{\liq}{1.5}
\newcommand{\vok}{2}

% Save text.
\newcommand{\lastsaved}{}
\newcommand{\textsave}[1]{\gdef\lastsaved{#1}#1}

\newcommand{\SonDiag}[2][0]{%
  \begin{tikzpicture}
    \textsave{.}
    \tikzset{
      normalseg/.style={fill=white},
      extrasyll/.style={circle, draw, fill=white},
      sylljoint/.style={diamond, draw, fill=white}
    }
    \node at (0,\plo) {P};
    \node at (0,\fri) {F};
    \node at (0,\nas) {N};
    \node at (0,\liq) {L};
    \node at (0,\vok) {V};

    % Draw the helper lines if required.
    \ifthenelse{\equal{#1}{0}}{}{%
      \foreach \y in {\plo, \fri, \nas, \liq,\vok} {%
	\draw [dotted, |-|] (0.25, \y) -- (#1.75, \y);
      }
    }

    \foreach [count=\x from 1, remember=\x as \lastx] \p / \y / \g in #2 {
      \ifthenelse{\equal{\y}{-1}}{\textsave{.}}{%

	% Draw the node, either plain, as Silbenbgelenk, or as extrasyllabic.
        \ifthenelse{\equal{\g}{1}}{%
	  \node (\x) [sylljoint] at (\x, \y) {\p};
	}{%
	  \ifthenelse{\equal{\g}{2}}{%
	    \node (\x) [extrasyll] at (\x, \y) {\p};
	  }{%
	    \node (\x) [normalseg] at (\x, \y) {\p};
	  }
	}

	% Draw the connection unless the previous node was not or was empty.
	\ifthenelse{\NOT\equal{\lastsaved}{.}}{%
	  \draw [->] (\lastx) to (\x);
	}{}
	\textsave{1}
      }
    }
    \aeundefinenumericnodes
  \end{tikzpicture}
}


% Meta ================================================================

\title[Deutsche Syntax]{Deutsche Syntax\\\TITLE}
\author{Roland Schäfer}
\institute{Institut für Germanistische Sprachwissenschaft\\Friedrich-Schiller-Universität Jena}
\date{Diese Version ist vom \today.\\\Zeile%
  \scriptsize \grau{stets aktuelle Fassungen: \url{https://github.com/rsling/VL-Deutsche-Syntax}}}

\begin{document}

\begingroup
\setbeamertemplate{navigation symbols}{}
\begin{frame}[noframenumbering]
 \titlepage
\end{frame}
\endgroup

\ifdefined\LECTURE
  \include{includes/\LECTURE}
\else

  \makeatletter
  \setbeamertemplate{section in sidebar}{\vspace{0.25\baselineskip}\vbox{%
      \beamer@sidebarformat{3pt}{section in sidebar}{\insertsectionhead}\vspace{-0.25\baselineskip}}}
  \setbeamertemplate{section in sidebar shaded}{\vspace{0.25\baselineskip}\vbox{%
      \beamer@sidebarformat{3pt}{section in sidebar shaded}{\insertsectionhead}\vspace{-0.25\baselineskip}}}
  \setbeamertemplate{subsection in sidebar}{\hspace{1em}\vbox{%
    \beamer@sidebarformat{3pt}{subsection in sidebar}{\insertsubsectionhead}\vspace{-0.5\baselineskip}}}
  \setbeamertemplate{subsection in sidebar shaded}{\hspace{1em}\vbox{%
      \beamer@sidebarformat{3pt}{subsection in sidebar shaded}{\insertsubsectionhead}\vspace{-0.5\baselineskip}}}
  \makeatother

%   \begin{frame}
%     {Stand der Überarbeitung}
%     \begin{center}
%       \Large Dieser Foliensatz ist erst bis\\
%       einschließlich \alert{Vorlesung 10}\\
%       für das Wintersemester 2019\slash 2020 überarbeitet.
%     \end{center}
%   \end{frame}

  \section{Sprache und Grammatik}
  \let\woopsi\section\let\section\subsection\let\subsection\subsubsection
  
\section{Organisation}

\begin{frame}
  {Roland Schäfer}
  \onslide<+->
  \begin{itemize}[<+->]
    \item seit WS 2022\slash 2023 Professur für Grammatik und Lexikon
    \item 2020--2022 Forschungsstelle an der HU Berlin
    \item 2018 habilitiert an der HU Berlin\\
      (Germanistische Linguistik und allgemeine Sprachwissenschaft)
    \item 2007--2022 Mitarbeiter an der FU Berlin
    \item 2008 promoviert an der Uni Göttingen (Englische Syntax)
    \item 2002--2007 Mitarbeiter in der Sprachwissenschaft in Göttingen
    \item Studium in Marburg (Sprachwissenschaft, Japanologie)
  \end{itemize}
  \Zeile
  \onslide<+->
  Bitte nennen Sie mich nicht Professor\ldots\ \onslide<+-> Wenn Sie es tun, dann bitte richtig:\\
  \url{https://rolandschaefer.net/regeln-fur-den-mailverkehr/}
\end{frame}

\begin{frame}
  {Forschung}
  \onslide<+->
  \onslide<+->
  Linguistik (des Deutschen)\\
  \Halbzeile
  \begin{itemize}[<+->]
    \item kognitiv fundierte Grammatik
    \item Morphosyntax und Graphematik
    \item grammatische Variation ("`Zweifelsfälle"')
    \item individuelle Variation
    \item Registervariation
    \item Epistemologie
  \end{itemize}
  \Zeile
  \onslide<+->
  Methoden\\
  \Halbzeile
  \begin{itemize}[<+->]
    \item Korpuserstellung und -analyse
    \item verhaltensbasierte Experimente
    \item Fragen der statistischen Inferenz
  \end{itemize}
\end{frame}

\begin{frame}
  {Ablauf und Inhalte der Vorlesung}
  \begin{itemize}
    \item 13 Sitzungen über Grammatik und Syntax des Deutschen
    \item Meine Inhalte entsprechen meiner \alert{\textit{Einführung in\\
      die grammatische Beschreibung des Deutschen}} \grau{\citep{Schaefer2018b}}
    \item \url{http://langsci-press.org/catalog/book/224} (\alert{open access})
      \vspace{\baselineskip}
    \item Bei Amazon für 20€\\
      \url{https://www.amazon.de/dp/3961101183/}
  \end{itemize}
\end{frame}

\begin{frame}
  {Fragen und Interaktion}
  \begin{itemize}
    \item Interaktion in einer VL ist immer schwierig!\\
      Ich versuche es ggf.\ trotzdem.
      \Zeile
    \item Wenn Sie Fragen zum Stoff oder zum Buch haben:
      \texttt{roland.schaefer@uni-jena.de}
      \Zeile
    \item Mein Youtube-Kanal (demnächst wieder lebendig):\\
      \url{https://www.youtube.com/channel/UCc0SUpRSVvU2jJxx4rRBdsg}
  \end{itemize}
\end{frame}

\begin{frame}
  {Der Plan für heute}
  \pause
  \begin{itemize}
    \item Grammatik
      \begin{itemize}
        \item Grammatik als System
        \item Kern und Peripherie des Systems
        \item Norm und Beschreibung, Regel und Regularität
      \end{itemize}
    \Zeile
    \item EGBD3: Kapitel 1
  \end{itemize}
\end{frame}


\section{Grammatik}

\begin{frame}
  {Deutsche Sätze erkennen und interpretieren}
  \pause
  \begin{exe}
    \ex Dies ist ein Satz.
  \pause
    \ex Satz dies ein ist.
  \pause
    \ex Kno kna knu.
  \pause
    \ex This is a sentence.
  \pause
    \vspace{\baselineskip}
    \ex Dies ist ein Satz
  \end{exe}
\end{frame}


\begin{frame}
  {Form und Bedeutung: Kompositionalität}
  \begin{exe}
    \ex Das ist ein Kneck.
    \pause
    \vspace{\baselineskip}
  \ex Jede Farbe ist ein Kurzwellenradio.
  \ex Der dichte Tank leckt.
\end{exe}
    \vspace{\baselineskip}
  \pause

  \Large\begin{block}{Kompositionalität}
    Die Bedeutung komplexer sprachlicher Ausdrücke ergibt sich aus der Bedeutung ihrer Teile und der Art ihrer grammatischen Kombination. 
    Diese Eigenschaft von Sprache nennt man Kompositionalität.
  \end{block}
\end{frame}

\begin{frame}
  {Grammatik als System und Grammatikalität}
  \pause

  \Large\begin{block}{Grammatik}
    Eine Grammatik ist ein \alert{System von Regularitäten}, nach denen aus einfachen Einheiten komplexe Einheiten einer Sprache gebildet werden.
  \end{block}
  \vspace{\baselineskip}

  \pause

  \begin{block}{Grammatikalität}
    Jede von einer bestimmten Grammatik beschriebene Symbolfolge ist \alert{grammatisch} relativ zu dieser Grammatik, alle anderen sind \alert{ungrammatisch}.
  \end{block}
\end{frame}

\begin{frame}
  {(Un)grammatisch ist nicht gleich (in)akzeptabel}
  \pause
  \begin{exe}
    \ex\begin{xlist}
      \ex Bäume wachsen werden hier so schnell nicht wieder.
      \pause
      \ex Touristen übernachten sollen dort schon im nächsten Sommer.
      \pause
      \ex Schweine sterben müssen hier nicht.
      \pause
      \ex Der letzte Zug vorbeigekommen ist hier 1957.
      \pause
      \ex Das Telefon geklingelt hat hier schon lange nicht mehr.
      \pause
      \ex Häuser gestanden haben hier schon immer.
      \pause
      \ex Ein Abstiegskandidat gewinnen konnte hier noch kein einziges Mal.
      \pause
      \ex Ein Außenseiter gewonnen hat hier erst letzte Woche.
      \pause
      \ex Die Heimmannschaft zu gewinnen scheint dort fast jedes Mal.
      \pause
      \ex Ein Außenseiter gewonnen zu haben scheint hier noch nie.
      \pause
      \ex Ein Außenseiter zu gewinnen versucht hat dort schon oft.
      \pause
      \ex Einige Außenseiter gewonnen haben dort schon im Laufe der Jahre.
    \end{xlist}
  \end{exe}
\end{frame}

\begin{frame}
  {Kern und Peripherie}
  \pause
\begin{exe}
  \ex\label{ex:kernundperipherie022}
    \begin{xlist}
      \ex \alert{Baum, Haus, Matte, Döner, Angst, Öl, Kutsche, \ldots}
      \ex \rot{System, Kapuze, Bovist, Schlamassel, Marmelade, Melodie, \ldots}
    \end{xlist}
    \pause
    \ex
    \begin{xlist}
      \ex \alert{geht, läuft, lacht, schwimmt, liest, \ldots}
      \ex \rot{kann, muss, will, darf, soll, mag}
    \end{xlist}
    \pause
    \ex
    \begin{xlist}
      \ex \alert{des Hundes, des Geistes, des Tisches, des Fußes, \ldots}
      \ex \rot{des Schweden, des Bären, des Prokuristen, des Phantasten, \ldots}
    \end{xlist}
  \end{exe}
  \pause
  \vspace{\baselineskip}
  \Large
  \centering
  \alert{Hohe Typenhäufigkeit} vs.\ \rot{niedrige Typenhäufigkeit}.  
\end{frame}

\begin{frame}
  {Zwei verschiedene Häufigkeiten}
  \pause
  \Large\begin{block}{Typenhäufigkeit}
    Wie viele \alert{verschiedene} Realisierungen (=~Typen)\\
    einer Sorte linguistischer Einheiten gibt es?
  \end{block}

  \pause
  \vspace{\baselineskip}
  
  \begin{block}{Tokenhäufigkeit}
    Wie häufig sind die \alert{ggf.\ identischen} Realisierungen\\
    (=~Tokens) einer Sorte linguistischer Einheiten?
  \end{block}
\end{frame}

\begin{frame}
  {Regel vs.\ Regularität bzw.\ Generalisierung}
  \pause
  \begin{exe}
    \ex
    \begin{xlist}
      \ex{Relativsätze und eingebettete \textit{w}-Sätze werden nicht\\
    durch Komplementierer eingeleitet.}
      \pause
      \ex{\textit{fragen} ist ein schwaches Verb.}
      \pause
      \ex{\textit{zurückschrecken} bildet das Perfekt mit dem Hilfsverb \textit{sein}.}
      \pause
      \ex{Im Aussagesatz steht vor dem finiten Verb genau ein Satzglied.}
      \pause
      \ex{In Kausalsätzen mit \textit{weil} steht das finite Verb an letzter Stelle.}
    \end{xlist}
  \end{exe}
\end{frame}


\begin{frame}
  {Normkorm? Regularitätenkonform?}
  \pause
  \begin{exe}
    \ex
    \begin{xlist}
      \ex Dann sieht man auf der ersten Seite \alert{wann, wo und wer} \rot{dass} kommt.
      \pause
      \ex Er \rot{frägt} nach der Uhrzeit.
      \pause
      \ex Man \rot{habe} zu jener Zeit nicht vor Morden \alert{zurückgeschreckt}.
      \pause
      \ex \rot{Der Universität} \alert{zum Jubiläum} gratulierte auch Bundesminister Dorothee Wilms, die in den fünfziger Jahren in Köln studiert hatte.
      \pause
      \ex Das ist Rindenmulch, \alert{weil} hier \rot{kommt} noch ein Weg.
    \end{xlist}
  \end{exe}
\end{frame}


\begin{frame}
  {Regel und Regularität}
  \pause
  \begin{block}{Regularität}
    Eine grammatische Regularität innerhalb eines Sprachsystems liegt dann vor, wenn sich Klassen von Symbolen unter vergleichbaren Bedingungen gleich (und damit vorhersagbar) verhalten.
  \end{block}

  \pause
  \vspace{0.5\baselineskip}

  \begin{block}{Regel}
    Eine grammatische Regel ist die Beschreibung einer Regularität, die in einem normativen Kontext geäußert wird.
  \end{block}

  \pause
  \vspace{0.5\baselineskip}
  
  \begin{block}{Generalisierung}
    Eine grammatische Generalisierung ist eine durch Beobachtung zustandegekommene Beschreibung einer Regularität.
  \end{block}
\end{frame}

\begin{frame}
  {Regel vs.\ Regularität bzw.\ Generalisierung}
  Was ist dann der Status dieser Feststellungen?\\
  \Zeile
  \begin{exe}
    \ex
    \begin{xlist}
      \ex{Relativsätze und eingebettete \textit{w}-Sätze werden nicht\\
    durch Komplementierer eingeleitet.}
      \ex{\textit{fragen} ist ein schwaches Verb.}
      \ex{\textit{zurückschrecken} bildet das Perfekt mit dem Hilfsverb \textit{sein}.}
      \ex{Im Aussagesatz steht vor dem finiten Verb genau ein Satzglied.}
      \ex{In Kausalsätzen mit \textit{weil} steht das finite Verb an letzter Stelle.}
    \end{xlist}
  \end{exe}
\end{frame}



\begin{frame}
  {Norm ist Beschreibung}
  \pause
  \begin{itemize}[<+->]
    \item Norm als Grundkonsens
    \item Sprache und Norm im Wandel
    \item Norm und Situation (Register, Stil, \dots)
    \item Variation in der Norm
      \vspace{\baselineskip}
    \item \alert{Wichtigkeit der Norm, insbesondere im schulischen Deutschunterricht}
  \end{itemize}
\end{frame}




  \let\subsection\section\let\section\woopsi
  \section{Sprache und Lehramt}
  \let\woopsi\section\let\section\subsection\let\subsection\subsubsection
  \section{Kongruenz}

Finden Sie im Textausschnitt \textit{Gang durch das Ried} fünf Fälle von Subjekt-Verb-Kongruenz und fünf Nominalphrasen (Substantiv mit mindestens einem vorausgehenden Artikel oder einem vorausgehenden Adjektiv), in denen Kongruenz herrscht.




\section{Verbvalenz}

Entscheiden Sie für die numerierten und unterstrichenen Phrasen im Textausschnitt \textit{Gang durch das Ried}, ob sie Ergänzungen oder Angaben sind.
Finden Sie dafür zunächst das Verb, von dem sie abhängen, und entscheiden Sie dann, ob es sich um Ergänzungen und Angaben handelt.
Im Fall von Ergänzungen geben Sie an, welches Merkmal \slash\ welche Form das Verb regiert.
Rechnen Sie damit, dass es in einigen Fällen schwer entscheidbar ist, ob es sich um eine Ergänzung oder eine Angabe handelt.
Wir kommen im letzten drittel der Vorlesung aber nochmal darauf zurück, um etwas mehr Klarheit zu schaffen.


Zur Wiederholung: Die wichtigsten Unterschiede zwischen valenzgebundenen Ergänzungen und Angaben sind die folgenden:

\Zeile

\begin{center}
  \begin{tabular}[h]{lll}
    \toprule
                         & \textbf{Ergänzungen} & \textbf{Angaben} \\
    \midrule
    \textbf{Semantik} & verbgebunden & verbunabhängig \\
    \textbf{Weglassbarkeit} & manchmal\slash oft obligatorisch & immer fakultativ \\
    \textbf{Kasus\slash Präposition\slash\ldots} & regiert & frei \\
    \textbf{Lizenzierung} & einmalig & iterierbar \\
    \midrule
    \textbf{Schultermini} & Subjekt, Objekte & adverbiale Bestimmung \\
    \bottomrule
  \end{tabular}
\end{center}

\newpage

\begin{center}
  \begin{tabular}[h]{cp{0.2\textwidth}lp{0.3\textwidth}}
    \toprule
    &\textbf{Verb} & \textbf{Status} & \textbf{regiertes Merkmal} \\
    \midrule
    &&& \\
     (1) & & \Square~Ergänzung\ \ \ \Square~Angabe & \\ \cline{2-2}\cline{4-4} 
    &&& \\
     (2) & & \Square~Ergänzung\ \ \ \Square~Angabe & \\ \cline{2-2}\cline{4-4} 
    &&& \\
     (3) & & \Square~Ergänzung\ \ \ \Square~Angabe & \\ \cline{2-2}\cline{4-4} 
    &&& \\
     (4) & & \Square~Ergänzung\ \ \ \Square~Angabe & \\ \cline{2-2}\cline{4-4} 
    &&& \\
     (5) & & \Square~Ergänzung\ \ \ \Square~Angabe & \\ \cline{2-2}\cline{4-4} 
    &&& \\
     (6) & & \Square~Ergänzung\ \ \ \Square~Angabe & \\ \cline{2-2}\cline{4-4} 
    &&& \\
     (7) & & \Square~Ergänzung\ \ \ \Square~Angabe & \\ \cline{2-2}\cline{4-4} 
    &&& \\
     (8) & & \Square~Ergänzung\ \ \ \Square~Angabe & \\ \cline{2-2}\cline{4-4} 
    &&& \\
     (9) & & \Square~Ergänzung\ \ \ \Square~Angabe & \\ \cline{2-2}\cline{4-4} 
    &&& \\
    (10) & & \Square~Ergänzung\ \ \ \Square~Angabe & \\ \cline{2-2}\cline{4-4} 
    &&& \\
    (11) & & \Square~Ergänzung\ \ \ \Square~Angabe & \\ \cline{2-2}\cline{4-4} 
    &&& \\
    (12) & & \Square~Ergänzung\ \ \ \Square~Angabe & \\ \cline{2-2}\cline{4-4} 
    &&& \\
    (13) & & \Square~Ergänzung\ \ \ \Square~Angabe & \\ \cline{2-2}\cline{4-4} 
    &&& \\
    (14) & & \Square~Ergänzung\ \ \ \Square~Angabe & \\ \cline{2-2}\cline{4-4} 
    &&& \\
    (15) & & \Square~Ergänzung\ \ \ \Square~Angabe & \\ \cline{2-2}\cline{4-4} 
    &&& \\
    (16) & & \Square~Ergänzung\ \ \ \Square~Angabe & \\ \cline{2-2}\cline{4-4} 
    &&& \\
    (17) & & \Square~Ergänzung\ \ \ \Square~Angabe & \\ \cline{2-2}\cline{4-4} 
    &&& \\
    (18) & & \Square~Ergänzung\ \ \ \Square~Angabe & \\ \cline{2-2}\cline{4-4} 
    &&& \\
    (19) & & \Square~Ergänzung\ \ \ \Square~Angabe & \\ \cline{2-2}\cline{4-4} 
    &&& \\
    (20) & & \Square~Ergänzung\ \ \ \Square~Angabe & \\ \cline{2-2}\cline{4-4} 
    &&& \\
    (21) & & \Square~Ergänzung\ \ \ \Square~Angabe & \\ \cline{2-2}\cline{4-4} 
    &&& \\
    (22) & & \Square~Ergänzung\ \ \ \Square~Angabe & \\ \cline{2-2}\cline{4-4} 
  \end{tabular}
\end{center}



\section{Text}

\begin{nohyphens}
  \begin{quote}
    \textbf{Elisabeth Langgässer: Gang durch das Ried (Anfang)}\\
    Verlag Jakob Hegner 1936, S. 1\\[0.5\baselineskip]

    Im Spätherbst des Jahres 1930 ging ein Mann (1)\ul{über das verlassene französische Lager, das früher ein deutsches gewesen war und sich zwischen der hessischen Hauptstadt, umschließenden Tannen- und Birkenwäldern und dem großen Sande dahinzieht}.
    (2)\ul{Es} nieselte langsam (3)\ul{vom Himmel herunter}, der Mann schlug (4)\ul{den Kragen der Jacke} hoch und rückte das Kappenschild noch tiefer in die Stirne.
    (5)\ul{Auf den breiten Kasernenstraßen}, die (6)\ul{durch leere Barackenreihen}, (7)\ul{an Stallungen, Vorratshäusern und Kantinen} vorüberführten, wuchs dichtes, grünbraunes Gras, das jeden Schritt verschluckte und den Wandernden wesenlos wie eine Traumgestalt machte, die, wenn sie auch rufen würde, (8)\ul{von niemand} gehört werden könnte.
    Noch vor kurzem hatten hier Feuerwerker aus Koblenz und Ludwigshafen den Übungsplatz abgesucht und die Blindgänger, Handgranaten und letzten Depots gesprengt – die Erde war damals zerstampft und (9)\ul{der Himmel} (10)\ul{von dem Echo jener dumpfen Schläge} erfüllt gewesen, die (11)\ul{bis in das Ried hinein} und noch weiterhin spürbar waren.

      Jetzt aber herrschte (12)\ul{Stille}, eine blöde Taubheit gleichsam, die sich wohl noch (13)\ul{der Töne} erinnert, doch so sehr mit ihnen gesättigt ist, daß sie nichts mehr vernehmen kann.
      Manche Fensterscheibe war da und dort durch die Erschütterung eingefallen und starrte gezackt wie ein schwarzer Stern aus der bröckelnden Mauerfüllung; (14)\ul{der rostigen Angel} enthoben, hing eine morsche Tür schief zu der eigenen Achse; eine andere schlug unaufhörlich, von dem Zugwind angetrieben, bis zur Hälfte der Schwelle vor, wo (15)\ul{ein üppiges Mooskissen} wucherte, (16)\ul{das} sie geräuschlos abfing.
      Auch ein paar fetzige Wellblechbaracken standen (17)\ul{neben den Backsteinbauten}; sie waren (18)\ul{rötlichgelb} angelaufen und (19)\ul{von der großen Versteigerung} vor Wochen übriggeblieben.

      Was diese Versteigerung anging, so konnte man damals glauben, in einer Stadt zu sein, die (20)\ul{von Erdkatastrophen} verschüttet gewesen und dann wieder ausgegraben und aufgebaut worden war: (21)\ul{unter freiem Himmel} stand, abgenutzt, das Inventar der Kasernen – alte Schränke, die jammervoll quietschten, verwanzte Betten und Öfen, welche glatt auseinanderfielen, ein paar Schemel mit starrenden Beinen, befleckte Bänke und Tische, (22)\ul{deren Holz}, wo es irgend anging, unzüchtig tätowiert war, nutzlose Eisenteile, die von Lumpensammlern hinausgefahren und auf halbem Weg wieder verloren wurden.
Nur einige Feldbettstellen waren ungefragt hiergeblieben und jene Wellblechbaracken, die, zerrissen, als ob eine Schere sie geschlitzt und geschnitten hätte, ja, teilweise schon zusammengebrochen, in dem weiten Gelände ruhten und den Eindruck riesiger Raupen oder Fabeltiere machten, welche rasselnd niedergesunken, doch immer noch gefährlich und voll tückischer Drohung sind.
Der Krieg hatte, wie ihm gemäß ist, wenn er irgendwo Abschied nimmt, seine leere Schale zurückgelassen, diese armselig rohen Kasernen, in denen er noch immer so gegenwärtig war, daß selbst die Allerärmsten sich nicht entschließen konnten, (23)\ul{in den verlassenen Höhlen einen Unterschlupf zu suchen}.
  \end{quote}
\end{nohyphens}


  
\fi

\makeatletter
\setcounter{lastpagemainpart}{\the\c@framenumber}
\makeatother

\appendix

\begin{frame}[allowframebreaks]
  {Literatur}
  \renewcommand*{\bibfont}{\footnotesize}
  \setbeamertemplate{bibliography item}{}
  \printbibliography
\end{frame}

\begin{frame}
  {Autor}
  \begin{block}{Kontakt}
    Prof.\ Dr.\ Roland Schäfer\\
    Institut für Germanistische Sprachwissenschaft\\
    Friedrich-Schiller-Universität Jena\\
    Fürstengraben 30\\
    07743 Jena\\[\baselineskip]
    \url{https://rolandschaefer.net}\\
    \texttt{roland.schaefer@uni-jena.de}
  \end{block}
\end{frame}

\begin{frame}
  {Lizenz}
  \begin{block}{Creative Commons BY-SA-3.0-DE}
    Dieses Werk ist unter einer Creative Commons Lizenz vom Typ \textit{Namensnennung - Weitergabe unter gleichen Bedingungen 3.0 Deutschland} zugänglich.
    Um eine Kopie dieser Lizenz einzusehen, konsultieren Sie \url{http://creativecommons.org/licenses/by-sa/3.0/de/} oder wenden Sie sich brieflich an Creative Commons, Postfach 1866, Mountain View, California, 94042, USA.
  \end{block}
\end{frame}

\mode<beamer>{\setcounter{framenumber}{\thelastpagemainpart}}

\end{document}
