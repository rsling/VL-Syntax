\def\GRAPHPATH{localgraphics}

\ifdefined\HANDOUT
  \documentclass[handout,aspectratio=1610,dvipsnames]{beamer}
  \def\GRAPHPATH{graphics}
\else
  \documentclass[aspectratio=1610,dvipsnames]{beamer}
\fi

\usepackage[margin=2cm]{geometry}
\usepackage[ngerman]{babel}

\usepackage{setspace}
\usepackage{booktabs}
\usepackage{array,graphics}
\usepackage{color}
\usepackage{soul}
\usepackage[linecolor=gray,backgroundcolor=yellow!50,textsize=tiny]{todonotes}
\usepackage[linguistics]{forest}
\usepackage{multirow}
\usepackage{longtable}
\usepackage{pifont}
\usepackage{wasysym}
\usepackage{langsci-gb4e}
\usepackage{soul}
\usepackage{enumitem}
\usepackage{marginnote}
\usepackage{ulem}
\usepackage{hyperref}
\usepackage{tikz}
\usetikzlibrary{arrows,positioning} 

\usepackage{fontspec}

\author{Prof.\ Dr.\ Roland Schäfer | Germanistische Linguistik FSU Jena}
\title{Syntax | \TITLE%
  \ifdefined\SOLUTIONS \gruen{\\Musterlösung} \fi
}
\date{Version 2024}

\definecolor{rot}{rgb}{0.7,0.2,0.0}
\newcommand{\rot}[1]{\textcolor{rot}{#1}}
\definecolor{blau}{rgb}{0.1,0.2,0.7}
\newcommand{\blau}[1]{\textcolor{blau}{#1}}
\definecolor{gruen}{rgb}{0.0,0.7,0.2}
\newcommand{\gruen}[1]{\textcolor{gruen}{#1}}
\definecolor{grau}{rgb}{0.6,0.6,0.6}
\newcommand{\grau}[1]{\textcolor{grau}{#1}}
\definecolor{orongsch}{RGB}{255,165,0}
\newcommand{\orongsch}[1]{\textcolor{orongsch}{#1}}
\definecolor{tuerkis}{RGB}{63,136,143}
\definecolor{braun}{RGB}{108,71,65}
\newcommand{\tuerkis}[1]{\textcolor{tuerkis}{#1}}
\newcommand{\braun}[1]{\textcolor{braun}{#1}}

\newcommand*\Rot{\rotatebox{75}}

\newcommand{\zB}{z.\,B.\ }
\newcommand{\ZB}{Z.\,B.\ }
\newcommand{\Sub}[1]{\ensuremath{_{\text{#1}}}}
\newcommand{\Up}[1]{\ensuremath{^{\text{#1}}}}
\newcommand{\UpSub}[2]{\ensuremath{^{\text{#1}}_{\text{#2}}}}
\newcommand{\Doppelzeile}{\vspace{2\baselineskip}}
\newcommand{\Zeile}{\vspace{\baselineskip}}
\newcommand{\Halbzeile}{\vspace{0.5\baselineskip}}
\newcommand{\Viertelzeile}{\vspace{0.25\baselineskip}}

\newcommand{\whyte}[1]{\textcolor{white}{#1}}

\newcommand{\Spur}[1]{t\Sub{#1}}
\newcommand{\Ti}{\Spur{1}}
\newcommand{\Tii}{\Spur{2}}
\newcommand{\Tiii}{\Spur{3}}
\newcommand{\Tiv}{\Spur{4}}
\newcommand*{\mybox}[1]{\framebox{#1}}
\newcommand\ol[1]{{\setul{-0.9em}{}\ul{#1}}}

\newenvironment{nohyphens}{%
  \par
  \hyphenpenalty=10000
  \exhyphenpenalty=10000
  \sloppy
}{\par}

\newcommand{\Lf}{
  \setlength{\itemsep}{1pt}
  \setlength{\parskip}{0pt}
  \setlength{\parsep}{0pt}
}

\forestset{
  Ephr/.style={draw, ellipse, thick, inner sep=2pt},
  Eobl/.style={draw, rounded corners, inner sep=5pt},
  Eopt/.style={draw, rounded corners, densely dashed, inner sep=5pt},
  Erec/.style={draw, rounded corners, double, inner sep=5pt},
  Eoptrec/.style={draw, rounded corners, densely dashed, double, inner sep=5pt},
  Ehd/.style={rounded corners, fill=gray, inner sep=5pt,
    delay={content=\whyte{##1}}
  },
  Emult/.style={for children={no edge}, for tree={l sep=0pt}},
  phrasenschema/.style={for tree={l sep=2em, s sep=2em}},
  sake/.style={tier=preterminal},
  ake/.style={
    tier=preterminal
    },
}

\forestset{
  decide/.style={draw, chamfered rectangle, inner sep=2pt},
  finall/.style={rounded corners, fill=gray, text=white},
  intrme/.style={draw, rounded corners},
  yes/.style={edge label={node[near end, above, sloped, font=\scriptsize]{Ja}}},
  no/.style={edge label={node[near end, above, sloped, font=\scriptsize]{Nein}}}
}


\useforestlibrary{edges}

\forestset{
  narroof/.style={roof, inner xsep=-0.25em, rounded corners},
  forky/.style={forked edge, fork sep-=7.5pt},
  bluetree/.style={for tree={blau}, for children={edge=blau}},
  orongschtree/.style={for tree={orongsch}, for children={edge=orongsch}},
  rottree/.style={for tree={rot}, for children={edge=rot}},
  gruentree/.style={for tree={gruen}, for children={edge=gruen}},
  tuerkistree/.style={for tree={tuerkis}, for children={edge=tuerkis}},
  brauntree/.style={for tree={braun}, for children={edge=braun}}, 
  grautree/.style={for tree={grau}, for children={edge=grau}}, 
  gruennode/.style={gruen, edge=gruen},
  graunode/.style={grau, edge=grau},
  whitearc/.style={for children={edge=white}},
}

\defaultfontfeatures{Ligatures=TeX,Numbers=OldStyle, Scale=MatchLowercase}
\setmainfont{Linux Libertine O}
\setsansfont{Linux Biolinum O}

\setlength{\parindent}{0pt}

\newenvironment{spread}
{%
  \newdimen\origiwspc%
  \newdimen\origiwstr%
  \origiwspc=\fontdimen2\font%
  \origiwstr=\fontdimen3\font%
  \fontdimen2\font=1em%
  \doublespacing%
}{%
  \fontdimen2\font=\origiwspc%
  \fontdimen3\font=\origiwstr%
}


\newcommand{\Sol}[1]{
  \ifdefined\SOLUTIONS
    \gruen{#1}
  \fi
}

\newcommand{\Solalt}[2]{
  \ifdefined\SOLUTIONS
    \gruen{#1}
  \else
    #2
  \fi
}

\newcounter{aufgabe}
\newcommand{\aufgabeginn}{\setcounter{aufgabe}{1}}
\newcommand{\aufg}{(\arabic{aufgabe})\ \stepcounter{aufgabe}}




\ifdefined\TITLE
  \title[Syntax | \StrSubstitute{\TITLE}{+}{ }]{Deutsche Syntax\\\StrSubstitute{\TITLE}{+}{ }}
\else
  \title[Deutsche Syntax]{Deutsche Syntax}
\fi

\author{Roland Schäfer}
\institute[FSU Jena]{Institut für Germanistische Sprachwissenschaft\\Friedrich-Schiller-Universität Jena}
\date[EGBD3]{Diese Version ist vom \today.\\\Zeile%
  \scriptsize \grau{stets aktuelle Fassungen: \url{https://github.com/rsling/VL-Deutsche-Syntax}}}

\begin{document}

\begingroup
  \setbeamertemplate{navigation symbols}{}
  \begin{frame}[noframenumbering,plain]
   \titlepage
  \end{frame}

  \ifdefined\TITLE
    \begin{frame}[noframenumbering,plain]
      \centering 
      \begin{minipage}[c]{0.975\textwidth}
      \begin{block}
        {\rot{Hinweise für diejenigen, die die Klausur bestehen möchten}}
        \begin{enumerate}
          \item Folien sind niemals selbsterklärend und nicht zum Selbststudium geeignet.\\
            Sie müssen sich die Videos ansehen und regelmäßig das Seminar besuchen.
          \item Ohne eine gründliche Lektüre der angegebenen Abschnitte des Buchs\\
            bestehen Sie die Klausur nicht.
            Das Buch definiert den Klausurstoff.
          \item Arbeiten Sie die entsprechenden Übungen im Buch durch.
            Nichts hilft\\
            Ihnen besser, um sich auf die Klausur vorzubereiten.
          \item \rot{Beginnen Sie spätestens jetzt mit dem Lernen.}
            \Zeile
          \item \rot{Langjähriger Erfahrungswert:
            Wenn Sie diese Hinweise nicht berücksichtigen, bestehen Sie die Klausur wahrscheinlich nicht.}
        \end{enumerate}
      \end{block}
      \end{minipage}
    \end{frame}
  \else
  \begin{frame}{Inhalt}
    \centering 
    \scalebox{0.9}{\begin{minipage}{\textwidth}
      \begin{multicols}{2}
        \tableofcontents
      \end{multicols}
    \end{minipage}}
    \end{frame}
  \fi
\endgroup

\ifdefined\TITLE
  \input{includes/\TITLE}
\else
  \section[Grammatik]{Grammarik}
  \let\woopsi\section\let\section\subsection\let\subsection\subsubsection
  
\section{Organisation}

\begin{frame}
  {Roland Schäfer}
  \onslide<+->
  \begin{itemize}[<+->]
    \item seit WS 2022\slash 2023 Professur für Grammatik und Lexikon
    \item 2020--2022 Forschungsstelle an der HU Berlin
    \item 2018 habilitiert an der HU Berlin\\
      (Germanistische Linguistik und allgemeine Sprachwissenschaft)
    \item 2007--2022 Mitarbeiter an der FU Berlin
    \item 2008 promoviert an der Uni Göttingen (Englische Syntax)
    \item 2002--2007 Mitarbeiter in der Sprachwissenschaft in Göttingen
    \item Studium in Marburg (Sprachwissenschaft, Japanologie)
  \end{itemize}
  \Zeile
  \onslide<+->
  Bitte nennen Sie mich nicht Professor\ldots\ \onslide<+-> Wenn Sie es tun, dann bitte richtig:\\
  \url{https://rolandschaefer.net/regeln-fur-den-mailverkehr/}
\end{frame}

\begin{frame}
  {Forschung}
  \onslide<+->
  \onslide<+->
  Linguistik (des Deutschen)\\
  \Halbzeile
  \begin{itemize}[<+->]
    \item kognitiv fundierte Grammatik
    \item Morphosyntax und Graphematik
    \item grammatische Variation ("`Zweifelsfälle"')
    \item individuelle Variation
    \item Registervariation
    \item Epistemologie
  \end{itemize}
  \Zeile
  \onslide<+->
  Methoden\\
  \Halbzeile
  \begin{itemize}[<+->]
    \item Korpuserstellung und -analyse
    \item verhaltensbasierte Experimente
    \item Fragen der statistischen Inferenz
  \end{itemize}
\end{frame}

\begin{frame}
  {Ablauf und Inhalte der Vorlesung}
  \begin{itemize}
    \item 13 Sitzungen über Grammatik und Syntax des Deutschen
    \item Meine Inhalte entsprechen meiner \alert{\textit{Einführung in\\
      die grammatische Beschreibung des Deutschen}} \grau{\citep{Schaefer2018b}}
    \item \url{http://langsci-press.org/catalog/book/224} (\alert{open access})
      \vspace{\baselineskip}
    \item Bei Amazon für 20€\\
      \url{https://www.amazon.de/dp/3961101183/}
  \end{itemize}
\end{frame}

\begin{frame}
  {Fragen und Interaktion}
  \begin{itemize}
    \item Interaktion in einer VL ist immer schwierig!\\
      Ich versuche es ggf.\ trotzdem.
      \Zeile
    \item Wenn Sie Fragen zum Stoff oder zum Buch haben:
      \texttt{roland.schaefer@uni-jena.de}
      \Zeile
    \item Mein Youtube-Kanal (demnächst wieder lebendig):\\
      \url{https://www.youtube.com/channel/UCc0SUpRSVvU2jJxx4rRBdsg}
  \end{itemize}
\end{frame}

\begin{frame}
  {Der Plan für heute}
  \pause
  \begin{itemize}
    \item Sprache
    \item Grammatik
    \item Grammatikalität
    \item Akzeptabilität
    \Zeile
    \item EGBD3: Kapitel 1
  \end{itemize}
\end{frame}


\section{Grammatik}

\begin{frame}
  {Deutsche Sätze erkennen und interpretieren}
  \pause
  \begin{exe}
    \ex Dies ist ein Satz.
  \pause
    \ex Satz dies ein ist.
  \pause
    \ex Kno kna knu.
  \pause
    \ex This is a sentence.
  \pause
    \vspace{\baselineskip}
    \ex Dies ist ein Satz
  \end{exe}
\end{frame}


\begin{frame}
  {Form und Bedeutung: Kompositionalität}
  \begin{exe}
    \ex Das ist ein Kneck.
    \pause
    \vspace{\baselineskip}
  \ex Jede Farbe ist ein Kurzwellenradio.
  \ex Der dichte Tank leckt.
\end{exe}
    \vspace{\baselineskip}
  \pause

  \Large\begin{block}{Kompositionalität}
    Die Bedeutung komplexer sprachlicher Ausdrücke ergibt sich aus der Bedeutung ihrer Teile und der Art ihrer grammatischen Kombination. 
    Diese Eigenschaft von Sprache nennt man Kompositionalität.
  \end{block}
\end{frame}

\begin{frame}
  {Grammatik als System und Grammatikalität}
  \pause

  \Large\begin{block}{Grammatik}
    Eine Grammatik ist ein \alert{System von Regularitäten}, nach denen aus einfachen Einheiten komplexe Einheiten einer Sprache gebildet werden.
  \end{block}
  \vspace{\baselineskip}

  \pause

  \begin{block}{Grammatikalität}
    Jede von einer bestimmten Grammatik beschriebene Symbolfolge ist \alert{grammatisch} relativ zu dieser Grammatik, alle anderen sind \alert{ungrammatisch}.
  \end{block}
\end{frame}

\begin{frame}
  {(Un)grammatisch ist nicht gleich (in)akzeptabel}
  \pause
  \begin{exe}
    \ex\begin{xlist}
      \ex Bäume wachsen werden hier so schnell nicht wieder.
      \pause
      \ex Touristen übernachten sollen dort schon im nächsten Sommer.
      \pause
      \ex Schweine sterben müssen hier nicht.
      \pause
      \ex Der letzte Zug vorbeigekommen ist hier 1957.
      \pause
      \ex Das Telefon geklingelt hat hier schon lange nicht mehr.
      \pause
      \ex Häuser gestanden haben hier schon immer.
      \pause
      \ex Ein Abstiegskandidat gewinnen konnte hier noch kein einziges Mal.
      \pause
      \ex Ein Außenseiter gewonnen hat hier erst letzte Woche.
      \pause
      \ex Die Heimmannschaft zu gewinnen scheint dort fast jedes Mal.
      \pause
      \ex Ein Außenseiter gewonnen zu haben scheint hier noch nie.
      \pause
      \ex Ein Außenseiter zu gewinnen versucht hat dort schon oft.
      \pause
      \ex Einige Außenseiter gewonnen haben dort schon im Laufe der Jahre.
    \end{xlist}
  \end{exe}
\end{frame}





  \let\subsection\section\let\section\woopsi

  \section[Grundbegriffe]{Grundbegriffe}
  \let\woopsi\section\let\section\subsection\let\subsection\subsubsection
  \section{Überblick}

\begin{frame}
  {Überblick}
  \onslide<+->
  \begin{itemize}[<+->]
    \item \alert{Strukturbildung} | große Einheiten aus kleinen Einheiten
    \Zeile
    \item \alert{Relationen} | Kongruenz und Valenz
    \Zeile
    \item \alert{Valenz} | Verbklassen und Ereignisbeschreibung
  \end{itemize}
\end{frame}

\section{Struktur}

\begin{frame}
  {Sprachliche Einheiten und ihre Bestandteile}
  \onslide<+->
  \onslide<+->
  Wichtig vor allem für die Syntax | \alert{Strukturbildung}\\
  \Zeile
  \begin{itemize}[<+->]
    \item\footnotesize \alert{Satz} \\
      {Nadezhda reißt die Hantel souveräner als andere Gewichtheberinnen.}
      \Halbzeile

    \item\footnotesize \alert{Satzteile} \\
      {Nadezhda | reißt | die Hantel | souveräner als andere Gewichtheberinnen}
      \Halbzeile

    \item\footnotesize \alert{Wörter} \\
      {Nadezhda | reißt | die | Hantel | souveräner | als | andere | Gewichtheberinnen}
      \Halbzeile

    \item\footnotesize \alert{Wortteile} \\
      {Nadezhda | reiß | t | d | ie | Hantel | souverän | er | als | ander | e | Gewicht | heb | er | inn | en}
      \Halbzeile

    \item\footnotesize \alert{Laute\slash Buchstaben} \\
      {N | a | d | e | z | h | d | a \ldots}
  \end{itemize}
\end{frame}


\begin{frame}
  {Syntaktische Strukturen}
  \onslide<+->
  \onslide<+->
  \begin{center}
  \resizebox{0.8\textwidth}{!}{\begin{forest}
    [Nadezhda reißt die Hantel souveräner als andere Gewichtheberinnen
      [Nadezhda]
      [reißt]
      [die Hantel
        [die]
        [Hantel]
      ]
      [souveräner als andere Gewichtheberinnen
        [souveräner]
        [als andere Gewichtheberinnen
          [als]
          [andere Gewichtheberinnen
            [andere]
            [Gewichtheberinnen]
          ]
        ]
      ]
    ]
  \end{forest}}
  \end{center}
\end{frame}


\begin{frame}
  {Struktur in der Morphologie}
  \onslide<+->
  \onslide<+->
  Auch innerhalb von Wörtern gibt es solche Strukturen.\\
  \Zeile
  \onslide<+->
  \centering 
    \begin{forest}
    [gegnerische
      [generisch
        [gegner]
        [isch, tier=terminal]
      ]
      [e, tier=terminal]
    ]
  \end{forest}
\end{frame}

\begin{frame}
  {Konstituenten}
  \onslide<+->
  \onslide<+->
  \centering 
  \begin{block}
    {Konstituenten einer Struktur}
    \textit{Konstituenten} einer Einheit sind die (meistens kleineren und höchstens genauso großen) Einheiten, aus denen eine Struktur besteht.    
  \end{block}
\end{frame}

\section{Rektion und Kongruenz}

\begin{frame}
  {Was sind Relationen?}
  \onslide<+->
  \onslide<+->
  \begin{exe}
    \ex\label{ex:rektionundkongruenz024}
    \begin{xlist}
      \ex{\label{ex:rektionundkongruenz025}[Martin] \alert{[zeigt] } \orongsch{[einen Schraubensprung]}.}
      \ex{\label{ex:rektionundkongruenz026}[Tina] \alert{[springt]} \gruen{[kraftvoll]}.}
    \end{xlist}
  \end{exe}
  \Zeile
  \begin{itemize}[<+->]
    \item \orongsch{\textit{einen Schraubensprung}} ist ein \orongsch{Objekt} zu \textit{zeigt}.
    \item \gruen{\textit{kraftvoll}} ist eine \gruen{adverbiale Bestimmung} zu \textit{springt}.
      \Zeile
    \item Es gibt kein Objekt und keine adverbiale Bestimmung\\
      ohne ein Verb im Satzkontext \ldots
    \item die Begriffe \orongsch{Objekt} und \gruen{adverbiale Bestimmung} sind also \alert{relational}.
  \end{itemize}
\end{frame}

\begin{frame}
  {Syntaktische Strukturen und morphologische Merkmale}
  \onslide<+->
  \onslide<+->
  \begin{center}
  \resizebox{0.8\textwidth}{!}{\begin{forest}
    [Nadezhda reißt die Hantel souveräner als andere Gewichtheberinnen
      [Nadezhda]
      [reißt]
      [die Hantel, alt=<3->{orongsch}{}
        [die, alt=<4->{orongsch}{}]
        [Hantel, alt=<5->{orongsch}{}]
      ]
      [souveräner als andere Gewichtheberinnen
        [souveräner]
        [als andere Gewichtheberinnen
          [als]
          [andere Gewichtheberinnen, alt=<6->{gruen}{}
            [andere, alt=<7->{gruen}{}]
            [Gewichtheberinnen, alt=<8->{gruen}{}]
          ]
        ]
      ]
    ]
  \end{forest}}
  \end{center}

  \Zeile
  \onslide<9->{Übereinstimmung von Merkmalen in syntaktischen Gruppen\\}
  \onslide<10->{\orongsch{Akkusativ Femininum Singular}} \onslide<11->{| \gruen{Nominativ Plural}}
\end{frame}

\begin{frame}
  {Kongruenz | NPs}
  \onslide<+->
  \onslide<+->
  \alert{Kongruenz} | Merkmalübereinstimmung in Nominalphrasen\\
  \Zeile
  \Zeile
  \centering 
  \onslide<+->
  \begin{tikzpicture}[node distance=1.5cm, auto]
    \node (context) {Wir möchten};
    \node [right=of context] (diesen) {\alert<5->{diesen}};
    \node[right=of diesen] (schönen) {\alert<4->{schönen}};
    \node[right=of schönen] (Sportwagen) {\alert<4->{Sportwagen}};
    \onslide<4->{\path[<->, trueblue, draw, bend left=30] (schönen) edge node {\scriptsize Akk Mask Sg} (Sportwagen);}
    \onslide<5->{\path[<->, trueblue, draw, bend left=60] (diesen) edge node[above] {\scriptsize Akk Mask Sg} (Sportwagen);}
    \onslide<6->{\path[<->, trueblue, draw, bend left=30] (diesen) edge node[above] {\scriptsize Akk Mask Sg} (schönen);}
  \end{tikzpicture}
\end{frame}

\begin{frame}
  {Kongruenz | Subjekt und finites Verb}
  \onslide<+->
  \onslide<+->
  \alert{Kongruenz} | Merkmalübereinstimmung zwischen Subjekt und finitem Verb\\
  \Zeile
  \Zeile
  \centering 
  \onslide<+->
  \begin{tikzpicture}[node distance=1cm, auto]
   \node                      (context)    {Ich glaube, dass};
   \node[right=of context]    (ihr)        {\alert<4->{ihr}};
   \node[right=of ihr]        (den)        {den};
   \node[right=of den]        (Wagen)      {Wagen};
   \node[right=of Wagen]      (anschieben) {anschieben};
   \node[right=of anschieben] (müsst)      {\alert<4->{müsst}};
   \onslide<4->{\path[<->, trueblue, draw, bend left=30]  (ihr) edge node {\small 2.~Per Pl} (müsst);}
  \end{tikzpicture}
\end{frame}

\begin{frame}
  {Rektion | Präpositionen}
  \onslide<+->
  \onslide<+->
  \gruen{Rektion} | Präpositionen bestimmen den Kasus von ganzen \alert{Nominalphrasen}\\
  \Zeile
  \Zeile
  \centering 
  \onslide<+->
  \begin{tikzpicture}[node distance=1cm, auto]
   \node                      (context)    {Wir fahren};
   \node[right=of context]    (mit)        {\gruen<4->{mit}};
   \node[right=of mit]        (dem)        {\alert<6->{dem}};
   \node[right=of dem]        (neuen)      {\alert<5->{neuen}};
   \node[right=of neuen]      (Wagen)      {\alert<4->{Wagen}};
   \node[right=of Wagen]      (rest)       {nach hause};
   \onslide<4->{\path[->, gruen, draw, bend right=30]  (mit) edge node[below] {Dat} (Wagen);}
   \onslide<5->{\path[<->, trueblue, draw, bend left=30]  (neuen) edge node {\footnotesize Dat Mask Sg} (Wagen);}
   \onslide<6->{\path[<->, trueblue, draw, bend left=30]  (dem) edge node {\footnotesize Dat Mask Sg} (neuen);}
  \end{tikzpicture}
\end{frame}

\begin{frame}
  {Rektion | Verben}
  \onslide<+->
  \onslide<+->
  \gruen{Rektion} | Verben bestimmen den Kasus von ganzen \alert{Nominalphrasen}\\
  \Zeile
  \Zeile
  \centering 
  \onslide<+->
  \begin{tikzpicture}[node distance=1cm, auto]
    \node                      (Nom)        {\alert<4->{Ich}};
    \node[right=of Nom]        (V)          {\gruen<4->{gab}};
    \node[right=of V]          (Dat)        {\alert<5->{dem netten Kollegen}};
    \node[right=of Dat]        (Akk)        {\alert<6->{den Stift}};
    \node[right=of Akk]        (rest)       {zurück};
    \onslide<4->{\path[->, gruen, draw, bend right=-30]  (V) edge node[below] {Nom} (Nom);}
    \onslide<5->{\path[->, gruen, draw, bend right=30]  (V) edge node[below] {Dat} (Dat);}
    \onslide<6->{\path[->, gruen, draw, bend right=60]  (V) edge node[below] {Akk} (Akk);}
  \end{tikzpicture}
\end{frame}


\section{Valenz}

\begin{frame}
  {Traditionelle Verbtypen}
  \pause
  \begin{itemize}[<+->]
    \item traditionelle Termini für Verbtypen (s.\ Kapitel 14 für Neuordnung)
      \Halbzeile
      \begin{itemize}[<+->]
        \item \alert{intransitiv}: regiert nur einen Nominativ (\textit{leben}, \textit{schlafen})
          \Viertelzeile
        \item \alert{transitiv}: regiert einen Nominativ und einen Akkusativ (\textit{sehen}, \textit{lesen})
          \Viertelzeile
        \item \alert{ditransitiv}: regiert zusätzlich einen Dativ (\textit{geben}, \textit{schicken})
          \Viertelzeile
        \item \alert{präpositional transitiv}: regiert Nom und PP (\textit{leiden +unter})
          \Viertelzeile
        \item \alert{präpositional ditransitiv}: regiert Nom, Akk, PP (\textit{schreiben +an})
          \Viertelzeile
        \item \ldots
          \Zeile
      \end{itemize}
    \item nur Abkürzungen für einige (von sehr viel mehr) \alert{Valenztypen}
  \end{itemize}
\end{frame}

\begin{frame}
  {Ergänzungen und Angaben}
  \pause
  Wo wollen wir denn hin?\\
  \pause\Halbzeile
  \begin{exe}
    \ex\label{ex:valenz034}
    \begin{xlist}
      \ex{Gabriele malt \alert{[ein Bild]}.}
      \pause
      \ex{Gabriele malt \gruen{[gerne]}.}
      \pause
      \ex{Gabriele malt \gruen{[den ganzen Tag]}.}
      \pause
      \ex{Gabriele malt \gruen{[ihrem Mann]} \rot{[zu figürlich]}.}
    \end{xlist}
  \end{exe}
  \pause\Halbzeile
  \begin{itemize}[<+->]
    \item \alert{[ein Bild]} mit besonderer Relation zum Verb
    \item "`Weglassbarkeit"' (Optionalität) nicht entscheidend
  \end{itemize}
\end{frame}

\begin{frame}
  {Lizenzierung}
  \pause
  \begin{exe}
    \ex 
    \begin{xlist}
      \ex[ ]{Gabriele isst \gruen{[den ganzen Tag]} Walnüsse.}
    \pause
      \ex[ ]{Gabriele läuft \gruen{[den ganzen Tag]}.}
      \pause
      \ex[ ]{Gabriele backt ihrer Schwester \gruen{[den ganzen Tag]} Stollen.}
      \pause
      \ex[ ]{Gabriele litt \gruen{[den ganzen Tag]} unter Sonnenbrand.}
    \end{xlist}
    \pause\Halbzeile
    \ex 
    \begin{xlist}
      \ex[*]{Gabriele isst \alert{[ein Bild]} Walnüsse.}
      \pause
      \ex[*]{Gabriele läuft \alert{[ein Bild]}.}
      \pause
      \ex[*]{Gabriele backt ihrer Schwester \alert{[ein Bild]} Stollen.}
      \pause
      \ex[*]{Gabriele litt \alert{[ein Bild]} unter Sonnenbrand. }
      \pause
    \end{xlist}
  \end{exe}
  \Halbzeile
  \begin{itemize}[<+->]
    \item \gruen{Angaben} sind verb-unspezifisch lizenziert
    \item \alert{Ergänzungen} sind verb(klassen)spezifisch \alert{genau einmal} lizenziert
    \item \rot{Valenz = Liste der Ergänzungen eines lexikalischen Worts}
  \end{itemize}
\end{frame}


\begin{frame}
  {Iterierbarkeit | Angaben sind beliebig stapelbar}
  \onslide<+->
  \onslide<+->
  \begin{exe}
    \ex[ ]{Wir müssen den Wagen\\
      \gruen{[jetzt]}\\
      \gruen{[mit aller Kraft]}\\
      \gruen{[vorsichtig]} anschieben.}
    \onslide<+->
    \ex[ ]{Wir essen \gruen{[schnell]}\\
    \gruen{[mit Appetit]}\\
    \gruen{[an einem Tisch]}\\
    \gruen{[mit der Gabel]}\\
    \alert{[einen Salat]}.}
    \onslide<+->
    \ex[*]{Wir essen \gruen{[schnell]}\\
    \rot{[ein Tofugericht]}\\
    \gruen{[mit Appetit]}\\
    \gruen{[an einem Tisch]}\\
    \gruen{[mit der Gabel]}\\
    \alert{[einen Salat]}.}
  \end{exe}
\end{frame}

\begin{frame}
  {Ergänzungen | Schnittstelle von Syntax und Semantik}
  \onslide<+->
  \onslide<+->
  Verbsemantik | Welche \alert{Rolle} spielen die von den Satzgliedern bezeichneten Dinge in der vom Verb beschriebenen Situation?\\
  \Zeile
  \onslide<+->
  Semantik von \alert{Ergänzungen} | \alert{abhängig} vom Verb\\
  \onslide<+->
  \Viertelzeile
  Semantik von \gruen{Angaben} | \gruen{unabhängig} vom Verb\\
  \Halbzeile
  \onslide<+->
  \begin{exe}
    \ex\label{ex:valenz071}
    \begin{xlist}
      \ex{\label{ex:valenz072}Ich lösche \alert{[den Ordner]} \gruen{[während der Hausdurchsuchung]}.}
      \onslide<+->
      \ex{\label{ex:valenz073}Ich mähe \alert{[den Rasen]} \gruen{[während der Ferien]}.}
      \onslide<+->
      \ex{\label{ex:valenz074}Ich fürchte \alert{[den Sturm]} \gruen{[während des Sommers]}.}
    \end{xlist}
  \end{exe}
\end{frame}


\begin{frame}
  {Valenz | Zusammenfassung}
  \onslide<+->
  \onslide<+->
  \begin{block}{Angaben}
    \alert{Angaben} sind grammatisch immer lizenziert und bringen\\
    ihre eigene semantische Rolle mit.\\
    \Halbzeile
    \grau{Sie können aber semantisch\slash pragmatisch inkompatibel sein.}
  \end{block}
  \Zeile
  \onslide<+->
  \begin{block}{Ergänzungen}
    \gruen{Ergänzungen} werden spezifisch vom Verb lizenziert und in ihrer semantischen Rolle\\
    vom Verb festgelegt. Jede dieser Rollen kann nur einmal vergeben werden.
  \end{block}
\end{frame}




\section{Zur nächsten Woche | Überblick}

\begin{frame}
  {Deutsche Syntax | Plan}
  \rot{Alle} angegebenen Kapitel\slash Abschnitte aus \rot{\citet{Schaefer2018b}} sind \rot{Klausurstoff}!\\
  \Halbzeile
  \begin{enumerate}
    \item Grammatik und Grammatik im Lehramt \rot{(Kapitel 1 und 3)}
    \item Grundbegriffe \rot{(Kapitel 2)}
    \item \alert{Wortklassen} \rot{(Kapitel 6)}
    \item Konstituenten und Satzglieder \rot{(Kapitel 11 und Abschnitt 12.1)}
    \item Nominalphrasen \rot{(Abschnitt 12.3)}
    \item Andere Phrasen \rot{(Abschnitte 12.2 und 12.4--12.7)}
    \item Verbphrasen und Verbkomplex \rot{(Abschnitte 12.8)}
    \item Sätze \rot{(Abschnitte 12.9 und 13.1--13.3)} 
    \item Nebensätze \rot{(Abschnitt 13.4)}
    \item Subjekte und Prädikate \rot{(Abschnitte 14.1--14.3)}
    \item Passive und Objekte \rot{(14.4 und 14.5)}
    \item Syntax infiniter Verbformen \rot{(Abschnitte 14.7--14.9)}
  \end{enumerate}
  \Halbzeile
  \centering 
  \url{https://langsci-press.org/catalog/book/224}
\end{frame}



  \let\subsection\section\let\section\woopsi

  \section[Wortklassen]{Wortklassen}
  \let\woopsi\section\let\section\subsection\let\subsection\subsubsection
  \section{Überblick}

\begin{frame}
  {Nächste Woche | Wortklassen}
  \onslide<+->
  \begin{itemize}[<+->]
    \item Was sind Wörter?
      \Halbzeile 
    \item Möglichkeiten, Wortklassen zu definieren
      \Halbzeile
    \item syntaktisch definierte Wortklassen
  \end{itemize}
\end{frame}

\section{Wörter}


\begin{frame}
  {Ebenen und Einheiten}
  \pause
  Kombinatorik von Wortbestandteilen und von Wörtern:
  \pause
  \Zeile
%  \begin{itemize}[<+->]
%    \item Wortakzent: \textit{\alert{\textbf{Sie}}ges\alert{säu}le}\\
%      $\rightarrow$ phonologisches\slash prosodisches Wort
%      \Zeile
%    \item Eigenschaften von Wörtern jenseits der Phonologie?
%  \end{itemize}
%  \Zeile
%  \pause
  \begin{exe}
    \ex
    \begin{xlist}
      \ex[]{Staat-es}
      \pause
      \ex[*]{Tür-es}
    \end{xlist}
    \pause
    \Zeile
    \ex
    \begin{xlist}
      \ex[]{Der Satz ist eine grammatische Einheit.}
      \pause
      \ex[*]{Die Satz ist eine grammatische Einheit.}
    \end{xlist}
  \end{exe}
\end{frame}

\begin{frame}
  {Wörter haben eine Bedeutung?}
  \pause
  \begin{exe}
    \ex \alert{Es} \alert{wird} schon wieder früh dunkel.
    \pause
    \ex Kristine denkt, \alert{dass} \alert{es} bald regnen \alert{wird}.
    \pause
    \ex Adrianna \alert{hat} gestern \alert{den} Keller inspiziert.
    \pause
    \ex Camilla \alert{und} Emma sehen \alert{sich} \alert{die} Fotos \alert{an}.
  \end{exe}
  \Zeile
  \pause
  \large
  Bedeutungstragende Wörter und \alert{Funktionswörter}
\end{frame}

\begin{frame}
  {Morphologie und Syntax}
  \pause
  \begin{itemize}[<+->]
    \item Kombinatorik für \alert{Wortbestandteile}: Morphologie
      \begin{itemize}[<+->]
        \item Wortbestandteile \zB mit \alert{Umlaut}: \textit{rot} -- \textit{röter}
        \item oder \alert{Ablaut}: \textit{heben} -- \textit{hob}
      \end{itemize}
    \item Kombinatorik für \alert{Wörter}: Syntax
      \Zeile
    \item \alert{Zirkuläre oder leere Definitionen?}
    \item \rot{Nein!} Prinzip: eigene Regularität → eigene Struktur
      \Zeile
    \item Wortbestandteile \alert{nicht trennbar}:
      \begin{itemize}
        \item \textit{heb-t}\\
          *\textit{heb mit Mühe t}
        \item \textit{Ge-hob-en-heit} \\
          *\textit{Gehoben anspruchsvolle heit}
        \item \textit{Sie geht schnell heim.}\\
          \textit{Schnell geht sie heim.}
      \end{itemize}
  \end{itemize}
\end{frame}



\section{Syntaktische Wörter}

\begin{frame}
  {Wort und Wortform I}
  \pause
  \begin{exe}
    \ex
    \begin{xlist}
      \ex (der) Tisch
      \pause
      \ex (den) Tisch
      \pause
      \ex (dem) Tisch\alert{e}
      \pause
      \ex (des) Tisch\alert{es}
      \pause
      \ex (die) Tisch\alert{e}
      \pause
      \ex (den) Tisch\alert{en}
    \end{xlist}
  \end{exe}
  \pause
  \begin{exe}
    \ex
    \begin{xlist}
      \ex Der \_\_\_\ ist voll hässlich.
      \pause
      \ex Ich kaufe den \_\_\_ nicht.
      \pause
      \ex Wir speisten am \_\_\_\ des Bundespräsidenten.
      \pause
      \ex Der Preis des \_\_\_\ ist eine Unverschämtheit.
      \pause
      \ex Die \_\_\_\ kosten nur noch die Hälfte.
      \pause
      \ex Mit den \_\_\_\ können wir nichts mehr anfangen.
    \end{xlist}
  \end{exe}
\end{frame}

\begin{frame}
  {Wort und Wortform II}
  \pause
  \begin{block}{Wortform}
    Eine \alert{Wortform} ist eine in syntaktischen Strukturen auftretende und in diesen Strukturen nicht weiter zu unterteilende Einheit.
    [\ldots]
  \end{block}
  \Zeile
  \pause
  \begin{block}{Lexikalisches Wort}
    Das (\alert{lexikalische}) \alert{Wort} ist eine Repräsentation von lexikalisch (bedeutungsmäßig) zusammengehörigen Wortformen.
    [\ldots]
  \end{block}
\end{frame}

\begin{frame}
  {Syntaktisches Wort}
  \onslide<+->
  \onslide<+->
  Ein \alert{syntaktisches Wort} ist eine \alert{Wortform} im syntaktischen Kontext.\\
  \Zeile
  \onslide<+->
  Ein syntaktisches Wort ist immer \alert{für alle Merkmale spezifiziert},\\
  auch wenn man ihm (morphologisch) nicht die volle Spezifikation ansieht.\\
  \Zeile
  \onslide<+->
  \begin{exe}
    \ex \alert{Ein [Mitglied]\Sub{Nom Sg Neut}} widersprach dem Beschluss.
    \onslide<+->
    \ex Wir überzeugten \alert{ein [Mitglied]\Sub{Akk Sg Neut}}, dem Beschluss zuzustimmen.
  \end{exe}
\end{frame}

\section{Methode}

\begin{frame}
  {Klassische Grundschul-Wortarten}
  \onslide<+->
    \begin{itemize}[<+->]
      \item Dingwort
      \item Tuwort, Tätigkeitswort
      \item Wiewort, Eigenschaftswort
      \item Umstandswort
    \end{itemize}
    \onslide<+->
    \Zeile
    Überwiegend \alert{bedeutungsbasiert}!
\end{frame}

\begin{frame}
  {Ein paar neue Wortarten nach Bedeutungen I}
  \pause
  \begin{itemize}[<+->]
    \item \alert{Bewegungsverben}: \textit{laufen}, \textit{springen}, \textit{fahren}, \dots
    \item \alert{Zustandsverben}: \textit{duften}, \textit{wohnen}, \textit{liegen}, \dots
      \Halbzeile
    \item \alert{Konkreta}: \textit{Haus}, \textit{Buch}, \textit{Blume}, \textit{Stier}, \dots
    \item \alert{Abstrakta}: \textit{Konzept}, \textit{Glaube}, \textit{Wunder}, \textit{Kausalität}, \dots
      \Halbzeile
    \item \alert{Zählsubstantive}: \textit{Keks}, \textit{Student}, \textit{Mikrobe}, \textit{Kneipe}, \dots
    \item \alert{Stoffsubstantive}: \textit{Wasser}, \textit{Wein}, \textit{Zement}, \textit{Mehl}, \dots
  \end{itemize}
\end{frame}

\begin{frame}
  {Ein paar neue Wortarten nach Bedeutungen II}
  \pause
  Aber Moment mal\dots\\
  \pause
  \Zeile
  \begin{exe}
   \ex
   \begin{xlist}
     \ex[ ]{\alert{Wein} kann lecker sein.}
     \ex[ ]{\alert{Ein Keks kann} lecker sein.}
     \ex[*]{\rot{Keks} kann lecker sein.}
     \ex[ ]{\alert{Kekse können} lecker sein.}
   \end{xlist}
    \pause
    \ex
    \begin{xlist}
      \ex Johanna hätte gerne \alert{einen Keks}.
      \ex Johanna hätte gerne \alert{einen Wein}.
    \end{xlist}
  \end{exe}
  \pause
  \Zeile
  Es gibt hier durchaus auch \alert{formale} Unterschiede.
\end{frame}


\begin{frame}
  {Syntaktische Klassifikation}
  \pause
  \begin{exe}
    \ex
    \begin{xlist}
      \ex[]{Ronnie spielt schnell \alert{und} präzise.}
      \pause
      \ex[*]{Ronnie spielt schnell \alert{obwohl} präzise.}
      \pause
      \ex[]{Ronnie \alert{und} Mark spielen eine gute Saison.}
      \pause
      \ex[*]{Ronnie \alert{obwohl} Mark spielen eine gute Saison.}
    \end{xlist}
    \pause
    \Zeile
    \ex
    \begin{xlist}
      \ex[]{Ronnie spielt herausragend,\\
        \alert{obwohl} der Leistungsdruck hoch ist.}
      \pause
      \ex[*]{Ronnie spielt herausragend,\\
        \alert{und} der Leistungsdruck hoch ist.}
    \end{xlist}
  \end{exe}
    \pause
    \Zeile
    \centering 
    Alles nur Bedeutung?
\end{frame}

\begin{frame}
  {Syntaktische Klassifikation}
  \pause
  \begin{center}
    \Large Wörter lassen sich in Kategorien einordnen, je nachdem\\
    \alert{in welchen syntaktischen Kontexten sie auftreten}.
  \end{center}
  \Zeile
  \pause
  \begin{itemize}[<+->]
    \item Konjunktionen: zwischen zwei gleichartigen Satzteilen
    \item Komplementierer: am Anfang bestimmter Nebensätze
  \end{itemize}
\end{frame}



\begin{frame}[fragile]
  {Filter}
  \onslide<+->
  \onslide<+->
  Mittels syntaktischer Klassifikation können wir den rechten Arm des Wortklassenbaums aufbauen (nicht-flektierbare Wörter).\\
  \Zeile
  \centering 
  \hspace{0.25\textwidth}\scalebox{0.6}{
    \begin{minipage}{0.5\textwidth}  
      \centering 
    \begin{forest}
      /tikz/every node/.append style={font=\footnotesize},
      for tree={l sep=2em, s sep=2.5em},
      [\textit{Wort}, intrme, {visible on=<3->}, for children={visible on=<4->}
        [{Hat  Numerus?}, decide, for children={visible on=<5->}
          [\textit{flektierbar}, intrme, yes, {visible on=<5->}, for children={visible on=<7->}
            [{Ist finit  flektierbar?}, decide, {visible on=<7->}, for children={visible on=<8->}
              [\textbf{Verb}, finall, yes, {visible on=<8->}]
              [\textit{Nomen}, intrme, no, {visible on=<9->}]
            ]
          ]
          [\textit{nicht flektierbar}, intrme, no, {visible on=<6->}, for children={visible on=<10->}
            [{Hat Valenz-\slash  Kasusrektion?}, decide, {visible on=<10->}, for children={visible on=<11->}
              [\textbf{Präposition}, finall, yes, {visible on=<11->}]
              [\textit{andere}, intrme, no, {visible on=<12->}]
            ]
          ]
        ]
      ]
    \end{forest}
   \end{minipage}
   }
\end{frame}


\section{Wortklassen}

\begin{frame}
  {Präpositionen flektieren nicht und regieren Kasus}
  \pause
  \begin{exe}
    \ex
    \begin{xlist}
      \ex{\alert<3->{Mit} \rot<4->{dem kaputten Rasen} ist nichts mehr anzufangen.}
      \pause
      \pause
      \pause
      \ex{\alert<6->{Angesichts} \rot<7->{des kaputten Rasens} wurde das Spiel abgesagt.}
    \end{xlist}
  \end{exe}
  \pause
  \pause
  \pause
  \Zeile
  \begin{block}{Rektion}
    In einer Rektionsrelation werden durch die regierende Einheit (das \alert{Regens}) Werte für bestimmte Merkmale\slash Werte (und damit ggf.\ auch die Form) beim regierten Element (dem \alert{Rectum}) verlangt.\\
  \end{block}
  \Zeile
  \pause
  \begin{block}{Präposition}
    Präpositionen kasusregieren eine obligatorische Nominalphrase.
  \end{block}
\end{frame}

\begin{frame}
  {Komplementierer}
  \pause
  \begin{exe}
    \ex
    \begin{xlist}
      \ex[]{Ich glaube, [\alert<3->{dass} dieser Nebensatz ein Verb \alert<4->{enthält}].}
      \ex[]{[\alert<5->{Während} die Spielzeit \alert<6->{läuft}], zählt jedes Tor.}
      \ex[]{Es fällt ihnen schwer [\rot<7->{zu laufen}].}
      \ex[\rot<10->{*}]{[\alert<8->{Obwohl} kein Tor \alert<9->{fiel}].}
    \end{xlist}
  \end{exe}
  \Zeile
  \onslide<11->
  \begin{block}{Komplementierer}
    Komplementierer leiten Nebensätze ein.\\
    Die Rede von der \textit{unterordnenden Konjunktion} ist ungeschickt.
  \end{block}
\end{frame}

\begin{frame}
  {Nicht-flektierbare Wörter im "`Vorfeld"'}
  \pause
  Was steht im unabhängigen Aussagesatz am Satzanfang?\\
  \pause
  {\rot{Antworten Sie nie mehr mit "`das Subjekt"'!}}
  \pause
  \begin{exe}
    \ex\label{ex:adverbenadkopulasundpartikeln038}
    \begin{xlist}
      \ex[ ]{\alert<5->{Gestern} hat Ronnie gewonnen.}
      \pause
      \pause
      \ex[ ]{\alert<7->{Erfreulicherweise} hat Ronnie gestern gewonnen.}
      \pause
      \pause
      \ex[ ]{\alert<9->{Oben} finden wir andere Beispiele.}
      \pause
      \pause
      \ex[*]{\alert<11->{Doch} ist das aber nicht das Ende der Saison.}
      \pause
      \pause
      \ex[*]{\alert<13->{Und} ist die Saison zuende.}
      \pause
      \pause
    \end{xlist}
    \ex\label{ex:adverbenadkopulasundpartikeln044} Das ist aber \alert{doch} nicht das Ende der Saison.
  \end{exe}
  \pause
  \Viertelzeile
  \begin{block}{Adverb}
    Adverben sind die übriggebliebenen nicht-flektierbaren Wörter,\\
    die im Vorfeld stehen können.
  \end{block}
\end{frame}


\begin{frame}
  {Konjunktionen}
  \onslide<+->
  \onslide<+->
  \begin{exe}
    \ex
    \begin{xlist}
      \ex Wir \alert{laufen} \rot{und} \alert{springen}.
      \ex Ich bin allergisch gegen \alert{Haselnüsse} \rot{und} \alert{Bananen}.
      \ex \alert{Kommst du jetzt} \rot{oder} \alert{sollen wir schon gehen}?
      \ex \alert{Erschöpft}, \rot{aber} \alert{zufrieden} lief sie über die Ziellinie.
    \end{xlist}
  \end{exe}
  \Zeile
  \onslide<+->
  \begin{block}{Kunjunktion}
    Eine Konjunktion (\textit{und}, \textit{oder}, \textit{aber}, \textit{sondern}, \ldots) verbindet zwei Konstituenten A und B, die sich syntaktisch gleich verhalten. Die Gesamtheit [A Konjunktion B] verhält sich ebenso.
  \end{block}
\end{frame}


\ifdefined\TITLE
  \section{Zur nächsten Woche | Überblick}

  \begin{frame}
    {Deutsche Syntax | Plan}
    \rot{Alle} angegebenen Kapitel\slash Abschnitte aus \rot{\citet{Schaefer2018b}} sind \rot{Klausurstoff}!\\
    \Halbzeile
    \begin{enumerate}
      \item Grammatik und Grammatik im Lehramt \rot{(Kapitel 1 und 3)}
      \item Grundbegriffe \rot{(Kapitel 2)}
      \item Wortklassen \rot{(Kapitel 6)}
      \item \alert{Konstituenten und Satzglieder} \rot{(Kapitel 11 und Abschnitt 12.1)}
      \item Nominalphrasen \rot{(Abschnitt 12.3)}
      \item Andere Phrasen \rot{(Abschnitte 12.2 und 12.4--12.7)}
      \item Verbphrasen und Verbkomplex \rot{(Abschnitte 12.8)}
      \item Sätze \rot{(Abschnitte 12.9 und 13.1--13.3)} 
      \item Nebensätze \rot{(Abschnitt 13.4)}
      \item Subjekte und Prädikate \rot{(Abschnitte 14.1--14.3)}
      \item Passive und Objekte \rot{(14.4 und 14.5)}
      \item Syntax infiniter Verbformen \rot{(Abschnitte 14.7--14.9)}
    \end{enumerate}
    \Halbzeile
    \centering 
    \url{https://langsci-press.org/catalog/book/224}
  \end{frame}
\fi

  \let\subsection\section\let\section\woopsi

  \section[Konstituenten]{Konstituen und Satzglieder}
  \let\woopsi\section\let\section\subsection\let\subsection\subsubsection
  

\section{Überblick}

\begin{frame}
  {Überblick: Konstituenten und Phrasen}
  \pause
  \begin{itemize}[<+->]
    \item Warum und wie syntaktische Analyse?
    \item syntaktische Generalisierungen formulieren
    \item größere und kleinere Teilstrukturen (Konstituenten) identifizieren
  \end{itemize}
\end{frame}


\section{Konstituenten}


\begin{frame}
  {Generalisierungen anhand von Wortklassen in der Syntax}
  \pause
  Denkbare Abstraktion für einen Satzbauplan anhand von Wortklassen:\\
  \Zeile
  \pause
  \begin{center}
    \begin{forest}
      [Satz
        [\it Ein]
        [\it Snookerball]
        [\it ist]
        [\it eine]
        [\it Kugel]
        [\it aus]
        [\it Kunststoff]
      ]
    \end{forest}\\
    \pause
    \Halbzeile
    \begin{center}
      →
    \end{center}
    \Halbzeile
    \begin{forest}
      [Satz
        [Art]
        [Subst]
        [Kopula-Verb]
        [Art]
        [Subst]
        [Prp]
        [Subst]
      ]
    \end{forest}        
  \end{center}
\end{frame}


\begin{frame}
  {"`Flache Beschreibungen"'}
  \pause
  \rot{Solche flachen Strukturbeschreibungen sind extrem ineffizient!}\\
  \Zeile
  \pause
  Aus Korpus mit \alert{über 1 Mrd.\ Wörtern} (DeReKo) \alert{alle Sätze} mit der Struktur\\
  von der vorherigen Folie (Art Subst Kopula Art Subst Prp Subst):\\
  \pause
  \Zeile
  \begin{exe}
    \ex
    \begin{xlist}
      \ex{Die Verlierer sind die Schulkinder in Weyerbusch.}
      \pause
      \ex{Die Vienne ist ein Fluss in Frankreich.}
      \pause
      \ex{Ein Baustein ist die Begegnung beim Spiel.}
      \pause
      \ex{Das Problem ist die Ortsdurchfahrt in Großsachsen.}
    \end{xlist}
  \end{exe}
\end{frame}

\begin{frame}
  {Viele ähnliche Strukturen auf einmal beschreiben}
  \pause
  Strukturen, die ähnlich, aber \alert{nicht genau} \\
  \alert{[Art Subst Kopula Art Subst Prp Subst]} sind:\\
  \pause
  \Zeile
  \begin{exe}
    \ex\label{ex:syntaktischestruktur013}
    \begin{xlist}
      \ex{\label{ex:syntaktischestruktur014} [Dieses Endspiel] ist [eine spannende Partie].}
      \pause
      \ex{\label{ex:syntaktischestruktur015} [Eine Hose] war [eine Hose].}
      \pause
      \ex{\label{ex:syntaktischestruktur016} [Sieger] wurde [ein Teilnehmer aus dem Vereinigten Königreich].}
      \pause
      \ex{\label{ex:syntaktischestruktur017} [Lemmy] ist [Ian Kilmister].}
    \end{xlist}
  \end{exe}
  \pause
  \Halbzeile
  \begin{itemize}[<+->]
    \item Diese Sätze sie sind \alert{gleich aufgebaut}.
    \item Sie haben jeweils \alert{drei Konstituenten} (= Bestandteile).
    \item Die Konstituenten haben intern teilweise abweichende Strukturen.
    \item Aber ihre unterschiedlich aufgebauten Konstituenten (Nominalphrasen)\\
      verhalten sich in diesen Sätzen jeweils gleich. 
  \end{itemize}
\end{frame}


\begin{frame}
  {Bauplan und Analyse}
  \pause
  Bauplan "`Kopula-Satz"' (vorläufig):\\
  \pause
  \Halbzeile
  \begin{center}
    \begin{forest}
      [Satz
        [NP]
        [Kopula-Verb]
        [NP]
      ]
    \end{forest}\\
    \pause
    \Zeile
    \raggedright
    Analyse auf Basis dieses Plans (vorläufig):\\
    \pause
    \Halbzeile
    \centering
    \begin{forest}
      [Satz
        [NP
          [\it Dieses Endspiel, narroof]
        ]
        [Kopula-Verb
          [\it ist]
        ]
        [NP
          [\it eine spannende Partie, narroof]
        ]
      ]
    \end{forest}
  \end{center}
\end{frame}


\begin{frame}
  {Konstituenten und Konstituententests}
  \pause
  {\Large \alert{Konstituententests sollen uns helfen, herauszufinden,\\
  wie wir Sätze in Konstituenten unterteilen wollen.}}\\
  \Zeile
  \pause
  \rot{Achtung!}
  \pause
  \Halbzeile
  \begin{itemize}[<+->]
    \item \rot{Konstituententests sind heuristisch!}
    \item unerwünschte Ergebnisse in beide Richtungen
    \item keine "`wahre Konstituentenstruktur"'
    \item theorieabhängig bzw.\ abhängig von gewählten Tests
    \Zeile
    \item Ziel: kompakte Beschreibung aller möglichen Strukturen
    \item gewiss: möglichst "`natürliche"' Analyse erwünscht
  \end{itemize}
\end{frame}

\begin{frame}
  {Pronominalisierungstest}
  \pause
  \begin{exe}
    \ex Mausi isst \alert<3->{den leckeren Marmorkuchen}.\\
    \pause
      \KTArr{PronTest} Mausi isst \alert{ihn}.
    \pause
    \ex{\label{ex:konstituententests025} \rot<5->{Mausi isst} den Marmorkuchen.\\
    \pause
      \KTArr{PronTest} \Ast \rot{Sie} den Marmorkuchen.}
    \pause
    \ex{\label{ex:konstituententests026} Mausi isst \alert<7->{den Marmorkuchen und das Eis mit Multebeeren}.\\
    \pause
    \KTArr{PronTest} Mausi isst \alert{sie}.}
  \end{exe}
  \pause
  \Halbzeile
  Pronominalausdrücke i.\,w.\,S.:
  \begin{exe}
    \ex{\label{ex:konstituententests027} Ich treffe euch \alert<9->{am Montag} \gruen<10->{in der Mensa}.\\
    \pause
    \KTArr{PronTest} Ich treffe euch \alert{dann} \gruen<10->{dort}.}
      \pause
      \pause
      \ex{\label{ex:konstituententests028} Er liest den Text \alert<12->{auf eine Art, die ich nicht ausstehen kann}.\\
      \pause
      \KTArr{PronTest} Er liest den Text \alert{so}.}
  \end{exe}
\end{frame}

\begin{frame}
  {Vorfeldtest\slash Bewegungstest}
  \pause
  \begin{exe}
    \ex
    \begin{xlist}
      \ex{Sarah sieht den Kuchen \alert<3->{durch das Fenster}.\\
        \pause
        \KTArr{VfTest} \alert{Durch das Fenster} sieht Sarah den Kuchen.}
      \pause
      \ex{Er versucht \alert{zu essen}.\\
        \pause
        \KTArr{VfTest} \alert<5->{Zu essen} versucht er.}
      \pause
      \ex{Sarah möchte gerne \alert{einen Kuchen backen}.\\
        \pause
        \KTArr{VfTest} \alert<7->{Einen Kuchen backen} möchte Sarah gerne.}
      \pause
      \ex{Sarah möchte \rot<9->{gerne einen} Kuchen backen.\\
        \pause
        \KTArr{VfTest} \Ast \rot{Gerne einen} möchte Sarah Kuchen backen.}
    \end{xlist}
  \end{exe}
  \pause
  \Halbzeile
  verallgemeinerter "`Bewegungstest"':\\
  \begin{exe}
    \ex\label{ex:konstituententests037}
    \begin{xlist}
      \ex{\label{ex:konstituententests038} Gestern hat \alert<11->{Elena} \gruen<11->{im Turmspringen} \orongsch<11->{eine Medaille} gewonnen.}
      \pause
      \ex{\label{ex:konstituententests039} Gestern hat \gruen{im Turmspringen} \alert{Elena} \orongsch{eine Medaille} gewonnen.}
      \pause
      \ex{\label{ex:konstituententests040} Gestern hat \gruen{im Turmspringen} \orongsch{eine Medaille} \alert{Elena} gewonnen.}
    \end{xlist}
  \end{exe}
\end{frame}

\begin{frame}
  {Koordinationstest}
  \pause
  \begin{exe}
    \ex\label{ex:konstituententests041}
    \begin{xlist}
      \ex Wir essen \alert<3->{einen Kuchen}.\\
      \pause
        \KTArr{KoorTest} Wir essen \alert{einen Kuchen} \gruen{und} \alert{ein Eis}.
      \pause
      \ex Wir \alert<5->{essen einen Kuchen}.\\
      \pause
        \KTArr{KoorTest} Wir \alert{essen einen Kuchen} \gruen{und} \alert{lesen ein Buch}.
      \pause
      \ex Sarah hat versucht, \alert<7->{einen Kuchen zu backen}.\\
      \pause
        \KTArr{KoorTest} Sarah hat versucht, \alert{einen Kuchen zu backen} \gruen{und} \\{}\alert{heimlich das Eis aufzuessen}.
      \pause
      \ex Wir sehen, dass \alert<9->{die Sonne scheint}.\\
      \pause
        \KTArr{KoorTest} Wir sehen, dass \alert{die Sonne scheint} \gruen{und} \\{}\alert{Mausi den Rasen mäht}.
    \end{xlist}
  \end{exe}
  \pause
  \begin{exe}
    \ex{\label{ex:konstituententests047} Der Kellner notiert, dass \rot<11->{meine Kollegin einen Salat} möchte.\\
    \pause
    \KTArr{KoorTest} Der Kellner notiert, dass \rot{meine Kollegin einen Salat}\\
    \gruen{und} \rot{mein Kollege einen Sojaburger} möchte.}
    \end{exe}
\end{frame}



\section{Satzglieder}

\begin{frame}
  {Satzglieder?}
  \pause
  \begin{exe}
    \ex
    \begin{xlist}
      \ex Sarah riecht den Kuchen \alert<3->{mit ihrer Nase}.\\
      \pause
        \KTArr{VfTest} \alert{Mit ihrer Nase} riecht Sarah den Kuchen.
        \pause
      \ex \KTArr{KoorTest} Sarah riecht den Kuchen\\
      {}\alert{mit ihrer Nase} und \alert{trotz des Durchzugs}.
    \end{xlist}
    \pause
    \ex
    \begin{xlist}
      \ex Sarah riecht den Kuchen \gruen<6->{mit der Sahne}.\\
      \pause
        \KTArr{VfTest} \Ast \rot{Mit der Sahne} riecht Sarah den Kuchen.
        \pause
      \ex \KTArr{KoorTest} Sarah riecht den Kuchen\\
      {}\alert{mit der Sahne} und \alert{mit den leckeren Rosinen}.
    \end{xlist}
  \end{exe}
  \pause
  \resizebox{0.9\textwidth}{!}{
    \begin{forest}
      [Satz
        [\it Sarah]
        [\it riecht]
        [\it den Kuchen]
        [\it mit ihrer Nase]
      ]
    \end{forest}\pause\begin{forest}
      [Satz
        [\it Sarah, tier=term]
        [\it riecht, tier=term]
        [Konstituente X
          [\it den Kuchen, tier=term]
          [\it mit der Sahne, tier=term]
        ]
      ]
    \end{forest}
  }
\end{frame}

\begin{frame}
  {Satzglieder als "`vorfeldfähige Konstituenten"'}
  \pause
  Ganz so einfach ist das nicht\ldots\\
  \Zeile
  \pause
  \begin{exe}
    \ex \rot{[Kaufen können]} möchte Alma die Wolldecke.
    \pause
    \ex \rot{[Über Syntax]} hat Sarah sich \alert{ein Buch} ausgeliehen.
  \end{exe}
  \Zeile
  \pause
  \alert{Wozu überhaupt den begriff des Satzglieds?}
  \begin{itemize}[<+->]
    \item in der Linguistik kaum von Interesse
    \item Sammelbegriff für "`Objekte und Adverbiale"'? -- \rot{Wozu?}
    \item Vorfeldfähigkeit? -- Wohl kaum, denn das wäre \rot{zirkulär} (und s.\,o.).
    \item Desambiguierung von Sätzen (s.\ Kuchen-Nase)? --\\
      \rot{Dabei hilft aber der Begriff "`Satzglied"' nicht.}
    \item Außerdem: \alert{Fördert das die Sprachkompetenz, oder kann das weg?}
  \end{itemize}
\end{frame}

\begin{frame}
  {Strukturelle Ambiguitäten und Kompositionalität}
  \pause
  \begin{exe}
    \ex{\label{ex:strukturelleambiguitaet060} Scully sieht den Außerirdischen mit dem Teleskop.}
  \end{exe}
  \pause
  \Halbzeile
  \begin{block}{Erinnerung: Kompositionalität}
    Die syntaktische Struktur ist die Basis für die Interpretation des Satzes (bzw.\ jedes syntaktisch komplexen Ausdrucks).
  \end{block}
  \pause
  \Halbzeile
  \begin{exe}
    \ex
    \begin{xlist}
      \ex Scully sieht \gruen<5->{[den Außerirdischen]} \orongsch<6->{[mit dem Teleskop]}.
      \pause
      \pause
      \pause
      \ex Scully sieht \alert<8->{[den Außerirdischen [mit dem Teleskop]]}.
    \end{xlist}
  \end{exe}
\end{frame}



\begin{frame}
  {Repräsentationsformat: Phrasenschemata}
  \pause
  \begin{itemize}[<+->]
    \item \alert{Grammatikalität = Konformität zu einer spezifischen Grammatik}
    \item Strukturen ohne spezifizierte Struktur: \rot{ungrammatisch}
      \Halbzeile
    \item Phrasenschemata = \alert{Baupläne} für zulässige Strukturen
    \item Strukturen = \alert{Bäume}
    \item Bei einer konkreten Analyse muss für jede Verzweigung im Baum\\
      ein Phrasenschema vorliegen, \rot{sonst ist die Analyse nicht zulässig}.
  \end{itemize}
  \pause
  \Halbzeile
  \centering
  \begin{multicols}{2}
    \footnotesize Das Schema:~\scalebox{0.6}{%
      \begin{forest}
      phrasenschema, baseline
      [NP, Ephr, calign=last
        [Artikel, Eopt, Emult
          [Pronomen, Eopt]
        ]
        [A, Eoptrec]
        [N, Ehd]
      ]
    \end{forest}
    \hspace{4em}
    }
    \onslide<8->{\footnotesize erlaubt~die~Analyse:~\scalebox{0.6}{%
      \begin{forest}
        [NP, calign=last, baseline
          [Artikel
            [\it ein]
          ]
          [A
            [\it leckerer]
          ]
          [A
            [\it geräucherter]
          ]
          [\textbf{N}
            [\it Tofu]
          ]
        ]
      \end{forest}
    }
  }
  \end{multicols}
\end{frame}

\section{Zur nächsten Woche | Überblick}

\begin{frame}
  {Deutsche Syntax | Plan}
  \rot{Alle} angegebenen Kapitel\slash Abschnitte aus \rot{\citet{Schaefer2018b}} sind \rot{Klausurstoff}!\\
  \Halbzeile
  \begin{enumerate}
    \item Grammatik und Grammatik im Lehramt \rot{(Kapitel 1 und 3)}
    \item Grundbegriffe \rot{(Kapitel 2)}
    \item Wortklassen \rot{(Kapitel 6)}
    \item Konstituenten und Satzglieder \rot{(Kapitel 11 und Abschnitt 12.1)}
    \item \alert{Nominalphrasen} \rot{(Abschnitt 12.3)}
    \item Andere Phrasen \rot{(Abschnitte 12.2 und 12.4--12.7)}
    \item Verbphrasen und Verbkomplex \rot{(Abschnitte 12.8)}
    \item Sätze \rot{(Abschnitte 12.9 und 13.1--13.3)} 
    \item Nebensätze \rot{(Abschnitt 13.4)}
    \item Subjekte und Prädikate \rot{(Abschnitte 14.1--14.3)}
    \item Passive und Objekte \rot{(14.4 und 14.5)}
    \item Syntax infiniter Verbformen \rot{(Abschnitte 14.7--14.9)}
  \end{enumerate}
  \Halbzeile
  \centering 
  \url{https://langsci-press.org/catalog/book/224}
\end{frame}




  \let\subsection\section\let\section\woopsi

  \section[Nominalphrasen]{Nominalphrasen}
  \let\woopsi\section\let\section\subsection\let\subsection\subsubsection
  

\section{Überblick}

\begin{frame}
  {Überblick: Konstituenten und Phrasen}
  \pause
  \begin{itemize}[<+->]
    \item Phrasen und Köpfe
    \item Strukur der deutschen \alert{Nominalphrase}
    \item (regierte) Attribute
      \Zeile
    \item \citet[Abschnitt~12.3]{Schaefer2018b}
  \end{itemize}
\end{frame}

\begin{frame}
  {Syntax und (bildungssprachliche) Funktion}
  \pause
  \begin{itemize}[<+->]
    \item \alert{hohe Komplexität} des syntaktischen Systems
    \item \rot{Regularitätensystem kaum vollständig explizit lernbar}
    \item überall \alert{starke Interaktion mit Semantik, Pragmatik usw.}
    \item \alert{Kompositionalität}
      \Zeile
    \item Der Versuch, Funktionen zu beschreiben, ohne Formsystem zu kennen,\\
      wäre in der Syntax völlig absurd.
      \Zeile
    \item reduzierte Syntax = erhebliche Einschränkung des Ausdrucks
    \item komplexe schriftsprachliche Syntax, ggf.\ \rot{Rezeptionsprobleme}
  \end{itemize}
\end{frame}


\section{Phrasentypen}

\begin{frame}
  {Jede Phrase hat genau einen Kopf}
  \pause
  \resizebox{\textwidth}{!}{
    \begin{tabular}{lll}
      \toprule
      \textbf{Kopf} & \textbf{Phrase} & \textbf{Beispiel} \\
      \midrule
      Nomen (Substantiv, Pronomen) & Nominalphrase (NP) & \textit{die tolle \alert{Auf"|führung}} \\
      Adjektiv & Adjektivphrase (AP) & \textit{sehr \alert{schön}} \\
      Präposition & Präpositionalphrase (PP) & \textit{\alert{in} der Uni} \\
      Adverb & Adverbphrase (AdvP) & \textit{total \alert{offensichtlich}} \\
      Verb & Verbphrase (VP) & \textit{Sarah den Kuchen gebacken \alert{hat}} \\
      Komplementierer & Komplementiererphrase (KP) & \textit{\alert{dass} es läuft} \\
      \bottomrule
    \end{tabular}
  } 
  \pause
  \Halbzeile
  \begin{itemize}[<+->]
    \item Der Kopf bestimmt den \alert{internen Aufbau} der Phrase.
    \item Der Kopf bestimmt die \alert{externen kategorialen Merkmale} der Phrase\\
      und so das syntaktische Verhalten der Phrase (Parallele: \alert{Kompositum}).
  \end{itemize}
\end{frame}


\begin{frame}
  {Wieviele Wortklassen? Wieviele Phrasentypen?}
  \pause
  \begin{itemize}[<+->]
    \item \alert{Phrasentyp: passend zur Wortklasse des Kopfes}
    \item maximal so viele Phrasentypen wie Wortklassen
    \item aber: nicht alle Wortklassen kopffähig (\alert{Funktionswörter})
      \Zeile
    \item heute nur der wahrscheinlich komplexeste nicht-satzförmige Phrasentyp:
      \begin{itemize}[<+->]
        \item Nominalphrase
      \end{itemize}
  \end{itemize}
\end{frame}

\section{Nominalphrasen}

\begin{frame}
  {Ziemlich volle NP-Struktur mit Substantiv-Kopf}
  \pause
  \centering
  \begin{forest}
    [NP, calign=child, calign child=3
      [Art
        [\it die]
      ]
      [AP
        [\it antiken, narroof]
      ]
      [\textbf{N}, tier=preterminal
        [\it Zahnbürsten]
      ]
      [NP, tier=preterminal
        [\it des Königs, narroof
        ]
      ]
      [RS
        [\it die nicht benutzt wurden, narroof]
      ]
    ]
  \end{forest}
  \pause
  \Zeile
  \begin{itemize}[<+->]
    \item \textit{die antiken Zahnbürsten}: \alert{Kongruenz}
    \item Baum über dem \alert{genusfesten} Kopf aufgebaut
    \item \alert{inneres Rechtsattribut} \textit{des Königs}
    \item \alert{Relativsatz} \textit{die nicht benutzt wurden}
  \end{itemize}
\end{frame}


\begin{frame}
  {Struktur mit pronominalem Kopf}
  \pause
  \centering
  \begin{forest}
    [NP, calign=child, calign child=1
      [\textbf{N}, tier=preterminal
        [\it einige]
      ]
      [NP, tier=preterminal
        [\it des Königs, narroof
        ]
      ]
      [RS
        [\it die geklaut wurden, narroof]
      ]
    ]
  \end{forest}
  \pause
  \Zeile
  \begin{itemize}[<+->]
    \item links vom Kopf: \rot{nichts}
    \item Determinierung erfolgt beim Pronomen \alert{im Kopf}.
    \item Determinierung schließt NP nach links ab.
    \item → \alert{Also kann links vom Pron-Kopf nichts stehen!}
  \end{itemize}
\end{frame}


\begin{frame}
  {Nominalphrase allgemein (Schema)}
  \pause
  \centering
  \begin{forest}
    phrasenschema
    [NP, Ephr
      [Art, Eopt, Emult, [NP\Sub{Genitiv}, Eopt]]
      [AP, Eopt, Erec]
      [N, Ehd, name=Nkopf]
      [innere Rechtsattribute, Eopt, Erec]
      {\draw [bend left=45, dashed,<-] (.south) to (Nkopf.south);}
      [RS, Eopt, Erec]
    ]
  \end{forest}
\end{frame}


\begin{frame}
  {Regierte Rechtsattribute}
  \pause
  \begin{exe}
    \ex die \gruen{Beachtung} \alert{[ihrer Lyrik]}
    \pause
    \ex mein \gruen{Wissen} \alert{[um die Bedeutung der komplexen Zahlen]}
    \pause
    \ex die \gruen{Überzeugung}, \alert{[dass die Quantenfeldtheorie \\
    die Welt korrekt beschreibt]}
    \pause
    \ex die \gruen{Frage}, \alert{[ob sich die Luftdruckanomalie von 2018 wiederholen wird]}
    \pause
    \ex die \gruen{Frage} \alert{[nach der möglichen Wiederholung der Luftdruckanomalie]}
  \end{exe}
  \pause
  \Halbzeile
  \begin{itemize}[<+->]
    \item typisch: postnominale Genitive, PPs, satzförmige Recta
  \end{itemize}
\end{frame}


\begin{frame}
  {Korrespondenzen zwischen Verben und Nomina(lisierungen)}
  \pause
  Viele Substantive entsprechen einem Verb mit bestimmten Rektionsanforderungen.\\
  \pause
  \Zeile
  \begin{exe}
    \ex\label{ex:rektionundvalenzindernp031}
    \begin{xlist}
      \ex{\label{ex:rektionundvalenzindernp032} \orongsch{Sarah} \alert{verziert} \gruen{[den Kuchen]}.}
      \pause
      \ex{\label{ex:rektionundvalenzindernp033} [Die \alert{Verzierung} \gruen{[des Kuchens]} \orongsch{[durch Sarah]}]}
      \pause
      \ex{\label{ex:rektionundvalenzindernp034} [Die \alert{Verzierung} \gruen{[von dem Kuchen]} \orongsch{[durch Sarah]}]}
    \end{xlist}
  \end{exe}
  \pause
  \Zeile
  \begin{itemize}[<+->]
    \item \gruen{Akkusativ} beim transitiven Verb $\Leftrightarrow$ \gruen{Genitiv}\slash\gruen{von-PP} beim Substantiv
    \item \orongsch{Nominativ} beim transitiven Verb $\Leftrightarrow$ \orongsch{durch-PP} beim Substantiv
    \item Beim nominalen Kopf: alle Ergänzungen optional
  \end{itemize}
\end{frame}


\begin{frame}
  {Alternative Korrespondenzen für Nominative}
  \pause
  \begin{exe}
    \ex\label{ex:rektionundvalenzindernp035}
    \begin{xlist}
      \ex{\label{ex:rektionundvalenzindernp036} \orongsch{[Sarah]} rettet [den Kuchen] [vor dem Anbrennen].}
      \pause
      \ex{\label{ex:rektionundvalenzindernp037} [\orongsch{[Sarahs]} Rettung [des Kuchens] [vor dem Anbrennen]]}
    \end{xlist}
  \end{exe}
  \pause
  \begin{itemize}[<+->]
    \item \orongsch{Nominativ} beim transitiven Verb $\Leftrightarrow$\\
      \orongsch{pränominaler Genitiv} beim Substantiv
  \end{itemize}
  \pause
  \Halbzeile
  \begin{exe}
    \ex[ ]{\gruen{[Die Schokolade]} wirkt gemütsaufhellend.}
    \pause
    \ex[ ]{[Die Wirkung \gruen{[der Schokolade]}] ist gemütsaufhellend.}
    \pause
    \ex[?]{[Die Wirkung \gruen{[von der Schokolade]}] ist gemütsaufhellend.}
    \pause
    \ex[*]{[\gruen{[Der Schokolade]} Wirkung] ist gemütsaufhellend.}
  \end{exe}
  \pause
  \begin{itemize}[<+->]
    \item \gruen{Nominativ} beim intransitiven Verb $\Leftrightarrow$\\
      \gruen{prä-\slash postnominaler Genitiv}\slash\gruen{von-PP} beim Substantiv
  \end{itemize}
\end{frame}


\begin{frame}
  {Komplexität der NP | Sätze und NPs}
  \onslide<+->
  \onslide<+->
  Die NP erreicht eine außergewöhnliche Komplexität,\\
  weil sich ganze Sätze als NP verpacken lassen.\\
  \onslide<+->
  \Zeile
  \begin{exe}
    \ex{ } \grau{Martinas Freundin ist wieder zuhause.}\\
      \rot{Martina} \alert{teilt} \gruen{ihr} \alert{mit}, \orongsch{dass die Pferde bereits gefüttert wurden}.
    \onslide<+->
    \Zeile
    \ex{ } [\rot{[Martinas]} \alert{Mitteilung} [\gruen{an ihre Freundin}, \grau{[die wieder zuhause ist]}],\\
      { }\orongsch{[dass die Pferde bereits gefüttert wurden]}],\\
      (kam gerade noch rechtzeitig.)
  \end{exe}
\end{frame}


\begin{frame}
  {Baum für die NP}
  \onslide<+->
  \onslide<+->
  \centering
  \begin{forest}
    [NP, calign=child, calign child=2, tier=root
      [NP, tier=subroot, rottree
        [\it Martinas, narroof, tier=terminal]
      ]
      [\textbf{N}, tier=subroot, bluetree
        [\it Mitteilung, tier=terminal]
      ]
      [PP, tier=subroot, gruen
        [P, calign=child, calign child=1, gruennode
          [\it an, tier=terminal, gruennode]
          [NP, calign=child, calign child=2, gruennode
            [Art, tier=preterminal, gruennode
              [\it ihre, tier=terminal, gruennode]
            ]
            [N, tier=preterminal, gruennode
              [\it Freundin, gruennode]
            ]
            [RS, tier=preterminal, grautree
              [\it die \ldots\ ist, narroof]
            ]
          ]
        ]
      ]
      [KP, tier=subroot, orongschtree
        [\it dass \ldots\ wurden, narroof, tier=terminal]
      ]
    ]
  \end{forest}

\end{frame}

\section{Vorschau}

\begin{frame}
  {Andere Phrasentypen}
  \onslide<+->
  \begin{itemize}[<+->]
    \item Adjektivphrasen
    \item Präpositionalphrasen
    \item Adverbphrasen
    \item Koordination
    \item Komplementiererphrase
      \Zeile
    \item \citet[12.2,12.4--12.7]{Schaefer2018b}
  \end{itemize}
\end{frame}

  \let\subsection\section\let\section\woopsi

  \section[Phrasen]{Andere Phrasen}
  \let\woopsi\section\let\section\subsection\let\subsection\subsubsection
  
\section{Überblick}

\begin{frame}
  {Andere Phrasentypen}
  \onslide<+->
  \begin{itemize}[<+->]
    \item Adjektivphrasen
    \item Präpositionalphrasen
    \item Adverbphrasen
    \item Koordination
    \item Komplementiererphrase
  \end{itemize}
\end{frame}

\section[AP]{Adjektivphrasen}

\begin{frame}
  {Gradierungselemente vor dem Adjektiv}
  \onslide<+->
  \onslide<+->
  \begin{exe}
    \ex[]{\label{ex:adjektivphrase081} die [\gruen{sehr} \alert{angenehme}] Stimmung}
    \ex[]{\label{ex:adjektivphrase082} die [\gruen{ziemlich} \alert{angenehme}] Stimmung}
    \ex[]{\label{ex:adjektivphrase083} die [\gruen{wenig} \alert{angenehme}] Stimmung}
      \onslide<+->
      \Zeile
      \ex[]{\label{ex:adjektivphrase085} die [\gruen{[über alle Maßen]} \alert{angenehme}] Stimmung}
      \ex[]{\label{ex:adjektivphrase086} die [\gruen{[ja mal wieder so rein gar nicht]} \alert{angenehme}] Stimmung}
  \end{exe}
\end{frame}

\begin{frame}
  {Modifizierer | noch vor Gradierungselementen}
  \onslide<+->
  \onslide<+->
  \begin{exe}
    \ex\label{ex:adjektivphrase087}
    \begin{xlist}
      \ex{\label{ex:adjektivphrase088} die [\braun{[seit gestern]} \gruen{sehr} \alert{angenehme}] Stimmung}
      \ex{\label{ex:adjektivphrase089} das [\braun{[in Hessen]} \gruen{überaus} \alert{beliebte}] Getränk}
    \end{xlist}
    \onslide<+->
    \Zeile
    \ex[*]{\label{ex:adjektivphrase090}die [\gruen{sehr} \braun{[seit gestern]} \alert{angenehme}] Stimmung}
  \end{exe}
\end{frame}

\begin{frame}
  {Adjektivphrase | Baumbeispiel}
  \onslide<+->
  \onslide<+->
  \centering
  \begin{forest}
    [AP, calign=last
      [PP, tier=preterminal
        [\it seit gestern, narroof]
      ]
      [Ptkl, tier=preterminal
        [\it sehr]
      ]
      [\bf A, tier=preterminal
        [\it angenehme]
      ]
    ]
  \end{forest}
\end{frame}

\begin{frame}
  {Ergänzungen in der AP}
  \onslide<+->
  \onslide<+->\begin{exe}
  \ex\label{ex:adjektivphrase092}
  \begin{xlist}
    \ex[]{\label{ex:adjektivphrase093} die [\tuerkis{[auf ihre Tochter]} \alert{stolze}] Frau}
    \ex[*]{die [\alert{stolze} \tuerkis{[auf ihre Tochter]}] Frau}
    \Zeile
    \onslide<+->
    \ex[]{die [\tuerkis{[über ihre Tochter]} \alert{verwunderte}] Frau}
    \ex[*]{die [\alert{verwunderte} \tuerkis{[über ihre Tochter]}] Frau}
    \onslide<+->
    \Zeile
    \ex[]{die [\tuerkis{[ihres Lieblingseises]} \alert{überdrüssige}] Frau}
    \ex[*]{die [\alert{überdrüssige} \tuerkis{[ihres Lieblingseises]}] Frau}
  \end{xlist}
\end{exe}
\end{frame}

\begin{frame}
  {Ziemlich volle AP}
  \onslide<+->
  \onslide<+->
  \centering
  \begin{forest}
    [AP, calign=last
      [PP, tier=preterminal
        [\it seit gestern, narroof]
      ]
      [PP, tier=preterminal
        [\it auf ihre Tochter, narroof, name=AufIhreTochter]
      ]
      [Ptkl, tier=preterminal
        [\it sehr]
      ]
      [\bf A, tier=preterminal
        [\it stolze]
        {\draw [->, bend left=30] (.south) to (AufIhreTochter);}
      ]
    ]
  \end{forest}
\end{frame}

\begin{frame}
  {Adjektivphrase | Schema}
  \onslide<+->
  \onslide<+->
  \centering 
  \begin{forest}
    phrasenschema
    [AP, Ephr, calign=child, calign child=2
      [Modifizierer, Eopt, Erec, Emult [Ergänzungen, Eopt, Erec, name=Apergaenzi]]
      [Gradierungselement, Eopt]
      [A, Ehd]
      {\draw [->, bend left=45] (.south) to (Apergaenzi.south);}
    ]
  \end{forest}
\end{frame}

\section[PP]{Präpositionalphrasen}

\begin{frame}
  {Präpositionalphrasen | Beispiele}
  \onslide<+->
  \onslide<+->
  Erinnerung | \alert{Präpositionen haben eine einstellige Valenz.}\\
  \onslide<+->
  \Zeile
  \begin{exe}
    \ex\label{ex:normalepp096}
    \begin{xlist}
      \ex{[\alert{Auf} \orongsch{[dem Tisch]}] steht Ischariots Skulptur.}
      \ex{[\gruen{[Einen Meter]} \alert{unter} \orongsch{[der Erde]}] ist die Skulptur versteckt.}
    \end{xlist}
    \onslide<+->
    \Zeile
    \ex{\label{ex:normalepp097} Seit der EM springt Christina [\gruen{weit} \alert{über} \orongsch{[ihrem früheren Niveau]}].}
  \end{exe}{}
\end{frame}

\begin{frame}
  {Baumbeispiel | PP mit Maßangabe}
  \onslide<+->
  \onslide<+->
  \centering
  \begin{forest}
    [PP, calign=child, calign child=2
      [NP, tier=preterminal
        [\it einen Meter, narroof]
      ]
      [\bf P, tier=preterminal
        [\it unter]
      ]
      [NP, tier=preterminal
        [\it der Erde, narroof]
      ]
    ]
  \end{forest}
\end{frame}


\begin{frame}
  {Präpositionalphrase | Schema}
  \onslide<+->
  \onslide<+->
  \centering
  \begin{forest}
    phrasenschema
    [PP, Ephr, calign=child, calign child=2
      [Modifizierer, Eopt]
      [P, Ehd, name=Ppkopf]
      [NP, Eobl]
      {\draw [<-, bend left=45] (.south) to (Ppkopf.south);}
    ]
  \end{forest}
\end{frame}


\section[AdvP]{Adverbphrasen}

\begin{frame}
  {Adverbphrasen}
  \onslide<+->
  \onslide<+->
  \alert{Adverben} | Präpositionen mit \alert{nullstelliger Valenz}.\\
  \Zeile
  \onslide<+->
  \begin{exe}
    \ex{\label{ex:adverbphrase106} Ischariot malt [\orongsch{sehr} \alert{oft}].}
      \Halbzeile
      \ex{\label{ex:adverbphrase107} Ischariot schwimmt [\orongsch{weit} \alert{draußen}].}
      \Halbzeile
      \ex{\label{ex:adverbphrase108} Ischariot verreist [\orongsch{sehr} \alert{wahrscheinlich}].}
  \end{exe}
\end{frame}


\begin{frame}
  {Baumbeispiel | AdvP mit Modifizierer}
  \onslide<+->
  \onslide<+->
  \centering
  \begin{forest}
    [AdvP, calign=last
      [Ptkl, tier=preterminal
        [\it sehr]
      ]
      [\bf Adv, tier=preterminal
        [\it oft]
      ]
    ]
  \end{forest}
\end{frame}


\begin{frame}
  {Adverbphrase | Schema}
  \onslide<+->
  \onslide<+->
  \centering
  \begin{forest}
    phrasenschema
    [AdvP, Ephr, calign=last
      [Modifizierer, Eopt]
      [Adv, Ehd]
    ]
  \end{forest}
\end{frame}



\section{Koordination}

\begin{frame}
  {Koordination | Beispiele}
  \onslide<+->
  \onslide<+->
  \alert{Koordination} | Gleiches mit Gleichem zu Gleichem verbinden.\\
  \onslide<+->
  \Zeile
  \begin{exe}
    \ex\label{ex:koordination006}
    \begin{xlist}
      \ex{Ihre Freundin möchte [\orongsch{Kuchen} \alert{und} \orongsch{Sahne}].}
      \Halbzeile
      \onslide<+->
      \ex{[\orongsch{[Es ist Sonntag]} \alert{und} \orongsch{[die Zeit wird knapp]}].}
      \Halbzeile
      \onslide<+->
      \ex{Hast du das Teepulver [\orongsch{auf} \alert{oder} \orongsch{neben}]\\
    den Tatami-Matten verstreut?}
    \end{xlist}
  \end{exe}
\end{frame}


\begin{frame}
  {Koordination von Substantiven (oder NPs?)}
  \centering
  \begin{forest}
    [\textbf{N}, calign=child, calign child=2
      [\textbf{N}, tier=preterminal
        [\it Kuchen]
      ]
      [Konj, tier=preterminal
        [\it und]
      ]
      [\textbf{N}, tier=preterminal
        [\it Sahne]
      ]
    ]
  \end{forest}
\end{frame}

\begin{frame}
  {Koordination von Sätzen}
  \centering
  \begin{forest}
    [S, calign=child, calign child=2
      [S, tier=preterminal
        [\it Es ist Sonntag, narroof]
      ]
      [Konj, tier=preterminal
        [\it und]
      ]
      [S, tier=preterminal
        [\it die Zeit wird knapp, narroof]
      ]
    ]
  \end{forest}
\end{frame}

\begin{frame}
  {Koordination von Präpositionen}
  \centering
  \begin{forest}
    [\textbf{P}, calign=child, calign child=2
      [\textbf{P}, tier=preterminal
        [\it auf]
      ]
      [Konj, tier=preterminal
        [\it oder]
      ]
      [\textbf{P}, tier=preterminal
        [\it neben]
      ]
    ]
  \end{forest}
\end{frame}

\begin{frame}
  {Koordination | Schema}
  \onslide<+->
  \onslide<+->
  Die Koordination selber ist kein Kopf!\\
  \onslide<+->
  \Zeile
  \centering
  \begin{forest}
    phrasenschema
    [$\kappa$, Ephr
      [$\kappa$, Eobl]
      [Konj, Eopt]
      [$\kappa$, Eobl]
    ]
  \end{forest}
\end{frame}


\section[KP]{Komplementiererphrase}

\begin{frame}
  {Komplementiererphrasen = eingeleitete Nebensätze}
  \pause
  \begin{exe}
    \ex\label{ex:komplementiererphrase111}
    \begin{xlist}
      \ex[]{\label{ex:komplementiererphrase112} Der Arzt möchte, [dass [der Privatpatient die Rechnung \alert{bezahlt}]].}
      \pause
      \ex[*]{\label{ex:komplementiererphrase113} Der Arzt möchte, [dass [der Privatpatient \rot{bezahlt} die Rechnung]].}
      \pause
      \ex[*]{\label{ex:komplementiererphrase114} Der Arzt möchte, [dass [\rot{bezahlt} der Privatpatient die Rechnung]].}
    \end{xlist}
  \end{exe}
  \pause
  \Halbzeile
  \centering
  \begin{forest}
    [KP, calign=first
      [\bf K, tier=preterminal
        [\it dass, name=Kpkopf]
      ]
      [\alert{VP}, tier=preterminal
        [\it der Kassenpatient \alert{geht}, narroof]
      ]
    ]
  \end{forest}\\
  \pause
  \Zeile
  \alert{Verb-Letzt-Stellung!}\\
\end{frame}



\begin{frame}
  {Komplementiererphrase | Schema}
  \begin{center}
    \begin{forest}
      phrasenschema
      [KP, Ephr, calign=first
        [K, Ehd, name=Kpkopf]
        [VP, Eobl]
        {\draw [bend left=45, <-] (.south) to (Kpkopf.south);}
      ]
    \end{forest}
  \end{center}
  \onslide<+->
  \Zeile
  \alert{Aber wie sieht die VP aus?}\\
  \Viertelzeile
  \onslide<+->
  \orongsch{Und was ist mit unabhängigen Sätzen?}
\end{frame}

\ifdefined\TITLE
  \section{Zur nächsten Woche | Überblick}

  \begin{frame}
    {Deutsche Syntax | Plan}
    \rot{Alle} angegebenen Kapitel\slash Abschnitte aus \rot{\citet{Schaefer2018b}} sind \rot{Klausurstoff}!\\
    \Halbzeile
    \begin{enumerate}
      \item Grammatik und Grammatik im Lehramt \rot{(Kapitel 1 und 3)}
      \item Grundbegriffe \rot{(Kapitel 2)}
      \item Wortklassen \rot{(Kapitel 6)}
      \item Konstituenten und Satzglieder \rot{(Kapitel 11 und Abschnitt 12.1)}
      \item Nominalphrasen \rot{(Abschnitt 12.3)}
      \item Andere Phrasen \rot{(Abschnitte 12.2 und 12.4--12.7)}
      \item \alert{Verbphrasen und Verbkomplex} \rot{(Abschnitte 12.8)}
      \item Sätze \rot{(Abschnitte 12.9 und 13.1--13.3)} 
      \item Nebensätze \rot{(Abschnitt 13.4)}
      \item Subjekte und Prädikate \rot{(Abschnitte 14.1--14.3)}
      \item Passive und Objekte \rot{(14.4 und 14.5)}
      \item Syntax infiniter Verbformen \rot{(Abschnitte 14.7--14.9)}
    \end{enumerate}
    \Halbzeile
    \centering 
    \url{https://langsci-press.org/catalog/book/224}
  \end{frame}
\fi

  \let\subsection\section\let\section\woopsi

  \section[Verbphrasen]{Verbphrasen und Verbkomplexe}
  \let\woopsi\section\let\section\subsection\let\subsection\subsubsection
  \section{Überblick}

\begin{frame}
  {Verbphrasen und Verbkomplexe}
  \onslide<+->
  \begin{itemize}[<+->]
    \item Verbphrasen mit Verb-Letzt-Stellung
    \item Scrambling | Stellungsfreiheit in der VP
      \Halbzeile
    \item Verbkomplexe | Verbketten am Ende der VP
      \Halbzeile
    \item systematische syntaktische Analysen
      \Zeile
    \item \citet[12.8]{Schaefer2018b}
  \end{itemize}
\end{frame}


\section{Verbphrasen}

\begin{frame}
  {Beispiele für Verbphrasen}
  \pause
  \begin{exe}
  \ex
    \begin{xlist}
      \ex{dass [Ischariot \alert{malt}]}
      \pause
      \Halbzeile
      \ex{dass [Ischariot [das Bild] \alert{malt}]}
      \pause
      \Halbzeile
      \ex{dass [Ischariot [dem Arzt] [das Bild] \alert{verkauft}]}
      \pause
      \Halbzeile
      \ex{dass [Ischariot [wahrscheinlich] [dem Arzt]\\
        { }[heimlich] [das Bild] schnell \alert{verkauft}]}
    \end{xlist}
  \end{exe}
\end{frame}

\begin{frame}
  {VP mit einstelliger Valenz}
  \centering 
  \scalebox{1}{
  \begin{forest}
    [VP, calign=last
      [NP, tier=preterminal
        [\it Ischariot, narroof]
      ]
      [\bf V, tier=preterminal
        [\it malt]
      ]
    ]
  \end{forest}
  }
\end{frame}

\begin{frame}
  {VP mit zweistelliger Valenz}
  \centering 
  \scalebox{1}{
  \begin{forest}
    l sep+=1em
    [VP, calign=last
      [NP, tier=preterminal
        [\it Ischariot, narroof]
      ]
      [NP, tier=preterminal
        [\it das Bild, narroof]
      ]
      [\bf V, tier=preterminal
        [\it malt]
      ]
    ]
  \end{forest}
  }
\end{frame}

\begin{frame}
  {VP mit dreistelliger Valenz}
  \centering 
  \scalebox{1}{
  \begin{forest}
    l sep+=2em
    [VP, calign=last
      [NP, tier=preterminal
        [\it Ischariot, narroof]
      ]
      [NP, tier=preterminal
        [\it dem Arzt, narroof]
      ]
      [NP, tier=preterminal
        [\it das Bild, narroof]
      ]
      [\bf V, tier=preterminal
        [\it verkauft]
      ]
    ]
  \end{forest}
  }
\end{frame}

\begin{frame}
  {VP mit einstelliger Valenz und Adverbialen}
  \centering 
  \scalebox{1}{
  \begin{forest}
    l sep+=3em
    [VP, calign=last
      [NP, tier=preterminal
        [\it Ischariot, narroof]
      ]
      [AdvP, tier=preterminal
        [\it wahrscheinlich, narroof]
      ]
      [NP, tier=preterminal
        [\it dem Arzt, narroof]
      ]
      [AdvP, tier=preterminal
        [\it heimlich, narroof]
      ]
      [NP, tier=preterminal
        [\it das Bild, narroof]
      ]
      [AdvP, tier=preterminal
        [\it schnell, narroof]
      ]
      [\bf V, tier=preterminal
        [\it verkauft]
      ]
    ]
  \end{forest}
  }
\end{frame}

\begin{frame}
  {Achtung! Scrambling!}
  \onslide<+->
  \onslide<+->
  \alert{Scrambling} | Die Phrasen innerhalb der VP können\\
  nahezu beliebig umsortiert werden.\\
  \Zeile
  \onslide<+->
  \begin{exe}
    \ex dass dem Arzt Ischariot wahrscheinlich\\
    schnell das Bild verkauft
    \ex dass Ischariot wahrscheinlich schnell\\
    dem Arzt das Bild verkauft
    \ex dass Ischariot wahrscheinlich\\
    das Bild schnell dem Arzt verkauft
    \ex \ldots
  \end{exe}
  \onslide<+->
  \Zeile
  Die Umstellungen haben \alert{semantische und pragmatische Effekte},\\
  aber syntaktisch sind sie alle möglich.
\end{frame}

\section{Verbkomplexe}

\begin{frame}
  {Warum Verbkomplexe?}
  \pause
  \begin{exe}
    \ex{\label{ex:verbkomplex121} dass der Junge ein Eis \alert{[isst]}}
    \pause
    \ex\label{ex:verbkomplex122}
    \begin{xlist}
      \ex{\label{ex:verbkomplex123} dass der Junge ein Eis \alert{[essen wird]}}
      \pause
      \ex{\label{ex:verbkomplex124} dass das Eis \alert{[gegessen wird]}}
      \pause
      \ex{\label{ex:verbkomplex125} dass die Freundin das Eis \alert{[kaufen wollen wird]}}
    \end{xlist}
  \end{exe}
  \Zeile
  \pause
  Deutsch: \alert{Verben werden miteinander kombiniert, um Tempora,\\
  Modalität, Diathese usw.\ zu kodieren.}\\
\end{frame}


\begin{frame}
  {Verbkomplexe und Statusrektion}
  \pause
  \Halbzeile
  \centering
  \scalebox{0.7}{
    \begin{forest}
      [\bf V, tier=preterminal
        [\it isst, baseline]
      ]
    \end{forest}
  }
  \hspace{1em}\scalebox{0.7}{
    \begin{forest}
      [\bf V\Sub{B+A}, calign=last, bluetree
        [\bf V\Sub{B}, tier=preterminal, gruentree
          [\it essen\\(1.~Status)]
        ]
        [\bf V\Sub{A}, tier=preterminal
          [\it wird, baseline]
          {\draw [->, trueblue, bend left=30] (.south) to (!uu11.south);}
        ]
      ]
    \end{forest}
  }
  \hspace{1em}\scalebox{0.7}{
    \begin{forest}
      [\bf V\Sub{B+A}, calign=last, bluetree
        [\bf V\Sub{B}, tier=preterminal, gruentree
          [\it gegessen\\(3.~Status)]
        ]
        [\bf V\Sub{A}, tier=preterminal
          [\it wird, baseline]
          {\draw [->, trueblue, bend left=30] (.south) to (!uu11.south);}
        ]
      ]
    \end{forest}
  }
  \hspace{1em}\scalebox{0.7}{
    \begin{forest}
      [\bf V\Sub{C+B+A}, calign=last, bluetree
        [\bf V\Sub{C+B}, calign=last, gruentree
          [\bf V\Sub{C}, tier=preterminal, orongschtree
            [\it kaufen\\(1.~Status)]
          ]
          [\bf V\Sub{B}, tier=preterminal
            [\it wollen\\(1.~Status)]
            {\draw [->, gruen, bend left=30] (.south) to (!uu11.south);}
          ]
        ]
        [\bf V\Sub{A}, tier=preterminal
          [\it wird, baseline]
          {\draw [->, trueblue, bend left=30] (.south) to (!uu121.south);}
        ]
      ]
    \end{forest}
  }
  \pause
  \Halbzeile
  \begin{itemize}[<+->]
    \item Buchstaben (im Buch Zahlen): \alert{Verb A} regiert \gruen{Verb B} regiert \orongsch{Verb C}
    \item Numerierung: Status
      \begin{itemize}[<+->]
        \item 1.~Status: Infinitiv ohne \textit{zu}
        \item 2.~Status: Infinitiv mit \textit{zu}
        \item 3.~Status: Partizip
      \end{itemize}
    \item infinite Verbformen: solche, die von anderen Verben regiert werden
%    \item "`Partizip 1"' keine infinite Verbform (Derivation zum Adjektiv)
  \end{itemize}
\end{frame}


\begin{frame}
  {Verbkomplex und Rektion in der VP}
  \pause
  Die Hilfsverben \textit{heben} die \alert{Valenz-Anforderungen}\\
  lexikalischer Verben zu sich \textit{an}.\\
  \pause
  \centering
  \adjustbox{max width=0.5\textwidth}{%
    \begin{forest}
      l sep+=2em
      [VP, calign=last
        [NP, tier=preterminal, name=subj
          [\it die Freundin, narroof]
        ]
        [NP, tier=preterminal, name=obj
          [\it das Eis, narroof]
        ]
        [\bf V\Sub{\orongsch{C}\gruen{+B}\alert{+A}}, calign=last, bluetree, name=CBAnode
          [\bf V\Sub{\orongsch{C}\gruen{+B}}, calign=last, gruentree, name=CBnode
            [\bf V\Sub{C}, tier=preterminal, orongschtree
              [\it kaufen]
            ]
            {\draw [->, orongsch, bend left=15] (.north) to node [above, near start] {\tiny{(Nom, Akk)}} (CBnode.west);}
            [\bf V\Sub{B}, tier=preterminal
              [\it wollen]
              {\draw [->, gruen, bend left=30] (.south) to node [below] {\footnotesize{1.~Status}} (!uu11.south);}
            ]
          ]
          {\draw [->, orongsch, bend left=15] (.north) to node [above, near start] {\tiny{(Nom, Akk)}} (CBAnode.west);}
          [\bf V\Sub{A}, tier=preterminal
            [\it wird]
            {\draw [->, trueblue, bend left=30] (.south) to node [below, near start] {\footnotesize{1.~Status}} (!uu121.south);}
          ]
        ]
        {\draw [->, orongsch, bend right=80] (.north) to node [above] {\footnotesize{Nom}} (subj.north);}
        {\draw [->, orongsch, bend right=60] (.north) to node [above] {\footnotesize{Akk}} (obj.north);}
      ]
    \end{forest}
  }
\end{frame}


\begin{frame}
  {Verbphrase und Verbkomplex | Schemata}
  \centering
  \begin{forest}
    phrasenschema
    [VP, Ephr, calign=last
      [Angaben, Emult, Eopt, Erec [Ergänzungen, Eopt, Emult, Erec, name=Vpergaenzi]]
      [V, Ehd]
      {\draw [->, bend left=90] (.south) to (Vpergaenzi.south);}
    ]
  \end{forest}
  \pause\hspace{2em}
  \begin{forest}
    phrasenschema
    [V\Sub{j+i}, Ephr, , calign=last
      [V\Sub{j}, Eopt, name=Vkkopf]
      [V\Sub{i}, Ehd]
      {\draw [->, bend left=30] (.south) to (Vkkopf.south);}
    ]
  \end{forest}
\end{frame}

\section{Analyse}

\begin{frame}
  {Systematische syntaktische Analyse | Schritt 1}
  \centering
  \begin{forest}
    [, phantom, s sep=1em
      [\bf K, tier=preterminal [\it dass]]
      [\bf N, tier=preterminal [\it Frida]]
      [Art, tier=preterminal [\it den]]
      [\bf A, tier=preterminal [\it heißen]]
      [\bf N, tier=preterminal [\it Kaffee]]
      [\bf Adv, tier=preterminal [\it gerne]]
      [\bf V, tier=preterminal [\it trinken]]
      [\bf V, tier=preterminal [\it möchte]]
    ]
  \end{forest}
\end{frame}

\begin{frame}
  {Systematische syntaktische Analyse | Schritt 2}
  \centering
  \begin{forest}
    [, phantom, s sep=1em
      [\bf K, tier=preterminal [\it dass]]
      [NP
        [\bf N, tier=preterminal [\it Frida]]
      ]
      [Art, tier=preterminal [\it den]]
      [AP
        [\bf A, tier=preterminal [\it heißen]]
      ]
      [\bf N, tier=preterminal [\it Kaffee]]
      [AdvP
        [\bf Adv, tier=preterminal [\it gerne]]
      ]
      [\bf V, tier=preterminal [\it trinken]]
      [\bf V, tier=preterminal [\it möchte]]
    ]
  \end{forest}
\end{frame}

\begin{frame}
  {Systematische syntaktische Analyse | Schritt 3}
  \centering
  \begin{forest}
    [, phantom, s sep=0.5em
      [\bf K, tier=preterminal [\it dass]]
      [NP
        [\bf N, tier=preterminal [\it Frida]]
      ]
      [NP, calign=last
        [Art [\it den, tier=terminal]]
        [AP
          [\bf A, tier=preterminal [\it heißen]]
        ]
        [\bf N, tier=preterminal [\it Kaffee]]
      ]
      [AdvP
        [\bf Adv, tier=preterminal [\it gerne]]
      ]
      [\bf V, tier=preterminal [\it trinken]]
      [\bf V, tier=preterminal [\it möchte, tier=terminal]]
    ]
  \end{forest}
\end{frame}

\begin{frame}
  {Systematische syntaktische Analyse | Schritt 4}
  \centering
  \begin{forest}
    [, phantom, s sep=0.5em
      [\bf K, tier=preterminal [\it dass]]
      [NP
        [\bf N, tier=preterminal [\it Frida]]
      ]
      [NP, calign=last
        [Art [\it den, tier=terminal]]
        [AP
          [\bf A, tier=preterminal [\it heißen]]
        ]
        [\bf N, tier=preterminal [\it Kaffee]]
      ]
      [AdvP
        [\bf Adv, tier=preterminal [\it gerne]]
      ]
      [\bf V, calign=last
        [\bf V, tier=preterminal [\it trinken]]
        [\bf V, tier=preterminal [\it möchte, tier=terminal]]
      ]
    ]
  \end{forest}
\end{frame}

\begin{frame}
  {Systematische syntaktische Analyse | Schritt 5}
  \centering
  \begin{forest}
    [, phantom, s sep=0.5em
      [\bf K, tier=preterminal [\it dass]]
      [VP, calign=last, l sep=4em
        [NP
          [\bf N, tier=preterminal [\it Frida]]
        ]
        [NP, calign=last
          [Art [\it den, tier=terminal]]
          [AP
            [\bf A, tier=preterminal [\it heißen]]
          ]
          [\bf N, tier=preterminal [\it Kaffee]]
        ]
        [AdvP
          [\bf Adv, tier=preterminal [\it gerne]]
        ]
        [\bf V, calign=last
          [\bf V, tier=preterminal [\it trinken]]
          [\bf V, tier=preterminal [\it möchte, tier=terminal]]
        ]
      ]
    ]
  \end{forest}
\end{frame}

\begin{frame}
  {Systematische syntaktische Analyse | Schritt 6}
  \centering
  \begin{forest}
    [KP, calign=first
      [\bf K, tier=preterminal [\it dass]]
      [VP, calign=last, l sep=4em
        [NP
          [\bf N, tier=preterminal [\it Frida]]
        ]
        [NP, calign=last
          [Art [\it den, tier=terminal]]
          [AP
            [\bf A, tier=preterminal [\it heißen]]
          ]
          [\bf N, tier=preterminal [\it Kaffee]]
        ]
        [AdvP
          [\bf Adv, tier=preterminal [\it gerne]]
        ]
        [\bf V, calign=last
          [\bf V, tier=preterminal [\it trinken]]
          [\bf V, tier=preterminal [\it möchte, tier=terminal]]
        ]
      ]
    ]
  \end{forest}
\end{frame}


\section{Vorschau}

\begin{frame}
  {Form und Funktion von Sätzen}
  \onslide<+->
  \begin{itemize}[<+->]
    \item Was ist ein "`unabhängiger Satz"'?\\
      \grau{Funktion unabhängiger Sätze}
    \item Hypotaxe und komplexe Sachverhalte
    \item Komplementatz, Adverbialsatz, Relativsatz\\
      in Relation zum Matrixsatz (Semantik)
      \Zeile
    \item Überblick über die Syntax des unabhängigen Satzes\\
      \grau{Feldermodell}
      \Zeile
    \item \citet[13.1, 13.2]{Schaefer2018b}
  \end{itemize}
\end{frame}

  \let\subsection\section\let\section\woopsi

  \section[Sätze]{Sätze}
  \let\woopsi\section\let\section\subsection\let\subsection\subsubsection
  \input{includes/08.+Sätze.tex}
  \let\subsection\section\let\section\woopsi

  \section[Nebensätze]{Nebensätze}
  \let\woopsi\section\let\section\subsection\let\subsection\subsubsection
  \input{includes/09.+Nebensätze.tex}
  \let\subsection\section\let\section\woopsi

  \section[Subjekte\slash Prädikate]{Subjekte und Prädikate}
  \let\woopsi\section\let\section\subsection\let\subsection\subsubsection
  \input{includes/10.+Subjekte+und+Prädikate.tex}
  \let\subsection\section\let\section\woopsi

  \section[Passive\slash Objekte]{Passive und Objekte}
  \let\woopsi\section\let\section\subsection\let\subsection\subsubsection
  
\section{Überblick}

\begin{frame}
  {Passive, Passivierbarkeit und Valenz}
  \begin{itemize}[<+->]
    \item Passivbildungen
    \item Passive als Test für den Ergänzungsstatus
      \Halbzeile
    \item direkte Objekte = Akkusativ-Ergänzungen
    \item indirekte Objekte = Dativ-Ergänzungen
    \item freie Dative = Dativ-Angaben
    \item Präpositionalobjekte = PP-Ergänzungen
  \end{itemize}
\end{frame}

\section{Passive}

\begin{frame}
  {Valenzänderungen | Vorbemerkung}
  \onslide<+->
  \onslide<+->
  \alert{Wir beschreiben Passivbildung als Valenzänderung\ldots}\\
  \Halbzeile
  \begin{itemize}[<+->]
    \item im Prinzip eine Art von \alert{Wortbildung}
    \item Valenz von \textit{kaufen} \{\alert{Nominativ-NP\Sub{1}}, \orongsch{Akkusativ-NP\Sub{2}}\}\\
      → Valenz des Passivs von \textit{kaufen} \{\orongsch{Nominativ-NP\Sub{2}}\}
      \Halbzeile
    \item andere Wortbildungsprozesse mit Valenzänderungen
      \begin{itemize}[<+->]
        \item Valenzanreicherung beim Applikativ \textit{be:}
        \item \textit{geh-en} → \textit{be:geh-en}
        \item Valenzänderung \{Nominativ-NP\Sub{1}\} → \{Nominativ-NP\Sub{1}, Akkusativ-NP\Sub{2}\}
        \item \textit{Ich gehe auf der Straße.} → \textit{Ich begehe die Straße.}
      \end{itemize}
  \end{itemize}
\end{frame}

\begin{frame}
  {\textit{werden}-Passiv oder Vorgangspassiv}
  \pause
  "`Nur transitive Verben können passiviert werden."'\pause\rot{--- Nein!}
  \pause
    \begin{exe}
    \addtolength\itemsep{-0.25\baselineskip}
      \ex\label{ex:werdenpassivundverbtypen110}
      \begin{xlist}\addtolength\itemsep{-0.5\baselineskip}
          \ex[ ]{\label{ex:werdenpassivundverbtypen111} \alert{Johan} wäscht \orongsch{den Wagen}.}
          \ex[ ]{\label{ex:werdenpassivundverbtypen112} \orongsch{Der Wagen} wird \alert{(von Johan)} gewaschen.}
      \end{xlist}
      \pause
      \ex\label{ex:werdenpassivundverbtypen113}
      \begin{xlist}\addtolength\itemsep{-0.5\baselineskip}
          \ex[ ]{\label{ex:werdenpassivundverbtypen114} \alert{Alma} schenkt \gruen{dem Schlossherrn} \orongsch{den Roman}.}
          \ex[ ]{\label{ex:werdenpassivundverbtypen115} \orongsch{Der Roman} wird \gruen{dem Schlossherrn} \alert{(von Alma)} geschenkt.}
      \end{xlist}
      \pause
      \ex\label{ex:werdenpassivundverbtypen116}
      \begin{xlist}\addtolength\itemsep{-0.5\baselineskip}
          \ex[ ]{\label{ex:werdenpassivundverbtypen117} \alert{Johan} bringt \orongsch{den Brief} zur Post.}
          \ex[ ]{\label{ex:werdenpassivundverbtypen118} \orongsch{Der Brief} wird \alert{(von Johan)} zur Post gebracht.}
      \end{xlist}
      \pause
      \ex\label{ex:werdenpassivundverbtypen119}
      \begin{xlist}\addtolength\itemsep{-0.5\baselineskip}
          \ex[ ]{\label{ex:werdenpassivundverbtypen120} \alert{Der Maler} dankt \gruen{den Fremden}.}
          \ex[ ]{\label{ex:werdenpassivundverbtypen121} \gruen{Den Fremden} wird \alert{(vom Maler)} gedankt.}
      \end{xlist}
      \pause
      \ex\label{ex:werdenpassivundverbtypen122}
      \begin{xlist}\addtolength\itemsep{-0.5\baselineskip}
          \ex[ ]{\label{ex:werdenpassivundverbtypen123} \alert{Johan} arbeitet hier immer montags.}
          \ex[ ]{\label{ex:werdenpassivundverbtypen124} Montags wird hier \alert{(von Johan)} immer gearbeitet.}
      \end{xlist}
      \pause
      \ex\label{ex:werdenpassivundverbtypen125}
      \begin{xlist}\addtolength\itemsep{-0.5\baselineskip}
          \ex[ ]{\label{ex:werdenpassivundverbtypen126} \alert{Der Ball} platzt bei zu hohem Druck.}
          \ex[*]{\label{ex:werdenpassivundverbtypen127} Bei zu hohem Druck wird \rot{(vom Ball)} geplatzt.}
      \end{xlist}
      \pause
      \ex\label{ex:werdenpassivundverbtypen128}
      \begin{xlist}\addtolength\itemsep{-0.5\baselineskip}
          \ex[ ]{\label{ex:werdenpassivundverbtypen129} \alert{Der Rottweiler} fällt \gruen{Michelle} auf.}
          \ex[*]{\label{ex:werdenpassivundverbtypen130} \alert{Michelle} wird \rot{(von dem Rottweiler)} aufgefallen.}
      \end{xlist}
    \end{exe}
\end{frame}

\begin{frame}
  {Was passiert beim Vorgangspassiv?}
  \pause
  \begin{itemize}[<+->]
    \item Auxiliar: \textit{werden}, Verbform: Partizip
    \item für Passivierbarkeit relevant: \alert{die Nominativ-Ergänzung!}
      \Halbzeile
    \item \alert{Passivierung = Valenzänderung}:
      \begin{itemize}[<+->]
        \item Nominativ-Ergänzung → optionale \textit{von}-PP-Angabe
        \item eventuelle Akkusativ-Ergänzung → obligatorische Nominativ-Ergänzung
        \item kein Akkusativ: kein "`Subjekt"' = keine Nom-Erg (\textit{es} ist positional)
        \item \grau{Dativ-Ergänzung → Dativ-Ergänzung (usw.)}
        \item \grau{Angaben: keine Änderung}
      \end{itemize}
    \Halbzeile
  \item \alert{nicht passivierbare Verben}?
    \begin{itemize}[<+->]
      \item {ohne }\rot{agentivische}\alert{ Nominativ-Ergänzung}
      \item Achtung! Gilt nur mit prototypischem Charakter\ldots
      \item Siehe Vertiefung 14.2 auf S.~439!
    \end{itemize}
  \end{itemize}
\end{frame}

\begin{frame}
  {Feinere Klassifikation von Verben}
  \onslide<+->
  \begin{itemize}[<+->]
    \item Neuklassifikation vor dem Hintergrund des Vorgangspassivs
    \item Wenn so eine Klassifikation einen Wert haben soll:\\
      \alert{Berücksichtigung der semantischen Rollen unabdinglich!}
    \item Bedingung für Vorgangs-Passiv: \alert{Nom\_Ag}
  \end{itemize} 
  \onslide<+-> 
  \Zeile
  \centering
  \scalebox{0.9}{\begin{tabular}{lllll}
    \toprule
    \textbf{Valenz} & \textbf{Passiv} & \textbf{Name} & \textbf{Beispiel} \\
    \midrule
    \alert{Nom\_Ag} & ja & Unergative & \textit{arbeiten} \\
    Nom & nein & Unakkusative & \textit{platzen} \\
    \alert{Nom\_Ag}, Akk & ja & Transitive & \textit{waschen} \\
    \alert{Nom\_Ag}, Dat & ja & unergative Dativverben & \textit{danken} \\
    Nom, Dat & nein & unakkusative Dativverben & \textit{auf"|fallen} \\
    \alert{Nom\_Ag}, Dat, Akk & ja & Ditransitive & \textit{geben} \\
    \bottomrule
  \end{tabular}}\\
  \raggedright
  \Zeile
  \onslide<+->
  Immer noch nichts als eine reine Bequemlichkeitsterminologie,\\
  um bestimmte (durchaus wichtige) Valenzmuster hervorzuheben.
\end{frame}


\begin{frame}
  {\textit{bekommen}-Passiv oder Rezipientenpassiv}
  \pause
  Es gibt nicht "`das Passiv im Deutschen"'.\\
  \Halbzeile
  \pause
  \begin{exe}
    \ex\label{ex:bekommenpassiv138}
    \begin{xlist}
      \ex[ ]{\small\label{ex:bekommenpassiv139} \gruen{Mein Kollege} bekommt \orongsch{den Wagen} \alert{(von Johan)} gewaschen.}
      \pause
      \ex[ ]{\small\label{ex:bekommenpassiv140} \gruen{Der Schlossherr} bekommt \orongsch{den Roman} \alert{(von Alma)} geschenkt.}
      \pause
      \ex[ ]{\small\label{ex:bekommenpassiv141} \gruen{Mein Kollege} bekommt \orongsch{den Brief} \alert{(von Johan)} zur Post gebracht.}
      \pause
      \ex[ ]{\small\label{ex:bekommenpassiv142} \gruen{Die Fremden} bekommen \alert{(von dem Maler)} gedankt.}
      \pause
      \ex[?]{\small\label{ex:bekommenpassiv143} \gruen{Mein Kollege} bekommt hier immer montags \alert{(von Johan)} gearbeitet.}
      \pause
      \ex[*]{\small\label{ex:bekommenpassiv144} \gruen{Mein Kollege} bekommt bei zu hohem Druck \rot{(von dem Ball)} geplatzt.}
      \pause
      \ex[*]{\small\label{ex:bekommenpassiv145} \gruen{Michelle} bekommt \rot{(von dem Rottweiler)} aufgefallen.}
    \end{xlist}
  \end{exe}
  \pause\Halbzeile
  \alert{Das ist eine Passivbildung, die genauso den Nom\_Ag betrifft\\
  wie das Vorgangspassiv.}
\end{frame}

\begin{frame}
  {Was passiert beim Rezipientenpassiv?}
  \pause
  Alles, was sich verglichen mit Vorgangspassiv nicht unterscheidet, grau.\\
  \Halbzeile
  \pause
  \begin{itemize}[<+->]
    \item Auxiliar: \textit{bekommen} (evtl.\ \textit{kriegen}), \grau{Verbform: Partizip}
    \item \grau{für Passivierbarkeit relevant: die Nominativ-Ergänzung!}
      \Halbzeile
    \item \grau{Passivierung = Valenzänderung}:
      \begin{itemize}[<+->]
        \item \grau{Nominativ-Ergänzung → optionale \textit{von}-PP-Angabe}
        \item eventuelle Akkusativ-Ergänzung: → Akkusativ-Ergänzung
        \item \alert{Dativ-Ergänzung → Nominativ-Ergänzung}
        \item \rot{kein Dativ: kein Rezipientenpassiv}
        \item \grau{Angaben: keine Änderung}
      \end{itemize}
    \Halbzeile
  \item \grau{nicht passivierbare Verben?}
    \begin{itemize}[<+->]
      \item \grau{ohne agentivische Nominativ-Ergänzung}
      \item \grau{Achtung! Gilt nur mit prototypischem Charakter\ldots}
      \item \grau{Siehe Vertiefung 14.2 auf S.~439!}
    \end{itemize}
  \end{itemize}
\end{frame}

\begin{frame}
  {Rezipientenpassiv bei unergativen Verben}
  \pause
  Warum war dieser Satz zweifelhaft?\\
  \begin{exe}
    \ex[?]{\small \gruen{Mein Kollege} bekommt hier immer montags \alert{(von Johan)} gearbeitet.}
  \end{exe}
  \pause
  \Halbzeile
  Ist der zugehörige Aktivsatz besser?\\
  \pause
  \begin{exe}
    \ex[?]{\small Montags arbeitet \alert{Johan} \gruen{meinem Kollegen} hier immer.}
  \end{exe}
  \pause
  \begin{itemize}[<+->]
    \item Nein.
    \item \alert{keine Frage des Rezipientenpassivs}
    \item bei diesen Verben: eher \textit{für}-PP
  \end{itemize}
\end{frame}


\section{Objekte und Valenz}

\begin{frame}
  {Direkte Objekte}
  \onslide<+->
  \onslide<+->
   Kaum anders als beim Subjekt.
  \begin{itemize}[<+->]
    \item \alert{Akkusativ-Ergänzungen zum Verb}
    \item \alert{oder Nebensätze an deren Stelle}
  \end{itemize}
  \onslide<+->
  \Halbzeile
  Und Doppelakkusative?\\
  \onslide<+->
  \begin{exe}
    \ex\label{ex:akkusativeunddirekteobjekte158}
    \begin{xlist}
      \ex[ ]{\label{ex:akkusativeunddirekteobjekte159} Ich lehre \alert{ihn} \orongsch{das Schwimmen}.}
      \onslide<+->
      \ex[*]{\label{ex:akkusativeunddirekteobjekte160} \orongsch{Das Schwimmen} wird \alert{ihn} gelehrt.}
      \onslide<+->
      \ex[*]{\label{ex:akkusativeunddirekteobjekte161} \alert{Er} wird \orongsch{das Schwimmen} gelehrt.}
      \onslide<+->
      \ex[ ]{\label{ex:akkusativeunddirekteobjekte161} Hier wird \orongsch{das Schwimmen} gelehrt.}
    \end{xlist}
  \end{exe}
  \begin{itemize}[<+->]
    \item unterschiedlicher Status der Akkusativ-Ergänzungen
    \item Die "`erste"' entspricht der normaler Transitiva.
    \item \grau{Korrektur zum Buch: Doppelakkusative bilden unpersönliche Passive.}
  \end{itemize} 
\end{frame}

\begin{frame}
  {Indirekte Objekte}
  \pause
  Welche Dative sind Ergänzungen (= Teil der Valenz)?\\
  \pause
  \Halbzeile
  \begin{exe}
    \ex\label{ex:dativeundindirekteobjekte166}
    \begin{xlist}
      \ex[ ]{\label{ex:dativeundindirekteobjekte167} \alert{Alma} gibt \gruen{ihm} heute ein Buch.}
      \pause
      \ex[ ]{\label{ex:dativeundindirekteobjekte168} \alert{Alma} fährt \gruen{mir} heute aber wieder schnell.}
      \pause
      \ex[ ]{\label{ex:dativeundindirekteobjekte169} \alert{Alma} mäht \gruen{mir} heute den Rasen.}
      \pause
      \ex[ ]{\label{ex:dativeundindirekteobjekte170} \alert{Alma} klopft \gruen{mir} heute auf die Schulter.}
    \end{xlist}
  \end{exe}
  \Halbzeile
  \pause
  Recht einfache Entscheidung, da wir Passiv\\
  als \alert{Valenzänderung} beschreiben:\\
  \pause
  \begin{exe}
    \ex\label{ex:dativeundindirekteobjekte171}
    \begin{xlist}
      \ex[ ]{\label{ex:dativeundindirekteobjekte172} \gruen{Er} bekommt \alert{von Alma} heute ein Buch gegeben.}
      \ex[*]{\label{ex:dativeundindirekteobjekte173} \rot{Ich} bekomme \alert{von Alma} heute aber wieder schnell gefahren.}
      \ex[ ]{\label{ex:dativeundindirekteobjekte174} \gruen{Ich} bekomme \alert{von Alma} heute den Rasen gemäht.}
      \ex[ ]{\label{ex:dativeundindirekteobjekte175} \gruen{Ich} bekomme \alert{von Alma} heute auf die Schulter geklopft.}
    \end{xlist}
  \end{exe}
\end{frame}

\begin{frame}
  {Die vier wichtigen verbabhängigen Dative}
  \pause
  \begin{exe}
    \ex\label{ex:dativeundindirekteobjekte166x}
    \begin{xlist}
      \ex{\label{ex:dativeundindirekteobjekte167x} Alma gibt \gruen{ihm} heute ein Buch.}
      \pause
      \ex{\label{ex:dativeundindirekteobjekte168x} Alma fährt \orongsch{mir} heute aber wieder schnell.}
      \pause
      \ex{\label{ex:dativeundindirekteobjekte169x} Alma mäht \alert{mir} heute den Rasen.}
      \pause
      \ex{\label{ex:dativeundindirekteobjekte170x} Alma klopft \alert{mir} heute auf die Schulter.}
    \end{xlist}
  \end{exe}
  \Halbzeile
  \pause
  \begin{itemize}[<+->]
    \item (\ref{ex:dativeundindirekteobjekte167x}) = \gruen{Ergänzung} bei ditransitivem Verb
    \item (\ref{ex:dativeundindirekteobjekte168x}) = \orongsch{Bewertungsdativ} (Angabe, im Vorfeld\slash direkt nach finitem Verb)
    \item (\ref{ex:dativeundindirekteobjekte169x}) = \alert{Nutznießerdativ} (\alert{Ergänzung per Valenzerweiterung})
    \item (\ref{ex:dativeundindirekteobjekte170x}) = \alert{Pertinenzdativ} (\alert{Ergänzung per Valenzerweiterung})
      \Halbzeile
    \item Bewertungsdativ, Nutznießerdativ und Pertinenzdativ\\
      nennt man auch \textit{freie Dative}.
  \end{itemize}
\end{frame}

\begin{frame}
  {Valenzveränderungen im Beispiel}
  \pause
  1.~Wir beginnen mit einem Verb mit \alert{Nom\_Ag} und einem \orongsch{Akk}:\\
  \pause
  \Halbzeile
  \begin{exe}
    \ex \alert{Alma} mäht \orongsch{den Rasen}.
  \end{exe}
  \Zeile
  \pause
  2.~Der \gruen{Nutznießerdativ} wird als Valenzerweiterung hinzugefügt:\\
  \pause
  \Halbzeile
  \begin{exe}
    \ex \alert{Alma} mäht \gruen{meinem Kollegen} \orongsch{den Rasen}.
  \end{exe}
  \Zeile
  \pause
  3.~Das Rezipientenpassiv (Valenzänderung) kann jetzt gebildet werden:
  \pause
  \Halbzeile
  \begin{exe}
    \ex \gruen{Mein Kollege} bekommt \alert{(von Alma)} \orongsch{den Rasen} gemäht.
  \end{exe}
\end{frame}

\begin{frame}
  {Präpositionalobjekte}
  \pause
  PP-Angabe vs.\ PP-Ergänzung: oft schwierig zu entscheiden.\\
  \Viertelzeile
  \pause
  \begin{exe}
    \ex\label{ex:ppergaenzungenundppangaben189}
    \begin{xlist}
      \ex{\label{ex:ppergaenzungenundppangaben190} Viele Menschen leiden \alert{unter Vorurteilen}.}
      \pause
      \ex{\label{ex:ppergaenzungenundppangaben191} Viele Menschen schwitzen \orongsch{unter Sonnenschirmen}.}
    \end{xlist}
  \end{exe}
  \Viertelzeile
  \pause
  \begin{itemize}[<+->]
    \item \alert{Ergänzungen}:
      \begin{itemize}[<+->]
        \item Semantik der PP nur verbgebunden interpretierbar
        \item = semantische Rolle der PP vom Verb zugewiesen
      \end{itemize}
    \item \orongsch{Angaben}:
      \begin{itemize}[<+->]
        \item Semantik der PP selbständig erschließbar (lokal unter)
        \item = "`semantische Rolle"' der PP von der Präposition zugewiesen
      \end{itemize}
      \Viertelzeile
    \item \alert{Sehen Sie, wie schnell man in der (Grund-)Schulgrammatik\\
      in gefährliche linguistische Fahrwasser gerät?}
    \item \rot{Wenn Sie dieses Wissen nicht haben, unterrichten Sie sehr leicht\\
      komplett Falsches, zumal wenn es im Lehrbuch falsch steht.}
  \end{itemize}
\end{frame}


\begin{frame}
  {Der umstrittene PP-Angaben-Test}
  \pause
  Die PP mit \textit{"`Dies geschieht PP."'} aus dem Satz auskoppeln.\\
  \Halbzeile
  \pause
  \begin{exe}
    \ex\label{ex:ppergaenzungenundppangaben192}
    \begin{xlist}
      \ex[*]{\label{ex:ppergaenzungenundppangaben193} Viele Menschen leiden.
      \rot{Dies geschieht unter Vorurteilen.}}
        \pause
      \ex[ ]{\label{ex:ppergaenzungenundppangaben194} Viele Menschen schwitzen.
      \alert{Dies geschieht unter Sonnenschirmen.}}
        \pause
      \ex[*]{\label{ex:ppergaenzungenundppangaben195} Mausi schickt einen Brief.
      \rot{Dies geschieht an ihre Mutter.}}
        \pause
      \ex[*]{\label{ex:ppergaenzungenundppangaben196} Mausi befindet sich.
      \rot{Dies geschieht in Hamburg.}}
        \pause
      \ex[?]{\label{ex:ppergaenzungenundppangaben197} Mausi liegt.
      \orongsch{Dies geschieht auf dem Bett.}}
    \end{xlist}
  \end{exe}
  \Halbzeile
  \pause
  \begin{itemize}[<+->]
    \item der beste Test, den es gibt
    \item trotz Problemen
    \item \rot{Verlangen Sie von Schüler*innen keine Entscheidungen,\\
    die Sie selber nicht operationalisieren können!}
  \end{itemize}
\end{frame}

\ifdefined\TITLE
  \section{Zur nächsten Woche | Überblick}

  \begin{frame}
    {Deutsche Syntax | Plan}
    \rot{Alle} angegebenen Kapitel\slash Abschnitte aus \rot{\citet{Schaefer2018b}} sind \rot{Klausurstoff}!\\
    \Halbzeile
    \begin{enumerate}
      \item Grammatik und Grammatik im Lehramt \rot{(Kapitel 1 und 3)}
      \item Grundbegriffe \rot{(Kapitel 2)}
      \item Wortklassen \rot{(Kapitel 6)}
      \item Konstituenten und Satzglieder \rot{(Kapitel 11 und Abschnitt 12.1)}
      \item Nominalphrasen \rot{(Abschnitt 12.3)}
      \item Andere Phrasen \rot{(Abschnitte 12.2 und 12.4--12.7)}
      \item Verbphrasen und Verbkomplex \rot{(Abschnitte 12.8)}
      \item Sätze \rot{(Abschnitte 12.9 und 13.1--13.3)} 
      \item Nebensätze \rot{(Abschnitt 13.4)}
      \item Subjekte und Prädikate \rot{(Abschnitte 14.1--14.3)}
      \item Passive und Objekte \rot{(14.4 und 14.5)}
      \item \alert{Syntax infiniter Verbformen} \rot{(Abschnitte 14.7--14.9)}
    \end{enumerate}
    \Halbzeile
    \centering 
    \url{https://langsci-press.org/catalog/book/224}
  \end{frame}
\fi

  \let\subsection\section\let\section\woopsi

  \section[Infinitivsyntax]{Syntax infiniter Verbformen}
  \let\woopsi\section\let\section\subsection\let\subsection\subsubsection
  
\section{Überblick}


\begin{frame}
  {Infinitivsyntax}
  \begin{itemize}[<+->]
    \item morphologische vs.\ analytische Tempora
    \item Ersatzinfinitiv und Oberfeldumstellung
      \Halbzeile
    \item kohärente und inkohärente Infinitive
    \item Modalverben und Halbmodale
    \item Kontrollverben
  \end{itemize}
\end{frame}

\section{Analytische Tempora}

\begin{frame}
  {Weitere Arten von Verben}
  \onslide<+->
  \onslide<+->
  Hilfs- und Modalverben mit besonderer Syntax und besonderer Formenbildung
  \onslide<+->
  \Halbzeile
  \begin{exe}
    \ex\label{ex:unterklassen072}
    \begin{xlist}
      \ex{\label{ex:unterklassen073} Frida \alert<9->{isst} den Marmorkuchen.}
      \onslide<+->
      \ex{\label{ex:unterklassen074} Frida \orongsch<10->{hat} den Marmorkuchen \alert<9->{gegessen}.}
      \onslide<+->
      \ex{\label{ex:unterklassen075} Der Marmorkuchen \orongsch<10->{wird} \alert<9->{gegessen}.}
      \onslide<+->
      \ex{\label{ex:unterklassen076} Frida \rot<11->{soll} den Marmorkuchen \alert<9->{essen}.}
      \onslide<+->
      \ex{\label{ex:unterklassen077} Dies hier \gruen<12->{ist} der leckere Marmorkuchen.}
      \onslide<+->
      \ex{\label{ex:unterklassen078} Der Marmorkuchen \gruen<12->{wird} lecker.}
    \end{xlist}
  \end{exe}
  \onslide<+->
  \Halbzeile
  \centering 
  \onslide<9->{\alert{Vollverben\slash lexikalische Verben}}\onslide<10->{, \orongsch{Hilfsverben}}\onslide<11->{, \rot{Modalverben}}\onslide<12->{, \gruen{Kopulaverben}}
\end{frame}

\begin{frame}
  {Welche Tempora hat das Deutsche?}
  \onslide<+->
  \onslide<+->
  Die Schulgrammatik lehrt \alert{sechs Tempusformen}, wir nur \rot{zwei}.\\
  \onslide<+->
  \Zeile
  \begin{center}
    \begin{tabular}[h]{lll}
      \textbf{Präsens}         & \textit{es \alert{geht}}                                     & \onslide<4->{\alert{synthetisch }} \\
      \textbf{Präteritum}      & \textit{es \alert{ging}}                                     & \onslide<4->{\alert{synthetisch }} \\
      && \\
      \textbf{Futur}         & \textit{es \orongsch{wird} \alert{gehen}}                    & \onslide<5->{\orongsch{analytisch }} \\
      && \\
      \textbf{Perfekt}         & \textit{es \orongsch{ist} \alert{gegangen}}                  & \onslide<5->{\orongsch{analytisch }} \\
      \textbf{Plusquamperfekt} & \textit{es \orongsch{war} \alert{gegangen}}                  & \onslide<5->{\orongsch{analytisch }} \\
      \textbf{Futurperfekt}         & \textit{es \orongsch{wird} \alert{gegangen} \orongsch{sein}} & \onslide<5->{\orongsch{analytisch }} \\
    \end{tabular}
  \end{center}
  \Zeile
  \begin{itemize}[<+->]
    \item Nur zwei werden als Form (\alert{synthetisch}) gebildet.
    \item Der Rest wird mit \orongsch{Hilfsverben} und \alert{infiniten Verbformen} (\orongsch{analytisch}) gebildet.
  \end{itemize}
\end{frame}

\begin{frame}
  {Präsens, Präteritum, Futur}
  \onslide<+->
  \begin{itemize}[<+->]
    \item Präsens
      \begin{itemize}[<+->]
        \item kein spezifischer Zeitbezug
        \item synthetische finite Form
      \end{itemize}
      \Viertelzeile
    \item Präteritum
      \begin{itemize}[<+->]
        \item Vergangenheitsbezug
        \item synthetische finite Form
      \end{itemize}
     \Viertelzeile 
    \item Futur
      \begin{itemize}[<+->]
        \item Zukunftsbezug oder Absichtserklärung
        \item analytische Form mit \rot{stets finitem} Hilfsverb
      \end{itemize}
  \end{itemize}
  \onslide<+->
  \Halbzeile
  \hspace{3em}\scalebox{0.8}{\begin{minipage}{\textwidth}
    \begin{exe}
      \onslide<11->{\ex[ ]{\ldots\ dass ich \alert{gehen werde}.}}
      \onslide<12->{\ex[*]{\ldots\ dass ich \rot{gehen werden} möchte.}}
      \onslide<13->{\ex[*]{\ldots\ dass ich \rot{gehen geworden} habe\slash bin.}}
      \onslide<14->{\ex[*]{\ldots\ dass ich \rot{gehen zu werden} habe.}}
    \end{exe}
  \end{minipage}}
\end{frame}

\begin{frame}
  {Perfekt}
  \onslide<+->
  \onslide<+->
  \rot{Das Perfekt ist nicht intrinsisch finit!}\\
  \onslide<+->
  \Viertelzeile
  Es kann daher im Infinitiv und in den drei finiten Tempora stehen.\\
  \Zeile
  \begin{itemize}[<+->]
    \item Hilfsverb \orongsch{sein} oder \orongsch{haben} + \alert{Partizip} des anderen Verbs
      \Halbzeile
    \item Infinitiv des Perfekts | \alert{gegangen} (Partizip) \orongsch{sein} (Inf des HVs)
    \item Präsens des Perfekts | \alert{gegangen} (Partizip) \orongsch{bin\slash bist\slash ist\slash\ldots} (Präs des HVs)
    \item Präteritum des Perfekts | \alert{gegangen} (Partizip) \orongsch{war\slash warst\slash\ldots} (Prät des HVs)
    \item Futur des Perfekts | \alert{gegangen} (Partizip) \orongsch{sein werde\slash wirst\slash wird\slash\ldots} (Futur des HVs)
  \end{itemize}
\end{frame}

\begin{frame}
  {Unterschiede zwischen Präteritum und Präsensperfekt}
  Stilistische Unterschiede\\
  \Halbzeile
  \begin{exe}
  \ex\label{ex:analytischetempora226}
  \begin{xlist}
    \ex{\label{ex:analytischetempora227} Das Pferd \alert{lief} im Kreis.}
    \ex{\label{ex:analytischetempora228} Das Pferd \orongsch{ist} im Kreis \alert{gelaufen}.}
  \end{xlist}
  \end{exe}
  \Zeile
  Semantische Unterschiede\\
  \Halbzeile
  \begin{exe}
  \ex\label{ex:analytischetempora229}
  \begin{xlist}
    \ex{\label{ex:analytischetempora230} Im Jahr 1993 \orongsch{hat} der Kommerz den Techno \alert{erobert}.}
    \ex{\label{ex:analytischetempora231} Im Jahr 1993 \alert{eroberte} der Kommerz den Techno.}
  \end{xlist}
  \onslide<+->
  \centering 
  Nicht alle Sprecher können die Lesarten differenzieren.
  \end{exe}
\end{frame}

\begin{frame}
  {Zusammenfassung | Finite Tempora und Perfekt}
  \onslide<+->
  \onslide<+->
  Klare Beziehungen zwischen den finiten Tempora und dem Perfekt\\
  \Zeile
  \begin{itemize}[<+->]
    \item Finite Tempora
      \begin{itemize}[<+->]
        \item Präsens | finite synthetische Form
        \item Präteritum | finite synthetische Form
        \item Futur (= Futur 1) | analytisch mit stets finitem Hilfsverb
      \end{itemize}
     \Zeile 
    \item \alert{Perfekta mit finiten Tempusformen des Hilfsverbs}
      \begin{itemize}[<+->]
        \item Präsensperfekt (= Perfekt) | Präsensform des Perfekts
        \item Präteritumsperfekt (= Plusquamperfekt) | Präteritalform des Perfekts
        \item Futurperfekt (= Futur 2) | Futur des Perfekts
      \end{itemize}
  \end{itemize}
\end{frame}


\begin{frame}
  {Analysen als Verbkomplex}
  \onslide<+->
  \onslide<+->
  Hilfsverben\slash Modalverben | \alert{Rektion des Status des anderen Verbs}\\
  \Halbzeile
  \centering
  \scalebox{0.85}{\begin{forest}
    [\textbf{V}, calign=last
      [\textbf{V}, calign=last
        [\textbf{V}, calign=last
          [\textbf{V}, tier=preterminal
            [\textit{behuft}]
          ]
          [\textbf{V}, tier=preterminal
            [\textit{gehabt}]
            {\draw [->, bend left=45] (.south) to node [below, midway] {\footnotesize\textsc{Status} (3)} (!uu11.south);}
          ]
        ]
        [\textbf{V}, tier=preterminal
          [\textit{haben}]
          {\draw [->, bend left=45] (.south) to node [below, midway] {\footnotesize\textsc{Status} (3)} (!uu121.south);}
        ]
      ]
      [\textbf{V}, tier=preterminal
        [\textit{will}]
        {\draw [->, bend left=45] (.south) to node [below, midway] {\footnotesize\textsc{Status} (1)} (!uu121.south);}
      ]
    ]
  \end{forest}}
\end{frame}




\begin{frame}
  {Nichtkanonische Infinitivrektion}
  \onslide<+->
  \onslide<+->
  Die sogenannte \alert{Oberfeldumstellung mit Ersatzinfinitiv}\\
  \Halbzeile
  \onslide<+->
  \begin{exe}
    \ex{\label{ex:ersatzinfinitivundoberfeldumstellung238} dass der Junge [\rot{hat} [[schwimmen] \rot{wollen}]]}
  \end{exe}
  \Zeile
  \onslide<+->
  \centering
  \scalebox{0.85}{\begin{forest}
    [\textbf{V}, calign=first
      [\textbf{V}, tier=preterminal
        [\textit{hat}\\1]
      ]
      [\textbf{V}, calign=last
        [\textbf{V}, tier=preterminal
          [\textit{schwimmen}\\3]
        ]
        [\textbf{V}, tier=preterminal
          [\textit{wollen}\\2]
          {\draw [->, bend left=20] (.south) to node [below, near end] {\footnotesize\textsc{Status} (1)} (!uu11.south);}
          {\draw [<-, bend left=60] (.south) to node [below, midway] {\footnotesize\textsc{Status} (1)} (!uuu11.south);}
        ]
      ]
    ]
  \end{forest}}
\end{frame}


\section{Infinitivsyntax}


\begin{frame}
  {Syntaktische Katgeorie von Infinitivphrasen}
  \onslide<+->
  \onslide<+->
  \alert{Infinitivphrasen mit Ergänzungen und Angaben} (\ref{ex:infvp}) vs.\ \orongsch{reine Infinitive} (\ref{ex:infv})\\
  \onslide<+->
  \Viertelzeile
  \begin{exe}
    \ex{\ldots\ dass Vanessa \alert{[das Pferd zu reiten]} scheint\label{ex:infvp}}
    \onslide<+->
    \ex{\ldots\ dass Vanessa \orongsch{[zu reiten]} scheint\label{ex:infv}}
  \end{exe}
  \onslide<+->
  \Halbzeile
  Da Infinitive kein Subjekt regieren, sind es VPs ohne Subjekt\\
  \Viertelzeile
  \centering 
  \onslide<+->
  \begin{forest}
    [VP, calign=last
      [NP
        [das Pferd, narroof]
      ]
      [V
        [\it zu reiten]
      ]
    ]
  \end{forest}
\end{frame}


\begin{frame}
  {Kommas bei \textit{Infinitvkonstruktionen}}
  \onslide<+->
  \onslide<+->
  Komma oder nich?
  \onslide<+->
  \begin{exe}
    \ex[*]{Nadezhda \rot{scheint}, die Kontrolle über die Hantel zu verlieren.}
    \ex[*]{Nadezhda \rot{will}, die Weltmeisterschaft gewinnen.}
    \ex[ ]{Nadezhda \alert{beschließt}, keine Steroide mehr einzunehmen.}
    \ex[?]{Nadezhda \alert{beschließt}\orongsch{,} zu trainieren.}
  \end{exe}
  \Zeile
  \begin{itemize}[<+->]
    \item \alert{Infinitivsyntax} ist der Schlüssel
    \item Komma nur bei \alert{inkohärenten Infinitiven}
  \end{itemize}
\end{frame}

\begin{frame}
  {(In)kohärente Infinitive}
  Kohärente und inkohärente Infinitivkonstruktionen\\
  \onslide<+->
  \Zeile
  \centering
  \scalebox{0.7}{\begin{minipage}{0.4\textwidth}
    \vspace{1.15cm}
    \begin{forest}
      [VP\Sub{1+2}, calign=last
        [NP, tier=preterminal
          [\textit{Vanessa}, narroof]
        ]
        [NP, tier=preterminal
          [\textit{die Pferde}, narroof]
        ]
        [\textbf{V\Sub{2+1}}, calign=last
          [\textbf{V\Sub{2}}, tier=preterminal
            [\textit{behufen}]
          ]
          [\textbf{V\Sub{1}}, tier=preterminal
            [\textit{will}]
          ]
        ]
      ]
    \end{forest}
  \end{minipage}}\hspace{0.1\textwidth}\scalebox{0.7}{\begin{minipage}{0.4\textwidth}
    \begin{forest}
      l sep+=3em, s sep+=2em
      [VP\Sub{1}, calign=last
        [NP, tier=preterminal
          [\textit{Vanessa}, narroof]
        ]
        [VP\Sub{2}, calign=last
          [NP, tier=preterminal
            [\textit{die Pferde}, narroof]
          ]
          [\textbf{V\Sub{2}}, tier=preterminal
            [\textit{zu behufen}]
          ]
        ]
        [\textbf{V\Sub{1}}, tier=preterminal
          [\textit{wünscht}]
        ]
      ]
    \end{forest}
  \end{minipage}}
\end{frame}


\begin{frame}
  {Test | Herausstellbarkeit}
  \onslide<+->
  \onslide<+->
  In der \rot{kohärenten} Konstruktion bildet der Infinitiv mit seinen Ergänzungen und Angaben keine Konstituente, also kann diese auf nicht nach rechts herausgestellt werden.\\
  \Zeile
  \onslide<+->
  \begin{exe}
    \ex[*]{Oma glaubt, dass Vanessa \rot{\Ti}\ will, \rot{[die Pferde behufen]\ORi}.}
  \end{exe}
  \onslide<+->
  \Zeile
  In der \gruen{inkohärenten} Konstruktion bildet der Infinitiv eine solche Konstituente.\\
  \Zeile
  \onslide<+->  
  \begin{exe}
    \ex[ ]{Oma glaubt, dass Vanessa \gruen{\Ti}\ wünscht, \gruen{[die Pferde zu behufen]\ORi}.}
  \end{exe}
\end{frame}


\begin{frame}
  {Halbmodale}
  \onslide<+->
  \onslide<+->
  Scheinbar gleich strukturiert | \gruen{wollen}, \orongsch{scheinen}, \alert{beschließen}\\
  \Halbzeile
  \onslide<+->
  \begin{exe}
  \ex
  \begin{xlist}
    \ex{dass der Hufschmied \gruen{das Pferd behufen will}.}
    \ex{dass der Hufschmied \orongsch{das Pferd zu behufen scheint}.}
    \ex{dass der Hufschmied \alert{das Pferd zu behufen beschließt}.}
  \end{xlist}
  \end{exe}
  \onslide<+->
  \Zeile
  Aber Abweichung bei der Extrahierbarkeit\\
  \Halbzeile
  \onslide<+->
  \begin{exe}
  \ex
  \begin{xlist}
    \ex[*]{dass der Hufschmied \gruen{\Ti}\ will, \gruen{[das Pferd behufen]\ORi}.}
    \ex[*]{dass der Hufschmied \orongsch{\Ti}\ scheint, \orongsch{[das Pferd zu behufen]\ORi}.}
    \ex[ ]{dass der Hufschmied \alert{\Ti}\ beschließt, \alert{[das Pferd zu behufen]\ORi}.}
  \end{xlist}
\end{exe}
\end{frame}

\begin{frame}
  {Halbmodale | \textit{scheinen} ohne Subjektrolle}
  \onslide<+->
  \onslide<+->
  Subjekt von \textit{scheinen} nicht erfragbar\\
  \onslide<+->
  \Halbzeile
  \begin{exe}
    \ex
    \begin{xlist}
      \ex[ ]{Frage: Wer \gruen{will} das Pferd behufen?\\
      Antwort: \gruen{Der Hufschmied will} das.}
      \onslide<+->
      \ex[*]{Frage: Wer \orongsch{scheint} das Pferd zu behufen?\\
      Antwort: \orongsch{Der Hufschmied scheint} das.}
      \onslide<+->
      \ex[ ]{Frage: Wer \alert{beschließt}, das Pferd zu behufen?\\
      Antwort: \alert{Der Hufschmied beschließt} das.}
    \end{xlist}
  \end{exe}
  \Zeile
  \onslide<+->
  Und \textit{scheinen} kann kein subjektloses Verb einbetten\\
  \Halbzeile
  \onslide<+->
  \begin{exe}
    \ex
    \begin{xlist}
      \ex[*]{Dem Hufschmied \gruen{will} grauen.}
      \onslide<+->
      \ex[ ]{Dem Hufschmied \orongsch{scheint} zu grauen}
      \onslide<+->
      \ex[*]{Dem Hufschmied \alert{beschließt} zu grauen.}
    \end{xlist}
  \end{exe}
\end{frame}


\begin{frame}
  {(In)kohärente Infinitve}
  \onslide<+->
  \onslide<+->
    \resizebox{1\textwidth}{!}{
    \begin{tabular}{lcllll}
      \lsptoprule
      & \multirow{2}{*}{\textbf{Status}} & \multirow{2}{*}{\textbf{Kohärenz}} & \textbf{eigenes} & \textbf{Subjekts-} \\
      & & & \textbf{Subjekt} & \textbf{Rolle} & \textbf{Beispiel}\\
      \midrule
      \textbf{Modalverben} & 1 & obl.\ kohärent & ja & Identität & \textit{wollen} \\
      \textbf{Halbmodalverben} & 2 & obl.\ kohärent & nein & nein & \textit{scheinen} \\
      \textbf{Kontrollverben} & 2 & \rot{opt.\ inkohärent} & ja & Kontrolle & \textit{beschließen} \\
      \lspbottomrule
    \end{tabular}
  }\\
  \Zeile
  \begin{itemize}[<+->]
    \item Nur \alert{inkohärente nachgestellte Infinitive} werden kommatiert!
    \item Sie gelten als satzwertig, aber die \rot{Inkohärenz ist leider nur optional}.
    \item Es kommen also nur \alert{Abhängige von Kontrollverben} infrage.
  \end{itemize}
  \onslide<+->
  \Viertelzeile
  \begin{exe}
    \ex[*]{Nadezhda \rot{scheint}, die Kontrolle über die Hantel zu verlieren.}
    \ex[*]{Nadezhda \rot{will}, die Weltmeisterschaft gewinnen.}
  \end{exe}
\end{frame}

\begin{frame}
  {(In)kohärente Infinitve}
  \onslide<+->
  \onslide<+->
  Was ist jetzt hiermit?\\
  \Halbzeile
  \onslide<+->
  \begin{exe}
    \ex[ ]{Nadezhda \alert{beschließt}, keine Steroide mehr einzunehmen.}
    \ex[?]{Nadezhda \alert{beschließt}\orongsch{,} zu trainieren.}
  \end{exe}
  \onslide<+->
  \Halbzeile
  \alert{Eindeutig inkohärent} | hinter die RSK versetzte Infinitive\\
  \Viertelzeile
  \onslide<+->
  \begin{exe}
    \ex \rot{\textbf{Inkohärent}}
    \begin{xlist}
      \ex[ ]{\ldots dass Nadezhda beschließt, keine Steroide mehr zu nehmen.}
      \ex[?]{\ldots dass Nadezhda keine Steroide mehr zu nehmen beschließt.}
    \end{xlist}
    \ex \alert{\textbf{Kohärent oder inkohärent}}
    \begin{xlist}
      \ex[ ]{\ldots dass Nadezhda zu trainieren beschließt.}
      \ex[ ]{\ldots dass Nadezhda beschließt zu trainieren.}
    \end{xlist}
  \end{exe}
\end{frame}


\begin{frame}
  {(In)kohärente Infinitve}
  Es liegt also an der syntaktischen Struktur.\\
  \Zeile
  \onslide<+->
  \begin{exe}
    \ex
    \begin{xlist}
      \ex[ ]{[Nadezhda]\Sub{2} \alert{[beschließt]\Sub{1}} [[t\Sub{2} \gruen{t\Sub{3}} \alert{[t\Sub{1}]\Sub{VK}}]\ \Sub{VP}\ \orongsch{,}\\
      {\hspace{1em}}\gruen{[keine Steroide mehr einzunehmen]\Sub{3}}]\Sub{VP}.}
        \Viertelzeile
      \ex[*]{[Nadezhda]\Sub{2} \rot{[beschließt]\Sub{1}}\\
      {\hspace{1em}}[t\Sub{2} [keine Steroide] [mehr] \rot{[einzunehmen t\Sub{1}]\Sub{VK}}\ ]\Sub{VP}.\label{ex:ohweia}}
    \end{xlist}
    \Halbzeile
    \ex
    \begin{xlist}
      \ex[ ]{[Nadezhda]\Sub{2} \alert{[beschließt]\Sub{1}}\ \orongsch{,} [[t\Sub{2} \gruen{t\Sub{3}} \alert{[t\Sub{1}]\Sub{VK}}\ ]\Sub{VP} \gruen{[zu trainieren]\Sub{3}}]\Sub{VP}.}
      \Viertelzeile
      \ex[ ]{[Nadezhda]\Sub{2} \tuerkis{[beschließt]\Sub{1}} [t\Sub{2} \tuerkis{[zu trainieren t\Sub{1}]\Sub{VK}}\ ]\Sub{VP}}
    \end{xlist}
  \end{exe}
  \Halbzeile
  \onslide<+->
  Füllen Sie den VK durch Hinzufügen von Hilfsverben auf,\\
  um das Phänomen noch deutlicher zu sehen.
\end{frame}

\begin{frame}
  {Bäume | Inkohärent}
  \onslide<+->
  \onslide<+->
  \rot{Inkohärent konstruiert}\\
  \Zeile
  \centering 
  \begin{forest}
    [S, calign=child, calign child=2
      [NP\Sub{2}, tier=pt
        [\it Nadezhda, narroof, tier=t]
      ]
      [V\Sub{1}, tier=pt
        [\it beschließt, tier=t]
      ]
      [VP, calign=child, calign child=1
        [VP, calign=child, calign child=3
          [\Tii, tier=t]
          [\rot{\Tiii}, tier=t]
          [\Ti, tier=t]
        ]
        [VP\Sub{3}, tier=pt, rottree
          [\it keine Steroide mehr einzunehmen, narroof, tier=t]
        ]
      ]
    ]
  \end{forest}
\end{frame}


\begin{frame}
  {Bäume | Inkohärent mit Hilfsverb}
  \onslide<+->
  \onslide<+->
  Dank des Verbs im Verbkomplex \rot{sieht man die Extraktion}\\
  \Zeile
  \centering 
  \begin{forest}
    [S, calign=child, calign child=2
      [NP\Sub{2}, tier=pt, tier=pt
        [\it Nadezhda, tier=t, narroof]
      ]
      [V\Sub{1}, tier=pt
        [\it hat, tier=t]
      ]
      [VP, , calign=child, calign child=1
        [VP, calign=last
          [\Tii, tier=t, forky]
          [\rot{\Tiii}, tier=t, forky]
          [V, calign=last
            [V, tier=pt
              [\it beschlossen, tier=t]
            ]
            [\Ti, tier=t]
          ]
        ]
        [VP\Sub{3}, tier=pt, rottree
          [\it keine Steroide mehr einzunehmen, narroof, tier=t]
        ]
      ]
    ]
  \end{forest}
\end{frame}


\begin{frame}
  {Bäume | Kohärent mit Hilfsverb}
  \onslide<+->
  \onslide<+->
  \rot{So gut wie ungrammatisch!}\\
  \Zeile
  \centering 
  \begin{forest}
    [S, calign=child, calign child=2
      [NP\Sub{2}, tier=pt
        [\it Nadezhda, tier=t, narroof]
      ]
      [V\Sub{1}, tier=pt
        [\it hat, tier=t]
      ]
      [VP, calign=last
        [\Tii, tier=t, forky]
        [NP, tier=pt
          [\it keine Steroide, narroof, tier=t]
        ]
        [AdvP, tier=pt
          [\it mehr, narroof, tier=t]
        ]
        [V, calign=last
          [V, calign=last
            [V
              [\it einzunehmen]
            ]
            [V, tier=pt
              [\it beschlossen, tier=t]
            ]
          ]
          [\Ti, tier=t]
        ]
      ]
    ]
  \end{forest}
\end{frame}

\begin{frame}
  {Bäume | Kohärent ohne Hilfsverb}
  \onslide<+->
  \onslide<+->
  Man kann daher davon ausgehen, dass diese Struktur auch nicht grammatisch ist.\\
  \onslide<+->
  \Viertelzeile
  Sie entspricht (\ref{ex:ohweia}), also der nicht kommatierten Version.\\
  \onslide<+->
  \Zeile
  \centering
  \begin{forest}
    [S, calign=child, calign child=2
      [NP\Sub{2}, tier=pt
        [\it Nadezhda, tier=t, narroof]
      ]
      [V\Sub{1}, tier=pt
        [\it beschließt, tier=t]
      ]
      [VP, calign=last
        [\Tii, tier=t, forky]
        [VP\Sub{3}, tier=pt, rottree
          [\it keine Steroide, narroof, tier=t]
        ]
        [AdvP, tier=pt
          [\it mehr, narroof]
        ]
        [V, calign=last
          [V, tier=pt
            [\it einzunehmen]
          ]
          [\Ti, tier=t]
        ]
      ]
    ]
  \end{forest}
\end{frame}

\section{Kontrollinfinitive}

\begin{frame}
  {\textit{zu}-Infinitive als Subjekte und Objekte}
  \onslide<+->
  \onslide<+->
  \alert{Controller} | Logisches Argument des Verbs, das die Bedeutung\\
  des fehlenden Subjekts des Infinitivs beisteuert\\
  \onslide<+->
  \Halbzeile
  \begin{exe}
  \ex\label{ex:infinitivkontrolle264}
  \begin{xlist}
    \ex{\label{ex:infinitivkontrolle265} [Das Geschirr \gruen{zu spülen}] \alert{nervt} Matthias. (Objektkontrolle)}\\
    \onslide<+->
    \Viertelzeile
    Matthias | der \alert{Genervte} (Objekt) und der \gruen{Spülende}\\
    \Halbzeile
    \onslide<+->
    \ex{\label{ex:infinitivkontrolle266} Doro wagt, [die Küche \gruen{zu betreten}]. (Subjektkontrolle)}\\
    \onslide<+->
    \Viertelzeile
    Doro | die \alert{Wagende} (Subjekt) und die \gruen{Betrende}
  \end{xlist}
\end{exe}
\Zeile
\onslide<+->
Auch mit Korrelat\\
\Halbzeile
\begin{exe}
  \ex\label{ex:infinitivkontrolle267}
  \begin{xlist}
    \ex{\label{ex:infinitivkontrolle268} Es nervt Matthias, [das Geschirr zu spülen].}
    \ex{\label{ex:infinitivkontrolle269} Doro wagt es, [die Küche zu betreten].}
  \end{xlist}
\end{exe}
\end{frame}

\begin{frame}
  {Kontrolle im Passiv}
  \onslide<+->
  \onslide<+->
  Kontrolle bleibt im Passiv erhalten | \alert{logische Valenz}, nicht Syntax\\
  \Halbzeile
  \onslide<+->
  \begin{exe}
  \ex\label{ex:infinitivkontrolle270}
  \begin{xlist}
    \ex{\label{ex:infinitivkontrolle271} Der Installateur hat gestern \alert{versucht}, die Küche \gruen{zu betreten}.}\\
    \onslide<+->
    \Viertelzeile
    der Installateur | der \alert{Versuchende} (Subjekt) und der \gruen{Betrende}\\
    \onslide<+->
    \Halbzeile
    \ex{\label{ex:infinitivkontrolle272} Gestern wurde (vom Installateur) versucht, die Küche zu betreten.}\\
    \Viertelzeile
    \onslide<+->
    der Installateur | der \alert{Versuchende} (Subjekt des Aktivs) und der \gruen{Betrende}\\
  \end{xlist}
\end{exe}
\end{frame}

\begin{frame}
  {Kontrolle}
  \begin{block}
    {Infinitivkontrolle}
Die \textit{Kontrollrelation} besteht zwischen einer nominalen Valenzstelle eines Verbs und einem von diesem Verb abhängigen (subjektlosen) \textit{zu}"=Infinitiv.
Die Bedeutung des nicht ausgedrückten Subjekts des abhängigen \textit{zu}"=Infinitivs wird dabei durch die mit der nominalen Valenzstelle verbundene Bedeutung beigesteuert.
  \end{block}
\end{frame}


\begin{frame}
  {Subjektinfinitive}
  \onslide<+->
  \onslide<+->
  Objektkontrolle präferiert\\
  \onslide<+->
  \Halbzeile
  \begin{exe}
  \ex\label{ex:infinitivkontrolle274}
  \begin{xlist}
    \ex{\label{ex:infinitivkontrolle275} Das Geschirr zu spülen, nervt \gruen{ihn}.\\
    Controller | \gruen{Akkusativobjekt}}
    \onslide<+->
    \Viertelzeile
    \ex{\label{ex:infinitivkontrolle276} Das Geschirr zu spülen, fällt \gruen{ihm} leicht.\\
    Controller | \gruen{Dativobjekt}}
    \onslide<+->
    \Viertelzeile
    \ex{\label{ex:infinitivkontrolle277} Das Geschirr zu spülen, beschert \gruen{ihm} einen zufriedenen Mitbewohner.\\
    Controller | \gruen{Dativobjekt}}
    \onslide<+->
    \Viertelzeile
    \ex{\label{ex:infinitivkontrolle278} Sich für Hilfe zu bedanken, freut \gruen{ihn} immer besonders.\\
    Controller | \gruen{Akkusativobjekt}}
  \end{xlist}
\end{exe}
\end{frame}

\begin{frame}
  {Objektinfinitive}
  \onslide<+->
  \onslide<+->
  Objektkontrolle präferiert, falls Objekte vorhanden\\
  \onslide<+->
  \Halbzeile
  \begin{exe}
  \ex\label{ex:infinitivkontrolle279}
  \begin{xlist}
    \ex{\label{ex:infinitivkontrolle280} \gruen{Er} wagt, die Küche zu betreten.\\
    Controller | \gruen{Subjekt}}
    \onslide<+->
    \Viertelzeile
    \ex{\label{ex:infinitivkontrolle281} Er bittet \gruen{seinen Mitbewohner}, das Geschirr zu spülen.\\
    Controller | \gruen{Akkusativobjekt}}
    \onslide<+->
    \Viertelzeile
    \ex{\label{ex:infinitivkontrolle282} Doro erlaubt \gruen{Matthias}, sich den Wagen zu leihen.\\
    Controller | \gruen{Dativobjekt}}
  \end{xlist}
\end{exe}
\end{frame}

\begin{frame}
  {Infinitivangaben}
  \onslide<+->
  \onslide<+->
  Immer Subjektkontrolle
  \begin{exe}
  \ex\label{ex:infinitivkontrolle283}
  \begin{xlist}
    \ex{\label{ex:infinitivkontrolle284} \gruen{Matthias} arbeitet, um Geld zu verdienen.\\
    Controller | \gruen{Subjekt}}
    \onslide<+->
    \Viertelzeile
    \ex{\label{ex:infinitivkontrolle285} \gruen{Matthias} begrüßt Doro, ohne aus der Rolle zu fallen.\\
    Controller | \gruen{Subjekt}}
    \onslide<+->
    \Viertelzeile
    \ex{\label{ex:infinitivkontrolle286} \gruen{Matthias} hilft Doro, anstatt untätig daneben zu stehen.\\
    Controller | \gruen{Subjekt}}
    \onslide<+->
    \Viertelzeile
    \ex{\label{ex:infinitivkontrolle287} \gruen{Matthias} bringt Doro den Wagen zurück, ohne den Lackschaden \\zu erwähnen.\\
    Controller | \gruen{Subjekt}}
  \end{xlist}
\end{exe}
\end{frame}


\section{Vor der Klausur | Überblick}

\begin{frame}
  {Deutsche Syntax | Plan}
  \rot{Alle} angegebenen Kapitel\slash Abschnitte aus \rot{\citet{Schaefer2018b}} sind \rot{Klausurstoff}!\\
  \Halbzeile
  \begin{enumerate}
    \item Grammatik und Grammatik im Lehramt \rot{(Kapitel 1 und 3)}
    \item Grundbegriffe \rot{(Kapitel 2)}
    \item Wortklassen \rot{(Kapitel 6)}
    \item Konstituenten und Satzglieder \rot{(Kapitel 11 und Abschnitt 12.1)}
    \item Nominalphrasen \rot{(Abschnitt 12.3)}
    \item Andere Phrasen \rot{(Abschnitte 12.2 und 12.4--12.7)}
    \item Verbphrasen und Verbkomplex \rot{(Abschnitte 12.8)}
    \item Sätze \rot{(Abschnitte 12.9 und 13.1--13.3)} 
    \item Nebensätze \rot{(Abschnitt 13.4)}
    \item Subjekte und Prädikate \rot{(Abschnitte 14.1--14.3)}
    \item Passive und Objekte \rot{(14.4 und 14.5)}
    \item Syntax infiniter Verbformen \rot{(Abschnitte 14.7--14.9)}
  \end{enumerate}
  \Halbzeile
  \centering 
  \url{https://langsci-press.org/catalog/book/224}
\end{frame}

  \let\subsection\section\let\section\woopsi

  \section{Vor der Klausur}

  \begin{frame}
    {Deutsche Syntax | Plan}
    \rot{Alle} angegebenen Kapitel\slash Abschnitte aus \rot{\citet{Schaefer2018b}} sind \rot{Klausurstoff}!\\
    \Halbzeile
    \begin{enumerate}
      \item Grammatik und Grammatik im Lehramt \rot{(Kapitel 1 und 3)}
      \item Grundbegriffe \rot{(Kapitel 2)}
      \item Wortklassen \rot{(Kapitel 6)}
      \item Konstituenten und Satzglieder \rot{(Kapitel 11 und Abschnitt 12.1)}
      \item Nominalphrasen \rot{(Abschnitt 12.3)}
      \item Andere Phrasen \rot{(Abschnitte 12.2 und 12.4--12.7)}
      \item Verbphrasen und Verbkomplex \rot{(Abschnitte 12.8)}
      \item Sätze \rot{(Abschnitte 12.9 und 13.1--13.3)} 
      \item Nebensätze \rot{(Abschnitt 13.4)}
      \item Subjekte und Prädikate \rot{(Abschnitte 14.1--14.3)}
      \item Passive und Objekte \rot{(14.4 und 14.5)}
      \item Syntax infiniter Verbformen \rot{(Abschnitte 14.7--14.9)}
    \end{enumerate}
    \Halbzeile
    \centering 
    \url{https://langsci-press.org/catalog/book/224}
  \end{frame}
\fi


\makeatletter
\setcounter{lastpagemainpart}{\the\c@framenumber}
\makeatother

\appendix

\begin{frame}[allowframebreaks]
  {Literatur}
  \renewcommand*{\bibfont}{\footnotesize}
  \setbeamertemplate{bibliography item}{}
  \printbibliography
\end{frame}

\begin{frame}
  {Autor}
  \begin{block}{Kontakt}
    Prof.\ Dr.\ Roland Schäfer\\
    Institut für Germanistische Sprachwissenschaft\\
    Friedrich-Schiller-Universität Jena\\
    Fürstengraben 30\\
    07743 Jena\\[\baselineskip]
    \url{https://rolandschaefer.net}\\
    \texttt{roland.schaefer@uni-jena.de}
  \end{block}
\end{frame}

\begin{frame}
  {Lizenz}
  \begin{block}{Creative Commons BY-SA-3.0-DE}
    Dieses Werk ist unter einer Creative Commons Lizenz vom Typ \textit{Namensnennung - Weitergabe unter gleichen Bedingungen 3.0 Deutschland} zugänglich.
    Um eine Kopie dieser Lizenz einzusehen, konsultieren Sie \url{http://creativecommons.org/licenses/by-sa/3.0/de/} oder wenden Sie sich brieflich an Creative Commons, Postfach 1866, Mountain View, California, 94042, USA.
  \end{block}
\end{frame}

\mode<beamer>{\setcounter{framenumber}{\thelastpagemainpart}}

\end{document}
